\documentclass{ximera}
\graphicspath{     %% setup a global graphics path
{./}               %% look in the same-level directory
{./pictures/}      %% look in graphics
{../pictures/}     %% look up one directory, then in graphics
%{../../pictures/} %% look up two directories, then in graphics
}

\author{Zack Reed}
\begin{document}

%matrix_neg_and_transpose
\begin{problem}
  For each matrix $A$, find the matrix $-A$ and $A^T$.
  \begin{enumerate}
    \item
    $A = \startmat{cc}
      1 & 2 \\
      2 & 1
    \stopmat$

    \begin{hint}
    
      In MATLAB, $A$ is defined as \texttt{A = [1, 2;2, 1]}, and you can use the \texttt{-} operator to negate a matrix. For example, \texttt{-A} will give you $-A$.

      Taking the transpose of a matrix in MATLAB is done by using the \texttt{'} operator. For example, \texttt{A'} will give you $A^T$. Alternatively, you can use the \texttt{transpose} function. For example, \texttt{transpose(A)} will also give you $A^T$.

      %Remember from property BLAH that $cA$ is found by multiplying each entry of $A$ by $c.

      %Remeber from property BLAH that $A^T$ is found by switching the rows and columns of $A$.

    \end{hint}


    \begin{enumerate}
      \item $-A = 
      \startmat{cc}
      \answer{-1} & \answer{-2} \\
      \answer{-2} & \answer{-1}
      \stopmat$

      \item $A^T =
      \startmat{cc}
      \answer{1} & \answer{2} \\
      \answer{2} & \answer{1}
      \stopmat$
    \end{enumerate}
  
\item
    $A = \startmat{cc}
      -2 & 3 \\
      0 & 2
    \stopmat$

    \begin{enumerate}
      \item $-A = 
      \startmat{cc}
      \answer{2} & \answer{-3} \\
      \answer{0} & \answer{-2}
      \stopmat$

      \item $A^T =
      \startmat{cc}
      \answer{-2} & \answer{0} \\
      \answer{3} & \answer{2}
      \stopmat$
    \end{enumerate}

\item
    $A = \startmat{ccc}
      0 & 1 & 2 \\
      1 & -1 & 3 \\
      4 & 2 & 0
    \stopmat$

    \begin{enumerate}
      \item $-A = 
      \startmat{ccc}
      \answer{0} & \answer{-1} & \answer{-2} \\
      \answer{-1} & \answer{1} & \answer{-3} \\
      \answer{-4} & \answer{-2} & \answer{0}
      \stopmat$

      \item $A^T =
      \startmat{ccc}
      \answer{0} & \answer{1} & \answer{4} \\
      \answer{1} & \answer{-1} & \answer{2} \\
      \answer{2} & \answer{3} & \answer{0}
      \stopmat$
    \end{enumerate}

  \end{enumerate}
  % \begin{sol}
  % \end{sol}
\end{problem}


\end{document}