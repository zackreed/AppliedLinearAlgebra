\documentclass{ximera}
\graphicspath{     %% setup a global graphics path
{./}               %% look in the same-level directory
{./pictures/}      %% look in graphics
{../pictures/}     %% look up one directory, then in graphics
%{../../pictures/} %% look up two directories, then in graphics
}

\author{Zack Reed}
%borrowed from selinger linear algebra
\begin{document}

%apple_pear_classify
\begin{problem}

The data that comprises the apple and pear fruit clusters in \href{https://ximera.osu.edu/appliedlinearalgebra/c1ChapterOne/Exercises/c1Homework/apple_pear_vect}{the previous problem} have been collected into two matrices, \texttt{Apple} and \texttt{Pear}, where each column of \texttt{Apple} and \texttt{Pear} is a vector representing a fruit measurement. Each row, therefore, must represent

\begin{multipleChoice}
  \choice{the measurement of a feature.}
  \choice[correct]{a feature being measured.}
  \choice{a fruit.}
\end{multipleChoice}

The matrices for \texttt{Apple} and \texttt{Pear} are given in the Module 1 course files, saved as \texttt{Apple.mat} and \texttt{Pear.mat}. Load the matrices into MATLAB and determine the dimensions of each matrix.

The matrix \texttt{Apple} has dimensions $\answer{3}$ rows by $\answer{68}$ columns. The matrix \texttt{Pear} has dimensions $\answer{3}$ rows by $\answer{75}$ columns.

Using the data, find the average pear measurement vector $\overline{P}$ and the average apple measurement vector $\overline{\texttt{Apple}}$. Using these averages, determine which of the following fruit should be classified as a pear, apple, or neither:

\begin{hint}[A note on answer entry]
  Enter the strings "apple" or "pear" or "neither" without quotes in your answers.
  
  The distance calculations have an error tolerance of 0.1, so if you're getting the answer wrong but think you're right, check your rounding, or check your work such as the average fruit vectors or the distance calculations. When in doubt, email your instructor!
\end{hint}

\begin{enumerate}
\item
The fruit
\begin{equation*}
  \vec{F}_1=\startmat{r}
    6.04 \\
    -.91 \\
    6.62
  \stopmat
\end{equation*}
is more likely to be a(n) \wordChoice{
  \choice[correct]{apple}
  \choice{pear}
  \choice{neither}
}, with a distance of $\answer[tolerance=.1]{2.99}$ from the average apple measurement vector and a distance of $\answer[tolerance=.1]{12.92}$ from the average pear measurement vector.

\item
The fruit
\begin{equation*}
  \vec{F}_2=\startmat{r}
    -2.19 \\
    7.75 \\
    6.75
  \stopmat
\end{equation*}
is more likely to be a(n) \wordChoice{
  \choice{apple}
  \choice[correct]{pear}
  \choice{neither}
}, with a distance of $\answer[tolerance=.1]{10.89}$ from the average apple measurement vector and a distance of $\answer[tolerance=.1]{4.94}$ from the average pear measurement vector.

\item
The fruit
\begin{equation*}
  \vec{F}_3=\startmat{r}
    7.1 \\
    -3.55 \\
    7.2
  \stopmat
\end{equation*}
is more likely to be a(n) \wordChoice{
  \choice{apple}
  \choice{pear}
  \choice[correct]{neither}
}, with a distance of $\answer[tolerance=.1]{15.91}$ from the average apple measurement vector and a distance of $\answer[tolerance=.1]{16.19}$ from the average pear measurement vector.

\item
The fruit
\begin{equation*}
  \vec{F}_4=\startmat{r}
    -2.76 \\
    4.25 \\
    2.57
  \stopmat
\end{equation*}
is more likely to be a(n) \wordChoice{
  \choice{apple}
  \choice{pear}
}, with a distance of $\answer[tolerance=.1]{10.98}$ from the average apple measurement vector and a distance of $\answer[tolerance=.1]{3.81}$ from the average pear measurement vector.

\end{enumerate}

{\bf Hint: Working in MATLAB}

  Make sure you're working in a folder that includes the +linalg course folder, then load the matrices.

  \begin{verbatim}
    load +linalg/Apple.mat
    load +linalg/Pear.mat
  \end{verbatim}

  To find the average column of a matrix in MATLAB, use the \texttt{mean} function. The second argument of mean specifies whether you want to average along the rows or columns. For example, to find the average column of matrix \texttt{Apple}, use the command \texttt{mean(Apple,2)}. For example, to find the average row of matrix \texttt{Pear}, use the command \texttt{mean(Pear,1)}.

  Warning: Double check that your vectors are all of the same dimension (i.e. 1x3 vs 3x1), as this will impact the distance calculations and difference vectors.



  
\end{problem}



\end{document}