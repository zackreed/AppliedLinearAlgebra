\documentclass{ximera}
\graphicspath{     %% setup a global graphics path
{./}               %% look in the same-level directory
{./pictures/}      %% look in graphics
{../pictures/}     %% look up one directory, then in graphics
%{../../pictures/} %% look up two directories, then in graphics
}

\author{Zack Reed}
%borrowed from selinger linear algebra
\begin{document}

%apple_pear_vect
\begin{exercise}

  Data has been collected on the measurements of two similar looking fruits, Packham pears and Conference pears. When the data is set in 3-dimensional space, one dimension per measurement type, the fruit measurements cluster into two distinct groups, much like in the applet below from the \href{https://ximera.osu.edu/appliedlinearalgebra/c1ChapterOne/learningActivities/m1LearningActivities/m1s1Vectors/vectorsAreEverywhereOne}{Chapter One Learning Activities}.

  \begin{center}
    %\pdfOnly{\includegraphics[width=.75\textwidth]{apple-pear.png}\\
    %$$\vec{F}\text{ is within the Apple cluster, and is thus an apple rather than a pear.}$$}
    \geogebra{qnghrmdg}{800}{600} %%https://www.geogebra.org/m/qnghrmdg
  \end{center}

  The Packham cluster of vectors is roughly spherical with center $\vec{P} = \startmat{c} 20 \\ 7 \\ 12 \stopmat$ and radius $r = 5$. The Conference cluster of vectors is roughly spherical with center $\vec{C} = \startmat{c} 10 \ 25 \ 9 \stopmat$ and radius $r = 7$.

  A new data point $\vec{F} = \startmat{c} 15 \\ 9 \\ 10 \stopmat$ has been collected. Determine if $\vec{F}$ is more likely to be a Packham pear or a Conference pear.
  
  $\vec{F}$ is more likely to be a \wordChoice{
    \choice[correct]{Packham}
    \choice{Conference}
    \choice{Unclassifiable}
    }
     pear. 
  
  \begin{hint}
  
  As you can see in the applet, the clusters are roughly separated by a plane. If new data points roughly fall within any one cluster (or just outside of it), the fruit is likely to be of that type. You might want to calculate the distance from $\vec{F}$ to the center of each cluster to help you decide.
  
  \end{hint}
  
  Another data point $\vec{G} = \startmat{c} 1 \\ 3 \\ 6 \stopmat$ has been collected.
  
  $\vec{G}$ is more likely to be a \wordChoice{
    \choice{Packham}
    \choice{Conference}
    \choice[correct]{Unclassifiable}
    }
    pear.
  
  The distance between $\vec{G}$ and the center of the Packham cluster is approximately $\answer[tolerance=.5]{20.322}$ and the distance between $\vec{G}$ and the center of the Conference cluster is approximately $\answer[tolerance=.5]{23.958}$.

\end{exercise}


\end{document}