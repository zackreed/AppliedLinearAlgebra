\documentclass{ximera}
\graphicspath{     %% setup a global graphics path
{./}               %% look in the same-level directory
{./pictures/}      %% look in graphics
{../pictures/}     %% look up one directory, then in graphics
%{../../pictures/} %% look up two directories, then in graphics
}

\author{Zack Reed}
%borrowed from selinger linear algebra
\begin{document}


\begin{exercise}

    For the following subspaces of
    $\R^n$, determine the dimension of the larger space in which the vectors live, determine the dimension of the subspace described, and find a basis for the subspace.

    (NOTE: More questions will appear after you correctly answer the currently visible questions)

    \begin{hint}
    
        This will take a few steps. You'll first need to interpret the subspace in terms of linear combinations of vectors, using the parameters $a, b, $ and $c$ to determine the spanning vectors. It will help to start with \texttt{syms a b c} and \texttt{assume(a^2+b^2+c^2>0)} before using the solve command.

        Then you'll need to check whether the spanning vectors are a basis for the subspace or not. 

        If they're not a basis, make a basis out of the first available linearly independent vectors.

        Don't factor nor simplify the basis vectors at all when entering them.

    \end{hint}

    \begin{enumerate}

    \item $S_1 =
      \set{\left.\startmat{c}
            4a+b-5c \\
            12a+6b-6c \\
            4a+4b+4c
          \stopmat ~\right\vert~a,b,c\text{in} \R}.$

          $S_1$ is a $\answer{2}$-dimensional subspace of $\R^{\answer{3}}$.

          \begin{problem}

            The basis vectors for $S_1$ are $u=\startmat{r} \answer{4} \\ \answer{12} \\ \answer{4} \stopmat$ and $v=\startmat{r} \answer{1} \\ \answer{6} \\ \answer{4} \stopmat$. (NOTE: Answer in order of the spanning vectors, pay attention to which vectors are given as linear combos of others when solving in MATLAB.)

          \end{problem}

          \item $S_2 =
          \set{\left.\startmat{c}
                2a+6b+7c \\
                -3a-9b-12c \\
                2a+6b+6c \\
                a+3b+3c
              \stopmat ~\right\vert~a,b,c\text{in} \R}.$
    
                $S_2$ is a $\answer{2}$-dimensional subspace of $\R^{\answer{4}}$.
    
                \begin{problem}
    
                  The basis vectors for $S_2$ are $u=\startmat{r} \answer{2} \\ \answer{-3} \\ \answer{2} \\ \answer{1} \stopmat$ and $v=\startmat{r} \answer{7} \\ \answer{-12} \\ \answer{6} \\ \answer{3} \stopmat$. (NOTE: Answer in order of the spanning vectors, pay attention to which vectors are given as linear combos of others when solving in MATLAB.)
    
                \end{problem}
    
                \item $S_3 =
                \set{\left.\startmat{c}
                      2a+b \\
                      6b-3a+3c \\
                      3b-6a+3c
                    \stopmat ~\right\vert~a,b,c\text{in} \R}.$
          
                      $S_3$ is a $\answer{3}$-dimensional vector space.
          
                      \begin{problem}
          
                        The basis vectors for $S_3$ are $u=\startmat{r} 
                        \answer{2} \\ \answer{-3} \\ \answer{-6} \stopmat$, $v=\startmat{r} \answer{1} \\ \answer{6} \\ \answer{3} \stopmat$, and $w=\startmat{r} \answer{0} \\ \answer{3} \\ \answer{3} \stopmat$. (NOTE: Answer in order of the spanning vectors, pay attention to which vectors are given as linear combos of others when solving in MATLAB.)
          
                      \end{problem}       

    \end{enumerate}    
    

\end{exercise}

\end{document}