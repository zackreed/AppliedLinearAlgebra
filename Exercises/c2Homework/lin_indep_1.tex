\documentclass{ximera}
\graphicspath{     %% setup a global graphics path
{./}               %% look in the same-level directory
{./pictures/}      %% look in graphics
{../pictures/}     %% look up one directory, then in graphics
%{../../pictures/} %% look up two directories, then in graphics
}

\author{Zack Reed}
%borrowed from selinger linear algebra
\begin{document}


\begin{exercise}

    True or false? Explain.
    \begin{enumerate}
    \item Every set of $5$ vectors in $\R^5$ \wordChoice{
        \choice{must be}
        \choice[correct]{could be}
        \choice{cannot be}
    } linearly independent.
        \begin{feedback}
          They vectors do not have to be. For example, the vectors
          $\set{\vect{0},\vect{0},\vect{0},\vect{0},\vect{0}}$ are
          linearly dependent. 
        \end{feedback}



    
    \item Every set of $4$ vectors in $\R^5$ \wordChoice{
      \choice{must be}
      \choice[correct]{could be}
      \choice{cannot be}
  } linearly independent.
    
    \begin{feedback}
      Not necessarily. For example, the vectors
      $\set{\vect{0},\vect{0},\vect{0},\vect{0}}$ are linearly
      dependent.
    \end{feedback}


    

    \item Every set of $6$ vectors in $\R^5$ \wordChoice{
      \choice[correct]{must be}
      \choice{could be}
      \choice{cannot be}
  } linearly dependent.

    \begin{feedback}
      If we have $5$ linearly independent vectors in $\R^5$, that set \emph{must} span $\R^5$. Hence, the addition of another vector must be in the span of the first $5$. If the first $5$ are not linearly independent, then they already are linearly dependent.
    \end{feedback}


    \item \wordChoice{
      \choice{Every}
      \choice{Some}
      \choice[correct]{No}
  } set of $4$ vectors spans $\RR^5$.

    \begin{feedback}
      Even if all vectors are linearly independent, the result is a $4$-dimensional subspace of $\RR^5$.
    \end{feedback}

    \item Every linearly independent set of $5$ vectors in $\R^5$ \wordChoice{
      \choice[correct]{must be}
      \choice{could be}
      \choice{cannot be}
  } a basis of $\R^5$.

    \begin{feedback}
      This satisfies the definition of a basis.
    \end{feedback}
  
    \item Every linearly independent set of $4$ vectors in $\R^5$ \wordChoice{
      \choice{must be}
      \choice{could be}
      \choice[correct]{cannot be}
  } a
    basis of $\R^5$.

    \begin{feedback}
      Again, the span is at most a $4$-dimensional subspace of $\RR^5$.
    \end{feedback}
  
  
    \item Every spanning set of $6$ vectors in $\R^5$ \wordChoice{
      \choice{must be}
      \choice{could be}
      \choice[correct]{cannot be}
  } a basis of
    $\R^5$.

    \begin{feedback}
      By definition a basis must be linearly independent. A set of $6$ vectors in $\RR^5$ must be linearly dependent.
    \end{feedback}


    \item Every linearly independent set of $4$ vectors in $\R^5$ \wordChoice{
      \choice[correct]{must}
      \choice{could}
      \choice{cannot}
  } span
    a $4$-dimensional subspace of $\R^5$.

    \begin{feedback}
      Linearly independent sets are not redundant, and hence you gain one dimension of variation for each vector.
    \end{feedback}
    \end{enumerate}

  \end{exercise}

\end{document}