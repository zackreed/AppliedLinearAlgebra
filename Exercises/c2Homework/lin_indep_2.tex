\documentclass{ximera}
\graphicspath{     %% setup a global graphics path
{./}               %% look in the same-level directory
{./pictures/}      %% look in graphics
{../pictures/}     %% look up one directory, then in graphics
%{../../pictures/} %% look up two directories, then in graphics
}

\author{Zack Reed}
%borrowed from selinger linear algebra
\begin{document}


\begin{exercise}

    Which of the following vectors are redundant? If there are redundant
vectors, write each of them as a linear combination of previous
vectors.

(NOTE: Take the vectors in order. You want to consider the first available linearly dependent vectors when filling in the answers. The first answer should not depend on MATLAB, but you might want to use MATLAB to determine the second dependent vector)

\begin{expandable}{stuff}{Hint}

    If it's not immediately obvious which vectors are linear combinations of others,that's totally okay! 
    
    You'll want to use MATLAB to solve the linear combination equation for the redundant vectors.

    First, use \texttt{assume(a^2+b^2+c^2>0)} to get rid of the trivial solution, then use \texttt{solve} to see if a linear combination of the vectors add to the zero vector.

\end{expandable}
\begin{equation*}
  \vect{u}_1 = \startmat{r} 1 \\ 0 \\ 1 \stopmat,\quad
  \vect{u}_2 = \startmat{r} 2 \\ 0 \\ 2 \stopmat,\quad
  \vect{u}_3 = \startmat{r} 1 \\ 2 \\ 1 \stopmat,\quad
  \vect{u}_4 = \startmat{r} 1 \\ 6 \\ 1 \stopmat.
\end{equation*}

  Vectors $\answer{2}$ and $\answer{4}$ are redundant. (Say the number, so vector $\vec{u}_3$ is vector $3$.)
  
  We have


  $\text{Vector }\answer{2}=\answer{2}\text{ times vector }\answer{1}$ and 
  
  $\text{Vector }\answer{4}=\answer{-2}\text{ times vector }\answer{1}+\answer{3}\text{ times vector }\answer{3}$.


\end{exercise}

\end{document}