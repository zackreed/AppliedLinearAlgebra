\documentclass{ximera}
\graphicspath{  %% When looking for images,
{./}            %% look here first,
{./pictures/}   %% then look for a pictures folder,
{../pictures/}  %% which may be a directory up.
{../../pictures/}  %% which may be a directory up.
{../../../pictures/}  %% which may be a directory up.
{../../../../pictures/}  %% which may be a directory up.
}

\usepackage{listings}
%\usepackage{circuitikz}
\usepackage{xcolor}
\usepackage{amsmath,amsthm}
\usepackage{subcaption}
\usepackage{graphicx}
\usepackage{tikz}
%\usepackage{tikz-3dplot}
\usepackage{amsfonts}
%\usepackage{mdframed} % For framing content
%\usepackage{tikz-cd}

  \renewcommand{\vector}[1]{\left\langle #1\right\rangle}
  \newcommand{\arrowvec}[1]{{\overset{\rightharpoonup}{#1}}}
  \newcommand{\ro}{\texttt{R}}%% row operation
  \newcommand{\dotp}{\bullet}%% dot product
  \renewcommand{\l}{\ell}
  \let\defaultAnswerFormat\answerFormatBoxed
  \usetikzlibrary{calc,bending}
  \tikzset{>=stealth}
  




%make a maroon color
\definecolor{maroon}{RGB}{128,0,0}
%make a dark blue color
\definecolor{darkblue}{RGB}{0,0,139}
%define the color fourier0 to be the maroon color
\definecolor{fourier0}{RGB}{128,0,0}
%define the color fourier1 to be the dark blue color
\definecolor{fourier1}{RGB}{0,0,139}
%define the color fourier 1t to be the light blue color
\definecolor{fourier1t}{RGB}{173,216,230}
%define the color fourier2 to be the dark green color
\definecolor{fourier2}{RGB}{0,100,0}
%define teh color fourier2t to be the light green color
\definecolor{fourier2t}{RGB}{144,238,144}
%define the color fourier3 to be the dark purple color
\definecolor{fourier3}{RGB}{128,0,128}
%define the color fourier3t to be the light purple color
\definecolor{fourier3t}{RGB}{221,160,221}
%define the color fourier0t to be the red color
\definecolor{fourier0t}{RGB}{255,0,0}
%define the color fourier4 to be the orange color
\definecolor{fourier4}{RGB}{255,165,0}
%define the color fourier4t to be the darker orange color
\definecolor{fourier4t}{RGB}{255,215,0}
%define the color fourier5 to be the yellow color
\definecolor{fourier5}{RGB}{255,255,0}
%define the color fourier5t to be the darker yellow color
\definecolor{fourier5t}{RGB}{255,255,100}
%define the color fourier6 to be the green color
\definecolor{fourier6}{RGB}{0,128,0}
%define the color fourier6t to be the darker green color
\definecolor{fourier6t}{RGB}{0,255,0}

%New commands for this doc for errors in copying
\newcommand{\eigenvar}{\lambda}
%\newcommand{\vect}[1]{\mathbf{#1}}
\renewcommand{\th}{^{\text{th}}}
\newcommand{\st}{^{\text{st}}}
\newcommand{\nd}{^{\text{nd}}}
\newcommand{\rd}{^{\text{rd}}}
\newcommand{\paren}[1]{\left(#1\right)}
\newcommand{\abs}[1]{\left|#1\right|}
\newcommand{\R}{\mathbb{R}}
\newcommand{\C}{\mathbb{C}}
\newcommand{\Hilb}{\mathbb{H}}
\newcommand{\qq}[1]{\text{#1}}
\newcommand{\Z}{\mathbb{Z}}
\newcommand{\N}{\mathbb{N}}
\newcommand{\q}[1]{\text{``#1''}}
%\newcommand{\mat}[1]{\begin{bmatrix}#1\end{bmatrix}}
\newcommand{\rref}{\text{reduced row echelon form}}
\newcommand{\ef}{\text{echelon form}}
\newcommand{\ohm}{\Omega}
\newcommand{\volt}{\text{V}}
\newcommand{\amp}{\text{A}}
\newcommand{\Seq}{\textbf{Seq}}
\newcommand{\Poly}{\textbf{P}}
\renewcommand{\quad}{\text{    }}
\newcommand{\roweq}{\simeq}
\newcommand{\rowop}{\simeq}
\newcommand{\rowswap}{\leftrightarrow}
\newcommand{\Mat}{\textbf{M}}
\newcommand{\Func}{\textbf{Func}}
\newcommand{\Hw}{\textbf{Hamming weight}}
\newcommand{\Hd}{\textbf{Hamming distance}}
\newcommand{\rank}{\text{rank}}
\newcommand{\longvect}[1]{\overrightarrow{#1}}
% Define the circled command
\newcommand{\circled}[1]{%
  \tikz[baseline=(char.base)]{
    \node[shape=circle,draw,inner sep=2pt,red,fill=red!20,text=black] (char) {#1};}%
}

% Define custom command \strikeh that just puts red text on the 2nd argument
\newcommand{\strikeh}[2]{\textcolor{red}{#2}}

% Define custom command \strikev that just puts red text on the 2nd argument
\newcommand{\strikev}[2]{\textcolor{red}{#2}}

%more new commands for this doc for errors in copying
\newcommand{\SI}{\text{SI}}
\newcommand{\kg}{\text{kg}}
\newcommand{\m}{\text{m}}
\newcommand{\s}{\text{s}}
\newcommand{\norm}[1]{\left\|#1\right\|}
\newcommand{\col}{\text{col}}
\newcommand{\sspan}{\text{span}}
\newcommand{\proj}{\text{proj}}
\newcommand{\set}[1]{\left\{#1\right\}}
\newcommand{\degC}{^\circ\text{C}}
\newcommand{\centroid}[1]{\overline{#1}}
\newcommand{\dotprod}{\boldsymbol{\cdot}}
%\newcommand{\coord}[1]{\begin{bmatrix}#1\end{bmatrix}}
\newcommand{\iprod}[1]{\langle #1 \rangle}
\newcommand{\adjoint}{^{*}}
\newcommand{\conjugate}[1]{\overline{#1}}
\newcommand{\eigenvarA}{\lambda}
\newcommand{\eigenvarB}{\mu}
\newcommand{\orth}{\perp}
\newcommand{\bigbracket}[1]{\left[#1\right]}
\newcommand{\textiff}{\text{ if and only if }}
\newcommand{\adj}{\text{adj}}
\newcommand{\ijth}{\emph{ij}^\text{th}}
\newcommand{\minor}[2]{M_{#2}}
\newcommand{\cofactor}{\text{C}}
\newcommand{\shift}{\textbf{shift}}
\newcommand{\startmat}[1]{
  \left[\begin{array}{#1}
}
\newcommand{\stopmat}{\end{array}\right]}
%a command to give a name to explorations and hints and theorems
\newcommand{\name}[1]{\begin{centering}\textbf{#1}\end{centering}}
\newcommand{\vect}[1]{\vec{#1}}
\newcommand{\dfn}[1]{\textbf{#1}}
\newcommand{\transpose}{\mathsf{T}}
\newcommand{\mtlb}[2][black]{\texttt{\textcolor{#1}{#2}}}
\newcommand{\RR}{\mathbb{R}} % Real numbers
\newcommand{\id}{\text{id}}
\newcommand{\coord}[1]{\langle#1\rangle}
\newcommand{\RREF}{\text{RREF}}
\newcommand{\Null}{\text{Null}}
\newcommand{\Nullity}{\text{Nullity}}
\newcommand{\Rank}{\text{Rank}}
\newcommand{\Col}{\text{Col}}
\newcommand{\Ef}{\text{EF}}
\newcommand{\boxprod}[3]{\abs{(#1\times#2)\cdot#3}}

\author{Zack Reed}
%borrowed from selinger linear algebra
\begin{document}

\section*{Practice Problems}
\begin{problem}
Determine whether each augmented matrix shown below is in reduced row-echelon form.
  \begin{problem}\label{prob:rrefmultchoice1}
  $$\left[\begin{array}{cccc|c} 
 0&1&1&0&2\\1&-3&0&1&4\\0&0&1&0&-1
 \end{array}\right]$$
 \begin{multipleChoice}
 \choice{Yes}
 \choice[correct]{No}
 \end{multipleChoice}
  \end{problem}
   
   \begin{problem}\label{prob:rrefmultchoice2}
  $$\left[\begin{array}{cc|c} 
 1&1&1\\0&1&0\\0&0&0
 \end{array}\right]$$
 \begin{multipleChoice}
 \choice{Yes}
 \choice[correct]{No}
 \end{multipleChoice}
  \end{problem}
   
  \begin{problem}\label{prob:rrefmultchoice3}
  $$\left[\begin{array}{cc|c} 
 1&0&1\\0&1&0\\0&0&0
 \end{array}\right]$$
 \begin{multipleChoice}
 \choice[correct]{Yes}
 \choice{No}
 \end{multipleChoice}
  \end{problem}
   
  \begin{problem}\label{prob:rrefmultchoice4}
  $$\left[\begin{array}{ccc|c} 
 1&0&1&0\\0&1&0&0\\0&0&0&1
 \end{array}\right]$$
 \begin{multipleChoice}
 \choice[correct]{Yes}
 \choice{No}
 \end{multipleChoice}
  \end{problem}
   
   \begin{problem}\label{prob:rrefmultchoice5}
  $$\left[\begin{array}{ccc|c} 
 1&1&0&2\\0&1&0&9\\0&0&1&-1
 \end{array}\right]$$
 \begin{multipleChoice}
 \choice{Yes}
 \choice[correct]{No}
 \end{multipleChoice}
  \end{problem}
   
\end{problem}
 
\begin{problem}\label{prob:rrefnosolutionssys}
Fill in the steps that lead to the reduced row-echelon form in Example \ref{ex:nosolutionssys}.
\end{problem}
 
\begin{problem}\label{prob:rreffinfmanysolutionsys}
Fill in the steps that lead to the reduced row-echelon form in Example \ref{ex:rrefinfmanysolutionssys}.
\end{problem}
 
\begin{problem} \label{prob:numberofsolutionsmultch}
Suppose a system of equations has the following reduced row-echelon form
  $$\left[\begin{array}{ccc|c} 
 1&0&0&4\\0&1&-3&-6\\0&0&0&1
 \end{array}\right]$$
 What can you say about the system?
 \begin{multipleChoice}
 \choice[correct]{The system is inconsistent}
 \choice{The system has infinitely many solutions}
 \choice{The system has a unique solution}
 \choice{We would have to examine the original system to make the final determination}
 \end{multipleChoice}
  \end{problem}
\begin{problem}
Solve each system of equations
\begin{problem}\label{prob:sys20solvesys1}
$$\begin{array}{ccccccc}
      x & +&3y&-&2z&= &-11 \\
     2x& +&y&+&4z&=&12\\
     x& -&y&-&z&=&0
    \end{array}$$
     
    Solution:
    $$( \answer{1}, \answer{-2}, \answer{3})$$
\end{problem}
 
\begin{problem}\label{prob:sys20solvesys2}
$$\begin{array}{ccccccc}
      3x & -&y&+&z&= &-5 \\
     x& +&2y&-&z&=&-3\\
     x& -&5y&+&3z&=&1
    \end{array}$$
     
    Solution: (Enter your answers in fraction form)
    $$\left(\answer{-13/7}-\answer{1/7}t, \answer{-4/7}+\answer{4/7}t, t\right)$$
\end{problem}
\end{problem}

\section*{Practice Problems}
 
\begin{problem}\label{prob:same_rref}
Show that applying Gauss-Jordan elimination to the matrix in Exploration \ref{init:gaussianelim2} yields the same reduced row-echelon form as the matrix we obtained in Exploration \ref{init:gaussianelim1}.
\end{problem}
 
\begin{problem}\label{prob:twowaystorref1}
Follow the indicated steps of the Gauss-Jordan algorithm to transform the matrix to its reduced row-echelon form.  Steps will unfold automatically as you enter correct answers.
 
$$\left[\begin{array}{ccc|c}  2&1&1&3\\-1&0&1&2\\1&1&-2&0
  \end{array}\right]$$
 
 \begin{prompt}  $\frac{1}{2}R_1\rightarrow R_1$.
$$ \left[\begin{array}{ccc|c}   \answer{1}&\answer{1/2}&\answer{1/2}&\answer{3/2}\\-1&0&1&2\\1&1&-2&0
  \end{array}\right]$$
 \end{prompt}
 
 \begin{problem}
 \begin{prompt} $R_1+R_2\rightarrow R_2$.
$$ \left[\begin{array}{ccc|c}  1&1/2&1/2&3/2\\\answer{0}&\answer{1/2}&\answer{3/2}&\answer{7/2}\\1&1&-2&0
  \end{array}\right]$$
 \end{prompt}
 \begin{problem}
 \begin{prompt} $R_3-R_1\rightarrow R_3$.
$$ \left[\begin{array}{ccc|c}   1&1/2&1/2&3/2\\0&1/2&3/2&7/2\\\answer{0}&\answer{1/2}&\answer{-5/2}&\answer{-3/2}
  \end{array}\right]$$
 \end{prompt}
  \begin{problem}
  \begin{prompt} $2R_2\rightarrow R_2$.
$$ \left[\begin{array}{ccc|c}   1&1/2&1/2&3/2\\\answer{0}&\answer{1}&\answer{3}&\answer{7}\\0&1/2&-5/2&-3/2
 \end{array}\right]$$
\end{prompt}
 \begin{problem}
 \begin{prompt} $R_3-\frac{1}{2}R_2\rightarrow R_3$.
$$ \left[\begin{array}{ccc|c} 
1&1/2&1/2&3/2\\0&1&3&7\\\answer{0}&\answer{0}&\answer{-4}&\answer{-5}
  \end{array}\right]$$
 \end{prompt}
 \begin{problem}
 \begin{prompt} $-\frac{1}{4}R_3\rightarrow R_3$.
$$\left[\begin{array}{ccc|c}    1&1/2&1/2&3/2\\0&1&3&7\\\answer{0}&\answer{0}&\answer{1}&\answer{5/4}
  \end{array}\right]$$
 \end{prompt}
 \begin{problem}
 \begin{prompt} $R_2-3R_3\rightarrow R_2$.
$$ \left[\begin{array}{ccc|c} 
1&1/2&1/2&3/2\\\answer{0}&\answer{1}&\answer{0}&\answer{13/4}\\0&0&1&5/4
  \end{array}\right]$$
 \end{prompt}
 \begin{problem}
 \begin{prompt} $R_1-\frac{1}{2}R_3\rightarrow R_1$.
$$ \left[\begin{array}{ccc|c} 
\answer{1}&\answer{1/2}&\answer{0}&\answer{7/8}\\0&1&0&13/4\\0&0&1&5/4
 \end{array}\right]$$
  \end{prompt}
  \begin{problem}
 \begin{prompt} $R_1-\frac{1}{2}R_2\rightarrow R_1$.
$$ \left[\begin{array}{ccc|c} 
 \answer{1}&\answer{0}&\answer{0}&\answer{-3/4}\\0&1&0&13/4\\0&0&1&5/4
  \end{array}\right]$$
  \end{prompt}
 \end{problem}
  \end{problem}
  \end{problem}
  \end{problem}
  \end{problem}
  \end{problem}
  \end{problem}
 \end{problem}
 \end{problem}
  
 
 
 
\begin{problem}
Find the rank of each matrix.
\begin{problem}\label{prob:rankofmat1}
$$A=\begin{bmatrix}4&3&-1\\-8&-6&2\end{bmatrix}$$
Answer:
 
$$\mbox{rank}(A)=\answer{1}$$
\end{problem}
 
\begin{problem}\label{prob:rankofmat2}
$$B=\begin{bmatrix}1&1\\2&-2\\3&-1\end{bmatrix}$$
Answer:
 
$$\mbox{rank}(B)=\answer{2}$$
\end{problem}
 
\begin{problem}\label{prob:rankofmat3}
$$C=\begin{bmatrix}1&0&1\\2&1&3\\0&1&-2\end{bmatrix}$$
Answer:
 
$$\mbox{rank}(C)=\answer{3}$$
\end{problem}
 
\begin{problem}\label{prob:rankofmat4}
$$D=\begin{bmatrix}1&1&2\\-1&-2&1\\1&0&5\end{bmatrix}$$
Answer:
 
$$\mbox{rank}(D)=\answer{2}$$
\end{problem}
\end{problem}
 
\begin{problem}\label{prob:rankofmat5}
Suppose $A$ is a $5\times 7$ matrix.  Which of the following can be true?
\begin{multipleChoice}
 \choice{$\mbox{rank}(A)=7$}
 \choice{$\mbox{rank}(A)=6$}
 \choice[correct]{$\mbox{rank}(A)=5$}
 \choice{All of the above}
 \end{multipleChoice}
\end{problem}
 
\begin{problem}\label{prob:4eq5un}
Suppose a linear system has $4$ equations and $5$ unknowns.  Which of the following is NOT a possibility?
\begin{multipleChoice}
  \choice[correct]{The system has a unique solution}
 \choice{The system has no solutions}
 \choice{The system has infinitely many solutions}
 \end{multipleChoice}
\end{problem}
 
\begin{problem}\label{prob:leadones}
Suppose $A$ is a matrix such that $\mbox{rref}(A)$ has $5$ leading $1's$.  What do we know to be true about $A$?  Select ALL that apply.
\begin{selectAll}
 \choice[correct]{$\mbox{rank}(A)=5$}
 \choice[correct]{$A$ has at least $25$ entries}
 \choice[correct]{Any row-echelon form of $A$ will have exactly $5$ nonzero rows}
 \choice{Some row-echelon forms of $A$ may have more than $5$ nonzero rows}
 \choice{Some row-echelon forms of $A$ may have less than $5$ nonzero rows}
 \end{selectAll}
\end{problem}
 
\begin{problem}\label{prob:rankaugvscoeff}
In this problem we will discuss how the rank of the {\it coefficient matrix} associated with a linear system compares to the rank of the {\it augmented matrix} associated with the system. 
\begin{enumerate}
\item Explain why the rank of the augmented matrix has to be greater than or equal to the rank of the coefficient matrix.
    \item Prove that for a {\it consistent} system the rank of the coefficient matrix will be the same as the rank of the {\it augmented} matrix.
    \item Give an example of an inconsistent system for which the rank of the augmented matrix is greater than the rank of the coefficient matrix.
    \item Can the rank of an augmented matrix be greater than the number of variables?
    \item Is the following statement true?
     
    ``If the rank of the augmented matrix associated with a linear system is greater than the rank of the coefficient matrix, then the system is inconsistent."
\end{enumerate}
\end{problem}

\end{document}