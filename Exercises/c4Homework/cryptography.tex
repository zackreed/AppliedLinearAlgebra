\documentclass{ximera}
\graphicspath{  %% When looking for images,
{./}            %% look here first,
{./pictures/}   %% then look for a pictures folder,
{../pictures/}  %% which may be a directory up.
{../../pictures/}  %% which may be a directory up.
{../../../pictures/}  %% which may be a directory up.
{../../../../pictures/}  %% which may be a directory up.
}

\usepackage{listings}
%\usepackage{circuitikz}
\usepackage{xcolor}
\usepackage{amsmath,amsthm}
\usepackage{subcaption}
\usepackage{graphicx}
\usepackage{tikz}
%\usepackage{tikz-3dplot}
\usepackage{amsfonts}
%\usepackage{mdframed} % For framing content
%\usepackage{tikz-cd}

  \renewcommand{\vector}[1]{\left\langle #1\right\rangle}
  \newcommand{\arrowvec}[1]{{\overset{\rightharpoonup}{#1}}}
  \newcommand{\ro}{\texttt{R}}%% row operation
  \newcommand{\dotp}{\bullet}%% dot product
  \renewcommand{\l}{\ell}
  \let\defaultAnswerFormat\answerFormatBoxed
  \usetikzlibrary{calc,bending}
  \tikzset{>=stealth}
  




%make a maroon color
\definecolor{maroon}{RGB}{128,0,0}
%make a dark blue color
\definecolor{darkblue}{RGB}{0,0,139}
%define the color fourier0 to be the maroon color
\definecolor{fourier0}{RGB}{128,0,0}
%define the color fourier1 to be the dark blue color
\definecolor{fourier1}{RGB}{0,0,139}
%define the color fourier 1t to be the light blue color
\definecolor{fourier1t}{RGB}{173,216,230}
%define the color fourier2 to be the dark green color
\definecolor{fourier2}{RGB}{0,100,0}
%define teh color fourier2t to be the light green color
\definecolor{fourier2t}{RGB}{144,238,144}
%define the color fourier3 to be the dark purple color
\definecolor{fourier3}{RGB}{128,0,128}
%define the color fourier3t to be the light purple color
\definecolor{fourier3t}{RGB}{221,160,221}
%define the color fourier0t to be the red color
\definecolor{fourier0t}{RGB}{255,0,0}
%define the color fourier4 to be the orange color
\definecolor{fourier4}{RGB}{255,165,0}
%define the color fourier4t to be the darker orange color
\definecolor{fourier4t}{RGB}{255,215,0}
%define the color fourier5 to be the yellow color
\definecolor{fourier5}{RGB}{255,255,0}
%define the color fourier5t to be the darker yellow color
\definecolor{fourier5t}{RGB}{255,255,100}
%define the color fourier6 to be the green color
\definecolor{fourier6}{RGB}{0,128,0}
%define the color fourier6t to be the darker green color
\definecolor{fourier6t}{RGB}{0,255,0}

%New commands for this doc for errors in copying
\newcommand{\eigenvar}{\lambda}
%\newcommand{\vect}[1]{\mathbf{#1}}
\renewcommand{\th}{^{\text{th}}}
\newcommand{\st}{^{\text{st}}}
\newcommand{\nd}{^{\text{nd}}}
\newcommand{\rd}{^{\text{rd}}}
\newcommand{\paren}[1]{\left(#1\right)}
\newcommand{\abs}[1]{\left|#1\right|}
\newcommand{\R}{\mathbb{R}}
\newcommand{\C}{\mathbb{C}}
\newcommand{\Hilb}{\mathbb{H}}
\newcommand{\qq}[1]{\text{#1}}
\newcommand{\Z}{\mathbb{Z}}
\newcommand{\N}{\mathbb{N}}
\newcommand{\q}[1]{\text{``#1''}}
%\newcommand{\mat}[1]{\begin{bmatrix}#1\end{bmatrix}}
\newcommand{\rref}{\text{reduced row echelon form}}
\newcommand{\ef}{\text{echelon form}}
\newcommand{\ohm}{\Omega}
\newcommand{\volt}{\text{V}}
\newcommand{\amp}{\text{A}}
\newcommand{\Seq}{\textbf{Seq}}
\newcommand{\Poly}{\textbf{P}}
\renewcommand{\quad}{\text{    }}
\newcommand{\roweq}{\simeq}
\newcommand{\rowop}{\simeq}
\newcommand{\rowswap}{\leftrightarrow}
\newcommand{\Mat}{\textbf{M}}
\newcommand{\Func}{\textbf{Func}}
\newcommand{\Hw}{\textbf{Hamming weight}}
\newcommand{\Hd}{\textbf{Hamming distance}}
\newcommand{\rank}{\text{rank}}
\newcommand{\longvect}[1]{\overrightarrow{#1}}
% Define the circled command
\newcommand{\circled}[1]{%
  \tikz[baseline=(char.base)]{
    \node[shape=circle,draw,inner sep=2pt,red,fill=red!20,text=black] (char) {#1};}%
}

% Define custom command \strikeh that just puts red text on the 2nd argument
\newcommand{\strikeh}[2]{\textcolor{red}{#2}}

% Define custom command \strikev that just puts red text on the 2nd argument
\newcommand{\strikev}[2]{\textcolor{red}{#2}}

%more new commands for this doc for errors in copying
\newcommand{\SI}{\text{SI}}
\newcommand{\kg}{\text{kg}}
\newcommand{\m}{\text{m}}
\newcommand{\s}{\text{s}}
\newcommand{\norm}[1]{\left\|#1\right\|}
\newcommand{\col}{\text{col}}
\newcommand{\sspan}{\text{span}}
\newcommand{\proj}{\text{proj}}
\newcommand{\set}[1]{\left\{#1\right\}}
\newcommand{\degC}{^\circ\text{C}}
\newcommand{\centroid}[1]{\overline{#1}}
\newcommand{\dotprod}{\boldsymbol{\cdot}}
%\newcommand{\coord}[1]{\begin{bmatrix}#1\end{bmatrix}}
\newcommand{\iprod}[1]{\langle #1 \rangle}
\newcommand{\adjoint}{^{*}}
\newcommand{\conjugate}[1]{\overline{#1}}
\newcommand{\eigenvarA}{\lambda}
\newcommand{\eigenvarB}{\mu}
\newcommand{\orth}{\perp}
\newcommand{\bigbracket}[1]{\left[#1\right]}
\newcommand{\textiff}{\text{ if and only if }}
\newcommand{\adj}{\text{adj}}
\newcommand{\ijth}{\emph{ij}^\text{th}}
\newcommand{\minor}[2]{M_{#2}}
\newcommand{\cofactor}{\text{C}}
\newcommand{\shift}{\textbf{shift}}
\newcommand{\startmat}[1]{
  \left[\begin{array}{#1}
}
\newcommand{\stopmat}{\end{array}\right]}
%a command to give a name to explorations and hints and theorems
\newcommand{\name}[1]{\begin{centering}\textbf{#1}\end{centering}}
\newcommand{\vect}[1]{\vec{#1}}
\newcommand{\dfn}[1]{\textbf{#1}}
\newcommand{\transpose}{\mathsf{T}}
\newcommand{\mtlb}[2][black]{\texttt{\textcolor{#1}{#2}}}
\newcommand{\RR}{\mathbb{R}} % Real numbers
\newcommand{\id}{\text{id}}
\newcommand{\coord}[1]{\langle#1\rangle}
\newcommand{\RREF}{\text{RREF}}
\newcommand{\Null}{\text{Null}}
\newcommand{\Nullity}{\text{Nullity}}
\newcommand{\Rank}{\text{Rank}}
\newcommand{\Col}{\text{Col}}
\newcommand{\Ef}{\text{EF}}
\newcommand{\boxprod}[3]{\abs{(#1\times#2)\cdot#3}}

\author{Zack Reed}
%borrowed from selinger linear algebra
\begin{document}

\begin{problem}

   You've intercepted an encrypted message from an opponent in a cryptography competition: ``VO,VO,QMYF NNESJKDDFSFWNT,,VIZIWUBPTG,Z DI''
   
    NOTE: More problems will appear after you successfully answer the following:

    \begin{enumerate}
        \item You know that your opponent uses a 3-dimensional Hill Cipher, and also that they always start their communication with  ``Yo yo yo.", which when encrypted reads ``VO,VO,QMY''
        
        This information \wordChoice{\choice{is}\choice[correct]{is not}} enough to break the encryption because the cipher text ``VO,VO,QMY'' makes a $3\times 3$ matrix of rank $\answer{2}$, and hence \wordChoice{\choice{is}\choice[correct]{is not}} invertible. Because of this, we \wordChoice{\choice{can}\choice[correct]{cannot}} solve for the inverse of the cipher. 


        \item You know that your opponent uses a 3-dimensional Hill Cipher, and also that they always end their communication with  ``lates fam.", which when encrypted reads ``AC,BBWFLGZ..''
        
        This information \wordChoice{\choice[correct]{is}\choice{is not}} enough to break the encryption because the cipher text ``AC,BBWFLGZ..'' lets you make a $3\times 3$ matrix of rank $\answer{3}$, and hence \wordChoice{\choice[correct]{is}\choice{is not}} invertible. Because of this, we \wordChoice{\choice[correct]{can}\choice{cannot}} solve for the inverse of the cipher. 

        \begin{problem}

            \begin{enumerate}
                \item First, use the known ending ``lates fam.'' in conjunction with the cipher text ``AC,BBWFLGZ..'' to hack the cipher by finding the inverse matrix of the hill cipher. 
                
                The inverse cipher $A^{-1}$ after conversion to the number system $0\rightarrow 28$ (using \texttt{linalg.mod\_p(M,29)} for a matrix $M$) is:

                $$A^{-1}=\begin{bmatrix}
                    \answer{2} & 0 &\answer{24}\\
                    \answer{0} & 6 & 23 \\
                    28 & \answer{17} & \answer{15}
                \end{bmatrix}.$$

                \item Now, translate the entirety of the intercepted message: ``VO,VO,QMYF NNESJKDDFSFWNT,,VIZIWUBPTG,Z DI''
                
                The full translated message reads ``Yo yo yo. $\answer[format=string]{catch me if you can.}$ Lates fam.''
            \end{enumerate}

            \begin{hint}
            
                You'll want to convert each matrix to the number system $0\rightarrow 28$ basically at each step of the way (i.e. convert to mod 29).

                After \texttt{linalg.string\_to\_vector()}, the numbers will be already in mod 29. Once you invert a matrix, or do a matrix multiplication, however, first convert to mod 29 before moving on. 

                For instance, after finding the inverse matrix $A^{-1}$, I'll want to convert to mod 29 before applying $A^{-1}$ to the encrypted matrix. 

                Example:

                \begin{verbatim}
                A=[1 0 1;
                2 1 3;
                1 0 0]
                A_inv=inv(A)
                A_inv=linalg.mod_p(A_inv,29)
                \end{verbatim}

                will find the inverse, $A^{-1}=\begin{bmatrix}
                    0 & 0 & 1\\-3 & 1 & 1\\1 & 0 & -1
                \end{bmatrix}$, which is not mod 29. Before doing anything else for cryptography, the final line of MATLAB converts $A^{-1}$ to mod 29 to yield $A^{-1}=\begin{bmatrix}
                    0 & 0 & 1\\ 26 & 1 & 1 \\ 1 & 0 & 28
                \end{bmatrix}.$

            \end{hint}

            
            
        \end{problem}


    \end{enumerate}

\end{problem}

\end{document}