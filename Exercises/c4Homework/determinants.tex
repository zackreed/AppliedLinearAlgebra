\documentclass{ximera}
\graphicspath{     %% setup a global graphics path
{./}               %% look in the same-level directory
{./pictures/}      %% look in graphics
{../pictures/}     %% look up one directory, then in graphics
%{../../pictures/} %% look up two directories, then in graphics
}

\author{Zack Reed}
%borrowed from selinger linear algebra
\begin{document}

\section*{Exercises}


  %Let
  %\begin{equation*}
  %  A = \startmat{cc}
  %    a & b \\
  %    c & d \\
  %  \stopmat.
  %\end{equation*}
  %An operation is done to get from $A$ to a matrix $B$. In each case,
  %identify which operation was done and explain how it will affect the
  %value of the determinant.
  %\begin{enumerate}
  %\item
  %  \begin{equation*}
  %    B = \startmat{cc}
  %      a & c \\
  %%      b & d \\
   %   \stopmat
   % \end{equation*}
  %\item
  %  \begin{equation*}
  %    B = \startmat{cc}
  %      c & d \\
  %      a & b \\
  %    \stopmat
  %  \end{equation*}
  %\item
  %  \begin{equation*}
  %    B = \startmat{cc}
  %      a   & b   \\
  %      a+c & b+d \\
  %    \stopmat
  %  \end{equation*}
  %\item
  %  \begin{equation*}
  %    B = \startmat{cc}
  %      a  & b  \\
  %      2c & 2d \\
  %    \stopmat
  %  \end{equation*}
  %\item
  %  \begin{equation*}
  %    B = \startmat{cc}
  %      b & a \\
  %      d & c \\
  %    \stopmat
  %  \end{equation*}
  %\end{enumerate}


\begin{problem}
  Let $A$ be an $n\times n$-matrix and suppose there are $n-1$ columns
  such that the remaining column is a linear combination of these $n-1$
  columns. 
  
  It must be that $\det(A) = \answer{0}$.

  (Hint: The columns are linearly dependent.)
\end{problem}

\begin{problem}

  Let $A$ be an $n\times n$-matrix. If $\det(A) \neq 0$ and
  $A\vect{x}=\vect{0}$, then $\vect{x}=\answer{0}$.

  (Hint: Think about the relationship between invertibility, nullity, and the determinant.)

\end{problem}


\begin{problem}
  An $n\times n$-matrix is called \textbf{nilpotent}%
  \index{nilpotent matrix}%
  \index{matrix!nilpotent} if there exists some positive integer $k$
  such that $A^k = 0$. If $A$ is a nilpotent matrix, then $\det(A)=\answer{0}$.
\end{problem}

\begin{problem}
  A square matrix is said to be \textbf{orthogonal}%
  \index{matrix!orthogonal}%
  \index{orthogonal matrix} if $A^TA = I$. Thus the inverse of an
  orthogonal matrix is its transpose. 
  
  If $A$ is orthogonal, then $\det(A)$ can only be either $\answer{1}$ or $\answer{-1}$ (your first answer must be greater than your second).
\end{problem}


\begin{problem}
  Find the determinant of
  \begin{equation*}
    A = \startmat{ccc}
      1 & 1 & 1 & 1\\
      1 & a & a^2 & a^3\\
      1 & b & b^2 & b^3\\
      1 & c & c^2 & c^3\\
    \stopmat.
  \end{equation*}
  For which values of $a$, $b$ and $c$ (and the relationships between these variables) is this matrix not invertible? Hint:
  the \texttt{simplify} command in MATLAB can help make sense of complicated polynomials. 

  Answer: $a-b=\answer{0}$, $a-c=\answer{0}$, $b-c=\answer{0}$, $a=\answer{1}$, $b=\answer{1}$, $c=\answer{1}$.
\end{problem}

\begin{problem}
  Find the following determinants.
  \begin{equation*}
    A=\startmat{rrr}
      1 & 2 & 3 \\
      4 & 5 & 6 \\
      7 & 8 & 9 \\
    \stopmat
    \quad
    B=\startmat{rrr}
      1  & 0 & 2 \\
      2  & 5 & 3 \\
      -1 & 0 & 0 \\
    \stopmat
    \quad
    C=\startmat{rrr}
      3 &  4 & 1 \\
      0 & -1 & 1 \\
      1 &  2 & 1 \\
    \stopmat
    \quad
    D=\startmat{rrr}
      0  & -2 &  1 \\
      4  & 1  & -3 \\
      -1 & 3  &  1 \\
    \stopmat
  \end{equation*}

  \begin{enumerate}
    \item $det(A)=\answer{0}$
    \item $det(B)=\answer{10}$
    \item $det(C)=\answer{-4}$
    \item $det(D)=\answer{15}$
  \end{enumerate}
\end{problem}

\end{document}