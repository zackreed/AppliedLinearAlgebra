\documentclass{ximera}
\graphicspath{     %% setup a global graphics path
{./}               %% look in the same-level directory
{./pictures/}      %% look in graphics
{../pictures/}     %% look up one directory, then in graphics
%{../../pictures/} %% look up two directories, then in graphics
}

\author{Zack Reed}
%borrowed from selinger linear algebra
\begin{document}

\section*{Exercises}


  Let
  \begin{equation*}
    A = \startmat{cc}
      a & b \\
      c & d \\
    \stopmat.
  \end{equation*}
  An operation is done to get from $A$ to a matrix $B$. In each case,
  identify which operation was done and explain how it will affect the
  value of the determinant.
  \begin{enumerate}
  \item
    \begin{equation*}
      B = \startmat{cc}
        a & c \\
        b & d \\
      \stopmat
    \end{equation*}
  \item
    \begin{equation*}
      B = \startmat{cc}
        c & d \\
        a & b \\
      \stopmat
    \end{equation*}
  \item
    \begin{equation*}
      B = \startmat{cc}
        a   & b   \\
        a+c & b+d \\
      \stopmat
    \end{equation*}
  \item
    \begin{equation*}
      B = \startmat{cc}
        a  & b  \\
        2c & 2d \\
      \stopmat
    \end{equation*}
  \item
    \begin{equation*}
      B = \startmat{cc}
        b & a \\
        d & c \\
      \stopmat
    \end{equation*}
  \end{enumerate}


\begin{example}
  Let $A$ be an $n\times n$-matrix and suppose there are $n-1$ rows
  such that the remaining row is a linear combination of these $n-1$
  rows. Show $\det(A) = 0$.
\end{example}

\begin{example}
  \label{ex:determinant3}
  Let $A$ be an $n\times n$-matrix. Show that if $\det(A) \neq 0$ and
  $A\vect{x}=\vect{0}$, then $\vect{x}=\vect{0}$.
\end{example}

\begin{example}
  Using only Theorems~\ref{thm:determinant-of-triangular-matrix} and
  {\ref{thm:determinant-of-product}}, show that
  $\det(kA) = k^n\det(A)$ for an $n\times n$-matrix $A$ and scalar
  $k$.
\end{example}

\begin{example}
  Construct two random $2\times 2$-matrices $A$ and $B$ and verify
  that $\det(A)\det(B) = \det(AB)$.
\end{example}


\begin{example}
  Is it true that $\det(A+B) = \det(A) + \det(B)$? If this is so,
  explain why. If it is not so, give a counterexample.
\end{example}

\begin{example}
  An $n\times n$-matrix is called \textbf{nilpotent}%
  \index{nilpotent matrix}%
  \index{matrix!nilpotent} if there exists some positive integer $k$
  such that $A^k = 0$. If $A$ is a nilpotent matrix, what are the
  possible values of $\det(A)$?
\end{example}

\begin{example}
  A square matrix is said to be \textbf{orthogonal}%
  \index{matrix!orthogonal}%
  \index{orthogonal matrix} if $A^TA = I$. Thus the inverse of an
  orthogonal matrix is its transpose. What are the possible values of
  $\det(A)$ if $A$ is an orthogonal matrix?
\end{example}



\begin{example}
  Let $A$ and $B$ be two $n\times n$-matrices. We say that $A$ is
  \textbf{similar}%
  \index{matrix!similar}%
  \index{similar matrices} to $B$, in symbols $A$ is similar to $B$, if there
  exists an invertible matrix $P$ such that $A = P^{-1}BP$. Show that
  if $A$ is similar to $B$, then $\det(A) = \det(B)$.
\end{example}


\begin{example}
  Find the determinant of
  \begin{equation*}
    A = \startmat{ccc}
      1 & 1 & 1 \\
      1 & a & a^2 \\
      1 & b & b^2 \\
    \stopmat.
  \end{equation*}
  For which values of $a$ and $b$ is this matrix invertible? Hint:
  after you compute the determinant, you can factor out $(a-1)$ and
  $(b-1)$ from it.
\end{example}



\begin{example}
  Assume $A$, $B$, and $C$ are $n\times n$-matrices and $ABC$ is
  invertible. Use determinants to show that each of $A,B$, and $C$ is
  invertible.
\end{example}

\begin{example}
  Suppose $A$ is an upper triangular matrix. Show that $A^{-1}$ exists
  if and only if all elements of the main diagonal are non-zero. Is it
  true that $A^{-1}$ will also be upper triangular? Explain. Could the
  same be concluded for lower triangular matrices?
\end{example}

\begin{example}
  Specify whether each statement is true or false. If true, provide a
  proof. If false, provide a counterexample.
  \begin{enumerate}
  \item If $A$ is a $3\times 3$-matrix with determinant zero, then one
    column must be a multiple of some other column.

  \item If any two columns of a square matrix are equal, then the
    determinant of the matrix equals zero.

  \item For two $n\times n$-matrices $A$ and $B$,
    $\det(A+B) = \det(A) + \det(B)$.

  \item For an $n\times n$-matrix $A$, $\det(3A) = 3\det(A)$.

  \item If $A^{-1}$ exists, then $\det(A^{-1}) = \det(A)^{-1}$.

  \item If $B$ is obtained by multiplying a single row of $A$ by $4$,
    then $\det(B) = 4\det(A)$.

  \item For an $n\times n$-matrix $A$, we have
    $\det(-A) = (-1)^n\det(A)$.

  \item If $A$ is a real $n\times n$-matrix, then
    $\det(A^TA) \geq 0$.

  \item If $A^k = 0$ for some positive integer $k$, then
    $\det(A) = 0$.

  \item If $A\vect{x} = 0$ for some $\vect{x} \neq 0$, then
    $\det(A) = 0$.
  \end{enumerate}
\end{example}



\begin{example} 
Find the determinants of the following matrices.
  \begin{equation*}
    (a)~\startmat{rr}
      1 & 3 \\
      0 & 2
    \stopmat
    \quad
    (b)~\startmat{rr}
      0 & 3 \\
      0 & 2
    \stopmat
    \quad
    (c)~\startmat{rr}
      4 & 3 \\
      6 & 2
    \stopmat
    \quad
    (d)~\startmat{rr}
      -3 & 4 \\
      -1 & 2
    \stopmat
  \end{equation*}
\end{example}


\begin{example}
  Find the following determinants.
  \begin{equation*}
    (a)~\startmat{rrr}
      1 & 2 & 3 \\
      4 & 5 & 6 \\
      7 & 8 & 9 \\
    \stopmat
    \quad
    (b)~\startmat{rrr}
      1  & 0 & 2 \\
      2  & 5 & 3 \\
      -1 & 0 & 0 \\
    \stopmat
    \quad
    (c)~\startmat{rrr}
      3 &  4 & 1 \\
      0 & -1 & 1 \\
      1 &  2 & 1 \\
    \stopmat
    \quad
    (d)~\startmat{rrr}
      0  & -2 &  1 \\
      4  & 1  & -3 \\
      -1 & 3  &  1 \\
    \stopmat
  \end{equation*}
\end{example}

\end{document}