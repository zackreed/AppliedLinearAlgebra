\documentclass{ximera}
\graphicspath{  %% When looking for images,
{./}            %% look here first,
{./pictures/}   %% then look for a pictures folder,
{../pictures/}  %% which may be a directory up.
{../../pictures/}  %% which may be a directory up.
{../../../pictures/}  %% which may be a directory up.
{../../../../pictures/}  %% which may be a directory up.
}

\usepackage{listings}
%\usepackage{circuitikz}
\usepackage{xcolor}
\usepackage{amsmath,amsthm}
\usepackage{subcaption}
\usepackage{graphicx}
\usepackage{tikz}
%\usepackage{tikz-3dplot}
\usepackage{amsfonts}
%\usepackage{mdframed} % For framing content
%\usepackage{tikz-cd}

  \renewcommand{\vector}[1]{\left\langle #1\right\rangle}
  \newcommand{\arrowvec}[1]{{\overset{\rightharpoonup}{#1}}}
  \newcommand{\ro}{\texttt{R}}%% row operation
  \newcommand{\dotp}{\bullet}%% dot product
  \renewcommand{\l}{\ell}
  \let\defaultAnswerFormat\answerFormatBoxed
  \usetikzlibrary{calc,bending}
  \tikzset{>=stealth}
  




%make a maroon color
\definecolor{maroon}{RGB}{128,0,0}
%make a dark blue color
\definecolor{darkblue}{RGB}{0,0,139}
%define the color fourier0 to be the maroon color
\definecolor{fourier0}{RGB}{128,0,0}
%define the color fourier1 to be the dark blue color
\definecolor{fourier1}{RGB}{0,0,139}
%define the color fourier 1t to be the light blue color
\definecolor{fourier1t}{RGB}{173,216,230}
%define the color fourier2 to be the dark green color
\definecolor{fourier2}{RGB}{0,100,0}
%define teh color fourier2t to be the light green color
\definecolor{fourier2t}{RGB}{144,238,144}
%define the color fourier3 to be the dark purple color
\definecolor{fourier3}{RGB}{128,0,128}
%define the color fourier3t to be the light purple color
\definecolor{fourier3t}{RGB}{221,160,221}
%define the color fourier0t to be the red color
\definecolor{fourier0t}{RGB}{255,0,0}
%define the color fourier4 to be the orange color
\definecolor{fourier4}{RGB}{255,165,0}
%define the color fourier4t to be the darker orange color
\definecolor{fourier4t}{RGB}{255,215,0}
%define the color fourier5 to be the yellow color
\definecolor{fourier5}{RGB}{255,255,0}
%define the color fourier5t to be the darker yellow color
\definecolor{fourier5t}{RGB}{255,255,100}
%define the color fourier6 to be the green color
\definecolor{fourier6}{RGB}{0,128,0}
%define the color fourier6t to be the darker green color
\definecolor{fourier6t}{RGB}{0,255,0}

%New commands for this doc for errors in copying
\newcommand{\eigenvar}{\lambda}
%\newcommand{\vect}[1]{\mathbf{#1}}
\renewcommand{\th}{^{\text{th}}}
\newcommand{\st}{^{\text{st}}}
\newcommand{\nd}{^{\text{nd}}}
\newcommand{\rd}{^{\text{rd}}}
\newcommand{\paren}[1]{\left(#1\right)}
\newcommand{\abs}[1]{\left|#1\right|}
\newcommand{\R}{\mathbb{R}}
\newcommand{\C}{\mathbb{C}}
\newcommand{\Hilb}{\mathbb{H}}
\newcommand{\qq}[1]{\text{#1}}
\newcommand{\Z}{\mathbb{Z}}
\newcommand{\N}{\mathbb{N}}
\newcommand{\q}[1]{\text{``#1''}}
%\newcommand{\mat}[1]{\begin{bmatrix}#1\end{bmatrix}}
\newcommand{\rref}{\text{reduced row echelon form}}
\newcommand{\ef}{\text{echelon form}}
\newcommand{\ohm}{\Omega}
\newcommand{\volt}{\text{V}}
\newcommand{\amp}{\text{A}}
\newcommand{\Seq}{\textbf{Seq}}
\newcommand{\Poly}{\textbf{P}}
\renewcommand{\quad}{\text{    }}
\newcommand{\roweq}{\simeq}
\newcommand{\rowop}{\simeq}
\newcommand{\rowswap}{\leftrightarrow}
\newcommand{\Mat}{\textbf{M}}
\newcommand{\Func}{\textbf{Func}}
\newcommand{\Hw}{\textbf{Hamming weight}}
\newcommand{\Hd}{\textbf{Hamming distance}}
\newcommand{\rank}{\text{rank}}
\newcommand{\longvect}[1]{\overrightarrow{#1}}
% Define the circled command
\newcommand{\circled}[1]{%
  \tikz[baseline=(char.base)]{
    \node[shape=circle,draw,inner sep=2pt,red,fill=red!20,text=black] (char) {#1};}%
}

% Define custom command \strikeh that just puts red text on the 2nd argument
\newcommand{\strikeh}[2]{\textcolor{red}{#2}}

% Define custom command \strikev that just puts red text on the 2nd argument
\newcommand{\strikev}[2]{\textcolor{red}{#2}}

%more new commands for this doc for errors in copying
\newcommand{\SI}{\text{SI}}
\newcommand{\kg}{\text{kg}}
\newcommand{\m}{\text{m}}
\newcommand{\s}{\text{s}}
\newcommand{\norm}[1]{\left\|#1\right\|}
\newcommand{\col}{\text{col}}
\newcommand{\sspan}{\text{span}}
\newcommand{\proj}{\text{proj}}
\newcommand{\set}[1]{\left\{#1\right\}}
\newcommand{\degC}{^\circ\text{C}}
\newcommand{\centroid}[1]{\overline{#1}}
\newcommand{\dotprod}{\boldsymbol{\cdot}}
%\newcommand{\coord}[1]{\begin{bmatrix}#1\end{bmatrix}}
\newcommand{\iprod}[1]{\langle #1 \rangle}
\newcommand{\adjoint}{^{*}}
\newcommand{\conjugate}[1]{\overline{#1}}
\newcommand{\eigenvarA}{\lambda}
\newcommand{\eigenvarB}{\mu}
\newcommand{\orth}{\perp}
\newcommand{\bigbracket}[1]{\left[#1\right]}
\newcommand{\textiff}{\text{ if and only if }}
\newcommand{\adj}{\text{adj}}
\newcommand{\ijth}{\emph{ij}^\text{th}}
\newcommand{\minor}[2]{M_{#2}}
\newcommand{\cofactor}{\text{C}}
\newcommand{\shift}{\textbf{shift}}
\newcommand{\startmat}[1]{
  \left[\begin{array}{#1}
}
\newcommand{\stopmat}{\end{array}\right]}
%a command to give a name to explorations and hints and theorems
\newcommand{\name}[1]{\begin{centering}\textbf{#1}\end{centering}}
\newcommand{\vect}[1]{\vec{#1}}
\newcommand{\dfn}[1]{\textbf{#1}}
\newcommand{\transpose}{\mathsf{T}}
\newcommand{\mtlb}[2][black]{\texttt{\textcolor{#1}{#2}}}
\newcommand{\RR}{\mathbb{R}} % Real numbers
\newcommand{\id}{\text{id}}
\newcommand{\coord}[1]{\langle#1\rangle}
\newcommand{\RREF}{\text{RREF}}
\newcommand{\Null}{\text{Null}}
\newcommand{\Nullity}{\text{Nullity}}
\newcommand{\Rank}{\text{Rank}}
\newcommand{\Col}{\text{Col}}
\newcommand{\Ef}{\text{EF}}
\newcommand{\boxprod}[3]{\abs{(#1\times#2)\cdot#3}}

\author{Zack Reed}
%borrowed from selinger linear algebra
\begin{document}

\section*{Exercises}


  Let
  \begin{equation*}
    A = \startmat{cc}
      a & b \\
      c & d \\
    \stopmat.
  \end{equation*}
  An operation is done to get from $A$ to a matrix $B$. In each case,
  identify which operation was done and explain how it will affect the
  value of the determinant.
  \begin{enumerate}
  \item
    \begin{equation*}
      B = \startmat{cc}
        a & c \\
        b & d \\
      \stopmat
    \end{equation*}
  \item
    \begin{equation*}
      B = \startmat{cc}
        c & d \\
        a & b \\
      \stopmat
    \end{equation*}
  \item
    \begin{equation*}
      B = \startmat{cc}
        a   & b   \\
        a+c & b+d \\
      \stopmat
    \end{equation*}
  \item
    \begin{equation*}
      B = \startmat{cc}
        a  & b  \\
        2c & 2d \\
      \stopmat
    \end{equation*}
  \item
    \begin{equation*}
      B = \startmat{cc}
        b & a \\
        d & c \\
      \stopmat
    \end{equation*}
  \end{enumerate}


\begin{example}
  Let $A$ be an $n\times n$-matrix and suppose there are $n-1$ rows
  such that the remaining row is a linear combination of these $n-1$
  rows. Show $\det(A) = 0$.
\end{example}

\begin{example}
  \label{ex:determinant3}
  Let $A$ be an $n\times n$-matrix. Show that if $\det(A) \neq 0$ and
  $A\vect{x}=\vect{0}$, then $\vect{x}=\vect{0}$.
\end{example}

\begin{example}
  Using only Theorems~\ref{thm:determinant-of-triangular-matrix} and
  {\ref{thm:determinant-of-product}}, show that
  $\det(kA) = k^n\det(A)$ for an $n\times n$-matrix $A$ and scalar
  $k$.
\end{example}

\begin{example}
  Construct two random $2\times 2$-matrices $A$ and $B$ and verify
  that $\det(A)\det(B) = \det(AB)$.
\end{example}


\begin{example}
  Is it true that $\det(A+B) = \det(A) + \det(B)$? If this is so,
  explain why. If it is not so, give a counterexample.
\end{example}

\begin{example}
  An $n\times n$-matrix is called \textbf{nilpotent}%
  \index{nilpotent matrix}%
  \index{matrix!nilpotent} if there exists some positive integer $k$
  such that $A^k = 0$. If $A$ is a nilpotent matrix, what are the
  possible values of $\det(A)$?
\end{example}

\begin{example}
  A square matrix is said to be \textbf{orthogonal}%
  \index{matrix!orthogonal}%
  \index{orthogonal matrix} if $A^TA = I$. Thus the inverse of an
  orthogonal matrix is its transpose. What are the possible values of
  $\det(A)$ if $A$ is an orthogonal matrix?
\end{example}



\begin{example}
  Let $A$ and $B$ be two $n\times n$-matrices. We say that $A$ is
  \textbf{similar}%
  \index{matrix!similar}%
  \index{similar matrices} to $B$, in symbols $A$ is similar to $B$, if there
  exists an invertible matrix $P$ such that $A = P^{-1}BP$. Show that
  if $A$ is similar to $B$, then $\det(A) = \det(B)$.
\end{example}


\begin{example}
  Find the determinant of
  \begin{equation*}
    A = \startmat{ccc}
      1 & 1 & 1 \\
      1 & a & a^2 \\
      1 & b & b^2 \\
    \stopmat.
  \end{equation*}
  For which values of $a$ and $b$ is this matrix invertible? Hint:
  after you compute the determinant, you can factor out $(a-1)$ and
  $(b-1)$ from it.
\end{example}



\begin{example}
  Assume $A$, $B$, and $C$ are $n\times n$-matrices and $ABC$ is
  invertible. Use determinants to show that each of $A,B$, and $C$ is
  invertible.
\end{example}

\begin{example}
  Suppose $A$ is an upper triangular matrix. Show that $A^{-1}$ exists
  if and only if all elements of the main diagonal are non-zero. Is it
  true that $A^{-1}$ will also be upper triangular? Explain. Could the
  same be concluded for lower triangular matrices?
\end{example}

\begin{example}
  Specify whether each statement is true or false. If true, provide a
  proof. If false, provide a counterexample.
  \begin{enumerate}
  \item If $A$ is a $3\times 3$-matrix with determinant zero, then one
    column must be a multiple of some other column.

  \item If any two columns of a square matrix are equal, then the
    determinant of the matrix equals zero.

  \item For two $n\times n$-matrices $A$ and $B$,
    $\det(A+B) = \det(A) + \det(B)$.

  \item For an $n\times n$-matrix $A$, $\det(3A) = 3\det(A)$.

  \item If $A^{-1}$ exists, then $\det(A^{-1}) = \det(A)^{-1}$.

  \item If $B$ is obtained by multiplying a single row of $A$ by $4$,
    then $\det(B) = 4\det(A)$.

  \item For an $n\times n$-matrix $A$, we have
    $\det(-A) = (-1)^n\det(A)$.

  \item If $A$ is a real $n\times n$-matrix, then
    $\det(A^TA) \geq 0$.

  \item If $A^k = 0$ for some positive integer $k$, then
    $\det(A) = 0$.

  \item If $A\vect{x} = 0$ for some $\vect{x} \neq 0$, then
    $\det(A) = 0$.
  \end{enumerate}
\end{example}



\begin{example} 
Find the determinants of the following matrices.
  \begin{equation*}
    (a)~\startmat{rr}
      1 & 3 \\
      0 & 2
    \stopmat
    \quad
    (b)~\startmat{rr}
      0 & 3 \\
      0 & 2
    \stopmat
    \quad
    (c)~\startmat{rr}
      4 & 3 \\
      6 & 2
    \stopmat
    \quad
    (d)~\startmat{rr}
      -3 & 4 \\
      -1 & 2
    \stopmat
  \end{equation*}
\end{example}


\begin{example}
  Find the following determinants.
  \begin{equation*}
    (a)~\startmat{rrr}
      1 & 2 & 3 \\
      4 & 5 & 6 \\
      7 & 8 & 9 \\
    \stopmat
    \quad
    (b)~\startmat{rrr}
      1  & 0 & 2 \\
      2  & 5 & 3 \\
      -1 & 0 & 0 \\
    \stopmat
    \quad
    (c)~\startmat{rrr}
      3 &  4 & 1 \\
      0 & -1 & 1 \\
      1 &  2 & 1 \\
    \stopmat
    \quad
    (d)~\startmat{rrr}
      0  & -2 &  1 \\
      4  & 1  & -3 \\
      -1 & 3  &  1 \\
    \stopmat
  \end{equation*}
\end{example}

\end{document}