\documentclass{ximera}
\graphicspath{     %% setup a global graphics path
{./}               %% look in the same-level directory
{./pictures/}      %% look in graphics
{../pictures/}     %% look up one directory, then in graphics
%{../../pictures/} %% look up two directories, then in graphics
}

\author{Zack Reed}
%borrowed from anna davis
\begin{document}

%\begin{problem}\label{prob:areasquareandparal} Let $S$ be a square determined by $\begin{bmatrix}2\\0\end{bmatrix}$ and $\begin{bmatrix}0\\2\end{bmatrix}$.  Let $P$ be a parallelogram determined by vectors $\begin{bmatrix}2\\5\end{bmatrix}$ and $\begin{bmatrix}0\\2\end{bmatrix}$. 
%  \begin{enumerate}
%  \item Sketch both figures in the same coordinate plane, and use geometry to explain why $S$ and $P$ have the same area.  Compute the area of $P$ using Formula \ref{form:areaofparallelogramdeterminant}.
%   
%  $$\text{Area of }P=\answer{4}$$
%  \item Suppose $M$ is the standard matrix of a linear transformation $T:\RR^2\rightarrow\RR^2$ such that $T(S)=P$.  Find $\det M$.
%  $$\det M=\answer{1}$$
%  \end{enumerate}
%\end{problem}

   
\begin{problem}\label{prob:volparallelepiped}
  Find the volume of a parallelepiped determined by
  $$\begin{bmatrix}0\\5\\4\end{bmatrix}\quad\begin{bmatrix}3\\1\\2\end{bmatrix}\quad\begin{bmatrix}1\\1\\6\end{bmatrix}$$
  Answer: $\mbox{Volume}=\answer{72}$
  \end{problem}
   
  \begin{problem}\label{prob:volparallelepiped0}
  Find the volume of a parallelepiped determined by
  $$\begin{bmatrix}1\\4\\-1\end{bmatrix}\quad\begin{bmatrix}3\\-2\\4\end{bmatrix}\quad\begin{bmatrix}5\\6\\2\end{bmatrix}$$
  Explain your result geometrically.
   
  Answer: $\mbox{Volume}=\answer{0}$
  \end{problem}

\end{document}