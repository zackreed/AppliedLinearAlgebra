\documentclass{ximera}
\graphicspath{     %% setup a global graphics path
{./}               %% look in the same-level directory
{./pictures/}      %% look in graphics
{../pictures/}     %% look up one directory, then in graphics
%{../../pictures/} %% look up two directories, then in graphics
}

\author{Zack Reed}
%borrowed from selinger linear algebra
\begin{document}

\section*{Exercises}

\begin{example}
  Find the cofactor matrix and the adjugate of each of the following
  matrices.
  \begin{equation*}
    A =
    \startmat{rr}
      1 & 0 \\
      3 & 2 \\
    \stopmat,
    \quad
    B =
    \startmat{rrr}
      1  & 0  & 1  \\
      -1 & 1  & -2 \\
      2  & -1 & 3  \\
    \stopmat,
    \quad
    C =
    \startmat{rrr}
      1 &  1 & 3 \\
      2 &  3 & 1 \\
      1 & -1 & 1 \\
    \stopmat.
  \end{equation*}
\end{example}

\begin{example}
  For each of the following matrices, determine whether it is
  invertible by checking whether the determinant is non-zero. If the
  determinant is non-zero, use the adjugate formula to find the
  inverse.
  \begin{equation*}
    A =
    \startmat{rrr}
      1 & 2 & 3 \\
      0 & 2 & 1 \\
      3 & 1 & 0 \\
    \stopmat,
    \quad
    B =
    \startmat{rrr}
      1 & 2 & 0 \\
      0 & 2 & 1 \\
      3 & 1 & 1 \\
    \stopmat,
    \quad
    C =
    \startmat{rrr}
      1 & 3 & 3 \\
      2 & 4 & 1 \\
      0 & 1 & 1 \\
    \stopmat,
    \quad
    D =
    \startmat{rrr}
      1 & 2 & 3 \\
      0 & 2 & 1 \\
      2 & 6 & 7 \\
    \stopmat,
    \quad
    E =
    \startmat{rrr}
      1 & 0 & 3 \\
      1 & 0 & 1 \\
      3 & 1 & 0 \\
    \stopmat.
  \end{equation*}
\end{example}

\begin{example}
  Determine whether each of the following matrices is invertible. If
  so, use the adjugate formula to find the inverse. If the inverse
  does not exist, explain why.
  \begin{equation*}
    A = \startmat{rr}
      1 & 1 \\
      1 & 2 \\
    \stopmat,
    \quad
    B = \startmat{rrr}
      1 & 2 & 3 \\
      0 & 2 & 1 \\
      4 & 1 & 1 \\
    \stopmat,
    \quad
    C = \startmat{rrr}
      1 & 2 & 1 \\
      2 & 3 & 0 \\
      0 & 1 & 2 \\
    \stopmat.
  \end{equation*}
\end{example}

\begin{example}
  Use the adjugate formula to find the inverse of the matrix
  \begin{equation*}
    A=\startmat{rrr}
      3 & 0 & 3 \\
      -1 & 2 & -3 \\
      -5 & 4 & -3
    \stopmat.
  \end{equation*}
\end{example}

\begin{example}
  Use the adjugate formula to find the inverse of the matrix
  \begin{equation*}
    A=\startmat{rrr}
      1 &  1 & 0 \\
      3 &  1 & 2 \\
      2 & -2 & 5 \\
    \stopmat.
  \end{equation*}
\end{example}

\begin{example}
  Consider the matrix
  \begin{equation*}
    A =
    \startmat{ccc}
      1 & 0 & 0 \\
      0 & \cos t & -\sin t \\
      0 & \sin t & \cos t \\
    \stopmat.
  \end{equation*}
  Does there exist a value of $t$ for which this matrix fails to be
  invertible? Explain.
\end{example}

\begin{example}
  Consider the matrix
  \begin{equation*}
    A =
    \startmat{rrr}
      1 & t & t^2 \\
      0 & 1 & 2t \\
      t & 0 & 2
    \stopmat.
  \end{equation*}
  Does there exist a value of $t$ for which this matrix fails to be
  invertible? Explain.
\end{example}

\begin{example}
  Consider the matrix
  \begin{equation*}
    A =
    \startmat{ccc}
      e^{t} & \cosh t & \sinh t \\
      e^{t} & \sinh t & \cosh t \\
      e^{t} & \cosh t & \sinh t
    \stopmat.
  \end{equation*}
  Does there exist a value of $t$ for which this matrix fails to be
  invertible? Explain.
\end{example}

\begin{example}
  Consider the matrix
  \begin{equation*}
    A =
    \startmat{ccc}
      e^{t} & e^{-t}\cos t & e^{-t}\sin t \\
      e^{t} & -e^{-t}\cos t-e^{-t}\sin t & -e^{-t}\sin t+e^{-t}\cos t \\
      e^{t} & 2e^{-t}\sin t & -2e^{-t}\cos t
    \stopmat.
  \end{equation*}
  Does there exist a value of $t$ for which this matrix fails to be
  invertible? Explain.
\end{example}

\begin{example}
  Use the adjugate formula to find the inverse of the matrix
  \begin{equation*}
    A=\startmat{ccc}
      e^{t} & 0 & 0 \\
      0 & \cos t & \sin t \\
      0 & \cos t-\sin t & \cos t+\sin t
    \stopmat.
  \end{equation*}
\end{example}

\begin{example}
  Find the inverse, if it exists, of the matrix
  \begin{equation*}
    A =
    \startmat{ccc}
      e^{t} & \cos t & \sin t \\
      e^{t} & -\sin t & \cos t \\
      e^{t} & -\cos t & -\sin t
    \stopmat.
  \end{equation*}
\end{example}

\end{document}