\documentclass{ximera}
\graphicspath{     %% setup a global graphics path
{./}               %% look in the same-level directory
{./pictures/}      %% look in graphics
{../pictures/}     %% look up one directory, then in graphics
%{../../pictures/} %% look up two directories, then in graphics
}

\author{Zack Reed}
%borrowed from selinger linear algebra
\begin{document}

  \begin{problem}
    Find the volume of the parallelepiped determined by the vectors
    $\startmat{r}
      1 \\
      -7 \\
      -5
    \stopmat$,
    $\startmat{r}
      1 \\
      -2 \\
      -6
    \stopmat$, and $\startmat{r}
      3 \\
      2 \\
      3
    \stopmat$.
    
    %$\startmat{r}1\\1\\3\stopmat
    %    \times \startmat{r}-7\\-2\\2\stopmat 
    %  \cdot \startmat{r}-5\\-6\\3\stopmat =
    %  \startmat{r}8\\-23\\5\stopmat
    %  \cdot \startmat{r}-5\\-6\\3\stopmat =
    The volume is $\answer{113}$.
    
  \end{problem}

  
  
  %\begin{problem} \label{exer-box-product-zero}
  %  What does it mean geometrically if the box product of three vectors
  %  equals zero?
  %  \begin{solution}
  %    It means that if you place them so that they all have their tails
  %    at the same point, the three will lie in the same plane.
  %  \end{solution}
  %\end{problem}
  
  
  \begin{problem}
    Find the area of the parallelogram determined by the vectors
    $\startmat{r}
      1 \\
      2 \\
      3
    \stopmat$, $\startmat{r}
      3 \\
      -2 \\
      1
    \stopmat$.

      $\startmat{r}
        1 \\
        2 \\
        3
      \stopmat \times
      \startmat{r}
        3 \\
        -2 \\
        1
      \stopmat = \startmat{r}
        \answer{8} \\
        \answer{8} \\
        \answer{-8}
      \stopmat$. 
      
      The area of the parallelogram is $\answer[tolerance=.5]{8\sqrt{3}}$.
    
  \end{problem}
  
  
  \begin{problem}
    Find the area of the parallelogram with vertices $(-2,3,1)$,
    $(2,1,1)$, $(1,2,-1)$, and $(5,0,-1)$.

    %\begin{solution}
    %  Let $P=(-2,3,1)$, $Q=(2,1,1)$, $R=(1,2,-1)$, and $S=(5,0,-1)$.
    %  We have
    %  $\longvect{PQ}\times\longvect{PR}
    %  =
    %  \startmat{r} 4 \\ -2 \\ 0 \stopmat
    %  \times
    %  \startmat{r} 3 \\ -1 \\ -2 \stopmat
    %  =
    %  \startmat{r} 5 \\ 8 \\ 2 \stopmat$.
    %  The area of the parallelogram is
    %  $\norm{\longvect{PQ}\times\longvect{PR}} = \sqrt{5^2+8^2+2^2} =
    The area is $\answer[tolerance=.5]{\sqrt{93}}$.
    %\end{solution}
  \end{problem}
  
  \begin{problem}
    Find the area of the triangle determined by the three points,
    $(1,0,3),(4,1,0)$ and $(-3,1,1)$.
    
    %  Let $P=(1,0,3)$, $Q=(4,1,0)$, and $R=(-3,1,1)$. 
    %  $\longvect{PQ}\times\longvect{PR}
    %  = \startmat{r}
    %    3 \\
    %    1 \\
    %    -3
    %  \stopmat \times \startmat{r}
    %    -4 \\
    %    1 \\
    %    -2
    %  \stopmat = \startmat{r}
    %    1 \\
    %    18 \\
    %    7
    %  \stopmat$. 
    
    The area of the triangle is
     % $\displaystyle\frac{1}{2}\norm{\longvect{PQ}\times\longvect{PR}} =
     % \frac{1}{2}\sqrt{1+18^2+7^2}=\frac{1}{2}
     $\answer[tolerance=1]{\sqrt{374}/2}$. 
    %\end{solution}
  \end{problem}
  
  %\begin{problem}
  %  Find the area of the triangle determined by the three points,
  %  $(1,2,3),(2,3,4)$ and $(3,4,5)$. Did something
  %  interesting happen here? What does it mean geometrically?
  %  \begin{solution}
  %    Let $P=(1,2,3)$, $Q=(2,3,4)$, and $R=(3,4,5)$. 
  %    $\longvect{PQ}\times\longvect{PR}
  %    = \startmat{r}
  %      1 \\ 1 \\ 1
  %    \stopmat \times \startmat{r}
  %      2 \\ 2 \\ 2
  %    \stopmat = \startmat{r}
  %      0 \\ 0 \\ 0
  %    \stopmat$.
  %    The area of the triangle is 0. It means the three points are on
  %    a line.
  %  \end{solution}
  %\end{problem}

\end{document}