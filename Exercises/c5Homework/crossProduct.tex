\documentclass{ximera}
\graphicspath{     %% setup a global graphics path
{./}               %% look in the same-level directory
{./pictures/}      %% look in graphics
{../pictures/}     %% look up one directory, then in graphics
%{../../pictures/} %% look up two directories, then in graphics
}

\author{Zack Reed}
%borrowed from selinger linear algebra
\begin{document}

  \begin{example}
    Find the volume of the parallelepiped determined by the vectors
    $\startmat{r}
      1 \\
      -7 \\
      -5
    \stopmat$,
    $\startmat{r}
      1 \\
      -2 \\
      -6
    \stopmat$, and $\startmat{r}
      3 \\
      2 \\
      3
    \stopmat$.
    \begin{solution}
      $\startmat{r}1\\1\\3\stopmat
        \times \startmat{r}-7\\-2\\2\stopmat 
      \cdot \startmat{r}-5\\-6\\3\stopmat =
      \startmat{r}8\\-23\\5\stopmat
      \cdot \startmat{r}-5\\-6\\3\stopmat =
      113$.
    \end{solution}
  \end{example}


  
  \begin{example}
    Which of the following systems of vectors
    $\vect{u},\vect{v},\vect{w}$ are right-handed? Which are
    left-handed? Which are coplanar?
    \begin{enumerate}
    \item $\vect{u}=\startmat{c}1\\0\\1\stopmat$, $\vect{v}=\startmat{c}1\\2\\0\stopmat$, $\vect{w}=\startmat{c}0\\0\\1\stopmat$.
    \item $\vect{u}=\startmat{c}0\\1\\1\stopmat$, $\vect{v}=\startmat{c}-1\\2\\0\stopmat$, $\vect{w}=\startmat{c}1\\1\\2\stopmat$.
    \item $\vect{u}=\startmat{c}1\\-1\\0\stopmat$, $\vect{v}=\startmat{c}1\\0\\1\stopmat$, $\vect{w}=\startmat{c}3\\1\\4\stopmat$.
    \item $\vect{u}=\startmat{c}1\\0\\0\stopmat$, $\vect{v}=\startmat{c}1\\2\\0\stopmat$, $\vect{w}=\startmat{c}2\\0\\-1\stopmat$.
    \end{enumerate}
    \begin{solution}
      \begin{enumerate}
      \item We have
        $\boxprod{\vect{u},\vect{v},\vect{w}} = (\vect{u}\times\vect{v})
        \cdot \vect{w} = \startmat{c}-2\\1\\2\stopmat\cdot\startmat{c}0\\0\\1\stopmat=2$, so
        the box product is positive. This means that the system of vectors
        $\vect{u},\vect{v},\vect{w}$ is right-handed.
      \item We have
        $\boxprod{\vect{u},\vect{v},\vect{w}} = (\vect{u}\times\vect{v})
        \cdot \vect{w} = \startmat{c}-2\\-1\\1\stopmat\cdot\startmat{c}1\\1\\2\stopmat=-1$, so
        the box product is negative and the system is left-handed.
      \item We have
        $\boxprod{\vect{u},\vect{v},\vect{w}} = (\vect{u}\times\vect{v})
        \cdot \vect{w} = \startmat{c}-1\\-1\\1\stopmat\cdot\startmat{c}3\\1\\4\stopmat=0$, so
        the box product is zero. This means that the volume of the
        parallelepiped spanned by $\vect{u},\vect{v},\vect{w}$ is empty,
        i.e., the vectors are coplanar.
      \item We have
        $\boxprod{\vect{u},\vect{v},\vect{w}} = (\vect{u}\times\vect{v})
        \cdot \vect{w} = \startmat{c}0\\0\\2\stopmat\cdot\startmat{c}2\\0\\-1\stopmat=-2$, so
        the box product is negative and the system is left-handed.
      \end{enumerate}
    \end{solution}
  \end{example}
  
  
  \begin{example} \label{exer-box-product-zero}
    What does it mean geometrically if the box product of three vectors
    equals zero?
    \begin{solution}
      It means that if you place them so that they all have their tails
      at the same point, the three will lie in the same plane.
    \end{solution}
  \end{example}
  
  
  \begin{example}
    Find the area of the parallelogram determined by the vectors
    $\startmat{r}
      1 \\
      2 \\
      3
    \stopmat$, $\startmat{r}
      3 \\
      -2 \\
      1
    \stopmat$.
    \begin{solution}
      $\startmat{r}
        1 \\
        2 \\
        3
      \stopmat \times
      \startmat{r}
        3 \\
        -2 \\
        1
      \stopmat = \startmat{r}
        8 \\
        8 \\
        -8
      \stopmat$. The area of the parallelogram is $8\sqrt{3}$.
    \end{solution}
  \end{example}
  
  \begin{example}
    Find the area of the parallelogram determined by the vectors
    $\startmat{r}
      1 \\
      0 \\
      3
    \stopmat$, $\startmat{r}
      4 \\
      -2 \\
      1
    \stopmat$.
    \begin{solution}
      $\startmat{r}
        1 \\
        0 \\
        3
      \stopmat \times
      \startmat{r}
        4 \\
        -2 \\
        1
      \stopmat = \startmat{r}
        6 \\
        11 \\
        -2
      \stopmat$. The area of the parallelogram is
      $\sqrt{36+121+4}=\sqrt{161}$. 
    \end{solution}
  \end{example}
  
  \begin{example}
    Find the area of the parallelogram with vertices $(-2,3,1)$,
    $(2,1,1)$, $(1,2,-1)$, and $(5,0,-1)$.
    \begin{solution}
      Let $P=(-2,3,1)$, $Q=(2,1,1)$, $R=(1,2,-1)$, and $S=(5,0,-1)$.
      We have
      $\longvect{PQ}\times\longvect{PR}
      =
      \startmat{r} 4 \\ -2 \\ 0 \stopmat
      \times
      \startmat{r} 3 \\ -1 \\ -2 \stopmat
      =
      \startmat{r} 5 \\ 8 \\ 2 \stopmat$.
      The area of the parallelogram is
      $\norm{\longvect{PQ}\times\longvect{PR}} = \sqrt{5^2+8^2+2^2} =
      \sqrt{93}$.
    \end{solution}
  \end{example}
  
  \begin{example}
    Find the area of the triangle determined by the three points,
    $(1,0,3),(4,1,0)$ and $(-3,1,1)$.
    \begin{solution}
      Let $P=(1,0,3)$, $Q=(4,1,0)$, and $R=(-3,1,1)$. 
      $\longvect{PQ}\times\longvect{PR}
      = \startmat{r}
        3 \\
        1 \\
        -3
      \stopmat \times \startmat{r}
        -4 \\
        1 \\
        -2
      \stopmat = \startmat{r}
        1 \\
        18 \\
        7
      \stopmat$. The area of the triangle is
      $\displaystyle\frac{1}{2}\norm{\longvect{PQ}\times\longvect{PR}} =
      \frac{1}{2}\sqrt{1+18^2+7^2}=\frac{1}{2}\sqrt{374}$. 
    \end{solution}
  \end{example}
  
  \begin{example}
    Find the area of the triangle determined by the three points,
    $(1,2,3),(2,3,4)$ and $(3,4,5)$. Did something
    interesting happen here? What does it mean geometrically?
    \begin{solution}
      Let $P=(1,2,3)$, $Q=(2,3,4)$, and $R=(3,4,5)$. 
      $\longvect{PQ}\times\longvect{PR}
      = \startmat{r}
        1 \\ 1 \\ 1
      \stopmat \times \startmat{r}
        2 \\ 2 \\ 2
      \stopmat = \startmat{r}
        0 \\ 0 \\ 0
      \stopmat$.
      The area of the triangle is 0. It means the three points are on
      a line.
    \end{solution}
  \end{example}

\end{document}