\documentclass{ximera}
\graphicspath{  %% When looking for images,
{./}            %% look here first,
{./pictures/}   %% then look for a pictures folder,
{../pictures/}  %% which may be a directory up.
{../../pictures/}  %% which may be a directory up.
{../../../pictures/}  %% which may be a directory up.
{../../../../pictures/}  %% which may be a directory up.
}

\usepackage{listings}
%\usepackage{circuitikz}
\usepackage{xcolor}
\usepackage{amsmath,amsthm}
\usepackage{subcaption}
\usepackage{graphicx}
\usepackage{tikz}
%\usepackage{tikz-3dplot}
\usepackage{amsfonts}
%\usepackage{mdframed} % For framing content
%\usepackage{tikz-cd}

  \renewcommand{\vector}[1]{\left\langle #1\right\rangle}
  \newcommand{\arrowvec}[1]{{\overset{\rightharpoonup}{#1}}}
  \newcommand{\ro}{\texttt{R}}%% row operation
  \newcommand{\dotp}{\bullet}%% dot product
  \renewcommand{\l}{\ell}
  \let\defaultAnswerFormat\answerFormatBoxed
  \usetikzlibrary{calc,bending}
  \tikzset{>=stealth}
  




%make a maroon color
\definecolor{maroon}{RGB}{128,0,0}
%make a dark blue color
\definecolor{darkblue}{RGB}{0,0,139}
%define the color fourier0 to be the maroon color
\definecolor{fourier0}{RGB}{128,0,0}
%define the color fourier1 to be the dark blue color
\definecolor{fourier1}{RGB}{0,0,139}
%define the color fourier 1t to be the light blue color
\definecolor{fourier1t}{RGB}{173,216,230}
%define the color fourier2 to be the dark green color
\definecolor{fourier2}{RGB}{0,100,0}
%define teh color fourier2t to be the light green color
\definecolor{fourier2t}{RGB}{144,238,144}
%define the color fourier3 to be the dark purple color
\definecolor{fourier3}{RGB}{128,0,128}
%define the color fourier3t to be the light purple color
\definecolor{fourier3t}{RGB}{221,160,221}
%define the color fourier0t to be the red color
\definecolor{fourier0t}{RGB}{255,0,0}
%define the color fourier4 to be the orange color
\definecolor{fourier4}{RGB}{255,165,0}
%define the color fourier4t to be the darker orange color
\definecolor{fourier4t}{RGB}{255,215,0}
%define the color fourier5 to be the yellow color
\definecolor{fourier5}{RGB}{255,255,0}
%define the color fourier5t to be the darker yellow color
\definecolor{fourier5t}{RGB}{255,255,100}
%define the color fourier6 to be the green color
\definecolor{fourier6}{RGB}{0,128,0}
%define the color fourier6t to be the darker green color
\definecolor{fourier6t}{RGB}{0,255,0}

%New commands for this doc for errors in copying
\newcommand{\eigenvar}{\lambda}
%\newcommand{\vect}[1]{\mathbf{#1}}
\renewcommand{\th}{^{\text{th}}}
\newcommand{\st}{^{\text{st}}}
\newcommand{\nd}{^{\text{nd}}}
\newcommand{\rd}{^{\text{rd}}}
\newcommand{\paren}[1]{\left(#1\right)}
\newcommand{\abs}[1]{\left|#1\right|}
\newcommand{\R}{\mathbb{R}}
\newcommand{\C}{\mathbb{C}}
\newcommand{\Hilb}{\mathbb{H}}
\newcommand{\qq}[1]{\text{#1}}
\newcommand{\Z}{\mathbb{Z}}
\newcommand{\N}{\mathbb{N}}
\newcommand{\q}[1]{\text{``#1''}}
%\newcommand{\mat}[1]{\begin{bmatrix}#1\end{bmatrix}}
\newcommand{\rref}{\text{reduced row echelon form}}
\newcommand{\ef}{\text{echelon form}}
\newcommand{\ohm}{\Omega}
\newcommand{\volt}{\text{V}}
\newcommand{\amp}{\text{A}}
\newcommand{\Seq}{\textbf{Seq}}
\newcommand{\Poly}{\textbf{P}}
\renewcommand{\quad}{\text{    }}
\newcommand{\roweq}{\simeq}
\newcommand{\rowop}{\simeq}
\newcommand{\rowswap}{\leftrightarrow}
\newcommand{\Mat}{\textbf{M}}
\newcommand{\Func}{\textbf{Func}}
\newcommand{\Hw}{\textbf{Hamming weight}}
\newcommand{\Hd}{\textbf{Hamming distance}}
\newcommand{\rank}{\text{rank}}
\newcommand{\longvect}[1]{\overrightarrow{#1}}
% Define the circled command
\newcommand{\circled}[1]{%
  \tikz[baseline=(char.base)]{
    \node[shape=circle,draw,inner sep=2pt,red,fill=red!20,text=black] (char) {#1};}%
}

% Define custom command \strikeh that just puts red text on the 2nd argument
\newcommand{\strikeh}[2]{\textcolor{red}{#2}}

% Define custom command \strikev that just puts red text on the 2nd argument
\newcommand{\strikev}[2]{\textcolor{red}{#2}}

%more new commands for this doc for errors in copying
\newcommand{\SI}{\text{SI}}
\newcommand{\kg}{\text{kg}}
\newcommand{\m}{\text{m}}
\newcommand{\s}{\text{s}}
\newcommand{\norm}[1]{\left\|#1\right\|}
\newcommand{\col}{\text{col}}
\newcommand{\sspan}{\text{span}}
\newcommand{\proj}{\text{proj}}
\newcommand{\set}[1]{\left\{#1\right\}}
\newcommand{\degC}{^\circ\text{C}}
\newcommand{\centroid}[1]{\overline{#1}}
\newcommand{\dotprod}{\boldsymbol{\cdot}}
%\newcommand{\coord}[1]{\begin{bmatrix}#1\end{bmatrix}}
\newcommand{\iprod}[1]{\langle #1 \rangle}
\newcommand{\adjoint}{^{*}}
\newcommand{\conjugate}[1]{\overline{#1}}
\newcommand{\eigenvarA}{\lambda}
\newcommand{\eigenvarB}{\mu}
\newcommand{\orth}{\perp}
\newcommand{\bigbracket}[1]{\left[#1\right]}
\newcommand{\textiff}{\text{ if and only if }}
\newcommand{\adj}{\text{adj}}
\newcommand{\ijth}{\emph{ij}^\text{th}}
\newcommand{\minor}[2]{M_{#2}}
\newcommand{\cofactor}{\text{C}}
\newcommand{\shift}{\textbf{shift}}
\newcommand{\startmat}[1]{
  \left[\begin{array}{#1}
}
\newcommand{\stopmat}{\end{array}\right]}
%a command to give a name to explorations and hints and theorems
\newcommand{\name}[1]{\begin{centering}\textbf{#1}\end{centering}}
\newcommand{\vect}[1]{\vec{#1}}
\newcommand{\dfn}[1]{\textbf{#1}}
\newcommand{\transpose}{\mathsf{T}}
\newcommand{\mtlb}[2][black]{\texttt{\textcolor{#1}{#2}}}
\newcommand{\RR}{\mathbb{R}} % Real numbers
\newcommand{\id}{\text{id}}
\newcommand{\coord}[1]{\langle#1\rangle}
\newcommand{\RREF}{\text{RREF}}
\newcommand{\Null}{\text{Null}}
\newcommand{\Nullity}{\text{Nullity}}
\newcommand{\Rank}{\text{Rank}}
\newcommand{\Col}{\text{Col}}
\newcommand{\Ef}{\text{EF}}
\newcommand{\boxprod}[3]{\abs{(#1\times#2)\cdot#3}}

\author{Zack Reed}
%borrowed from selinger linear algebra
\begin{document}

  \begin{example}
    Find the volume of the parallelepiped determined by the vectors
    $\startmat{r}
      1 \\
      -7 \\
      -5
    \stopmat$,
    $\startmat{r}
      1 \\
      -2 \\
      -6
    \stopmat$, and $\startmat{r}
      3 \\
      2 \\
      3
    \stopmat$.
    \begin{solution}
      $\startmat{r}1\\1\\3\stopmat
        \times \startmat{r}-7\\-2\\2\stopmat 
      \cdot \startmat{r}-5\\-6\\3\stopmat =
      \startmat{r}8\\-23\\5\stopmat
      \cdot \startmat{r}-5\\-6\\3\stopmat =
      113$.
    \end{solution}
  \end{example}


  
  \begin{example}
    Which of the following systems of vectors
    $\vect{u},\vect{v},\vect{w}$ are right-handed? Which are
    left-handed? Which are coplanar?
    \begin{enumerate}
    \item $\vect{u}=\startmat{c}1\\0\\1\stopmat$, $\vect{v}=\startmat{c}1\\2\\0\stopmat$, $\vect{w}=\startmat{c}0\\0\\1\stopmat$.
    \item $\vect{u}=\startmat{c}0\\1\\1\stopmat$, $\vect{v}=\startmat{c}-1\\2\\0\stopmat$, $\vect{w}=\startmat{c}1\\1\\2\stopmat$.
    \item $\vect{u}=\startmat{c}1\\-1\\0\stopmat$, $\vect{v}=\startmat{c}1\\0\\1\stopmat$, $\vect{w}=\startmat{c}3\\1\\4\stopmat$.
    \item $\vect{u}=\startmat{c}1\\0\\0\stopmat$, $\vect{v}=\startmat{c}1\\2\\0\stopmat$, $\vect{w}=\startmat{c}2\\0\\-1\stopmat$.
    \end{enumerate}
    \begin{solution}
      \begin{enumerate}
      \item We have
        $\boxprod{\vect{u},\vect{v},\vect{w}} = (\vect{u}\times\vect{v})
        \cdot \vect{w} = \startmat{c}-2\\1\\2\stopmat\cdot\startmat{c}0\\0\\1\stopmat=2$, so
        the box product is positive. This means that the system of vectors
        $\vect{u},\vect{v},\vect{w}$ is right-handed.
      \item We have
        $\boxprod{\vect{u},\vect{v},\vect{w}} = (\vect{u}\times\vect{v})
        \cdot \vect{w} = \startmat{c}-2\\-1\\1\stopmat\cdot\startmat{c}1\\1\\2\stopmat=-1$, so
        the box product is negative and the system is left-handed.
      \item We have
        $\boxprod{\vect{u},\vect{v},\vect{w}} = (\vect{u}\times\vect{v})
        \cdot \vect{w} = \startmat{c}-1\\-1\\1\stopmat\cdot\startmat{c}3\\1\\4\stopmat=0$, so
        the box product is zero. This means that the volume of the
        parallelepiped spanned by $\vect{u},\vect{v},\vect{w}$ is empty,
        i.e., the vectors are coplanar.
      \item We have
        $\boxprod{\vect{u},\vect{v},\vect{w}} = (\vect{u}\times\vect{v})
        \cdot \vect{w} = \startmat{c}0\\0\\2\stopmat\cdot\startmat{c}2\\0\\-1\stopmat=-2$, so
        the box product is negative and the system is left-handed.
      \end{enumerate}
    \end{solution}
  \end{example}
  
  
  \begin{example} \label{exer-box-product-zero}
    What does it mean geometrically if the box product of three vectors
    equals zero?
    \begin{solution}
      It means that if you place them so that they all have their tails
      at the same point, the three will lie in the same plane.
    \end{solution}
  \end{example}
  
  
  \begin{example}
    Find the area of the parallelogram determined by the vectors
    $\startmat{r}
      1 \\
      2 \\
      3
    \stopmat$, $\startmat{r}
      3 \\
      -2 \\
      1
    \stopmat$.
    \begin{solution}
      $\startmat{r}
        1 \\
        2 \\
        3
      \stopmat \times
      \startmat{r}
        3 \\
        -2 \\
        1
      \stopmat = \startmat{r}
        8 \\
        8 \\
        -8
      \stopmat$. The area of the parallelogram is $8\sqrt{3}$.
    \end{solution}
  \end{example}
  
  \begin{example}
    Find the area of the parallelogram determined by the vectors
    $\startmat{r}
      1 \\
      0 \\
      3
    \stopmat$, $\startmat{r}
      4 \\
      -2 \\
      1
    \stopmat$.
    \begin{solution}
      $\startmat{r}
        1 \\
        0 \\
        3
      \stopmat \times
      \startmat{r}
        4 \\
        -2 \\
        1
      \stopmat = \startmat{r}
        6 \\
        11 \\
        -2
      \stopmat$. The area of the parallelogram is
      $\sqrt{36+121+4}=\sqrt{161}$. 
    \end{solution}
  \end{example}
  
  \begin{example}
    Find the area of the parallelogram with vertices $(-2,3,1)$,
    $(2,1,1)$, $(1,2,-1)$, and $(5,0,-1)$.
    \begin{solution}
      Let $P=(-2,3,1)$, $Q=(2,1,1)$, $R=(1,2,-1)$, and $S=(5,0,-1)$.
      We have
      $\longvect{PQ}\times\longvect{PR}
      =
      \startmat{r} 4 \\ -2 \\ 0 \stopmat
      \times
      \startmat{r} 3 \\ -1 \\ -2 \stopmat
      =
      \startmat{r} 5 \\ 8 \\ 2 \stopmat$.
      The area of the parallelogram is
      $\norm{\longvect{PQ}\times\longvect{PR}} = \sqrt{5^2+8^2+2^2} =
      \sqrt{93}$.
    \end{solution}
  \end{example}
  
  \begin{example}
    Find the area of the triangle determined by the three points,
    $(1,0,3),(4,1,0)$ and $(-3,1,1)$.
    \begin{solution}
      Let $P=(1,0,3)$, $Q=(4,1,0)$, and $R=(-3,1,1)$. 
      $\longvect{PQ}\times\longvect{PR}
      = \startmat{r}
        3 \\
        1 \\
        -3
      \stopmat \times \startmat{r}
        -4 \\
        1 \\
        -2
      \stopmat = \startmat{r}
        1 \\
        18 \\
        7
      \stopmat$. The area of the triangle is
      $\displaystyle\frac{1}{2}\norm{\longvect{PQ}\times\longvect{PR}} =
      \frac{1}{2}\sqrt{1+18^2+7^2}=\frac{1}{2}\sqrt{374}$. 
    \end{solution}
  \end{example}
  
  \begin{example}
    Find the area of the triangle determined by the three points,
    $(1,2,3),(2,3,4)$ and $(3,4,5)$. Did something
    interesting happen here? What does it mean geometrically?
    \begin{solution}
      Let $P=(1,2,3)$, $Q=(2,3,4)$, and $R=(3,4,5)$. 
      $\longvect{PQ}\times\longvect{PR}
      = \startmat{r}
        1 \\ 1 \\ 1
      \stopmat \times \startmat{r}
        2 \\ 2 \\ 2
      \stopmat = \startmat{r}
        0 \\ 0 \\ 0
      \stopmat$.
      The area of the triangle is 0. It means the three points are on
      a line.
    \end{solution}
  \end{example}

\end{document}