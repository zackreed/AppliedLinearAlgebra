\documentclass{ximera}
\graphicspath{     %% setup a global graphics path
{./}               %% look in the same-level directory
{./pictures/}      %% look in graphics
{../pictures/}     %% look up one directory, then in graphics
%{../../pictures/} %% look up two directories, then in graphics
}

\author{Zack Reed}
%borrowed from selinger linear algebra
\begin{document}

\begin{example}
    Find $\startmat{r}
      1 \\
      2 \\
      3 \\
      4
    \stopmat \dotprod \startmat{r}
      2 \\
      0 \\
      1 \\
      3
    \stopmat$.
    \begin{solution}
      $\startmat{r}
        1 \\
        2 \\
        3 \\
        4
      \stopmat \dotprod \startmat{r}
        2 \\
        0 \\
        1 \\
        3
      \stopmat = 17$.
    \end{solution}
  \end{example}
  

  
  \begin{example}
    Find $\cos \theta$ where $\theta$ is the angle between the vectors
    \begin{equation*}
      \vect{u}
      =
      \startmat{r}
        3 \\
        -1 \\
        -1
      \stopmat,
      \vect{v}
      =
      \startmat{r}
        1 \\
        4 \\
        2
      \stopmat
    \end{equation*}
    \begin{solution}
      $\displaystyle\cos\theta=\frac{\startmat{ccc}
          3 & -1 & -1
        \stopmat^T \dotprod
        \startmat{ccc}
          1 & 4 & 2
        \stopmat^T}
      {\sqrt{9+1+1}\sqrt{1+16+4}}= \frac{-3}{\sqrt{11}\sqrt{21}}$.
    \end{solution}
  \end{example}
  
  \begin{example}
    Find $\proj_{\vect{v}}(\vect{w})$ where
    $\vect{w}=\startmat{r}
      1 \\
      0 \\
      -2
    \stopmat$ and $\vect{v}=\startmat{r}
      1 \\
      2 \\
      3
    \stopmat$.
    \begin{solution}
      $\displaystyle
      \def\arraystretch{1.2}
      \frac{\vect{v}\dotprod \vect{w}}{\vect{v}\dotprod
        \vect{v}}\,\vect{v}=\frac{-5}{14}\startmat{r}
        1 \\
        2 \\
        3
      \stopmat =\startmat{c}
        -\frac{5}{14} \\
        -\frac{5}{7} \\
        -\frac{15}{14}
      \stopmat$.
    \end{solution}
  \end{example}
  
  \begin{example}
    Decompose the vector $\vect{v}$ into $\vect{v}=\vect{a}+\vect{b}$
    where $\vect{a}$ is parallel to $\vect{u}$ and $\vect{b}$ is
    orthogonal to $\vect{u}$.
    \begin{equation*}
      \vect{v} = \startmat{r} 3\\2\\-5 \stopmat,\quad
      \vect{u} = \startmat{r} 1\\-1\\2 \stopmat.
    \end{equation*}
  
    \begin{solution}
      \begin{equation*}
        \vect{a} ~=~ \startmat{c} -3/2\\3/2\\-3 \stopmat,\quad
        \vect{b} ~=~ \startmat{c} 9/2\\1/2\\-2 \stopmat.
      \end{equation*}
    \end{solution}
  \end{example}

\end{document}