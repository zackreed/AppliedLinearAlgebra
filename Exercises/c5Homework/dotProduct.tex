\documentclass{ximera}
\graphicspath{     %% setup a global graphics path
{./}               %% look in the same-level directory
{./pictures/}      %% look in graphics
{../pictures/}     %% look up one directory, then in graphics
%{../../pictures/} %% look up two directories, then in graphics
}

\author{Zack Reed}
%borrowed from selinger linear algebra
\begin{document}

\begin{example}
    Find $\startmat{r}
      1 \\
      2 \\
      3 \\
      4
    \stopmat \dotprod \startmat{r}
      2 \\
      0 \\
      1 \\
      3
    \stopmat$.
    \begin{solution}
      $\startmat{r}
        1 \\
        2 \\
        3 \\
        4
      \stopmat \dotprod \startmat{r}
        2 \\
        0 \\
        1 \\
        3
      \stopmat = 17$.
    \end{solution}
  \end{example}
  
  \begin{example}
    Let $\vect{a},\vect{b}$ be vectors. Show that
    $(\vect{a}\dotprod \vect{b})
    = \frac{1}{4}(\norm{\vect{a}+\vect{b}}^2
    - \norm{\vect{a}-\vect{b}}^2)$.
    % \begin{solution}
    % \end{solution}
  \end{example}
  
  \begin{example}
    Using the properties of the dot product, prove the parallelogram
    identity:
    \begin{equation*}
      \norm{\vect{a}+\vect{b}}^2+\norm{\vect{a}-\vect{b}}^2
      = 2\norm{\vect{a}}^2+2\norm{\vect{b}}^2
    \end{equation*}
    % \begin{solution}
    % \end{solution}
  \end{example}
  
  \begin{example}
    Find $\cos \theta$ where $\theta$ is the angle between the vectors
    \begin{equation*}
      \vect{u}
      =
      \startmat{r}
        3 \\
        -1 \\
        -1
      \stopmat,
      \vect{v}
      =
      \startmat{r}
        1 \\
        4 \\
        2
      \stopmat
    \end{equation*}
    \begin{solution}
      $\displaystyle\cos\theta=\frac{\startmat{ccc}
          3 & -1 & -1
        \stopmat^T \dotprod
        \startmat{ccc}
          1 & 4 & 2
        \stopmat^T}
      {\sqrt{9+1+1}\sqrt{1+16+4}}= \frac{-3}{\sqrt{11}\sqrt{21}}$.
    \end{solution}
  \end{example}
  
  \begin{example}
    Find $\proj_{\vect{v}}(\vect{w})$ where
    $\vect{w}=\startmat{r}
      1 \\
      0 \\
      -2
    \stopmat$ and $\vect{v}=\startmat{r}
      1 \\
      2 \\
      3
    \stopmat$.
    \begin{solution}
      $\displaystyle
      \def\arraystretch{1.2}
      \frac{\vect{v}\dotprod \vect{w}}{\vect{v}\dotprod
        \vect{v}}\,\vect{v}=\frac{-5}{14}\startmat{r}
        1 \\
        2 \\
        3
      \stopmat =\startmat{c}
        -\frac{5}{14} \\
        -\frac{5}{7} \\
        -\frac{15}{14}
      \stopmat$.
    \end{solution}
  \end{example}
  
  \begin{example}
    Decompose the vector $\vect{v}$ into $\vect{v}=\vect{a}+\vect{b}$
    where $\vect{a}$ is parallel to $\vect{u}$ and $\vect{b}$ is
    orthogonal to $\vect{u}$.
    \begin{equation*}
      \vect{v} = \startmat{r} 3\\2\\-5 \stopmat,\quad
      \vect{u} = \startmat{r} 1\\-1\\2 \stopmat.
    \end{equation*}
  
    \begin{solution}
      \begin{equation*}
        \vect{a} ~=~ \startmat{c} -3/2\\3/2\\-3 \stopmat,\quad
        \vect{b} ~=~ \startmat{c} 9/2\\1/2\\-2 \stopmat.
      \end{equation*}
    \end{solution}
  \end{example}
  
  \begin{example}\label{perp-linear}
    Let $\vect{v},\vect{w},\vect{u}$ be vectors. Show that
    $(\vect{w}+\vect{u})_{\perp}=\vect{w}_{\perp}+\vect{u}_{\perp}$,
    where $\vect{w}_{\perp}=\vect{w}-\proj_{\vect{v}}(\vect{w})$.
    \begin{solution}
      \begin{eqnarray*}
        \vect{w}-\proj_{\vect{v}}(\vect{w})+\vect{u}-\proj_{\vect{v}}(\vect{u})
        &=& \vect{w}+\vect{u}-(\proj_{\vect{v}}(\vect{w})+\proj_{\vect{v}}(\vect{u})) \\
        &=&\vect{w}+\vect{u}-\proj_{\vect{v}}(\vect{w}+\vect{u}).
      \end{eqnarray*}
      This follows because
      \begin{eqnarray*}
        \proj_{\vect{v}}(\vect{w})+\proj_{\vect{v}}(\vect{u})
        &=& \frac{\vect{v}\dotprod\vect{w}}{\norm{\vect{v}}^2}\vect{v}
            + \frac{\vect{v}\dotprod\vect{u}}{\norm{\vect{v}}^2}\vect{v} \\
        &=& \frac{\vect{v}\dotprod(\vect{w}+\vect{u})}{\norm{\vect{v}}^2}\vect{v} \\
        &=& \proj_{\vect{v}}(\vect{w}+\vect{u}).
      \end{eqnarray*}
    \end{solution}
  \end{example}
  
  \begin{example}
    Show that
    \begin{equation*}
      \vect{u}\dotprod(\vect{v}-\proj_{\vect{u}}(\vect{v}))=0
    \end{equation*}
    and conclude every vector in $\R^n$ can be written as the sum of
    two vectors, one which is orthogonal and one which is parallel to
    the given vector.
    \begin{solution}
      $\displaystyle\vect{u}\dotprod(\vect{v}-\proj_{\vect{u}}(\vect{v}))
      = \vect{u} \dotprod \vect{v}-\frac{\vect{u}\cdot\vect{v}}{\norm{\vect{u}}^2}\vect{u}\dotprod\vect{u}
      = \vect{u} \dotprod \vect{v}
      - \vect{u} \dotprod \vect{v}
      =0$. Therefore, we can write
      $\vect{v} = (\vect{v} - \proj_{\vect{u}}(\vect{v})) +
      \proj_{\vect{u}}(\vect{v})$. The first is orthogonal to
      $\vect{u}$ and the second is a multiple of $\vect{u}$ so is
      parallel to $\vect{u}$.
    \end{solution}
  \end{example}

\end{document}