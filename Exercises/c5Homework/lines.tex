\documentclass{ximera}
\graphicspath{     %% setup a global graphics path
{./}               %% look in the same-level directory
{./pictures/}      %% look in graphics
{../pictures/}     %% look up one directory, then in graphics
%{../../pictures/} %% look up two directories, then in graphics
}

\author{Zack Reed}
%borrowed from selinger linear algebra
\begin{document}

\section*{Exercises}

\begin{exercise}
  Find the vector equation for the line through $(-7,6,0)$ and
  $(-1,1,4)$. Then, find the parametric equations for this
  line.
\end{exercise}

\begin{exercise}
  Find parametric equations for the line through the point
  $(7,7,1)$ with direction vector
  $\vect{d} = \startmat{c} 1 \\ 6 \\ 2 \stopmat$.
\end{exercise}

\begin{exercise}
  Parametric equations of the line are
  \begin{equation*}
    \begin{array}{c}
      x = t+2, \\
      y = 6-3t, \\
      z = -t-6.
    \end{array}
  \end{equation*}
  Find a direction vector for the line and a point on the line.
\end{exercise}

\begin{exercise}
  The equation of a line in two dimensions is written as $y=x-5$. Find
  a vector equation for this line.
\end{exercise}

\begin{exercise}
  Find parametric equations for the line through $(6, 5, -2, 3)$
  and $(5, 1, 2, 1)$.
\end{exercise}

\begin{exercise}
  Consider the following vector equation for a line in $\R^3$:
  \begin{equation*}
    \startmat{c} x\\y\\z \stopmat
    = \startmat{r} 1\\2\\0 \stopmat
    + t\,\startmat{r} 1\\0\\1 \stopmat.
  \end{equation*}
  Find a new vector equation for the same line by doing the change of
  parameter $t=2-s$.
  \begin{solution}
    We have
    \begin{equation*}
      \startmat{c} x\\y\\z \stopmat
      = \startmat{r} 1\\2\\0 \stopmat
      + (2-s)\,\startmat{c} 1\\0\\1 \stopmat
      = \startmat{r} 3\\2\\2 \stopmat
      + s\,\startmat{r} -1\\0\\-1 \stopmat.
    \end{equation*}
  \end{solution}
\end{exercise}

\begin{exercise}
  Consider the line given by the following parametric equations:
  \begin{equation*}
    \begin{array}{c}
      x = 2t+2, \\
      y = 5-4t, \\
      z= -t-3.
    \end{array}
  \end{equation*}
  Find symmetric equations for the line.
\end{exercise}

\begin{exercise}
  Find the point on the line segment from $P = (-4, 7, 5)$ to
  $Q = (2, -2, -3)$ which is $\frac{1}{7}$ of the way from $P$
  to $Q$.
\end{exercise}


\begin{exercise}
  Determine whether the lines
  \begin{equation*}
    \startmat{c} x \\ y \\ z \stopmat
    = \startmat{c} 1 \\ 1 \\ 2 \stopmat
    + t \startmat{c} 1 \\ 2 \\ 2 \stopmat
    \quad\mbox{and}\quad
    \startmat{c} x \\ y \\ z \stopmat
    = \startmat{c} 1 \\ -1 \\ -4 \stopmat
    + s \startmat{c} 1 \\ 1 \\ -1 \stopmat
  \end{equation*}
  intersect. If yes, find the point of intersection.
\end{exercise}

\begin{exercise}
  Determine whether the lines
  \begin{equation*}
    \startmat{c} x \\ y \\ z \stopmat
    = \startmat{c} 2 \\ -1 \\ 0 \stopmat
    + t \startmat{c} 1 \\ 3 \\ 2 \stopmat
    \quad\mbox{and}\quad
    \startmat{c} x \\ y \\ z \stopmat
    = \startmat{c} 1 \\ 1 \\ 3 \stopmat
    + s \startmat{c} 1 \\ 2 \\ 0 \stopmat
  \end{equation*}
  intersect. If yes, find the point of intersection.
\end{exercise}

\begin{exercise}
  Find the angle between the two lines
  \begin{equation*}
    \startmat{r} x \\ y \\ z \stopmat
    = \startmat{r} 3 \\ 0 \\ 1 \stopmat
    + t \startmat{r} 3 \\ -3 \\ 0 \stopmat
    \quad\mbox{and}\quad
    \startmat{r} x \\ y \\ z \stopmat
    = \startmat{r} 3 \\ 0 \\ 1 \stopmat
    + s \startmat{r} -1 \\ 2 \\ 2 \stopmat.
  \end{equation*}
\end{exercise}

\begin{exercise} Let $P = (1,2,3)$ be a point in $\R^3$. Let $L$ be the line
  through the point $P_0 = (1, 4, 5)$ with direction vector
  $\vect{d} = \startmat{c} 1 \\ -1 \\
    1 \stopmat$. Find the shortest distance from $P$ to
  $L$, and find the point $Q$ on $L$ that is closest to $P$.
\end{exercise}

\begin{exercise} Let $P = (0,2,1)$ be a point in $\R^3$. Let $L$ be the line through the points $P_0 = (1, 1, 1)$ and $P_1 = (4, 1, 2)$. Find the shortest distance from $P$ to $L$, and find the point $Q$ on $L$ that is closest to $P$.
\end{exercise}


\end{document}