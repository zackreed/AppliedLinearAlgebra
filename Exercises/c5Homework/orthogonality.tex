\documentclass{ximera}
\graphicspath{     %% setup a global graphics path
{./}               %% look in the same-level directory
{./pictures/}      %% look in graphics
{../pictures/}     %% look up one directory, then in graphics
%{../../pictures/} %% look up two directories, then in graphics
}

\author{Zack Reed}
%borrowed from selinger linear algebra
\begin{document}

\section*{Exercises}

\begin{exercise}
  Let $A=\startmat{ccc} 1 & 1 & 0 \\ 1 & 2 & 0 \\ 0 & 0 & 2 \stopmat$, and consider $\R^3$ with the inner product given by
  $\iprod{\vect{u},\vect{v}} = \vect{u}^TA\vect{v}$. Which of the
  following vectors are orthogonal to each other?
  \begin{equation*}
    \vect{u}_1 = \startmat{c}  1 \\  1 \\  1 \stopmat,\quad
    \vect{u}_2 = \startmat{c} -1 \\  2 \\ -2 \stopmat,\quad
    \vect{u}_3 = \startmat{c}  7 \\ -5 \\ -2 \stopmat,\quad
    \vect{u}_4 = \startmat{c} 10 \\ -2 \\ -7 \stopmat.
  \end{equation*}
  \begin{solution}
    We have
    $\iprod{\vect{u}_1,\vect{u}_2} = \vect{u}_1^T A \vect{u}_2 = 0$,
    $\iprod{\vect{u}_1,\vect{u}_3} = \vect{u}_1^T A \vect{u}_3 = -5$,
    $\iprod{\vect{u}_1,\vect{u}_4} = \vect{u}_1^T A \vect{u}_4 = 0$,
    $\iprod{\vect{u}_2,\vect{u}_3} = \vect{u}_2^T A \vect{u}_3 = 0$,
    $\iprod{\vect{u}_2,\vect{u}_4} = \vect{u}_2^T A \vect{u}_4 = 32$,
    and
    $\iprod{\vect{u}_3,\vect{u}_4} = \vect{u}_3^T A \vect{u}_4 = 54$.
    Therefore, $\vect{u}_1\orth\vect{u}_2$,
    $\vect{u}_1\orth\vect{u}_4$, and $\vect{u}_2\orth\vect{u}_3$. None
    of the other pairs of vectors are orthogonal.
  \end{solution}
\end{exercise}

\begin{exercise}
  On $C[-1,1]$, which of the following functions are orthogonal to each other?
  \begin{equation*}
    f_1(x) = x,\quad
    f_2(x) = x^2,\quad
    f_3(x) = x^3-x,\quad
    f_4(x) = 1-x^4.
  \end{equation*}
  \begin{solution}
    $f_1\orth f_2$, $f_1\orth f_4$, $f_2\orth f_3$, and $f_3\orth f_4$.
  \end{solution}
\end{exercise}

\begin{exercise}
  Consider $\R^3$ as an inner product space with the usual dot
  product.  For each of the following bases of $\R^3$, state whether
  it is orthonormal, orthogonal, or neither.
  \begin{enumerate}
  \item $\set{
      \startmat{r} 1 \\ 0 \\ 0 \stopmat,
      \startmat{r} 0 \\ 1 \\ 0 \stopmat,
      \startmat{r} 0 \\ 0 \\ 1 \stopmat
    }$.
  \item $\set{
      \startmat{r} 1 \\ 0 \\ 1 \stopmat,
      \startmat{r} 0 \\ 1 \\ 1 \stopmat,
      \startmat{r} 0 \\ 0 \\ 1 \stopmat
    }$.
  \item $\set{
      \startmat{r}  1 \\ 0 \\ 2 \stopmat,
      \startmat{r}  0 \\ 1 \\ 0 \stopmat,
      \startmat{r} -2 \\ 0 \\ 1 \stopmat
    }$.
  \item $\def\arraystretch{1.2}
    \set{
      \startmat{r} \frac{3}{5} \\ \frac{4}{5} \\ 0 \stopmat,
      \startmat{r} 0 \\ 0 \\ -1 \stopmat,
      \startmat{r} \frac{4}{5} \\ -\frac{3}{5} \\ 0 \stopmat
    }$.
  \end{enumerate}
  \begin{solution}
    (a) Orthonormal (therefore also orthogonal). (b) Neither
    orthogonal nor orthonormal. (c) Orthogonal (not orthonormal). (d)
    Orthonormal (therefore also orthogonal).
  \end{solution}
\end{exercise}

\begin{exercise}
  Suppose $B=\set{\vect{u}_1,\vect{u}_2,\vect{u}_3}$ is an orthogonal
  basis for an inner product space $V$, such that
  $\norm{\vect{u}_1}=2$, $\norm{\vect{u}_2}=\sqrt{3}$, and
  $\norm{\vect{u}_3}=\sqrt{5}$.  Moreover, suppose that
  $\vect{v}\in V$ is a vector such that
  $\iprod{\vect{v},\vect{u}_1} = 1$,
  $\iprod{\vect{v},\vect{u}_2} = 2$, and
  $\iprod{\vect{v},\vect{u}_3} = -4$.  Find the coordinates of
  $\vect{v}$ with respect to $B$.
  \begin{solution}
    $\vect{v} = \frac{1}{4}\vect{u}_1 + \frac{2}{3}\vect{u}_2 - \frac{4}{5}\vect{u}_3$.
  \end{solution}
\end{exercise}

\begin{exercise}
  Suppose $B=\set{\vect{u}_1,\vect{u}_2,\vect{u}_3}$ is an
  orthogonal basis of $\R^3$. We have been told that
  \begin{equation*}
    \vect{u}_1 = \startmat{c} 1 \\ 1 \\ 0 \stopmat,
  \end{equation*}
  but it is not known what $\vect{u}_2$ and $\vect{u}_3$ are. Find the
  first coordinate of the vector
  \begin{equation*}
    \vect{v} = \startmat{c} 1 \\ 0 \\ 2 \stopmat
  \end{equation*}
  with respect to the basis $B$.
  \begin{solution}
    We have $\vect{v} = a_1\vect{u}_1 + a_2\vect{u}_2 + a_3\vect{u}_3$
    where
    $a_1 =
    \frac{\iprod{\vect{u}_1,\vect{v}}}{\iprod{\vect{u}_1,\vect{u}_1}}
    = \frac{1}{2}$. So the first coordinate is $\frac{1}{2}$.
  \end{solution}
\end{exercise}


\end{document}