\documentclass{ximera}
\graphicspath{     %% setup a global graphics path
{./}               %% look in the same-level directory
{./pictures/}      %% look in graphics
{../pictures/}     %% look up one directory, then in graphics
%{../../pictures/} %% look up two directories, then in graphics
}

\author{Zack Reed}
%borrowed from selinger linear algebra
\begin{document}


\section*{Exercises}

\begin{exercise}
  Find vector and parametric equations for the plane through the
  points $P = (0,1,1)$, $Q = (-1,2,1)$, and $R = (1,1,2)$.
\end{exercise}

\begin{exercise}
  Consider the following vector equation for a plane in $\R^4$:
  \begin{equation*}
    \startmat{c} x\\y\\z\\w \stopmat
    = \startmat{r} 1\\2\\0\\0 \stopmat
    + t\,\startmat{r} 1\\0\\0\\1 \stopmat
    + s\,\startmat{r} -1\\-1\\1\\0 \stopmat.
  \end{equation*}
  Find a new vector equation for the same plane by doing the change of
  parameters%
  \index{plane!change of parameters}%
  \index{change of parameters!plane}
  $t=1-r_1$, $s=r_1+r_2$.
  \begin{solution}
    We have
    \begin{equation*}
      \startmat{c} x\\y\\z\\w \stopmat
      = \startmat{r} 1\\2\\0\\0 \stopmat
      + (1-r_1)\,\startmat{r} 1\\0\\0\\1 \stopmat
      + (r_1+r_2)\,\startmat{r} -1\\-1\\1\\0 \stopmat
      = \startmat{r} 2\\2\\0\\1 \stopmat
      + r_1\,\startmat{r} -2\\-1\\1\\-1 \stopmat
      + r_2\,\startmat{r} -1\\-1\\1\\0 \stopmat.
    \end{equation*}
  \end{solution}
\end{exercise}

\begin{exercise}
  Determine which of the following points lie on the plane through the
  points $P = (2,6,1)$, $R = (1,4,1)$, and $Q = (1,2,-1)$.
  \begin{enumerate}
  \item $S_1=(1,2,4)$.
  \item $S_2=(1,5,2)$.
  \item $S_3=(0,0,0)$.
  \end{enumerate}
\end{exercise}

\begin{exercise}
  Use cross products to find the normal vector to the plane going
  through the points $P=(1,2,3)$, $Q=(-2,1,8)$ and $R=(2,2,2)$.
\end{exercise}

\begin{exercise}
  Find normal and standard equations of the plane through the
  point $P=(1,1,2)$ and orthogonal to $\vect{n}=\startmat{c} 1 \\ 0 \\ -1 \stopmat$.
\end{exercise}

\begin{exercise}
  Find normal and standard equations for the plane through the points
  $P = (2,1,0)$, $Q=(1,-1,0)$, and $R=(1,1,-1)$.
\end{exercise}

\begin{exercise}
  Find a vector equation for the plane $2x+y-z=1$.
\end{exercise}

\begin{exercise}
  Find the intersection between the planes $x+3y+4z=3$ and $2x+5y-z=2$.
  Is the intersection a line, a plane, or empty?
\end{exercise}

\begin{exercise}
  Find the intersection of the line
  \begin{equation*}
    \startmat{r} x \\ y \\ z \stopmat
    = \startmat{r} 1 \\ 1 \\ 1 \stopmat
    + t \startmat{r} 2 \\ -1 \\ 2 \stopmat
  \end{equation*}
  and the plane $x+3y+z = 6$.
  Is the intersection a point, a line, or empty?
\end{exercise}

\begin{exercise}
  Find the angle between the planes $x+y=5$ and $2x+y-z=4$.
\end{exercise}

\begin{exercise}
  Find the angle between the line
  \begin{equation*}
    \startmat{r} x \\ y \\ z \stopmat
    = \startmat{r} 0 \\ 3 \\ 7 \stopmat
    + t \startmat{r} 1 \\ 1 \\ 4 \stopmat
  \end{equation*}
  and the plane $4x+7y+4z = 6$.
\end{exercise}

\begin{exercise}
  Find the shortest distance from the point $P = (1,1,-1)$ to the plane
  given by $x + 2y + 2z = 6$, and find the point $Q$ on the plane
  that is closest to $P$.
\end{exercise}

\begin{exercise}
  Use Exercise~\ref{exer-box-product-zero} to find an equation of a
  plane containing the two vectors $\vect{p}$ and $\vect{q}$ and the
  point $0$. \textbf{Hint:} If $(x,y,z)$ is a point in this
  plane, the volume of the parallelepiped determined by $(x,y,z)$
  and the vectors $\vect{p}$, $\vect{q}$ equals 0.
  \begin{solution}
    $\vect{x}\dotprod (\vect{a}\times \vect{b}) =0$.
  \end{solution}
\end{exercise}


\end{document}