\documentclass{ximera}
\graphicspath{  %% When looking for images,
{./}            %% look here first,
{./pictures/}   %% then look for a pictures folder,
{../pictures/}  %% which may be a directory up.
{../../pictures/}  %% which may be a directory up.
{../../../pictures/}  %% which may be a directory up.
{../../../../pictures/}  %% which may be a directory up.
}

\usepackage{listings}
%\usepackage{circuitikz}
\usepackage{xcolor}
\usepackage{amsmath,amsthm}
\usepackage{subcaption}
\usepackage{graphicx}
\usepackage{tikz}
%\usepackage{tikz-3dplot}
\usepackage{amsfonts}
%\usepackage{mdframed} % For framing content
%\usepackage{tikz-cd}

  \renewcommand{\vector}[1]{\left\langle #1\right\rangle}
  \newcommand{\arrowvec}[1]{{\overset{\rightharpoonup}{#1}}}
  \newcommand{\ro}{\texttt{R}}%% row operation
  \newcommand{\dotp}{\bullet}%% dot product
  \renewcommand{\l}{\ell}
  \let\defaultAnswerFormat\answerFormatBoxed
  \usetikzlibrary{calc,bending}
  \tikzset{>=stealth}
  




%make a maroon color
\definecolor{maroon}{RGB}{128,0,0}
%make a dark blue color
\definecolor{darkblue}{RGB}{0,0,139}
%define the color fourier0 to be the maroon color
\definecolor{fourier0}{RGB}{128,0,0}
%define the color fourier1 to be the dark blue color
\definecolor{fourier1}{RGB}{0,0,139}
%define the color fourier 1t to be the light blue color
\definecolor{fourier1t}{RGB}{173,216,230}
%define the color fourier2 to be the dark green color
\definecolor{fourier2}{RGB}{0,100,0}
%define teh color fourier2t to be the light green color
\definecolor{fourier2t}{RGB}{144,238,144}
%define the color fourier3 to be the dark purple color
\definecolor{fourier3}{RGB}{128,0,128}
%define the color fourier3t to be the light purple color
\definecolor{fourier3t}{RGB}{221,160,221}
%define the color fourier0t to be the red color
\definecolor{fourier0t}{RGB}{255,0,0}
%define the color fourier4 to be the orange color
\definecolor{fourier4}{RGB}{255,165,0}
%define the color fourier4t to be the darker orange color
\definecolor{fourier4t}{RGB}{255,215,0}
%define the color fourier5 to be the yellow color
\definecolor{fourier5}{RGB}{255,255,0}
%define the color fourier5t to be the darker yellow color
\definecolor{fourier5t}{RGB}{255,255,100}
%define the color fourier6 to be the green color
\definecolor{fourier6}{RGB}{0,128,0}
%define the color fourier6t to be the darker green color
\definecolor{fourier6t}{RGB}{0,255,0}

%New commands for this doc for errors in copying
\newcommand{\eigenvar}{\lambda}
%\newcommand{\vect}[1]{\mathbf{#1}}
\renewcommand{\th}{^{\text{th}}}
\newcommand{\st}{^{\text{st}}}
\newcommand{\nd}{^{\text{nd}}}
\newcommand{\rd}{^{\text{rd}}}
\newcommand{\paren}[1]{\left(#1\right)}
\newcommand{\abs}[1]{\left|#1\right|}
\newcommand{\R}{\mathbb{R}}
\newcommand{\C}{\mathbb{C}}
\newcommand{\Hilb}{\mathbb{H}}
\newcommand{\qq}[1]{\text{#1}}
\newcommand{\Z}{\mathbb{Z}}
\newcommand{\N}{\mathbb{N}}
\newcommand{\q}[1]{\text{``#1''}}
%\newcommand{\mat}[1]{\begin{bmatrix}#1\end{bmatrix}}
\newcommand{\rref}{\text{reduced row echelon form}}
\newcommand{\ef}{\text{echelon form}}
\newcommand{\ohm}{\Omega}
\newcommand{\volt}{\text{V}}
\newcommand{\amp}{\text{A}}
\newcommand{\Seq}{\textbf{Seq}}
\newcommand{\Poly}{\textbf{P}}
\renewcommand{\quad}{\text{    }}
\newcommand{\roweq}{\simeq}
\newcommand{\rowop}{\simeq}
\newcommand{\rowswap}{\leftrightarrow}
\newcommand{\Mat}{\textbf{M}}
\newcommand{\Func}{\textbf{Func}}
\newcommand{\Hw}{\textbf{Hamming weight}}
\newcommand{\Hd}{\textbf{Hamming distance}}
\newcommand{\rank}{\text{rank}}
\newcommand{\longvect}[1]{\overrightarrow{#1}}
% Define the circled command
\newcommand{\circled}[1]{%
  \tikz[baseline=(char.base)]{
    \node[shape=circle,draw,inner sep=2pt,red,fill=red!20,text=black] (char) {#1};}%
}

% Define custom command \strikeh that just puts red text on the 2nd argument
\newcommand{\strikeh}[2]{\textcolor{red}{#2}}

% Define custom command \strikev that just puts red text on the 2nd argument
\newcommand{\strikev}[2]{\textcolor{red}{#2}}

%more new commands for this doc for errors in copying
\newcommand{\SI}{\text{SI}}
\newcommand{\kg}{\text{kg}}
\newcommand{\m}{\text{m}}
\newcommand{\s}{\text{s}}
\newcommand{\norm}[1]{\left\|#1\right\|}
\newcommand{\col}{\text{col}}
\newcommand{\sspan}{\text{span}}
\newcommand{\proj}{\text{proj}}
\newcommand{\set}[1]{\left\{#1\right\}}
\newcommand{\degC}{^\circ\text{C}}
\newcommand{\centroid}[1]{\overline{#1}}
\newcommand{\dotprod}{\boldsymbol{\cdot}}
%\newcommand{\coord}[1]{\begin{bmatrix}#1\end{bmatrix}}
\newcommand{\iprod}[1]{\langle #1 \rangle}
\newcommand{\adjoint}{^{*}}
\newcommand{\conjugate}[1]{\overline{#1}}
\newcommand{\eigenvarA}{\lambda}
\newcommand{\eigenvarB}{\mu}
\newcommand{\orth}{\perp}
\newcommand{\bigbracket}[1]{\left[#1\right]}
\newcommand{\textiff}{\text{ if and only if }}
\newcommand{\adj}{\text{adj}}
\newcommand{\ijth}{\emph{ij}^\text{th}}
\newcommand{\minor}[2]{M_{#2}}
\newcommand{\cofactor}{\text{C}}
\newcommand{\shift}{\textbf{shift}}
\newcommand{\startmat}[1]{
  \left[\begin{array}{#1}
}
\newcommand{\stopmat}{\end{array}\right]}
%a command to give a name to explorations and hints and theorems
\newcommand{\name}[1]{\begin{centering}\textbf{#1}\end{centering}}
\newcommand{\vect}[1]{\vec{#1}}
\newcommand{\dfn}[1]{\textbf{#1}}
\newcommand{\transpose}{\mathsf{T}}
\newcommand{\mtlb}[2][black]{\texttt{\textcolor{#1}{#2}}}
\newcommand{\RR}{\mathbb{R}} % Real numbers
\newcommand{\id}{\text{id}}
\newcommand{\coord}[1]{\langle#1\rangle}
\newcommand{\RREF}{\text{RREF}}
\newcommand{\Null}{\text{Null}}
\newcommand{\Nullity}{\text{Nullity}}
\newcommand{\Rank}{\text{Rank}}
\newcommand{\Col}{\text{Col}}
\newcommand{\Ef}{\text{EF}}
\newcommand{\boxprod}[3]{\abs{(#1\times#2)\cdot#3}}

\author{Zack Reed}
%borrowed from selinger linear algebra
\begin{document}


\section*{Exercises}

\begin{exercise}
  Find vector and parametric equations for the plane through the
  points $P = (0,1,1)$, $Q = (-1,2,1)$, and $R = (1,1,2)$.
\end{exercise}

\begin{exercise}
  Consider the following vector equation for a plane in $\R^4$:
  \begin{equation*}
    \startmat{c} x\\y\\z\\w \stopmat
    = \startmat{r} 1\\2\\0\\0 \stopmat
    + t\,\startmat{r} 1\\0\\0\\1 \stopmat
    + s\,\startmat{r} -1\\-1\\1\\0 \stopmat.
  \end{equation*}
  Find a new vector equation for the same plane by doing the change of
  parameters%
  \index{plane!change of parameters}%
  \index{change of parameters!plane}
  $t=1-r_1$, $s=r_1+r_2$.
  \begin{solution}
    We have
    \begin{equation*}
      \startmat{c} x\\y\\z\\w \stopmat
      = \startmat{r} 1\\2\\0\\0 \stopmat
      + (1-r_1)\,\startmat{r} 1\\0\\0\\1 \stopmat
      + (r_1+r_2)\,\startmat{r} -1\\-1\\1\\0 \stopmat
      = \startmat{r} 2\\2\\0\\1 \stopmat
      + r_1\,\startmat{r} -2\\-1\\1\\-1 \stopmat
      + r_2\,\startmat{r} -1\\-1\\1\\0 \stopmat.
    \end{equation*}
  \end{solution}
\end{exercise}

\begin{exercise}
  Determine which of the following points lie on the plane through the
  points $P = (2,6,1)$, $R = (1,4,1)$, and $Q = (1,2,-1)$.
  \begin{enumerate}
  \item $S_1=(1,2,4)$.
  \item $S_2=(1,5,2)$.
  \item $S_3=(0,0,0)$.
  \end{enumerate}
\end{exercise}

\begin{exercise}
  Use cross products to find the normal vector to the plane going
  through the points $P=(1,2,3)$, $Q=(-2,1,8)$ and $R=(2,2,2)$.
\end{exercise}

\begin{exercise}
  Find normal and standard equations of the plane through the
  point $P=(1,1,2)$ and orthogonal to $\vect{n}=\startmat{c} 1 \\ 0 \\ -1 \stopmat$.
\end{exercise}

\begin{exercise}
  Find normal and standard equations for the plane through the points
  $P = (2,1,0)$, $Q=(1,-1,0)$, and $R=(1,1,-1)$.
\end{exercise}

\begin{exercise}
  Find a vector equation for the plane $2x+y-z=1$.
\end{exercise}

\begin{exercise}
  Find the intersection between the planes $x+3y+4z=3$ and $2x+5y-z=2$.
  Is the intersection a line, a plane, or empty?
\end{exercise}

\begin{exercise}
  Find the intersection of the line
  \begin{equation*}
    \startmat{r} x \\ y \\ z \stopmat
    = \startmat{r} 1 \\ 1 \\ 1 \stopmat
    + t \startmat{r} 2 \\ -1 \\ 2 \stopmat
  \end{equation*}
  and the plane $x+3y+z = 6$.
  Is the intersection a point, a line, or empty?
\end{exercise}

\begin{exercise}
  Find the angle between the planes $x+y=5$ and $2x+y-z=4$.
\end{exercise}

\begin{exercise}
  Find the angle between the line
  \begin{equation*}
    \startmat{r} x \\ y \\ z \stopmat
    = \startmat{r} 0 \\ 3 \\ 7 \stopmat
    + t \startmat{r} 1 \\ 1 \\ 4 \stopmat
  \end{equation*}
  and the plane $4x+7y+4z = 6$.
\end{exercise}

\begin{exercise}
  Find the shortest distance from the point $P = (1,1,-1)$ to the plane
  given by $x + 2y + 2z = 6$, and find the point $Q$ on the plane
  that is closest to $P$.
\end{exercise}

\begin{exercise}
  Use Exercise~\ref{exer-box-product-zero} to find an equation of a
  plane containing the two vectors $\vect{p}$ and $\vect{q}$ and the
  point $0$. \textbf{Hint:} If $(x,y,z)$ is a point in this
  plane, the volume of the parallelepiped determined by $(x,y,z)$
  and the vectors $\vect{p}$, $\vect{q}$ equals 0.
  \begin{solution}
    $\vect{x}\dotprod (\vect{a}\times \vect{b}) =0$.
  \end{solution}
\end{exercise}


\end{document}