\documentclass{ximera}
\graphicspath{     %% setup a global graphics path
{./}               %% look in the same-level directory
{./pictures/}      %% look in graphics
{../pictures/}     %% look up one directory, then in graphics
%{../../pictures/} %% look up two directories, then in graphics
}

\author{Zack Reed}
%borrowed from selinger linear algebra
\begin{document}

A key utility of diagonalization was that matrix powers $A^k$ could be computed easily if $A$ was diagonalizable.

This was because if $A$ was diagonalizable, then $A=PDP^{-1}$ for an invertible matrix $P$ and diagonal matrix $D$. Then, by cancellation of $P$ and $P^{-1}$ through repeated multiplication, $A^k=PD^kP^{-1}$.

Now, we extend this to analytic functions of matrices. 

From calculus, recall that a function is called \textbf{analytic} if it can be defined by a power series. For example:
\begin{eqnarray*}
  e^{x} &=& 1 + x + \frac{1}{2}x^2 + \frac{1}{3!}x^3 + \frac{1}{4!}x^4 + \ldots \\
  \sin x &=& x - \frac{1}{3!}x^3 + \frac{1}{5!}x^5 - \frac{1}{7!}x^7 \pm \ldots \\
  \cos x &=& 1 - \frac{1}{2}x^2 + \frac{1}{4!}x^4 - \frac{1}{6!}x^6 \pm \ldots
\end{eqnarray*}

We know from calculus that the above power series converge for all
real numbers $x$. Since it makes sense to compute the $n\th$ power of
a square matrix, in principle it also makes sense to plug a matrix
into a power series. For a square matrix $A$, we can define

\begin{eqnarray*}
  e^{A} &=& I + A + \frac{1}{2}A^2 + \frac{1}{3!}A^3 + \frac{1}{4!}A^4 + \ldots \\
  \sin A &=& A - \frac{1}{3!}A^3 + \frac{1}{5!}A^5 - \frac{1}{7!}A^7 \pm \ldots \\
  \cos A &=& I - \frac{1}{2}A^2 + \frac{1}{4!}A^4 - \frac{1}{6!}A^6 \pm \ldots
\end{eqnarray*}

In fact, if we apply reasoning very simlar to the matrix-powers argument above, we can indeed accurately define analytic functions on diagonalizable matrices as the following:

If $f$ is an analytic function and $A$ is a diagonalizable matrix with eigen decomposition $PDP^{-1}$, then the matrix 

$$f(A)=P*f(D)*P^{-1}$$

is well-defined, where if $\lambda_i$ are the diagonal elements of $D$, then $f(D)$ is a diagonal matrix with diagonal entries $f(\lambda_i)$.

Use this result to find $e^A$ and $\cos(A)$ for the following matrices. 

\begin{problem}
  
  \begin{enumerate}
    \item $$A = \startmat{rr}
      1 & 0 \\
      0 & 2 \\
    \stopmat,$$
    
    $e^A=\begin{bmatrix}
      \answer{e} & \answer{1}\\
      \answer{1} & \answer{e^2}
    \end{bmatrix}$ and $\cos(A)=\begin{bmatrix}
      \answer{cos(1)} & \answer{1}\\
      \answer{1} & \answer{cos(2)}
    \end{bmatrix}$


    \item $$B = \startmat{rrr}
      0 & 1 & 1 \\
      1 & 0 & -1 \\
      -1 & 1 & 2 \\
    \stopmat,$$

    $e^B=\begin{bmatrix}
      \answer[tolerance=.1]{1}&\answer[tolerance=.1]{e-1}&\answer[tolerance=.1]{e-1}\\
      \answer[tolerance=.1]{e-1}& \answer[tolerance=.1]{1} & \answer[tolerance=.1]{1-e}\\
      \answer[tolerance=.1]{1-e}& \answer[tolerance=.1]{e-1} & \answer[tolerance=.1]{2e-1}
    \end{bmatrix}$ and $\cos(B)=\begin{bmatrix}
      \answer[tolerance=.1]{1}&\answer[tolerance=.1]{cos(1)-1}&\answer[tolerance=.1]{cos(1)-1}\\
      \answer[tolerance=.1]{cos(1)-1}& \answer[tolerance=.1]{1} & \answer[tolerance=.1]{1-cos(1)}\\
      \answer[tolerance=.1]{1-cos(1)}& \answer[tolerance=.1]{cos(1)-1} & \answer[tolerance=.1]{2*cos(1)-2}
    \end{bmatrix}$


  \end{enumerate}

  
\end{problem}

%\begin{problem}
%  Use matrix exponentials to find the solution to the
%  first-order linear differential equation
%  \begin{equation*}
%    \startmat{c}
%      x \\
%      y
%    \stopmat^{\prime} = \startmat{rr}
%      0 & -1 \\
%      6 & 5
%    \stopmat \startmat{c}
%      x \\
%      y
%    \stopmat
%  \end{equation*}
%  with initial value
%  \begin{equation*}
%    \startmat{c}
%      x(0) \\
%      y(0)
%    \stopmat = \startmat{r}
%      2 \\
%      2
%    \stopmat.
%  \end{equation*}
%  Hint: form the matrix exponential $e^{At}$ and then the solution is
%  $e^{At}\,\vect{v}_0$, where $\vect{v}_0$ is the initial vector.
%  \begin{solution}
%    The solution is
%    \begin{equation*}
%      e^{At}C = \startmat{c}
%        8e^{2t} - 6e^{3t} \\
%        18e^{3t} - 16e^{2t}
%      \stopmat.
%    \end{equation*}
%  \end{solution}
%\end{problem}

\end{document}