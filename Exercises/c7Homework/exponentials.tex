\documentclass{ximera}
\graphicspath{  %% When looking for images,
{./}            %% look here first,
{./pictures/}   %% then look for a pictures folder,
{../pictures/}  %% which may be a directory up.
{../../pictures/}  %% which may be a directory up.
{../../../pictures/}  %% which may be a directory up.
{../../../../pictures/}  %% which may be a directory up.
}

\usepackage{listings}
%\usepackage{circuitikz}
\usepackage{xcolor}
\usepackage{amsmath,amsthm}
\usepackage{subcaption}
\usepackage{graphicx}
\usepackage{tikz}
%\usepackage{tikz-3dplot}
\usepackage{amsfonts}
%\usepackage{mdframed} % For framing content
%\usepackage{tikz-cd}

  \renewcommand{\vector}[1]{\left\langle #1\right\rangle}
  \newcommand{\arrowvec}[1]{{\overset{\rightharpoonup}{#1}}}
  \newcommand{\ro}{\texttt{R}}%% row operation
  \newcommand{\dotp}{\bullet}%% dot product
  \renewcommand{\l}{\ell}
  \let\defaultAnswerFormat\answerFormatBoxed
  \usetikzlibrary{calc,bending}
  \tikzset{>=stealth}
  




%make a maroon color
\definecolor{maroon}{RGB}{128,0,0}
%make a dark blue color
\definecolor{darkblue}{RGB}{0,0,139}
%define the color fourier0 to be the maroon color
\definecolor{fourier0}{RGB}{128,0,0}
%define the color fourier1 to be the dark blue color
\definecolor{fourier1}{RGB}{0,0,139}
%define the color fourier 1t to be the light blue color
\definecolor{fourier1t}{RGB}{173,216,230}
%define the color fourier2 to be the dark green color
\definecolor{fourier2}{RGB}{0,100,0}
%define teh color fourier2t to be the light green color
\definecolor{fourier2t}{RGB}{144,238,144}
%define the color fourier3 to be the dark purple color
\definecolor{fourier3}{RGB}{128,0,128}
%define the color fourier3t to be the light purple color
\definecolor{fourier3t}{RGB}{221,160,221}
%define the color fourier0t to be the red color
\definecolor{fourier0t}{RGB}{255,0,0}
%define the color fourier4 to be the orange color
\definecolor{fourier4}{RGB}{255,165,0}
%define the color fourier4t to be the darker orange color
\definecolor{fourier4t}{RGB}{255,215,0}
%define the color fourier5 to be the yellow color
\definecolor{fourier5}{RGB}{255,255,0}
%define the color fourier5t to be the darker yellow color
\definecolor{fourier5t}{RGB}{255,255,100}
%define the color fourier6 to be the green color
\definecolor{fourier6}{RGB}{0,128,0}
%define the color fourier6t to be the darker green color
\definecolor{fourier6t}{RGB}{0,255,0}

%New commands for this doc for errors in copying
\newcommand{\eigenvar}{\lambda}
%\newcommand{\vect}[1]{\mathbf{#1}}
\renewcommand{\th}{^{\text{th}}}
\newcommand{\st}{^{\text{st}}}
\newcommand{\nd}{^{\text{nd}}}
\newcommand{\rd}{^{\text{rd}}}
\newcommand{\paren}[1]{\left(#1\right)}
\newcommand{\abs}[1]{\left|#1\right|}
\newcommand{\R}{\mathbb{R}}
\newcommand{\C}{\mathbb{C}}
\newcommand{\Hilb}{\mathbb{H}}
\newcommand{\qq}[1]{\text{#1}}
\newcommand{\Z}{\mathbb{Z}}
\newcommand{\N}{\mathbb{N}}
\newcommand{\q}[1]{\text{``#1''}}
%\newcommand{\mat}[1]{\begin{bmatrix}#1\end{bmatrix}}
\newcommand{\rref}{\text{reduced row echelon form}}
\newcommand{\ef}{\text{echelon form}}
\newcommand{\ohm}{\Omega}
\newcommand{\volt}{\text{V}}
\newcommand{\amp}{\text{A}}
\newcommand{\Seq}{\textbf{Seq}}
\newcommand{\Poly}{\textbf{P}}
\renewcommand{\quad}{\text{    }}
\newcommand{\roweq}{\simeq}
\newcommand{\rowop}{\simeq}
\newcommand{\rowswap}{\leftrightarrow}
\newcommand{\Mat}{\textbf{M}}
\newcommand{\Func}{\textbf{Func}}
\newcommand{\Hw}{\textbf{Hamming weight}}
\newcommand{\Hd}{\textbf{Hamming distance}}
\newcommand{\rank}{\text{rank}}
\newcommand{\longvect}[1]{\overrightarrow{#1}}
% Define the circled command
\newcommand{\circled}[1]{%
  \tikz[baseline=(char.base)]{
    \node[shape=circle,draw,inner sep=2pt,red,fill=red!20,text=black] (char) {#1};}%
}

% Define custom command \strikeh that just puts red text on the 2nd argument
\newcommand{\strikeh}[2]{\textcolor{red}{#2}}

% Define custom command \strikev that just puts red text on the 2nd argument
\newcommand{\strikev}[2]{\textcolor{red}{#2}}

%more new commands for this doc for errors in copying
\newcommand{\SI}{\text{SI}}
\newcommand{\kg}{\text{kg}}
\newcommand{\m}{\text{m}}
\newcommand{\s}{\text{s}}
\newcommand{\norm}[1]{\left\|#1\right\|}
\newcommand{\col}{\text{col}}
\newcommand{\sspan}{\text{span}}
\newcommand{\proj}{\text{proj}}
\newcommand{\set}[1]{\left\{#1\right\}}
\newcommand{\degC}{^\circ\text{C}}
\newcommand{\centroid}[1]{\overline{#1}}
\newcommand{\dotprod}{\boldsymbol{\cdot}}
%\newcommand{\coord}[1]{\begin{bmatrix}#1\end{bmatrix}}
\newcommand{\iprod}[1]{\langle #1 \rangle}
\newcommand{\adjoint}{^{*}}
\newcommand{\conjugate}[1]{\overline{#1}}
\newcommand{\eigenvarA}{\lambda}
\newcommand{\eigenvarB}{\mu}
\newcommand{\orth}{\perp}
\newcommand{\bigbracket}[1]{\left[#1\right]}
\newcommand{\textiff}{\text{ if and only if }}
\newcommand{\adj}{\text{adj}}
\newcommand{\ijth}{\emph{ij}^\text{th}}
\newcommand{\minor}[2]{M_{#2}}
\newcommand{\cofactor}{\text{C}}
\newcommand{\shift}{\textbf{shift}}
\newcommand{\startmat}[1]{
  \left[\begin{array}{#1}
}
\newcommand{\stopmat}{\end{array}\right]}
%a command to give a name to explorations and hints and theorems
\newcommand{\name}[1]{\begin{centering}\textbf{#1}\end{centering}}
\newcommand{\vect}[1]{\vec{#1}}
\newcommand{\dfn}[1]{\textbf{#1}}
\newcommand{\transpose}{\mathsf{T}}
\newcommand{\mtlb}[2][black]{\texttt{\textcolor{#1}{#2}}}
\newcommand{\RR}{\mathbb{R}} % Real numbers
\newcommand{\id}{\text{id}}
\newcommand{\coord}[1]{\langle#1\rangle}
\newcommand{\RREF}{\text{RREF}}
\newcommand{\Null}{\text{Null}}
\newcommand{\Nullity}{\text{Nullity}}
\newcommand{\Rank}{\text{Rank}}
\newcommand{\Col}{\text{Col}}
\newcommand{\Ef}{\text{EF}}
\newcommand{\boxprod}[3]{\abs{(#1\times#2)\cdot#3}}

\author{Zack Reed}
%borrowed from selinger linear algebra
\begin{document}

A utility of diagonalization was that matrix powers $A^k$ could be computed easily if $A$ was diagonalizable.

This was because if $A$ was diagonalizable, then $A=PDP^{-1}$ for an invertible matrix $P$ and diagonal matrix $D$. Then, by cancellation of $P$ and $P^{-1}$ through repeated multiplication, $A^k=PD^kP^{-1}$.

Now, we extend this to analytic functions of matrices. 

From calculus, recall that a function is called \textbf{analytic} if it can be defined by a power series. For example:
\begin{eqnarray*}
  e^{x} &=& 1 + x + \frac{1}{2}x^2 + \frac{1}{3!}x^3 + \frac{1}{4!}x^4 + \ldots \\
  \sin x &=& x - \frac{1}{3!}x^3 + \frac{1}{5!}x^5 - \frac{1}{7!}x^7 \pm \ldots \\
  \cos x &=& 1 - \frac{1}{2}x^2 + \frac{1}{4!}x^4 - \frac{1}{6!}x^6 \pm \ldots
\end{eqnarray*}

We know from calculus that the above power series converge for all
real numbers $x$. Since it makes sense to compute the $n\th$ power of
a square matrix, in principle it also makes sense to plug a matrix
into a power series. For a square matrix $A$, we can define

\begin{eqnarray*}
  e^{A} &=& I + A + \frac{1}{2}A^2 + \frac{1}{3!}A^3 + \frac{1}{4!}A^4 + \ldots \\
  \sin A &=& A - \frac{1}{3!}A^3 + \frac{1}{5!}A^5 - \frac{1}{7!}A^7 \pm \ldots \\
  \cos A &=& I - \frac{1}{2}A^2 + \frac{1}{4!}A^4 - \frac{1}{6!}A^6 \pm \ldots
\end{eqnarray*}

In fact, if we apply reasoning very simlar to the matrix-powers argument above, we can indeed accurately define analytic functions on diagonalizable matrices as the following:

If $f$ is an analytic function and $A$ is a diagonalizable matrix with eigen decomposition $PDP^{-1}$, then the matrix 

$$f(A)=P*f(D)*P^{-1}$$

is well-defined, where if $\lambda_i$ are the diagonal elements of $D$, then $f(D)$ is a diagonal matrix with diagonal entries $f(\lambda_i)$.

Use this result to find $e^A$ and $\cos(A)$ for the following matrices. 

\begin{problem}
  
  \begin{enumerate}
    \item $$A = \startmat{rr}
      1 & 0 \\
      0 & 2 \\
    \stopmat,$$
    
    $e^A=\begin{bmatrix}
      \answer{e} & \answer{1}\\
      \answer{1} & \answer{e^2}
    \end{bmatrix}$ and $\cos(A)=\begin{bmatrix}
      \answer{cos(1)} & \answer{1}\\
      \answer{1} & \answer{cos(2)}
    \end{bmatrix}$


    \item $$B = \startmat{rrr}
      0 & 1 & 1 \\
      1 & 0 & -1 \\
      -1 & 1 & 2 \\
    \stopmat,$$

    $e^B=\begin{bmatrix}
      \answer{3}&\answer{e-1}&\answer{e-1}\\
      \answer{e-1}& \answer{0} & \answer{1-e}\\
      \answer{3-e}& \answer{e-1} & \answer{2e-2}
    \end{bmatrix}$ and $\cos(B)=\begin{bmatrix}
      \answer{3}&\answer{cos(1)-1}&\answer{cos(1)-1}\\
      \answer{cos(1)-1}& \answer{0} & \answer{1-cos(1)}\\
      \answer{3-cos(1)}& \answer{cos(1)-1} & \answer{2*cos(1)-2}
    \end{bmatrix}$


  \end{enumerate}

  
\end{problem}

%\begin{problem}
%  Use matrix exponentials to find the solution to the
%  first-order linear differential equation
%  \begin{equation*}
%    \startmat{c}
%      x \\
%      y
%    \stopmat^{\prime} = \startmat{rr}
%      0 & -1 \\
%      6 & 5
%    \stopmat \startmat{c}
%      x \\
%      y
%    \stopmat
%  \end{equation*}
%  with initial value
%  \begin{equation*}
%    \startmat{c}
%      x(0) \\
%      y(0)
%    \stopmat = \startmat{r}
%      2 \\
%      2
%    \stopmat.
%  \end{equation*}
%  Hint: form the matrix exponential $e^{At}$ and then the solution is
%  $e^{At}\,\vect{v}_0$, where $\vect{v}_0$ is the initial vector.
%  \begin{solution}
%    The solution is
%    \begin{equation*}
%      e^{At}C = \startmat{c}
%        8e^{2t} - 6e^{3t} \\
%        18e^{3t} - 16e^{2t}
%      \stopmat.
%    \end{equation*}
%  \end{solution}
%\end{problem}

\end{document}