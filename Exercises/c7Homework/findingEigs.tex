\documentclass{ximera}
\graphicspath{     %% setup a global graphics path
{./}               %% look in the same-level directory
{./pictures/}      %% look in graphics
{../pictures/}     %% look up one directory, then in graphics
%{../../pictures/} %% look up two directories, then in graphics
}

\author{Zack Reed}
%borrowed from selinger linear algebra
\begin{document}

\begin{exercise}
  Find the characteristic polynomial of the matrix
  \begin{equation*}
    \startmat{rr}
      3 &  -2 \\
      1 &   0 \\
    \stopmat.
  \end{equation*}
  Use the quadratic formula to find the eigenvalues.
  % \begin{solution}
  % \end{solution}
\end{exercise}

\begin{exercise}
  Find the characteristic polynomial, eigenvalues, and basic
  eigenvectors of the matrix
  \begin{equation*}
    \startmat{rr}
      9 &  10 \\
      -5 &  -6 \\
    \stopmat.
  \end{equation*}
\end{exercise}

\begin{exercise}
  Find the characteristic polynomial, eigenvalues, and basic
  eigenvectors of the matrix
  \begin{equation*}
    \startmat{rrr}
      0 &   3 &  -1 \\
      -2 &   4 &  -2 \\
      2 &  -3 &   3 \\
    \stopmat.
  \end{equation*}
  One eigenvalue is $1$.
\end{exercise}

\begin{exercise}
  Find the characteristic polynomial, eigenvalues, and basic
  eigenvectors of the matrix
  \begin{equation*}
    \startmat{rrr}
      3 &   0 &  -2 \\
      -2 &   1 &   2 \\
      0 &   0 &   1 \\
    \stopmat.
  \end{equation*}
  One eigenvalue is $3$.
  % \begin{solution}
  % \end{solution}
\end{exercise}

\begin{exercise}
  Find the characteristic polynomial, eigenvalues, and basic
  eigenvectors of the matrix
  \begin{equation*}
    \startmat{rrr}
      9 & 2 & 8 \\
      2 & -6 & -2 \\
      -8 & 2 & -5 \\
    \stopmat.
  \end{equation*}
  One eigenvalue is $-3$.
  % \begin{solution}
  % \end{solution}
\end{exercise}

\begin{exercise}
  Which of the following matrices have no real eigenvalue?
  \begin{equation*}
    A = \startmat{rr}
      1 & 1 \\
      1 & -1 \\
    \stopmat,
    \quad
    B = \startmat{rr}
      1 & -1 \\
      1 & 1 \\
    \stopmat,
    \quad
    C = \startmat{rr}
      0 & 1 \\
      1 & 0 \\
    \stopmat.
  \end{equation*}
\end{exercise}

\begin{exercise}
  Find the eigenvalues and eigenvectors of the following triangular
  matrix:
  \begin{equation*}
    \startmat{rrr}
      3 & 2 & 2 \\
      0 & 1 & -2 \\
      0 & 0 & -1 \\
    \stopmat.
  \end{equation*}
\end{exercise}

\begin{exercise}
  Is it possible for a non-zero matrix to have only $0$ as an eigenvalue?
  \vspace{1mm}
  \begin{solution}
    Yes. $\startmat{cc}
      0 & 1 \\
      0 & 0
    \stopmat$ works.
  \end{solution}
\end{exercise}

\section*{Exercises}

\begin{exercise}
  Let
  \begin{equation*}
    A = \startmat{rr}
      5 & 7 \\
      -4 & 3 \\
    \stopmat.
  \end{equation*}
  Find the characteristic polynomial $p(\eigenvar)$, and compute
  $p(A)$.
\end{exercise}

\begin{exercise}
  Let
  \begin{equation*}
    A = \startmat{rrr}
      1 & 2 & 0 \\
      0 & 2 & -1 \\
      0 & 1 & 4 \\
    \stopmat.
  \end{equation*}
  Find the characteristic polynomial $p(\eigenvar)$, and compute
  $p(A)$.
\end{exercise}

\begin{exercise}
  \begin{enumerate}
  \item Let $A$ be a $2\times 2$-matrix. Prove that $A^2$ is a linear
    combination of $A$ and $I$.
  \item Give an example of a $3\times 3$-matrix $A$ such that $A^2$
    is not a linear combination of $A$ and $I$.
  \end{enumerate}
  \begin{solution}
    \begin{enumerate}
    \item The characteristic polynomial of $A$ is a quadratic
      polynomial, and therefore it is of the form
      $p(\eigenvar) = \eigenvar^2 + b\eigenvar + c$, for some
      $r,s\in\R$. By the Cayley-Hamilton theorem, $p(A)=0$, therefore
      $A^2 = -bA - cI$. This proves that $A^2$ is a linear combination
      of $A$ and $I$.
    \item $A=\startmat{rrr}
        0 & 1 & 0 \\
        0 & 0 & 1 \\
        0 & 0 & 0 \\
      \stopmat$.
    \end{enumerate}
  \end{solution}
\end{exercise}


\end{document}