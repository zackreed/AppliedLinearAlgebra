\documentclass{ximera}
\graphicspath{     %% setup a global graphics path
{./}               %% look in the same-level directory
{./pictures/}      %% look in graphics
{../pictures/}     %% look up one directory, then in graphics
%{../../pictures/} %% look up two directories, then in graphics
}

\author{Zack Reed}
%borrowed from selinger linear algebra
\begin{document}

\begin{problem}

Let $C=\begin{bmatrix} 2 & 1 & 1\\ 1 & 2 & 1\\ 1 & 1 & 2\end{bmatrix}$.

\begin{enumerate}
    
\item Show that $\begin{bmatrix} 1\\1\\1 \end{bmatrix}$ is an eigenvector of $C$.  What is its corresponding eigenvalue?
$\answer{4}$

    
\item Show that $\begin{bmatrix} 1\\-1\\0 \end{bmatrix}$ is an eigenvector of $C$.  What is its corresponding eigenvalue?
$\answer{1}$

    
\item Show that $\begin{bmatrix} 1\\1\\-2 \end{bmatrix}$ is an eigenvector of $C$.  What is its corresponding eigenvalue?
$\answer{1}$
\end{enumerate}
\end{problem}
    
    
%\begin{problem}\label{prob:eigprojmatrix}
%Let $Q=\begin{bmatrix} 0& 0\\ 0&1\end{bmatrix}$.  Note that $Q$ takes any vector in $\RR^2$ and projects it onto the $y$-axis.  Which vectors in $\RR^2$ would be eigenvectors, and what are the corresponding eigenvalues?
%\end{problem}
    
    
\begin{problem}\label{ex:eigsrotation2}
Returning to Example \ref{ex:eigsrotation}, let $M=\begin{bmatrix}
\frac{\sqrt{2}}{2} & -\frac{\sqrt{2}}{2}\\
\frac{\sqrt{2}}{2} & \frac{\sqrt{2}}{2}
\end{bmatrix}$.  Show that $\begin{bmatrix} \frac{\sqrt{2}}{2}\\ -\frac{\sqrt{2}}{2} i \end{bmatrix}$ is an eigenvector of $M$.  What is its corresponding eigenvalue?
    
$$\answer[tolerance=.5]{\frac{\sqrt{2}}{2}}+\answer[tolerance=.5]{\frac{\sqrt{2}}{2}}i$$
\end{problem}
    
    
\begin{problem}\label{prob:eigenvalgeometry}
Arguing geometrically, identify the linear transformation whose standard matrix has eigenvalues $\lambda_1=1$ and $\lambda_2=-1$.
\begin{multipleChoice}
\choice{Vertical Shear}
    \choice{Horizontal Shear}
    \choice{Counterclockwise Rotation through a $90^\circ$ angle}
    \choice[correct]{Reflection About the line $y=mx$}
    \choice{Horizontal Stretch}
    \choice{Vertical Stretch}
    \end{multipleChoice}
\end{problem}
    
\begin{problem}\label{prob:eigvalvectorsofdiagmat} Let $A=\begin{bmatrix} 3&0&0\\0&3&0\\0&0&3\end{bmatrix}$.  
  
\begin{enumerate}
  \item What is the single eigenvalue for this matrix? $\lambda=\answer{3}$.
  \item What is the orthonormal basis for the eigenspace of $\lambda$? (Write the vectors ordered first by the first component, then by the second component. So if you were ordering the vectors $\vec{u}=\begin{bmatrix}
  1\\
  5\\
  5
  \end{bmatrix}$, $\vec{v}=\begin{bmatrix}
    2\\
    0\\
    1
  \end{bmatrix}$, and $\vec{w}=\begin{bmatrix}
    2\\
    5\\
    5
  \end{bmatrix}$, the order would be $\vec{w},\vec{v},\vec{u}$).


  Answer:

  The orthonormal basis for the eigenspace of $\lambda$ is $\begin{bmatrix}
    \answer{1}\\
    \answer{0}\\
    \answer{0}
  \end{bmatrix}$, $\begin{bmatrix}
    \answer{0}\\
    \answer{1}\\
    \answer{0}
  \end{bmatrix}$, $\begin{bmatrix}
    \answer{0}\\
    \answer{0}\\
    \answer{1}
  \end{bmatrix}$

\end{enumerate}


\end{problem}
    
\begin{problem}\label{prob:rotmatrixrealeig1}The rotation matrix in Example \ref{ex:eigsrotation} has complex eigenvectors and eigenvalues.  Think geometrically to find an example of a (non-identity) rotation matrix with real eigenvectors and eigenvalues. 
    
Enter degree measure between 0 and 360.
    
Answer:  Rotation through $\answer{180}$ degrees.
\end{problem}
    
\begin{problem}\label{prob:eigenmultchoice}
Can an eigenvalue have multiple eigenvectors associated with it?
\begin{multipleChoice}
    \choice[correct]{Yes}
    \choice{No}
    \end{multipleChoice}
        
    Can an eigenvector have multiple eigenvalues associated with it?
    \begin{multipleChoice}
    \choice{Yes}
    \choice[correct]{No}
    \end{multipleChoice}
\end{problem}

\end{document}