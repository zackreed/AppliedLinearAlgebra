\documentclass{ximera}
\graphicspath{     %% setup a global graphics path
{./}               %% look in the same-level directory
{./pictures/}      %% look in graphics
{../pictures/}     %% look up one directory, then in graphics
%{../../pictures/} %% look up two directories, then in graphics
}

\author{Zack Reed}
%borrowed from selinger linear algebra
\begin{document}

\section*{Exercises}

\begin{exercise}
  Determine by direct calculation (i.e., without calculating the
  characteristic polynomial or the eigenvalues) which of the following
  matrices are positive definite, positive semidefinite, or neither.
  \begin{equation*}
    A = \startmat{rr}
      1 & 1 \\
      1 & 1 \\
    \stopmat,\quad
    B = \startmat{rr}
      1  & -2 \\
      -2 &  5 \\
    \stopmat,\quad
    C = \startmat{rr}
      2 & 3 \\
      1 & 3 \\
    \stopmat,\quad
    D = \startmat{rrr}
      2 & 1 & 1 \\
      1 & 1 & 0 \\
      1 & 0 & 1 \\
    \stopmat,\quad
    E = \startmat{rrr}
      1 & 0 &  0 \\
      0 & 2 &  0 \\
      0 & 0 & -1 \\
    \stopmat.
  \end{equation*}
  \begin{solution}
    (a) Positive semidefinite. (b) Positive definite. (c) Not
    symmetric (therefore neither). (d) Positive semidefinite. (e)
    Neither.
  \end{solution}
\end{exercise}

\begin{exercise}
  Calculate the eigenvalues of each symmetric matrix, then determine
  for each matrix whether it is positive definite, positive
  semidefinite, or neither.
  \begin{equation*}
    A = \startmat{rr}
      2 & 2 \\
      2 & 5 \\
    \stopmat,\quad
    B = \startmat{rr}
      4  & -6 \\
      -6 &  9 \\
    \stopmat,\quad
    C = \startmat{rrr}
      2 &  1 &  0 \\
      1 &  1 & -1 \\
      0 & -1 &  2 \\
    \stopmat,\quad
    D = \startmat{rrr}
      3 & -4 &  2 \\
     -4 &  4 &  0 \\
      2 &  0 &  4 \\
    \stopmat.
  \end{equation*}
  is positive definite, positive semidefinite, or neither.
  \begin{solution}
    (a) Eigenvalues: $\set{1,6}$. Positive definite.
    (b) Eigenvalues: $\set{0,13}$. Positive semidefinite.
    (c) Eigenvalues: $\set{0,2,3}$. Positive semidefinite.
    (d) Eigenvalues: $\set{-1,4,8}$. Neither.
  \end{solution}
\end{exercise}

\begin{exercise}
  Use Descartes' rule of signs to determine which of the following
  matrices are positive definite and/or positive semidefinite.
  \begin{equation*}
    A = \startmat{cc}
      2 & 2 \\
      2 & 1 \\
    \stopmat,
    \quad
    B = \startmat{ccc}
      2  & -1 & 0 \\
      -1 &  1 & 0 \\
      0  &  0 & 1 \\
    \stopmat,
    \quad
    C = \startmat{cccc}
      2  & 1 & -1 &  1 \\
      1  & 1 &  0 &  0 \\
      -1 & 0 &  2 & -1 \\
      1  & 0 & -1 &  1 \\
    \stopmat.
  \end{equation*}
  \begin{solution}
    The characteristic polynomials are:
    \begin{equation*}
      \begin{array}{lcl}
        \det(A-\eigenvar I) &=& \eigenvar^2 - 3\eigenvar - 2, \\
        \det(B-\eigenvar I) &=& -\eigenvar^3 + 4\eigenvar^2 - 4\eigenvar + 1, \\
        \det(C-\eigenvar I) &=& \eigenvar^4 - 6\eigenvar^3 + 9\eigenvar^2 - 3\eigenvar + 0.
      \end{array}
    \end{equation*}
    For $A$, the coefficients are not weakly alternating, so $A$ is
    not positive semidefinite. For $B$, the coefficients are strongly
    alternating, so $B$ is positive definite. For $C$, the
    coefficients are weakly, but not strongly alternating, so $C$ is
    positive semidefinite, but not positive definite.
  \end{solution}
\end{exercise}


\end{document}