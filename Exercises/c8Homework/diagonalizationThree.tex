\documentclass{ximera}
\graphicspath{     %% setup a global graphics path
{./}               %% look in the same-level directory
{./pictures/}      %% look in graphics
{../pictures/}     %% look up one directory, then in graphics
%{../../pictures/} %% look up two directories, then in graphics
}

\author{Zack Reed}
%borrowed from selinger linear algebra
\begin{document}


  \begin{problem}
  In this exercise you will "fill in the details" of Example \ref{ex:diagonalizematrix}.
  \begin{problem}\label{prob:ex:diagonalizematrix1}
  Find the eigenvalues of matrix
  \begin{equation*}
  A=\begin{bmatrix}
  2 & 0 & 0 \\
  1 & 4 & -1 \\
  -2 & -4 & 4
  \end{bmatrix}
  \end{equation*}
  \end{problem}
  \begin{problem}\label{prob:ex:diagonalizematrix2}
  Find a basis for each eigenspace of the matrix $A$.
  \end{problem}
  \begin{problem}\label{prob:ex:diagonalizematrix3}
  Compute the inverse of \begin{equation*}
  P=
  \begin{bmatrix}
  -2 & 1 & 0 \\
  1 & 0 & 1 \\
  0 & 1 & -2
  \end{bmatrix}
  \end{equation*}
  \end{problem}
  \begin{problem}\label{prob:ex:diagonalizematrix5}
  Compute $D=P^{-1}AP$
  \end{problem}
  \begin{problem}\label{prob:ex:diagonalizematrix4}
  Show that computing the inverse of $P$ is not really necessary by comparing the products  $PD$ and $AP$.
  \end{problem}
    \end{problem}
   
  \begin{problem}
  In each case, decide whether the matrix $A$ is diagonalizable. If so, find $P$ such that $P^{-1}AP$ is diagonal.
   
  \begin{problem}\label{prb:diagonalizable}
  \begin{enumerate}
   
  \item $\begin{bmatrix}
  1 & 0 & 0 \\
  1 & 2 & 1 \\
  0 & 0 & 1
  \end{bmatrix}$
  \wordChoice{[correct]\choice{Diagonalizable}\choice{Not Diagonalizable}}
  \item $\begin{bmatrix}
  3 &  0 & 6 \\
  0 & -3 & 0 \\
  5 &  0 & 2
  \end{bmatrix}$
  \wordChoice{[correct]\choice{Diagonalizable}\choice{Not Diagonalizable}}
  \item $\begin{bmatrix}
   3 &  1 &  6 \\
   2 &  1 &  0 \\
  -1 &  0 & -3
  \end{bmatrix}$
  \wordChoice{[correct]\choice{Diagonalizable}\choice{Not Diagonalizable}}
  \item $\begin{bmatrix}
  4 & 0 & 0 \\
  0 & 2 & 2 \\
  2 & 3 & 1
  \end{bmatrix}$
  \wordChoice{\choice{Diagonalizable}\choice[correct]{Not Diagonalizable}}
  \end{enumerate}
  \end{problem}
   
  %\setcounter{enumi}{1}
  %problem  Yes, $ P =
  %\leftB \begin{array}{rrr}
  %-1 & 0 & 6 \\
  % 0 & 1 & 0 \\
  % 1 & 0 & 5
  %\end{array} \rightB$, $P^{-1}AP =
  %\leftB \begin{array}{rrr}
  %-3 &  0 & 0 \\
  % 0 & -3 & 0 \\
  % 0 &  0 & 8
  %\end{array} \rightB$
   
  %\setcounter{enumi}{3}
  %problem  No, $c_{A}(x) = (x + 1)(x - 4)^{2}$ so $\lambda = 4$ has multiplicity 2. But $\func{dim}(E_{4}) = 1$ so Theorem~\ref{thm:016250} applies.
  \end{problem}
   
  \begin{problem}\label{prb:upper_triangular_case}
  Let $A$ denote an $n \times n$ upper triangular matrix.
  \begin{enumerate}
   
  \item If all the main diagonal entries of $A$ are distinct, show that $A$ is diagonalizable.
   
  \item If all the main diagonal entries of $A$ are equal, show that $A$ is diagonalizable only if it is \textit{already} diagonal.
  Click the arrow to see the answer.
  \begin{expandable}
  The eigenvalues of $A$ are all equal (they are the diagonal elements), so if $P^{-1}AP = D$ is diagonal, then $D = \lambda I$. Hence $A = P^{-1}(\lambda I)P = \lambda I$.
  \end{expandable}
   
  \item Show that $\begin{bmatrix}
  1 & 0 & 1 \\
  0 & 1 & 0 \\
  0 & 0 & 2
  \end{bmatrix}$ is diagonalizable but that $\begin{bmatrix}
   1 & 1 & 0 \\
   0 & 1 & 0 \\
   0 & 0 & 2
   \end{bmatrix}$ is not diagonalizable.
    
  \end{enumerate}
   
  \end{problem}
   
  \begin{problem}\label{prb:det_and_tr_diagonalizable}
  Let $A$ be a diagonalizable $n \times n$ matrix with eigenvalues $\lambda_{1}, \lambda_{2}, \dots, \lambda_{n}$ (including multiplicities). Show that:
   
  \begin{enumerate}
  \item $\det A = \lambda_{1}\lambda_{2}\cdots \lambda_{n}$
  \item $\mbox{tr} A = \lambda_{1} + \lambda_{2} + \cdots + \lambda_{n}$
  \end{enumerate}
  \end{problem}
   
  \begin{problem}\label{prb:A_sim_A^T_diagonalizable}
  \begin{enumerate}
   
  \item Show that two diagonalizable matrices are similar if and only if they have the same eigenvalues with the same multiplicities.
   
  \item If $A$ is diagonalizable, show that $A \sim A^{T}$.
   
  \item Show that $A \sim A^{T}$ if
   $A = \begin{bmatrix}
   1 & 1 \\
   0 & 1
   \end{bmatrix}$
   
  \end{enumerate}
  \end{problem}

\end{document}