\documentclass{ximera}
\graphicspath{  %% When looking for images,
{./}            %% look here first,
{./pictures/}   %% then look for a pictures folder,
{../pictures/}  %% which may be a directory up.
{../../pictures/}  %% which may be a directory up.
{../../../pictures/}  %% which may be a directory up.
{../../../../pictures/}  %% which may be a directory up.
}

\usepackage{listings}
%\usepackage{circuitikz}
\usepackage{xcolor}
\usepackage{amsmath,amsthm}
\usepackage{subcaption}
\usepackage{graphicx}
\usepackage{tikz}
%\usepackage{tikz-3dplot}
\usepackage{amsfonts}
%\usepackage{mdframed} % For framing content
%\usepackage{tikz-cd}

  \renewcommand{\vector}[1]{\left\langle #1\right\rangle}
  \newcommand{\arrowvec}[1]{{\overset{\rightharpoonup}{#1}}}
  \newcommand{\ro}{\texttt{R}}%% row operation
  \newcommand{\dotp}{\bullet}%% dot product
  \renewcommand{\l}{\ell}
  \let\defaultAnswerFormat\answerFormatBoxed
  \usetikzlibrary{calc,bending}
  \tikzset{>=stealth}
  




%make a maroon color
\definecolor{maroon}{RGB}{128,0,0}
%make a dark blue color
\definecolor{darkblue}{RGB}{0,0,139}
%define the color fourier0 to be the maroon color
\definecolor{fourier0}{RGB}{128,0,0}
%define the color fourier1 to be the dark blue color
\definecolor{fourier1}{RGB}{0,0,139}
%define the color fourier 1t to be the light blue color
\definecolor{fourier1t}{RGB}{173,216,230}
%define the color fourier2 to be the dark green color
\definecolor{fourier2}{RGB}{0,100,0}
%define teh color fourier2t to be the light green color
\definecolor{fourier2t}{RGB}{144,238,144}
%define the color fourier3 to be the dark purple color
\definecolor{fourier3}{RGB}{128,0,128}
%define the color fourier3t to be the light purple color
\definecolor{fourier3t}{RGB}{221,160,221}
%define the color fourier0t to be the red color
\definecolor{fourier0t}{RGB}{255,0,0}
%define the color fourier4 to be the orange color
\definecolor{fourier4}{RGB}{255,165,0}
%define the color fourier4t to be the darker orange color
\definecolor{fourier4t}{RGB}{255,215,0}
%define the color fourier5 to be the yellow color
\definecolor{fourier5}{RGB}{255,255,0}
%define the color fourier5t to be the darker yellow color
\definecolor{fourier5t}{RGB}{255,255,100}
%define the color fourier6 to be the green color
\definecolor{fourier6}{RGB}{0,128,0}
%define the color fourier6t to be the darker green color
\definecolor{fourier6t}{RGB}{0,255,0}

%New commands for this doc for errors in copying
\newcommand{\eigenvar}{\lambda}
%\newcommand{\vect}[1]{\mathbf{#1}}
\renewcommand{\th}{^{\text{th}}}
\newcommand{\st}{^{\text{st}}}
\newcommand{\nd}{^{\text{nd}}}
\newcommand{\rd}{^{\text{rd}}}
\newcommand{\paren}[1]{\left(#1\right)}
\newcommand{\abs}[1]{\left|#1\right|}
\newcommand{\R}{\mathbb{R}}
\newcommand{\C}{\mathbb{C}}
\newcommand{\Hilb}{\mathbb{H}}
\newcommand{\qq}[1]{\text{#1}}
\newcommand{\Z}{\mathbb{Z}}
\newcommand{\N}{\mathbb{N}}
\newcommand{\q}[1]{\text{``#1''}}
%\newcommand{\mat}[1]{\begin{bmatrix}#1\end{bmatrix}}
\newcommand{\rref}{\text{reduced row echelon form}}
\newcommand{\ef}{\text{echelon form}}
\newcommand{\ohm}{\Omega}
\newcommand{\volt}{\text{V}}
\newcommand{\amp}{\text{A}}
\newcommand{\Seq}{\textbf{Seq}}
\newcommand{\Poly}{\textbf{P}}
\renewcommand{\quad}{\text{    }}
\newcommand{\roweq}{\simeq}
\newcommand{\rowop}{\simeq}
\newcommand{\rowswap}{\leftrightarrow}
\newcommand{\Mat}{\textbf{M}}
\newcommand{\Func}{\textbf{Func}}
\newcommand{\Hw}{\textbf{Hamming weight}}
\newcommand{\Hd}{\textbf{Hamming distance}}
\newcommand{\rank}{\text{rank}}
\newcommand{\longvect}[1]{\overrightarrow{#1}}
% Define the circled command
\newcommand{\circled}[1]{%
  \tikz[baseline=(char.base)]{
    \node[shape=circle,draw,inner sep=2pt,red,fill=red!20,text=black] (char) {#1};}%
}

% Define custom command \strikeh that just puts red text on the 2nd argument
\newcommand{\strikeh}[2]{\textcolor{red}{#2}}

% Define custom command \strikev that just puts red text on the 2nd argument
\newcommand{\strikev}[2]{\textcolor{red}{#2}}

%more new commands for this doc for errors in copying
\newcommand{\SI}{\text{SI}}
\newcommand{\kg}{\text{kg}}
\newcommand{\m}{\text{m}}
\newcommand{\s}{\text{s}}
\newcommand{\norm}[1]{\left\|#1\right\|}
\newcommand{\col}{\text{col}}
\newcommand{\sspan}{\text{span}}
\newcommand{\proj}{\text{proj}}
\newcommand{\set}[1]{\left\{#1\right\}}
\newcommand{\degC}{^\circ\text{C}}
\newcommand{\centroid}[1]{\overline{#1}}
\newcommand{\dotprod}{\boldsymbol{\cdot}}
%\newcommand{\coord}[1]{\begin{bmatrix}#1\end{bmatrix}}
\newcommand{\iprod}[1]{\langle #1 \rangle}
\newcommand{\adjoint}{^{*}}
\newcommand{\conjugate}[1]{\overline{#1}}
\newcommand{\eigenvarA}{\lambda}
\newcommand{\eigenvarB}{\mu}
\newcommand{\orth}{\perp}
\newcommand{\bigbracket}[1]{\left[#1\right]}
\newcommand{\textiff}{\text{ if and only if }}
\newcommand{\adj}{\text{adj}}
\newcommand{\ijth}{\emph{ij}^\text{th}}
\newcommand{\minor}[2]{M_{#2}}
\newcommand{\cofactor}{\text{C}}
\newcommand{\shift}{\textbf{shift}}
\newcommand{\startmat}[1]{
  \left[\begin{array}{#1}
}
\newcommand{\stopmat}{\end{array}\right]}
%a command to give a name to explorations and hints and theorems
\newcommand{\name}[1]{\begin{centering}\textbf{#1}\end{centering}}
\newcommand{\vect}[1]{\vec{#1}}
\newcommand{\dfn}[1]{\textbf{#1}}
\newcommand{\transpose}{\mathsf{T}}
\newcommand{\mtlb}[2][black]{\texttt{\textcolor{#1}{#2}}}
\newcommand{\RR}{\mathbb{R}} % Real numbers
\newcommand{\id}{\text{id}}
\newcommand{\coord}[1]{\langle#1\rangle}
\newcommand{\RREF}{\text{RREF}}
\newcommand{\Null}{\text{Null}}
\newcommand{\Nullity}{\text{Nullity}}
\newcommand{\Rank}{\text{Rank}}
\newcommand{\Col}{\text{Col}}
\newcommand{\Ef}{\text{EF}}
\newcommand{\boxprod}[3]{\abs{(#1\times#2)\cdot#3}}

\author{Zack Reed}
%borrowed from selinger linear algebra
\begin{document}


  \begin{problem}
  In this exercise you will "fill in the details" of Example \ref{ex:diagonalizematrix}.
  \begin{problem}\label{prob:ex:diagonalizematrix1}
  Find the eigenvalues of matrix
  \begin{equation*}
  A=\begin{bmatrix}
  2 & 0 & 0 \\
  1 & 4 & -1 \\
  -2 & -4 & 4
  \end{bmatrix}
  \end{equation*}
  \end{problem}
  \begin{problem}\label{prob:ex:diagonalizematrix2}
  Find a basis for each eigenspace of the matrix $A$.
  \end{problem}
  \begin{problem}\label{prob:ex:diagonalizematrix3}
  Compute the inverse of \begin{equation*}
  P=
  \begin{bmatrix}
  -2 & 1 & 0 \\
  1 & 0 & 1 \\
  0 & 1 & -2
  \end{bmatrix}
  \end{equation*}
  \end{problem}
  \begin{problem}\label{prob:ex:diagonalizematrix5}
  Compute $D=P^{-1}AP$
  \end{problem}
  \begin{problem}\label{prob:ex:diagonalizematrix4}
  Show that computing the inverse of $P$ is not really necessary by comparing the products  $PD$ and $AP$.
  \end{problem}
    \end{problem}
   
  \begin{problem}
  In each case, decide whether the matrix $A$ is diagonalizable. If so, find $P$ such that $P^{-1}AP$ is diagonal.
   
  \begin{problem}\label{prb:diagonalizable}
  \begin{enumerate}
   
  \item $\begin{bmatrix}
  1 & 0 & 0 \\
  1 & 2 & 1 \\
  0 & 0 & 1
  \end{bmatrix}$
  \wordChoice{[correct]\choice{Diagonalizable}\choice{Not Diagonalizable}}
  \item $\begin{bmatrix}
  3 &  0 & 6 \\
  0 & -3 & 0 \\
  5 &  0 & 2
  \end{bmatrix}$
  \wordChoice{[correct]\choice{Diagonalizable}\choice{Not Diagonalizable}}
  \item $\begin{bmatrix}
   3 &  1 &  6 \\
   2 &  1 &  0 \\
  -1 &  0 & -3
  \end{bmatrix}$
  \wordChoice{[correct]\choice{Diagonalizable}\choice{Not Diagonalizable}}
  \item $\begin{bmatrix}
  4 & 0 & 0 \\
  0 & 2 & 2 \\
  2 & 3 & 1
  \end{bmatrix}$
  \wordChoice{\choice{Diagonalizable}\choice[correct]{Not Diagonalizable}}
  \end{enumerate}
  \end{problem}
   
  %\setcounter{enumi}{1}
  %problem  Yes, $ P =
  %\leftB \begin{array}{rrr}
  %-1 & 0 & 6 \\
  % 0 & 1 & 0 \\
  % 1 & 0 & 5
  %\end{array} \rightB$, $P^{-1}AP =
  %\leftB \begin{array}{rrr}
  %-3 &  0 & 0 \\
  % 0 & -3 & 0 \\
  % 0 &  0 & 8
  %\end{array} \rightB$
   
  %\setcounter{enumi}{3}
  %problem  No, $c_{A}(x) = (x + 1)(x - 4)^{2}$ so $\lambda = 4$ has multiplicity 2. But $\func{dim}(E_{4}) = 1$ so Theorem~\ref{thm:016250} applies.
  \end{problem}
   
  \begin{problem}\label{prb:upper_triangular_case}
  Let $A$ denote an $n \times n$ upper triangular matrix.
  \begin{enumerate}
   
  \item If all the main diagonal entries of $A$ are distinct, show that $A$ is diagonalizable.
   
  \item If all the main diagonal entries of $A$ are equal, show that $A$ is diagonalizable only if it is \textit{already} diagonal.
  Click the arrow to see the answer.
  \begin{expandable}
  The eigenvalues of $A$ are all equal (they are the diagonal elements), so if $P^{-1}AP = D$ is diagonal, then $D = \lambda I$. Hence $A = P^{-1}(\lambda I)P = \lambda I$.
  \end{expandable}
   
  \item Show that $\begin{bmatrix}
  1 & 0 & 1 \\
  0 & 1 & 0 \\
  0 & 0 & 2
  \end{bmatrix}$ is diagonalizable but that $\begin{bmatrix}
   1 & 1 & 0 \\
   0 & 1 & 0 \\
   0 & 0 & 2
   \end{bmatrix}$ is not diagonalizable.
    
  \end{enumerate}
   
  \end{problem}
   
  \begin{problem}\label{prb:det_and_tr_diagonalizable}
  Let $A$ be a diagonalizable $n \times n$ matrix with eigenvalues $\lambda_{1}, \lambda_{2}, \dots, \lambda_{n}$ (including multiplicities). Show that:
   
  \begin{enumerate}
  \item $\det A = \lambda_{1}\lambda_{2}\cdots \lambda_{n}$
  \item $\mbox{tr} A = \lambda_{1} + \lambda_{2} + \cdots + \lambda_{n}$
  \end{enumerate}
  \end{problem}
   
  \begin{problem}\label{prb:A_sim_A^T_diagonalizable}
  \begin{enumerate}
   
  \item Show that two diagonalizable matrices are similar if and only if they have the same eigenvalues with the same multiplicities.
   
  \item If $A$ is diagonalizable, show that $A \sim A^{T}$.
   
  \item Show that $A \sim A^{T}$ if
   $A = \begin{bmatrix}
   1 & 1 \\
   0 & 1
   \end{bmatrix}$
   
  \end{enumerate}
  \end{problem}

\end{document}