\documentclass{ximera}
\graphicspath{     %% setup a global graphics path
{./}               %% look in the same-level directory
{./pictures/}      %% look in graphics
{../pictures/}     %% look up one directory, then in graphics
%{../../pictures/} %% look up two directories, then in graphics
}

\author{Zack Reed}
%borrowed from selinger linear algebra
\begin{document}


%\begin{example}
%  Let $A = \startmat{rr}
%    1 & 2 \\
%    2 & 1
%  \stopmat$. Find $A^{10}$ by diagonalization.
%  First we write $A = PDP^{-1}$.
%  \begin{equation*}
%    \def\arraystretch{1.3}
%    \startmat{rr}
%      1 & 2 \\
%      2 & 1
%    \stopmat
%    =
%    \startmat{rr}
%      -1 & 1 \\
%      1 & 1
%    \stopmat
%    \startmat{rr}
%      -1 & 0 \\
%      0 & 3
%    \stopmat
%    \startmat{rr}
%      -\frac{1}{2} & \frac{1}{2} \\
%      \frac{1}{2} & \frac{1}{2}
%    \stopmat
%  \end{equation*}
%  Therefore $A^{10} = PD^{10}P^{-1}$.
%  \begin{equation*}
%    \def\arraystretch{1.3}
%    \startmat{rr}
%      1 & 2 \\
%      2 & 1
%    \stopmat^{10}
%    ~=~
%    \startmat{rr}
%      -1 & 1 \\
%      1 & 1
%    \stopmat
%    \startmat{rr}
%      (-1)^{10} & 0 \\
%      0 & 3^{10}
%    \stopmat
%    \startmat{rr}
%      -\frac{1}{2} & \frac{1}{2} \\
%      \frac{1}{2} & \frac{1}{2}
%    \stopmat \\
%    ~=~
%    \frac{1}{2}\startmat{cc}
%      3^{10}+1 & 3^{10}-1 \\
%      3^{10}-1 & 3^{10}+1
%    \stopmat.
%  \end{equation*}
%\end{example}

\begin{example}
  Let $A = \startmat{rrr}
    1 & -2 & -1 \\
    2 & -1 & 1 \\
    -2 & 3 & 1
  \stopmat$. Find $A^{1/3}$ by diagonalization. (Hint: You'll want to use the command \texttt{nthroot(n,3)} for the cubic root of a number $n$.)

  Answer (round to 2 decimal places, but you should actually verify that your cubic root actually multiplies to $A$):

  $A^{1/3}=\begin{bmatrix}
    \answer{1} & \answer[tolerance=.1]{-1.38} & \answer[tolerance=.1]{-.38}\\
    \answer{2} & \answer[tolerance=.1]{-.88} & \answer[tolerance=.1]{1.12}\\
    \answer{-2} & \answer[tolerance=.1]{2.14} & \answer[tolerance=.1]{.14}
  \end{bmatrix}$
\end{example}

\begin{example}
  Let $A = \startmat{rr}
    -5 & -6 \\
    9 & 10 \\
  \stopmat$. Find a square root of $A$, i.e., find a matrix $B$
  such that $B^2=A$.
  
  Answer:

  $B=\startmat{rr}
    \answer{-1} & \answer{-2} \\
    \answer{3}  & \answer{4} \\
  \stopmat$.
\end{example}

\begin{example}
  Let $A = \startmat{rrr}
    -2 & 0 & 6 \\
    -3 & 1 & 6 \\
    -3 & 0 & 7 \\
  \stopmat$. 
  
  Find a square root of $A$.
  
  
  Answer:
  
  $A = \startmat{rrr}
    \answer{0}  &  \answer{0} & \answer{2} \\
    \answer{-1} &  \answer{1} & \answer{2} \\
    \answer{-1} &  \answer{0} & \answer{3} \\
  \stopmat$.
\end{example}

\end{document}