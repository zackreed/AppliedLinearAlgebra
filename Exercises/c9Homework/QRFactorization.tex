\documentclass{ximera}
\graphicspath{     %% setup a global graphics path
{./}               %% look in the same-level directory
{./pictures/}      %% look in graphics
{../pictures/}     %% look up one directory, then in graphics
%{../../pictures/} %% look up two directories, then in graphics
}

\author{Zack Reed}
%borrowed from selinger linear algebra
\begin{document}

    
\begin{problem}\label{prob:findQR}
In each case find the QR-factorization of $A$.
    
\begin{enumerate}
\item $A = \left[ \begin{array}{rr}
1 & -1 \\
-1 & 0
\end{array}\right]$
\item $A = \left[ \begin{array}{rr}
2 & 1 \\
1 &1
\end{array}\right]$
%\item $A = \left[ \begin{array}{rrr}
%1 & 1 & 1 \\
%1 & 1 & 0 \\
%1 & 0 & 0 \\
%0 & 0 & 0
%\end{array}\right]$
%\item $A = \left[ \begin{array}{rrr}
%1 & 1 & 0 \\
%-1 & 0 & 1 \\
%0 & 1 & 1 \\
%1 & -1 & 0
%\end{array}\right]$
\end{enumerate}


\begin{enumerate}
    \item  $Q = \answer{1/\sqrt{5}}\left[ \begin{array}{rr}
    \answer{2} & \answer{-1} \\
    \answer{1} & \answer{2}
    \end{array}\right]$,
        $R = \answer{1/\sqrt{5}}\left[ \begin{array}{rr}
    \answer{5} & \answer{3} \\
    \answer{0} & \answer{1}
    \end{array}\right]$
        
    \item  $Q = \answer{1/\sqrt{3}}\left[ \begin{array}{rrr}
    \answer{1} & \answer{1} & \answer{0} \\
    \answer{-1} & \answer{0} & \answer{1} \\
    \answer{0} & \answer{1} & \answer{1} \\
    \answer{1} & \answer{-1} & \answer{1}
    \end{array}\right]$, \\
    $R = \answer{1/\sqrt{3}}\left[ \begin{array}{rrr}
    \answer{3} & \answer{0} & \answer{-1} \\
    \answer{0} & \answer{3} & \answer{1} \\
    \answer{0} & \answer{0} & \answer{2}
    \end{array}\right]$
\end{enumerate}

\end{problem}


    
%\begin{problem}\label{prob:indcolumns}
%Let $A$ and $B$ denote matrices.
    
    
%\begin{enumerate}
%\item If $A$ and $B$ have independent columns, show that $AB$ has independent columns. %[\textit{Hint}: Theorem~\ref{thm:015672}.]
    
%\item Show that $A$ has a QR-factorization if and only if $A$ has independent columns.
%\begin{hint}
%If $A$ has a QR-factorization, use (a). For the converse use Theorem~\ref{thm:025133}.
%\end{hint}
    
%\item If $AB$ has a QR-factorization, show that the same is true of $B$ but not necessarily $A$.
%\end{enumerate}
%\begin{hint}
%Consider $AA^{T}$ where $A = \left[ \begin{array}{rrr}
%1 & 0 & 0 \\
%1 & 1 & 1
%\end{array}\right]$.
%\end{hint}
    
%\end{problem}
    
%\begin{problem}
%In each case, find the exact eigenvalues and then approximate them using the QR-algorithm.
    
    
%\begin{enumerate}
%\item $A = \left[ \begin{array}{rr}
%1 & 1 \\
%1 & 0
%\end{array}\right]$
%\item $A = \left[ \begin{array}{rr}
%3 & 1 \\
%1 & 0
%\end{array}\right]$
%\end{enumerate}
    


\end{document}