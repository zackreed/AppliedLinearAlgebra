\documentclass{ximera}
\graphicspath{     %% setup a global graphics path
{./}               %% look in the same-level directory
{./pictures/}      %% look in graphics
{../pictures/}     %% look up one directory, then in graphics
%{../../pictures/} %% look up two directories, then in graphics
}

\author{Zack Reed}
%borrowed from selinger linear algebra
\begin{document}

\section*{Exercises}

\begin{exercise}\label{ex:orthogonal-basis-r3}
  In $\R^3$ with the usual dot product, find an orthogonal basis for
  \begin{equation*}
    \sspan\set{
      \startmat{c} 1 \\ 2 \\ 3 \stopmat,~
      \startmat{c} 2 \\ 6 \\ 0 \stopmat
    }.
  \end{equation*}
  \begin{solution}
    $\vect{u}_1 = \startmat{c} 1 \\ 2 \\ 3 \stopmat$,
    $\vect{u}_2 = \startmat{c} 1 \\ 4 \\ -3 \stopmat$.
  \end{solution}
\end{exercise}

\begin{exercise}
  In $\R^4$ with the usual dot product, find an orthogonal basis for
  \begin{equation*}
    \sspan\set{
      \startmat{c} 0 \\ 1 \\ 1 \\  0 \stopmat,~
      \startmat{c} 3 \\ 0 \\ 2 \\ -1 \stopmat
    }.
  \end{equation*}
  \begin{solution}
    $\vect{u}_1 = \startmat{c} 0 \\ 1 \\ 1 \\ 0 \stopmat$,
    $\vect{u}_2 = \startmat{c} 3 \\ -1 \\ 1 \\ -1 \stopmat$.
  \end{solution}
\end{exercise}

\begin{exercise}
  In $\R^4$ with the usual dot product, find an orthogonal basis for
  \begin{equation*}
    \sspan\set{
      \startmat{c} 1 \\ 0 \\ 1 \\  0 \stopmat,~
      \startmat{c} 1 \\ 3 \\ 1 \\ -1 \stopmat,~
      \startmat{c} 2 \\ 4 \\ 2 \\  2 \stopmat
    }.
  \end{equation*}
  \begin{solution}
    $\vect{u}_1 = \startmat{c} 1 \\ 0 \\ 1 \\ 0 \stopmat$,
    $\vect{u}_2 = \startmat{c} 0 \\ 3 \\ 0 \\ -1 \stopmat$,
    $\vect{u}_3 = \startmat{c} 0 \\ 1 \\ 0 \\ 3 \stopmat$.
  \end{solution}
\end{exercise}

\begin{exercise}
  Let
  \begin{equation*}
    A = \startmat{ccc}
      3 & -1 & 0 \\
      -1 & 5 & 2 \\
      0 & 2 & 3 \\
    \stopmat,
  \end{equation*}
  and consider the vector space $\R^3$ with the inner product given by
  $\iprod{\vect{v},\vect{w}} = \vect{v}^T A\vect{w}$.
  Let
  \begin{equation*}
    \vect{v}_1 = \startmat{c} 1 \\ 0 \\ 2 \stopmat,
    \quad
    \vect{v}_2 = \startmat{c} -1 \\ 1 \\ -5 \stopmat,
    \quad\mbox{and}\quad
    \vect{v}_3 = \startmat{c} 2 \\ 2 \\ 3 \stopmat.
  \end{equation*}
  Apply the Gram-Schmidt procedure to
  $\vect{v}_1,\vect{v}_2,\vect{v}_3$ to find an orthogonal basis
  $\set{\vect{u}_1,\vect{u}_2,\vect{u}_3}$ for $\R^3$ with respect to
  the above inner product.
  \begin{solution}
    $\vect{u}_1 = \startmat{c} 1 \\ 0 \\ 2 \stopmat$,
    $\vect{u}_2 = \startmat{c} 1 \\ 1 \\ -1 \stopmat$,
    $\vect{u}_3 = \startmat{c} -1 \\ 1 \\ 0 \stopmat$.
  \end{solution}
\end{exercise}

\begin{exercise}
  Let
  \begin{equation*}
    A = \startmat{cccc}
      3 & 2 & 0 & 0 \\
      2 & 5 & 1 & 0 \\
      0 & 1 & 3 & 1 \\
      0 & 0 & 1 & 3 \\
    \stopmat,
  \end{equation*}
  and consider the vector space $\R^4$ with the inner product given by
  $\iprod{\vect{v},\vect{w}} = \vect{v}^T A\vect{w}$.
  Let
  \begin{equation*}
    \vect{v}_1 = \startmat{c} 1 \\ -1 \\ 1 \\ 1 \stopmat,
    \quad
    \vect{v}_2 = \startmat{c} 2 \\ 0 \\ 0 \\ 2 \stopmat,
    \quad\mbox{and}\quad
    \vect{v}_3 = \startmat{c} 2 \\ 0 \\ 2 \\ 3 \stopmat,
  \end{equation*}
  and let $W=\sspan\set{\vect{v}_1,\vect{v}_2,\vect{v}_3}$.  Apply the
  Gram-Schmidt procedure to $\vect{v}_1,\vect{v}_2,\vect{v}_3$ to find
  an orthogonal basis $\set{\vect{u}_1,\vect{u}_2,\vect{u}_3}$ for $W$
  with respect to the above inner product.
  \begin{solution}
    $\vect{u}_1 = \startmat{c} 1 \\ -1 \\ 1 \\ 1 \stopmat$,
    $\vect{u}_2 = \startmat{c} 1 \\ 1 \\ -1 \\ 1 \stopmat$,
    $\vect{u}_3 = \startmat{c} -1 \\ 1 \\ 1 \\ 0 \stopmat$.
  \end{solution}
\end{exercise}

\begin{exercise}
  Find an orthonormal basis for the subspace of $\R^3$ from Exercise~\ref{ex:orthogonal-basis-r3}.
  \begin{solution}
    The orthonormal basis is $\displaystyle\set{
      \frac{\vect{u}_1}{\norm{\vect{u}_1}},
      \frac{\vect{u}_2}{\norm{\vect{u}_2}}
    } = \set{
      \frac{1}{\sqrt{14}} \startmat{c} 1 \\ 2 \\ 3 \stopmat,
      \frac{1}{\sqrt{26}} \startmat{c} 1 \\ 4 \\
        -3 \stopmat
    }$.
  \end{solution}
\end{exercise}


\end{document}