\documentclass{ximera}
\graphicspath{     %% setup a global graphics path
{./}               %% look in the same-level directory
{./pictures/}      %% look in graphics
{../pictures/}     %% look up one directory, then in graphics
%{../../pictures/} %% look up two directories, then in graphics
}

\author{Zack Reed}
%borrowed from selinger linear algebra
\begin{document}


\begin{problem}\label{ex:orthogonal-basis-r3}
  In $\R^3$ with the usual dot product, find an orthogonal basis for
  $$\startmat{c} 1 \\ 2 \\ 3 \stopmat,\startmat{c} 2 \\ 6 \\ 0 \stopmat
    .$$

    (Note: Use Graham-Schmidt to find the orthogonal basis.)
 

  Answer:
  
    $\vect{u}_1 = \startmat{c} 1 \\ 2 \\ 3 \stopmat$,
    $\vect{u}_2 = \startmat{c} \answer{1} \\ \answer{4} \\ \answer{-3} \stopmat$.

\end{problem}

\begin{problem}
  In $\R^4$ with the usual dot product, find an orthogonal basis for

  $$\startmat{c} 0 \\ 1 \\ 1 \\ 0 \stopmat,\startmat{c} 3 \\ 0 \\ 2 \\ -1 \stopmat
    .$$

  (Note: Use Graham-Schmidt to find the orthogonal basis.)

  Answer:

    $\vect{u}_1 = \startmat{c} 0 \\ 1 \\ 1 \\ 0 \stopmat$,
    $\vect{u}_2 = \startmat{c} \answer{3} \\ \answer{-1} \\ \answer{1} \\ \answer{-1} \stopmat$.

\end{problem}

\begin{problem}
  In $\R^4$ with the usual dot product, find an orthogonal basis for

  $$\startmat{c} 1 \\ 0 \\ 1 \\ 0 \stopmat,\startmat{c} 1 \\ 3 \\ 1 \\ -1 \stopmat
    ,\startmat{c} 2 \\ 4 \\ 2 \\ 2 \stopmat
    .$$

  (Note: Use Graham-Schmidt to find the orthogonal basis.)

  Answer:

    $\vect{u}_1 = \startmat{c} 1 \\ 0 \\ 1 \\ 0 \stopmat$,
    $\vect{u}_2 = \startmat{c} \answer{0} \\ \answer{3} \\ \answer{0} \\ \answer{-1} \stopmat$,
    $\vect{u}_3 = \startmat{c} \answer{0} \\ \answer{1} \\ \answer{0} \\ \answer{3} \stopmat$.

\end{problem}

% \begin{problem}
%   Let
%   \begin{equation*}
%     A = \startmat{ccc}
%       \answer{3} & \answer{-1} & \answer{0} \\
%       \answer{-1} & \answer{5} & \answer{2} \\
%       \answer{0} & \answer{2} & \answer{3} \\
%     \stopmat,
%   \end{equation*}
%   and consider the vector space $\R^3$ with the inner product given by
%   $\iprod{\vect{v},\vect{w}} = \vect{v}^T A\vect{w}$.
%   Let
%   \begin{equation*}
%     \vect{v}_1 = \startmat{c} \answer{1} \\ \answer{0} \\ \answer{2} \stopmat,
%     \quad
%     \vect{v}_2 = \startmat{c} \answer{-1} \\ \answer{1} \\ \answer{-5} \stopmat,
%     \quad\mbox{and}\quad
%     \vect{v}_3 = \startmat{c} \answer{2} \\ \answer{2} \\ \answer{3} \stopmat.
%   \end{equation*}
%   Apply the Gram-Schmidt procedure to
%   $\vect{v}_1,\vect{v}_2,\vect{v}_3$ to find an orthogonal basis
%   $\set{\vect{u}_1,\vect{u}_2,\vect{u}_3}$ for $\R^3$ with respect to
%   the above inner product.

%   Answer:

%     $\vect{u}_1 = \startmat{c} \answer{1} \\ \answer{0} \\ \answer{2} \stopmat$,
%     $\vect{u}_2 = \startmat{c} \answer{1} \\ \answer{1} \\ \answer{-1} \stopmat$,
%     $\vect{u}_3 = \startmat{c} \answer{-1} \\ \answer{1} \\ \answer{0} \stopmat$.

% \end{problem}

% \begin{problem}
%   Let
%   \begin{equation*}
%     A = \startmat{cccc}
%       \answer{3} & \answer{2} & \answer{0} & \answer{0} \\
%       \answer{2} & \answer{5} & \answer{1} & \answer{0} \\
%       \answer{0} & \answer{1} & \answer{3} & \answer{1} \\
%       \answer{0} & \answer{0} & \answer{1} & \answer{3} \\
%     \stopmat,
%   \end{equation*}
%   and consider the vector space $\R^4$ with the inner product given by
%   $\iprod{\vect{v},\vect{w}} = \vect{v}^T A\vect{w}$.
%   Let
%   \begin{equation*}
%     \vect{v}_1 = \startmat{c} \answer{1} \\ \answer{-1} \\ \answer{1} \\ \answer{1} \stopmat,
%     \quad
%     \vect{v}_2 = \startmat{c} \answer{2} \\ \answer{0} \\ \answer{0} \\ \answer{2} \stopmat,
%     \quad\mbox{and}\quad
%     \vect{v}_3 = \startmat{c} \answer{2} \\ \answer{0} \\ \answer{2} \\ \answer{3} \stopmat,
%   \end{equation*}
%   and let $W=\sspan\set{\vect{v}_1,\vect{v}_2,\vect{v}_3}$.  Apply the
%   Gram-Schmidt procedure to $\vect{v}_1,\vect{v}_2,\vect{v}_3$ to find
%   an orthogonal basis $\set{\vect{u}_1,\vect{u}_2,\vect{u}_3}$ for $W$
%   with respect to the above inner product.

%   Answer:


%     $\vect{u}_1 = \startmat{c} \answer{1} \\ \answer{-1} \\ \answer{1} \\ \answer{1} \stopmat$,
%     $\vect{u}_2 = \startmat{c} \answer{1} \\ \answer{1} \\ \answer{-1} \\ \answer{1} \stopmat$,
%     $\vect{u}_3 = \startmat{c} \answer{-1} \\ \answer{1} \\ \answer{1} \\ \answer{0} \stopmat$.

% \end{problem}


\end{document}