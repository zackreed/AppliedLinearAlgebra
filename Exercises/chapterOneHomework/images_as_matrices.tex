\documentclass{ximera}
\graphicspath{     %% setup a global graphics path
{./}               %% look in the same-level directory
{./pictures/}      %% look in graphics
{../pictures/}     %% look up one directory, then in graphics
%{../../pictures/} %% look up two directories, then in graphics
}

\author{Zack Reed}
\begin{document}

%images_as_matrices
\begin{problem}

  Suppose that the matrix $A$ gives the grayscale values of the image of a lion below:

  \begin{figure}[h]
    \centering
      \includegraphics[width=.5\textwidth]{test_image.jpg}
      \caption{Original Image (Option A)}
      \label{fig:original}
  \end{figure}
  
  Which image(s) below correspond to $A^T$? (Select all that apply)
  
  \begin{hint}
  
    Remember that the transpose of a matrix is obtained by switching the rows and columns of the matrix.

    Also remember that for an image, each pixel is an entry in a matrix, where the number (from 0 to 255) represents how dark or light the pixel is. So this $1080 \times 1920$ image is given by a $1080 \times 1920$ matrix of numbers.

  \end{hint}

  \begin{selectAll}
  
    \choice{Option A}
    \choice{Option B}
    \choice[correct]{Option C}
    \choice{Option D}
  
  \end{selectAll}
  
  Which image(s) below correspond to $(A^T)^T$? (Select all that apply)

  \begin{hint}
  
    Remember that $(A^T)$ is a new matrix, so $(A^T)^T$ is the transpose of that new matrix.

  \end{hint}
  
  \begin{selectAll}
  
    \choice[correct]{Option A}
    \choice{Option B}
    \choice{Option C}
    \choice{Option D}
  
  \end{selectAll}

    %Show images as the transpose of matrices using includegraphics on test_image.jpg
  
    \begin{figure}[h]
      \centering
        \includegraphics[width=.5\textwidth]{test_image.jpg}
        \caption{Original Image (Option A)}
        \label{fig:original}
    \end{figure}
  
      \begin{figure}[h]
        \centering
        \includegraphics[width=.5\textwidth]{test_image_rot_2.jpg}
        \caption{Option B}
        \label{fig:optionA}
    \end{figure}

    %make two vertical figures in a row out of test_image_rot_1 and test_image_rot_3
    \begin{figure}[h]
      \centering
      \begin{minipage}{.5\textwidth}
        \centering
        \includegraphics[width=.4\textwidth]{test_image_rot_1.jpg}
        \caption{Option C}
        \label{fig:optionC}
      \end{minipage}%
      \begin{minipage}{.5\textwidth}
        \centering
        \includegraphics[width=.4\textwidth]{test_image_rot_3.jpg}
        \caption{Option D}
        \label{fig:optionD}
      \end{minipage}
    \end{figure}
  
  
  \end{problem}


\end{document}