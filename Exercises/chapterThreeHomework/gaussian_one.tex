\documentclass{ximera}
\graphicspath{     %% setup a global graphics path
{./}               %% look in the same-level directory
{./pictures/}      %% look in graphics
{../pictures/}     %% look up one directory, then in graphics
%{../../pictures/} %% look up two directories, then in graphics
}

\author{Zack Reed}
%borrowed from selinger linear algebra
\begin{document}

\begin{exercise}

Consider the following augmented matrices in which $\ast$ denotes an
arbitrary number and $\blacksquare$ denotes a non-zero number. Determine
whether the given augmented matrices correspond to consistent systems of equations. If consistent, is the
solution unique?

(NOTE: After you enter whether solutions do, might, or don't exist, follow up questions will then render.)

\begin{enumerate}
    \item \begin{equation*}
    \startmat{ccccc|c}
      \blacksquare & \ast & \ast & \ast & \ast & \ast \\
      0 & \blacksquare & \ast & \ast & 0 & \ast \\
      0 & 0 & \blacksquare & \ast & \ast & \ast \\
      0 & 0 & 0 & 0 & \blacksquare & \ast
    \stopmat
\end{equation*}

The solution \wordChoice{
    \choice[correct]{does}
    \choice{might}
    \choice{doesn't}} exist.

    \begin{problem}
    
        The solution \wordChoice{
            \choice{is}
            \choice[correct]{isn't}} unique.

    \end{problem}

\item \begin{equation*}
    \startmat{ccc|c}
      \blacksquare & \ast & \ast & \ast \\
      0 & \blacksquare & \ast & \ast \\
      0 & 0 & \blacksquare & \ast
    \stopmat
\end{equation*}

%A solution exists and is unique.
The solution \wordChoice{
    \choice[correct]{does}
    \choice{might}
    \choice{doesn't}} exist.

    \begin{problem}
    
        The solution \wordChoice{
            \choice[correct]{is}
            \choice{isn't}} unique.

    \end{problem}

\item \begin{equation*}
    \startmat{ccccc|c}
      \blacksquare & \ast & \ast & \ast & \ast & \ast \\
      0 & \blacksquare & \ast & \ast & 0 & \ast \\
      0 & 0 & 0 & 0 & \blacksquare & 0 \\
      0 & 0 & 0 & 0 & \ast & \blacksquare
    \stopmat
\end{equation*}

%There might be a solution. If so, there are infinitely many.
The solution \wordChoice{
    \choice{does}
    \choice[correct]{might}
    \choice{doesn't}} exist.

    \begin{problem}
    
        If it exists, there is(are) \wordChoice{
            \choice{one}
            \choice[correct]{infinitely many}} solution(s).

    \end{problem}


\end{enumerate}

\end{exercise}
\end{document}