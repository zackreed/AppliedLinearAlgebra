\documentclass{ximera}
\graphicspath{     %% setup a global graphics path
{./}               %% look in the same-level directory
{./pictures/}      %% look in graphics
{../pictures/}     %% look up one directory, then in graphics
%{../../pictures/} %% look up two directories, then in graphics
}

\author{Zack Reed}
%borrowed from selinger linear algebra
\begin{document}

%unit_vectors
\begin{exercise}

    Consider the following diagram of four circuits.
    \begin{center}
      %\scalebox{0.8}{
       % \begin{circuitikz}[american, scale=0.7] \draw
        %  (0,0) to [battery1, v^= $5\volt$~~] (0,4)
         % (0,0) to [R = $2 \ohm$] (4,0)
         % to [R = $5 \ohm$] (4,4)
         % (0,4) to [R =$3 \ohm$] (4,4)
         %% (6,4) to [battery1, v_= \raisebox{1ex}{$20\volt$}] (4,4)
          %(6,4) to [R = $1 \ohm$] (8,4)
         % to [R = $1 \ohm$] (8,0)
         % (4,0) to [R = $6 \ohm$] (8,0)
         % to [R = $3 \ohm$] (8,-4)
         %% to [R = $2 \ohm$] (4,-4)
          %to [R = $1 \ohm$] (4,0)
         % (4,-4)to [R = $4 \ohm$] (0,-4)
         % (0,0 )to [battery1, v_= $10\volt$~~] (0,-4)
         % (2,2) node[scale=3]{$\circlearrowleft$}
         % (2,2) node{$I_2$}
         % (6,2) node[scale=3]{$\circlearrowleft$}
         % (6,2) node{$I_3$}
         % (6,-2) node[scale=3]{$\circlearrowleft$}
         % (6,-2) node{$I_4$}
         % (2,-2) node[scale=3]{$\circlearrowleft$}
         % (2,-2) node{$I_1$}
         % ;
        %\end{circuitikz}
      %}
      %include graphics of circuits_four
      \includegraphics{circuits_four.png}
    \end{center}
    The current in amperes in the four circuits is denoted by $I_1$,
    $I_2$, $I_3$, and $I_4$. It is understood that a positive
    current means a current flowing in the counterclockwise direction. If
    $I_k$ ends up being negative, then it just means the current flows
    in the clockwise direction.  In the above diagram, the top left
    circuit should give the equation
    \begin{equation*}
      2I_2 - 2I_1+5I_2 - 5I_3+3I_2=5.
    \end{equation*}
    Write equations for each of the other three circuits and then give a solution
    to the resulting system of equations.
    %\begin{sol}
      The other three equations are
  
      \begin{eqnarray*}
        \answer{4}I_1 + I_1 - I_4 + \answer{2}I_1 - \answer{2}I_2 &=& \answer{-10} \\
        \answer{6}I_3 - \answer{6}I_4 + I_3 + I_3 + \answer{5}I_3 - \answer{5}I_2 &=& \answer{-20} \\
        \answer{2}I_4 + \answer{3}I_4 + \answer{6}I_4 - \answer{6}I_3 + I_4 - I_1 &=& \answer{0}.
      \end{eqnarray*}

        The solution to the system of equations is (in fraction form)
  
      \begin{eqnarray*}
        I_1 &=& -\frac{\answer{750}}{\answer{373}} \amp \\
        I_2 &=& -\frac{\answer{1421}}{\answer{1119}} \amp \\
        I_3 &=& -\frac{\answer{3061}}{\answer{1119}} \amp \\
        I_4 &=& -\frac{\answer{1718}}{\answer{1119}} \amp.
      \end{eqnarray*}
  
  
    %\end{sol}

\end{exercise}

\end{document}