\documentclass{ximera}
\graphicspath{     %% setup a global graphics path
{./}               %% look in the same-level directory
{./pictures/}      %% look in graphics
{../pictures/}     %% look up one directory, then in graphics
%{../../pictures/} %% look up two directories, then in graphics
}

\author{Zack Reed}
%borrowed from selinger linear algebra
\begin{document}

%unit_vectors
\begin{problem}

    Find $I_1$, $I_2$, and $I_3$, the counterclockwise currents in
    amperes in the three circuits of the following diagram.
  
    \begin{center}
      \scalebox{0.8}{
        \begin{circuitikz}[american, scale=0.7] \draw
          (0,0) to [battery1, v^= $10\volt$~~] (0,4)
          (0,0) to [R = $2 \ohm$] (4,0)
          to [R = $5 \ohm$] (4,4)
          (0,4) to [R =$3 \ohm$] (4,4)
          (6,4) to [battery1, v_= \raisebox{1ex}{$12\volt$}] (4,4)
          (6,4) to [R = $7 \ohm$] (8,4)
          to [R = $3 \ohm$] (8,0)
          (4,0) to [R = $1 \ohm$] (8,0)
          to [R = $4 \ohm$] (8,-4)
          to [R = $4 \ohm$] (4,-4)
          to [R = $2 \ohm$] (4,0)
          (2,2) node[scale=3]{$\circlearrowleft$}
          (2,2) node{$I_1$}
          (6,2) node[scale=3]{$\circlearrowleft$}
          (6,2) node{$I_2$}
          (6,-2) node[scale=3]{$\circlearrowleft$}
          (6,-2) node{$I_3$}
          ;
        \end{circuitikz}
      }
    \end{center}
  
    %\begin{sol}
  
      \begin{eqnarray*}
        \answer{2}I_1 + \answer{5}I_1 + \answer{3}I_1 - \answer{5}I_2 &=& \answer{10} \\
        I_2 - I_3 + \answer{3}I_2 + \answer{7}I_2 + \answer{5}I_2 - \answer{5}I_1 &=& \answer{-12} \\
        \answer{2}I_3 + \answer{4}I_3 + \answer{4}I_3 + I_3 - I_2 &=& \answer{0}.
      \end{eqnarray*}

        The solution to the system of equations is (in fraction form)
  
      \begin{equation*}
        I_1 = \frac{\answer{218}}{\answer{295}},\quad
        I_2 = -\frac{\answer{154}}{\answer{295}},\quad
        I_3 = -\frac{\answer{14}}{\answer{295}}.
      \end{equation*}
  
    %\end{sol}

\end{problem}

\end{document}