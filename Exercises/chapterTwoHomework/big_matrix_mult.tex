\documentclass{ximera}
\graphicspath{     %% setup a global graphics path
{./}               %% look in the same-level directory
{./pictures/}      %% look in graphics
{../pictures/}     %% look up one directory, then in graphics
%{../../pictures/} %% look up two directories, then in graphics
}

\author{Zack Reed}
%borrowed from selinger linear algebra
\begin{document}


\begin{problem}
    
    % Matrix A
    Let matrix \( A \) be defined as follows:
    \[
    A = \startmat{rrrrrrrrrrrr}
        1 & -5 & 3 & 4 & -2 & 8 & 0 & 7 & -3 & 6 & 5 & -1 \\
        0 & 9 & 6 & 5 & -7 & 2 & 8 & -4 & 3 & 2 & -5 & 1 \\
        7 & 8 & -1 & 6 & 3 & -9 & 2 & -8 & 4 & 0 & 5 & 6 \\
        2 & -2 & 0 & 3 & 7 & -4 & -5 & 1 & -6 & 8 & -3 & 9 \\
        5 & 4 & -8 & -7 & 2 & 3 & 9 & 1 & -5 & 6 & -2 & 0 \\
        -6 & 3 & 9 & 8 & 4 & 0 & -7 & 5 & 6 & -1 & 2 & 3 \\
        0 & 7 & 2 & -5 & 8 & 9 & -3 & 4 & -1 & 6 & -4 & 7 \\
        1 & -6 & 5 & 2 & 0 & -8 & 4 & -7 & 3 & 5 & 9 & -3 \\
        9 & 4 & 3 & 1 & -5 & 6 & -7 & 0 & 8 & 2 & -6 & 1 \\
        -8 & 2 & -1 & 0 & 5 & 7 & 9 & 3 & -4 & 6 & 8 & -2 \\
        4 & 5 & -6 & 3 & -7 & 0 & 8 & 2 & 1 & -4 & 6 & 9 \\
        7 & -3 & 0 & 9 & -6 & 5 & 2 & 1 & 8 & 4 & 0 & -5
    \stopmat
    \]
    
    \begin{hint}
    The MATLAB code for entering matrix \( A \):
    \begin{verbatim}
    A = [
        1 -5 3 4 -2 8 0 7 -3 6 5 -1;
        0 9 6 5 -7 2 8 -4 3 2 -5 1;
        7 8 -1 6 3 -9 2 -8 4 0 5 6;
        2 -2 0 3 7 -4 -5 1 -6 8 -3 9;
        5 4 -8 -7 2 3 9 1 -5 6 -2 0;
        -6 3 9 8 4 0 -7 5 6 -1 2 3;
        0 7 2 -5 8 9 -3 4 -1 6 -4 7;
        1 -6 5 2 0 -8 4 -7 3 5 9 -3;
        9 4 3 1 -5 6 -7 0 8 2 -6 1;
        -8 2 -1 0 5 7 9 3 -4 6 8 -2;
        4 5 -6 3 -7 0 8 2 1 -4 6 9;
        7 -3 0 9 -6 5 2 1 8 4 0 -5
    ];
    \end{verbatim}

    \end{hint}
    
    % Matrix B
    Let matrix \( B \) be defined as follows:
    \[
    B = \startmat{rrrrrrrrrrrr}
        -4 & 7 & 1 & 5 & -2 & 9 & 0 & 3 & 6 & -8 & 4 & 2 \\
        8 & 2 & -9 & 6 & 0 & 4 & -1 & 7 & -3 & 5 & -6 & 9 \\
        3 & 4 & 8 & 1 & -7 & 5 & 9 & -2 & 0 & 6 & -1 & -5 \\
        -2 & 1 & 0 & 9 & 8 & -4 & 3 & 7 & 2 & 5 & -3 & 6 \\
        9 & -6 & 4 & 0 & 7 & -8 & 1 & 3 & -5 & 6 & 2 & -1 \\
        5 & 3 & -8 & 7 & 2 & 6 & 9 & -4 & 0 & 1 & 5 & -9 \\
        -7 & 9 & 0 & 4 & 6 & 3 & 8 & -1 & 7 & -5 & 2 & 6 \\
        1 & -3 & 5 & -7 & 2 & 9 & 6 & 0 & -2 & 8 & 3 & 1 \\
        6 & 4 & -2 & 3 & 7 & -1 & 9 & 8 & 5 & 0 & 2 & -6 \\
        0 & -8 & 9 & 2 & 5 & 4 & -3 & 1 & 7 & 6 & 8 & -2 \\
        2 & -5 & 6 & 9 & 4 & -3 & 0 & 7 & 1 & 5 & -9 & 8 \\
        8 & 3 & -4 & 0 & 9 & 6 & -5 & 2 & 4 & -7 & 1 & 3
    \stopmat
    \]
    
    \begin{hint}
    The MATLAB code for entering matrix \( B \):
    \begin{verbatim}
    B = [
        -4 7 1 5 -2 9 0 3 6 -8 4 2;
        8 2 -9 6 0 4 -1 7 -3 5 -6 9;
        3 4 8 1 -7 5 9 -2 0 6 -1 -5;
        -2 1 0 9 8 -4 3 7 2 5 -3 6;
        9 -6 4 0 7 -8 1 3 -5 6 2 -1;
        5 3 -8 7 2 6 9 -4 0 1 5 -9;
        -7 9 0 4 6 3 8 -1 7 -5 2 6;
        1 -3 5 -7 2 9 6 0 -2 8 3 1;
        6 4 -2 3 7 -1 9 8 5 0 2 -6;
        0 -8 9 2 5 4 -3 1 7 6 8 -2;
        2 -5 6 9 4 -3 0 7 1 5 -9 8;
        8 3 -4 0 9 6 -5 2 4 -7 1 3
    ];
    \end{verbatim}
    \end{hint}

The provided hints give the MATLAB code for entering matrices \( A \) and \( B \).

Compute the following entries of the given matrix products:

%list various products and powers and ask for random entries. The answers are of the form $M(i,j)=\answer{number}$

\begin{enumerate}

\item $A^2(3,5)=\answer{-136}$
\item $B^2(4,6)=\answer{-3}$
\item $AB(5,8)=\answer{-53}$
\item $BA(7,9)=\answer{127}$
\item $A^3B^2(2,3)=\answer{78392}$
\item $B^3A^2(6,7)=\answer{726074}$
\item $A^2B^3(8,9)=\answer{-298394}$

\end{enumerate}
    

\end{problem}

\end{document}