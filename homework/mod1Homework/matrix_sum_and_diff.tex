\documentclass{ximera}
\graphicspath{     %% setup a global graphics path
{./}               %% look in the same-level directory
{./pictures/}      %% look in graphics
{../pictures/}     %% look up one directory, then in graphics
%{../../pictures/} %% look up two directories, then in graphics
}

\author{Zack Reed}
\begin{document}

%matrix_sum_and_diff
\begin{problem}
    For the following pairs of matrices, does the sum $A+B$ and difference $A-B$ make sense? (i.e. is an element-by-element calculation possible?) If so, calculate them.
    \begin{enumerate}
    \item
      $A = \startmat{cc}
        1 & 0 \\
        0 & 1
      \stopmat$,\quad
      $B = \startmat{cc}
        0 & 1 \\
        1 & 0
      \stopmat$.
  
      
  
        $A+B$ and $A-B$ $\answer{\text{are}}$ possible because 
  
        \begin{selectAll}
          \choice[correct]{they have the same dimensions}
          \choice{they have different dimensions}
          \choice{each element of $A$ cannot be matched with any element of $B$}
          \choice[correct]{each element of $A$ can be matched with an element of $B$ in the same location}
          \choice{elements of $A$ cannot be matched with elements of $B$ in the same location and vice versa}
        \end{selectAll}

        \begin{hint}
        
          You add matrices by adding the corresponding elements. For example, if $A = \startmat{cc} 1 & 2 \\ 3 & 4 \stopmat$ and $B = \startmat{cc} 5 & 6 \\ 7 & 8 \stopmat$, then $A+B = \startmat{cc} 1+5 & 2+6 \\ 3+7 & 4+8 \stopmat = \startmat{cc} 6 & 8 \\ 10 & 12 \stopmat$.

          In MATLAB, you can add matrices with the command \texttt{A+B}. The matrix $A$ is \texttt{[1 0; 0 1]} and the matrix $B$ is \texttt{[0 1; 1 0]}.

        \end{hint}
  
        $A+B = \startmat{cc}
          \answer{1} & \answer{1} \\
          \answer{1} & \answer{1}
        \stopmat$,\quad
  
        $A-B = \startmat{cc}
          \answer{1} & \answer{-1} \\
          \answer{-1} & \answer{1}
        \stopmat$.
  
  
  
    \item
      $A = \startmat{ccc}
        2 & 1 & 2 \\
        1 & 1 & 0
      \stopmat$,\quad
      $B = \startmat{ccc}
        -1 & 0 & 3 \\
        0 & 1 & 4
      \stopmat$.
  
      
  
        $A+B$ and $A-B$ $\answer{\text{are}}$ possible because 
  
        \begin{selectAll}
          \choice[correct]{they have the same dimensions}
          \choice{they have different dimensions}
          \choice{each element of $A$ cannot be matched with any element of $B$}
          \choice[correct]{each element of $A$ can be matched with an element of $B$ in the same location}
          \choice{elements of $A$ cannot be matched with elements of $B$ in the same location and vice versa}
        \end{selectAll}


  
        $A+B = \startmat{ccc}
          \answer{1} & \answer{1} & \answer{5} \\
          \answer{1} & \answer{2} & \answer{4}
        \stopmat$,\quad
  
        $A-B = \startmat{ccc}
          \answer{3} & \answer{1} & \answer{-1} \\
          \answer{1} & \answer{0} & \answer{-4}
        \stopmat$.
  
  
    \item
      $A = \startmat{cc}
        1 & 0 \\
        -2 & 3 \\
        4 & 2
      \stopmat$,\quad
      $B = \startmat{ccc}
        2 & 7 & -1 \\
        0 & 3 & 4
      \stopmat$.
  
  
      
  
        $A+B$ and $A-B$ $\answer{\text{are not}}$ possible because 
  
        \begin{selectAll}
          \choice{they have the same dimensions}
          \choice[correct]{they have different dimensions}
          \choice{each element of $A$ cannot be matched with any element of $B$}
          \choice{each element of $A$ can be matched with an element of $B$ in the same location}
          \choice[correct]{elements of $A$ cannot be matched with elements of $B$ in the same location and vice versa}
        \end{selectAll}
        
        \begin{problem}
        If we instead took $B^T$ and $A^T$ then $A+B^T$ and $A^T-B$ would be possible.

        \begin{hint}
        
          The transpose of a matrix is a new matrix with the rows and columns swapped. For example, if $A = \startmat{cc} 1 & 2 \\ 3 & 4 \stopmat$, then $A^T = \startmat{cc} 1 & 3 \\ 2 & 4 \stopmat$.

          In MATLAB, you can find the transpose of a matrix with the command \texttt{A'}, or \texttt{transpose(A)}.
        \end{hint}
  
        $A+B^T = \startmat{cc}
          \answer{0} & \answer{4}  \\
          \answer{5} & \answer{6}  \\
          \answer{6} & \answer{2} 
        \stopmat$,\quad

        $A^T-B = \startmat{ccc}
          \answer{-1} & \answer{-9} & \answer{5} \\
          \answer{0} & \answer{0} & \answer{-2}
        \stopmat$.

        \end{problem}

  \end{enumerate}
\end{problem}

\end{document}