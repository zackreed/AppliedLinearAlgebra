\documentclass{ximera}
\graphicspath{     %% setup a global graphics path
{./}               %% look in the same-level directory
{./pictures/}      %% look in graphics
{../pictures/}     %% look up one directory, then in graphics
%{../../pictures/} %% look up two directories, then in graphics
}

\author{Zack Reed}
%borrowed from mooculus calculus 2 https://ximera.osu.edu/mooculus/calculus2/differentialEquations/digInDifferentialEquations.tex
\title{Learning Activity: }
\begin{document}
\begin{abstract}


\end{abstract}
\maketitle

\section{Derivatives and Differential Equations}

A \textit{differential equation}\index{differential equation} is
simply an equation with a derivative in it. Here is an example:
\[
a\cdot f''(x) + b\cdot f'(x) + c\cdot f(x) = g(x).
\]
\begin{question}
  What is a differential equation?
  \begin{multipleChoice}
    \choice{An equation that you take the derivative of.}
    \choice[correct]{An equation that relates the rate of a function to other values.}
    \choice{It is a formula for the slope of a tangent line at a given point.} 
  \end{multipleChoice}
\end{question}

When a mathematician solves a differential equation, they are finding
\textit{functions} satisfying the equation.
\begin{question}
  Which of the following functions solve the differential equation
  \[
  f^{(4)}(x) = f(x)?
  \]
  \begin{hint}
    Remember, $f^{(4)}$ is the fourth derivative of $f$.
  \end{hint}
  \begin{selectAll}
    \choice[correct]{$f(x) = \sin(x)$}
    \choice{$f(x) = x^2$}
    \choice[correct]{$f(x) = e^x$}
    \choice[correct]{$f(x) = e^{-x}$}
    \choice{$f(x) = \tan(x)$}
  \end{selectAll}
  \begin{feedback}
    It turns out that the complete solution to this differential equation
    is $c_1\sin(x)+c_2\cos(x)+c_3e^x+c_4e^{-x}$.  In other words, every
    solution to this differential equation can be written in this form.
    You should check that these are all solutions (for example $f(x) =
    \sin(x)+3\cos(x)-7e^x+\pi e^{-x}$ is a solution).  Proving that these
    are \textbf{all} of the solutions is beyond the scope of this course.
  \end{feedback}
\end{question}

\section{Approximating Solutions to Differential Equations}

\section{Derivatives in Multiple Variables}

\section{Approximating Solutions to the Heat Equation}

\section{Poisson's Equation and Electrostatics}

%\section*{Text Source}
%The text in this section is a compilation of material from Section # Author's \href{link}{title} (license).
 
%Ken Kuttler, {\it  A First Course in Linear Algebra}, Lyryx 2017, Open Edition, p. 362-364.
 
%Many of the Practice Problems are Exercises from
%W. Keith Nicholson, {\it Linear Algebra with Applications}, Lyryx 2018, Open Edition, pp. 298-310.


\end{document}