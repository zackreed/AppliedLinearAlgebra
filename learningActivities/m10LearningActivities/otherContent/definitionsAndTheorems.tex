\documentclass{ximera}
\graphicspath{     %% setup a global graphics path
{./}               %% look in the same-level directory
{./pictures/}      %% look in graphics
{../pictures/}     %% look up one directory, then in graphics
%{../../pictures/} %% look up two directories, then in graphics
}

\author{Zack Reed}
%borrowed from selinger linear algebra
\title{Definitions, Theorems, Formulas, and More}

\begin{document}
\begin{abstract}

\end{abstract}
\maketitle

While reading through a textbook and gaining new tools, such as definitions, theorems, and formulas, progressively as the content is presented is very important. It can, however, make it hard to reference these tools outside of specific chapter or section contexts when needed for problem solving.

Here is a list of all relevant definitions, theorems, formulas, and other information that you might want to have on hand while you do homework and problem solve.

The content is organized by topic (as much as possible): 1) vectors, 2) matrices, 3) vector spaces. If you have trouble finding a particular bit of information, try first searching according to how it's used (e.g. to help with vectors, matrices, vector spaces, etc.). Also, there is likely more content here than is presented in the Learning Activities, feel free to use whatever tools you have available to solve problems, but the problems will only need the material from the Learning Activities.

\section{Vectors}

\begin{definition}{Scalars}

    For now, we're going to think of \textbf{scalars} as real numbers. $1$, $100$, $\pi$, $e^2$, are all examples of scalars.

  \end{definition}

  \begin{definition}{Vectors as Addition and Scalar Multiplication}

    Vectors, thought of most abstractly, are quantities where addition and scalar multiplication make sense. We'll get more precise about this later, but for now we're going to call a collection of vectors a \textit{vector space} if for $\vec{v}$ and $\vec{w}$ in the space, $a\vec{v}+b\vec{w}$ is also in the space for all scalars $a$ and $b$. We would call $\vec{v}$ and $\vec{w}$ \textit{vectors}.

  \end{definition}


\end{document}