\documentclass{ximera}
\graphicspath{     %% setup a global graphics path
{./}               %% look in the same-level directory
{./pictures/}      %% look in graphics
{../pictures/}     %% look up one directory, then in graphics
%{../../pictures/} %% look up two directories, then in graphics
}

\author{Zack Reed}
%borrowed from selinger linear algebra
\title{More Practice Problems}
\begin{document}
\begin{abstract}

    In this learning activity, you will be 
\end{abstract}
\maketitle



\section*{Vectors, Matrices, and Linear Combinations}

\begin{example}
    Let $A=\begin{bmatrix} 1 & 2 \\ 3 & 4 \end{bmatrix}$ and $B=\begin{bmatrix} 5 & 6 \\ 7 & 8 \end{bmatrix}$. 
    \begin{enumerate}
        \item If $\vec{v}$ is the first row of $A$ and $\vec{w}$ is the second column of $B$, then $2\vec{v}+3\vec{w} = \begin{bmatrix} \answer{20} \\ \answer{28} \end{bmatrix}$. (Note: Make both of the vectors into either column or row vectors before performing the operation.)
        
        \item If $\vec{v}$ is the first column of $A$ and $\vec{w}$ is the second row of $B$, then $2\vec{v}+3\vec{w} = \begin{bmatrix} \answer{23} \\ \answer{30} \end{bmatrix}$. (Note: Make both of the vectors into either column or row vectors before performing the operation.)
        
        \item If $\vec{v}$ is the first row of $A$ and $\vec{w}$ is the second row of $B$, then $2\vec{v}+3\vec{w} = \begin{bmatrix} \answer{23} \\ \answer{28} \end{bmatrix}$. (Note: Make both of the vectors into either column or row vectors before performing the operation.)
        
        \item If $\vec{v}$ is the first column of $A$ and $\vec{w}$ is the second column of $B$, then $2\vec{v}+3\vec{w} = \begin{bmatrix} \answer{20} \\ \answer{30} \end{bmatrix}$.(Note: Make both of the vectors into either column or row vectors before performing the operation.)
    \end{enumerate}
\end{example}

\begin{example}
    Which of these vectors in $\RR^4$ are linear combinations of the vectors

    \[
        \vec{v}_1 = \begin{bmatrix} 1 \\ 2 \\ 3 \\ -4 \end{bmatrix}, \vec{v}_2 = \begin{bmatrix} -2 \\ 3 \\ 4 \\ 5 \end{bmatrix}, \vec{v}_3 = \begin{bmatrix} 3 \\ -4 \\ 5 \\ 6 \end{bmatrix}?
    \]

    \begin{selectAll}
        \choice{$\begin{bmatrix} -5 \\ -3.5 \\ 2 \\ 1.5 \end{bmatrix}$}
        \choice[correct]{$\begin{bmatrix} -6\\1\\-20\\-9 \end{bmatrix}$}
        \choice[correct]{$\begin{bmatrix} 5\\-4\\-5\\-14 \end{bmatrix}$}
        \choice{$\begin{bmatrix} -6 \\ 15 \\ 0 \\ -15 \end{bmatrix}$}
        \choice[correct]{$\begin{bmatrix} 6\\-5\\4\\-3 \end{bmatrix}$}
        \choice{$\begin{bmatrix} 6 \\ 7 \\ 8 \\ 9 \end{bmatrix}$}
    \end{selectAll}
\end{example}

\section*{Span and Bases}

\begin{example}
Find bases for the following spaces.

\begin{enumerate}
    \item A basis of unit vectors for the range of the matrix \[
A = \begin{bmatrix} 
-5.2224 & -4.0484 & -4.9298 & -3.4443 \\ 
-1.9373 & -2.4145 & -0.9249 & -1.1244 \\ 
2.7016 & 1.9515 & 2.4393 & 1.6093 \\ 
4.7306 & 2.3337 & 7.3003 & 4.5229 
\end{bmatrix}
\]

\textbf{MATLAB Code:}

\begin{verbatim}
A = [-5.2224 -4.0484 -4.9298 -3.4443; 
-1.9373 -2.4145 -0.9249 -1.1244; 
2.7016 1.9515 2.4393 1.6093; 
4.7306 2.3337 7.3003 4.5229];
\end{verbatim}

    \textbf{Answer:}

    A basis for the range of the matrix $A$ (with the first coordinate of each vector given) is 

    \[
b_1 = \begin{bmatrix} -0.6703 \\ \answer[tolerance=.01]{-0.2486} \\ \answer[tolerance=.01]{0.3468} \\ \answer[tolerance=.01]{0.6072} \end{bmatrix}, \quad
b_2 = \begin{bmatrix} -0.7216 \\ \answer[tolerance=.01]{-0.4304} \\ \answer[tolerance=.01]{0.3479} \\ \answer[tolerance=.01]{0.4160} \end{bmatrix}, \quad
b_3 = \begin{bmatrix} -0.5366 \\ \answer[tolerance=.01]{-0.1007} \\ \answer[tolerance=.01]{0.2655} \\ \answer[tolerance=.01]{0.7946} \end{bmatrix}
\]

    Note, other bases are possible. This takes the appropriate vectors from the matrix $A$ and makes them unit vectors, in the order they appear.

    \item A unit vector basis for the null space of the previous matrix is \[
b_1 = \begin{bmatrix} -0.3608 \\ \answer[tolerance=.01]{0.4008} \\ \answer[tolerance=.01]{0.5174} \\ \answer[tolerance=.01]{0.6645} \end{bmatrix}
\]


    
    \item An orthonormal basis for the range of the matrix \[
B = \begin{bmatrix} 
-8 & 5 & -5 \\ 
35 & 65 & 9 
\end{bmatrix}
\]

    \textbf{Answer:} $B$ has rank $\answer{2}$, and is a $2\times 3$ matrix. So the range is $\RR^{\answer{2}}$. Hence, \wordChoice{\choice[correct]{any}\choice{no}\choice{only some}} orthonormal basis for $\RR^{\answer{2}}$ will work.

    \item An orthonormal basis for the null space of the previous matrix.
    
    \textbf{Solution:}

    If we take the SVD of $B$, the third column of $S$ is $\begin{bmatrix}
        \answer{0}\\\answer{0}
    \end{bmatrix}$ which tells us that the third basis vector in $V$ spans the null space of $B$. Since $B$ has rank $\answer{2}$ then the null space of $B$ is $\answer{1}$-dimensional. Since $V$ gives an orthonormal basis for $\RR^3$, we can form the orthonormal basis for the null space of $B$ by the vector

    $$b_1=\begin{bmatrix}
        \answer[tolerance=.01]{-0.4660} \\
        \answer[tolerance=.01]{0.1297} \\
        \answer[tolerance=.01]{0.8752}
        \end{bmatrix}$$
    
    \item A basis of unit vectors for the eigenspace of the matrix \[
\begin{bmatrix} 
5.2713 & 0.0352 & 0.6612 & 0.1352 & -1.4593 & -0.0311 & 0.2197 & 0.7133 \\ 
0.0352 & 5.5933 & -0.2701 & -0.2153 & -0.8639 & -0.8503 & 0.5779 & -0.4228 \\ 
0.6612 & -0.2701 & 5.0584 & 0.3328 & 0.9953 & -0.5785 & -0.1108 & -1.2348 \\ 
0.1352 & -0.2153 & 0.3328 & 4.2787 & -0.3371 & -0.0054 & 0.7662 & 0.5801 \\ 
-1.4593 & -0.8639 & 0.9953 & -0.3371 & 4.3892 & -1.0024 & 0.3618 & 0.0643 \\ 
-0.0311 & -0.8503 & -0.5785 & -0.0054 & -1.0024 & 4.3743 & 0.6907 & -1.1862 \\ 
0.2197 & 0.5779 & -0.1108 & 0.7662 & 0.3618 & 0.6907 & 5.5454 & 0.2893 \\ 
0.7133 & -0.4228 & -1.2348 & 0.5801 & 0.0643 & -1.1862 & 0.2893 & 4.4894 
\end{bmatrix}
\] that corresponds to the eigenvalue of highest algebraic multiplicity.

\textbf{MATLAB Code:}
\begin{verbatim}
A = [5.2713 0.0352 0.6612 0.1352 -1.4593 -0.0311 0.2197 0.7133;
0.0352 5.5933 -0.2701 -0.2153 -0.8639 -0.8503 0.5779 -0.4228;
0.6612 -0.2701 5.0584 0.3328 0.9953 -0.5785 -0.1108 -1.2348;
0.1352 -0.2153 0.3328 4.2787 -0.3371 -0.0054 0.7662 0.5801;
-1.4593 -0.8639 0.9953 -0.3371 4.3892 -1.0024 0.3618 0.0643;
-0.0311 -0.8503 -0.5785 -0.0054 -1.0024 4.3743 0.6907 -1.1862;
0.2197 0.5779 -0.1108 0.7662 0.3618 0.6907 5.5454 0.2893;
0.7133 -0.4228 -1.2348 0.5801 0.0643 -1.1862 0.2893 4.4894];
\end{verbatim}


\textbf{Answer:}

The eigenvalue of highest multiplicity is $\lambda=\answer{6}$. It has algebraic multiplicity $\answer{4}$. It has geometric multiplicity $\answer{4}$, meaning the eigenvectors for $\lambda$ form a $\answer{4}$-dimensional subspace of $\RR^8$. A basis for this subspace (ordered by the first coordinate of each vector) and comprised only of unit vectors is

\[
b_1 = \begin{bmatrix} 
    \answer[tolerance=.01]{0.0471} \\ 
-0.0973 \\ 
0.2240 \\ 
-0.0576 \\ 
\answer[tolerance=.01]{-0.2443} \\ 
0.6548 \\ 
0.1308 \\ 
-0.6552 
\end{bmatrix}, \quad
b_2 = \begin{bmatrix} 
-0.1775 \\ 
\answer[tolerance=.01]{-0.5789} \\ 
\answer[tolerance=.01]{-0.3995} \\ 
-0.0210 \\ 
\answer[tolerance=.01]{-0.1796} \\ 
0.6529 \\ 
0.0124 \\ 
-0.1212 
\end{bmatrix}, \quad
b_3 = \begin{bmatrix} 
\answer[tolerance=.01]{-0.5632} \\ 
\answer[tolerance=.01]{0.0453} \\ 
-0.4700 \\ 
\answer[tolerance=.01]{-0.1151} \\ 
-0.0618 \\ 
0.5284 \\ 
\answer[tolerance=.01]{0.2654} \\ 
\answer[tolerance=.01]{-0.3053} 
\end{bmatrix}, \quad
b_4 = \begin{bmatrix} 
\answer[tolerance=.01]{0.3474} \\ 
-0.5025 \\ 
0.3365 \\ 
\answer[tolerance=.01]{0.4078} \\ 
\answer[tolerance=.01]{0.2344} \\ 
\answer[tolerance=.01]{-0.0815} \\ 
0.4131 \\ 
0.3393 
\end{bmatrix}
\]


\end{enumerate}

\end{example}



\section*{Linear Transformations}

\section*{Matrix Operations}


\section*{Systems of Equations}

\section*{Gaussian Elimination}


\section*{Matrix Inverses}

\begin{example}
    Which of the following matrices are invertible? Verify this with at least two methods, one of which should be the determinant.

    \begin{enumerate}
        \item 
    \end{enumerate}
\end{example}

\section*{Range and Null Space}

\section*{Determinants}


\section*{Dot Products}

\section*{Projections}

\section*{Orthonormal Bases}

\section*{Cross Products}

\section*{Lines and Planes}


\section*{SVD}


\section*{Eigenvalues and Eigenvectors}

\begin{example}
    Find the eigenvalues and eigenvectors of the matrix 

    
\end{example}

\section*{Multiplicity and Eigenspace}

\begin{example}
    What is the dimension of the eigenspace corresponding to the eigenvalue with the highest algebraic multiplicity for the matrix 



\end{example}




\end{document}