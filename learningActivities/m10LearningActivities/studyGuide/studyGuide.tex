\documentclass{ximera}
\graphicspath{  %% When looking for images,
{./}            %% look here first,
{./pictures/}   %% then look for a pictures folder,
{../pictures/}  %% which may be a directory up.
{../../pictures/}  %% which may be a directory up.
{../../../pictures/}  %% which may be a directory up.
{../../../../pictures/}  %% which may be a directory up.
}

\usepackage{listings}
%\usepackage{circuitikz}
\usepackage{xcolor}
\usepackage{amsmath,amsthm}
\usepackage{subcaption}
\usepackage{graphicx}
\usepackage{tikz}
%\usepackage{tikz-3dplot}
\usepackage{amsfonts}
%\usepackage{mdframed} % For framing content
%\usepackage{tikz-cd}

  \renewcommand{\vector}[1]{\left\langle #1\right\rangle}
  \newcommand{\arrowvec}[1]{{\overset{\rightharpoonup}{#1}}}
  \newcommand{\ro}{\texttt{R}}%% row operation
  \newcommand{\dotp}{\bullet}%% dot product
  \renewcommand{\l}{\ell}
  \let\defaultAnswerFormat\answerFormatBoxed
  \usetikzlibrary{calc,bending}
  \tikzset{>=stealth}
  




%make a maroon color
\definecolor{maroon}{RGB}{128,0,0}
%make a dark blue color
\definecolor{darkblue}{RGB}{0,0,139}
%define the color fourier0 to be the maroon color
\definecolor{fourier0}{RGB}{128,0,0}
%define the color fourier1 to be the dark blue color
\definecolor{fourier1}{RGB}{0,0,139}
%define the color fourier 1t to be the light blue color
\definecolor{fourier1t}{RGB}{173,216,230}
%define the color fourier2 to be the dark green color
\definecolor{fourier2}{RGB}{0,100,0}
%define teh color fourier2t to be the light green color
\definecolor{fourier2t}{RGB}{144,238,144}
%define the color fourier3 to be the dark purple color
\definecolor{fourier3}{RGB}{128,0,128}
%define the color fourier3t to be the light purple color
\definecolor{fourier3t}{RGB}{221,160,221}
%define the color fourier0t to be the red color
\definecolor{fourier0t}{RGB}{255,0,0}
%define the color fourier4 to be the orange color
\definecolor{fourier4}{RGB}{255,165,0}
%define the color fourier4t to be the darker orange color
\definecolor{fourier4t}{RGB}{255,215,0}
%define the color fourier5 to be the yellow color
\definecolor{fourier5}{RGB}{255,255,0}
%define the color fourier5t to be the darker yellow color
\definecolor{fourier5t}{RGB}{255,255,100}
%define the color fourier6 to be the green color
\definecolor{fourier6}{RGB}{0,128,0}
%define the color fourier6t to be the darker green color
\definecolor{fourier6t}{RGB}{0,255,0}

%New commands for this doc for errors in copying
\newcommand{\eigenvar}{\lambda}
%\newcommand{\vect}[1]{\mathbf{#1}}
\renewcommand{\th}{^{\text{th}}}
\newcommand{\st}{^{\text{st}}}
\newcommand{\nd}{^{\text{nd}}}
\newcommand{\rd}{^{\text{rd}}}
\newcommand{\paren}[1]{\left(#1\right)}
\newcommand{\abs}[1]{\left|#1\right|}
\newcommand{\R}{\mathbb{R}}
\newcommand{\C}{\mathbb{C}}
\newcommand{\Hilb}{\mathbb{H}}
\newcommand{\qq}[1]{\text{#1}}
\newcommand{\Z}{\mathbb{Z}}
\newcommand{\N}{\mathbb{N}}
\newcommand{\q}[1]{\text{``#1''}}
%\newcommand{\mat}[1]{\begin{bmatrix}#1\end{bmatrix}}
\newcommand{\rref}{\text{reduced row echelon form}}
\newcommand{\ef}{\text{echelon form}}
\newcommand{\ohm}{\Omega}
\newcommand{\volt}{\text{V}}
\newcommand{\amp}{\text{A}}
\newcommand{\Seq}{\textbf{Seq}}
\newcommand{\Poly}{\textbf{P}}
\renewcommand{\quad}{\text{    }}
\newcommand{\roweq}{\simeq}
\newcommand{\rowop}{\simeq}
\newcommand{\rowswap}{\leftrightarrow}
\newcommand{\Mat}{\textbf{M}}
\newcommand{\Func}{\textbf{Func}}
\newcommand{\Hw}{\textbf{Hamming weight}}
\newcommand{\Hd}{\textbf{Hamming distance}}
\newcommand{\rank}{\text{rank}}
\newcommand{\longvect}[1]{\overrightarrow{#1}}
% Define the circled command
\newcommand{\circled}[1]{%
  \tikz[baseline=(char.base)]{
    \node[shape=circle,draw,inner sep=2pt,red,fill=red!20,text=black] (char) {#1};}%
}

% Define custom command \strikeh that just puts red text on the 2nd argument
\newcommand{\strikeh}[2]{\textcolor{red}{#2}}

% Define custom command \strikev that just puts red text on the 2nd argument
\newcommand{\strikev}[2]{\textcolor{red}{#2}}

%more new commands for this doc for errors in copying
\newcommand{\SI}{\text{SI}}
\newcommand{\kg}{\text{kg}}
\newcommand{\m}{\text{m}}
\newcommand{\s}{\text{s}}
\newcommand{\norm}[1]{\left\|#1\right\|}
\newcommand{\col}{\text{col}}
\newcommand{\sspan}{\text{span}}
\newcommand{\proj}{\text{proj}}
\newcommand{\set}[1]{\left\{#1\right\}}
\newcommand{\degC}{^\circ\text{C}}
\newcommand{\centroid}[1]{\overline{#1}}
\newcommand{\dotprod}{\boldsymbol{\cdot}}
%\newcommand{\coord}[1]{\begin{bmatrix}#1\end{bmatrix}}
\newcommand{\iprod}[1]{\langle #1 \rangle}
\newcommand{\adjoint}{^{*}}
\newcommand{\conjugate}[1]{\overline{#1}}
\newcommand{\eigenvarA}{\lambda}
\newcommand{\eigenvarB}{\mu}
\newcommand{\orth}{\perp}
\newcommand{\bigbracket}[1]{\left[#1\right]}
\newcommand{\textiff}{\text{ if and only if }}
\newcommand{\adj}{\text{adj}}
\newcommand{\ijth}{\emph{ij}^\text{th}}
\newcommand{\minor}[2]{M_{#2}}
\newcommand{\cofactor}{\text{C}}
\newcommand{\shift}{\textbf{shift}}
\newcommand{\startmat}[1]{
  \left[\begin{array}{#1}
}
\newcommand{\stopmat}{\end{array}\right]}
%a command to give a name to explorations and hints and theorems
\newcommand{\name}[1]{\begin{centering}\textbf{#1}\end{centering}}
\newcommand{\vect}[1]{\vec{#1}}
\newcommand{\dfn}[1]{\textbf{#1}}
\newcommand{\transpose}{\mathsf{T}}
\newcommand{\mtlb}[2][black]{\texttt{\textcolor{#1}{#2}}}
\newcommand{\RR}{\mathbb{R}} % Real numbers
\newcommand{\id}{\text{id}}
\newcommand{\coord}[1]{\langle#1\rangle}
\newcommand{\RREF}{\text{RREF}}
\newcommand{\Null}{\text{Null}}
\newcommand{\Nullity}{\text{Nullity}}
\newcommand{\Rank}{\text{Rank}}
\newcommand{\Col}{\text{Col}}
\newcommand{\Ef}{\text{EF}}
\newcommand{\boxprod}[3]{\abs{(#1\times#2)\cdot#3}}

\author{Zack Reed}
%borrowed from selinger linear algebra
\title{More Practice Problems}
\begin{document}
\begin{abstract}

    In this learning activity, you will be 
\end{abstract}
\maketitle



\section*{Vectors, Matrices, and Linear Combinations}

\begin{example}\label{ex:basesprob}
    Let $A=\begin{bmatrix} 1 & 2 \\ 3 & 4 \end{bmatrix}$ and $B=\begin{bmatrix} 5 & 6 \\ 7 & 8 \end{bmatrix}$. 
    \begin{enumerate}
        \item If $\vec{v}$ is the first row of $A$ and $\vec{w}$ is the second column of $B$, then $2\vec{v}+3\vec{w} = \begin{bmatrix} \answer{20} \\ \answer{28} \end{bmatrix}$. (Note: Make both of the vectors into either column or row vectors before performing the operation.)
        
        \item If $\vec{v}$ is the first column of $A$ and $\vec{w}$ is the second row of $B$, then $2\vec{v}+3\vec{w} = \begin{bmatrix} \answer{23} \\ \answer{30} \end{bmatrix}$. (Note: Make both of the vectors into either column or row vectors before performing the operation.)
        
        \item If $\vec{v}$ is the first row of $A$ and $\vec{w}$ is the second row of $B$, then $2\vec{v}+3\vec{w} = \begin{bmatrix} \answer{23} \\ \answer{28} \end{bmatrix}$. (Note: Make both of the vectors into either column or row vectors before performing the operation.)
        
        \item If $\vec{v}$ is the first column of $A$ and $\vec{w}$ is the second column of $B$, then $2\vec{v}+3\vec{w} = \begin{bmatrix} \answer{20} \\ \answer{30} \end{bmatrix}$.(Note: Make both of the vectors into either column or row vectors before performing the operation.)
    \end{enumerate}
\end{example}

\begin{example}
    Which of these vectors in $\RR^4$ are linear combinations of the vectors

    \[
        \vec{v}_1 = \begin{bmatrix} 1 \\ 2 \\ 3 \\ -4 \end{bmatrix}, \vec{v}_2 = \begin{bmatrix} -2 \\ 3 \\ 4 \\ 5 \end{bmatrix}, \vec{v}_3 = \begin{bmatrix} 3 \\ -4 \\ 5 \\ 6 \end{bmatrix}?
    \]

    \begin{selectAll}
        \choice{$\begin{bmatrix} -5 \\ -3.5 \\ 2 \\ 1.5 \end{bmatrix}$}
        \choice[correct]{$\begin{bmatrix} -6\\1\\-20\\-9 \end{bmatrix}$}
        \choice[correct]{$\begin{bmatrix} 5\\-4\\-5\\-14 \end{bmatrix}$}
        \choice[correct]{$\begin{bmatrix} -6 \\ 15 \\ 0 \\ -15 \end{bmatrix}$}
        \choice[correct]{$\begin{bmatrix} 6\\-5\\4\\-3 \end{bmatrix}$}
        \choice{$\begin{bmatrix} 6 \\ 7 \\ 8 \\ 9 \end{bmatrix}$}
    \end{selectAll}
\end{example}

\section*{Span and Bases}

\begin{example}
Find bases for the following spaces.

\begin{enumerate}
    \item A basis of unit vectors for the range of the matrix \[
A = \begin{bmatrix} 
-5.2224 & -4.0484 & -4.9298 & -3.4443 \\ 
-1.9373 & -2.4145 & -0.9249 & -1.1244 \\ 
2.7016 & 1.9515 & 2.4393 & 1.6093 \\ 
4.7306 & 2.3337 & 7.3003 & 4.5229 
\end{bmatrix}
\]

\textbf{MATLAB Code:}

\begin{verbatim}
A = [-5.2224 -4.0484 -4.9298 -3.4443; 
-1.9373 -2.4145 -0.9249 -1.1244; 
2.7016 1.9515 2.4393 1.6093; 
4.7306 2.3337 7.3003 4.5229];
\end{verbatim}

    \textbf{Answer:}

    A basis for the range of the matrix $A$ (with the first coordinate of each vector given) is 

    \[
b_1 = \begin{bmatrix} -0.6703 \\ \answer[tolerance=.01]{-0.2486} \\ \answer[tolerance=.01]{0.3468} \\ \answer[tolerance=.01]{0.6072} \end{bmatrix}, \quad
b_2 = \begin{bmatrix} -0.7216 \\ \answer[tolerance=.01]{-0.4304} \\ \answer[tolerance=.01]{0.3479} \\ \answer[tolerance=.01]{0.4160} \end{bmatrix}, \quad
b_3 = \begin{bmatrix} -0.5366 \\ \answer[tolerance=.01]{-0.1007} \\ \answer[tolerance=.01]{0.2655} \\ \answer[tolerance=.01]{0.7946} \end{bmatrix}
\]

    Note, other bases are possible. This takes the appropriate vectors from the matrix $A$ and makes them unit vectors, in the order they appear.

    \item A unit vector basis for the null space of the previous matrix is \[
b_1 = \begin{bmatrix} -0.3608 \\ \answer[tolerance=.01]{0.4008} \\ \answer[tolerance=.01]{0.5174} \\ \answer[tolerance=.01]{0.6645} \end{bmatrix}
\]


    
    \item An orthonormal basis for the range of the matrix \[
B = \begin{bmatrix} 
-8 & 5 & -5 \\ 
35 & 65 & 9 
\end{bmatrix}
\]

    \textbf{Answer:} $B$ has rank $\answer{2}$, and is a $2\times 3$ matrix. So the range is $\RR^{\answer{2}}$. Hence, \wordChoice{\choice[correct]{any}\choice{no}\choice{only some}} orthonormal basis for $\RR^{\answer{2}}$ will work.

    \item An orthonormal basis for the null space of the previous matrix.
    
    \textbf{Solution:}

    If we take the SVD of $B$, the third column of $S$ is $\begin{bmatrix}
        \answer{0}\\\answer{0}
    \end{bmatrix}$ which tells us that the third basis vector in $V$ spans the null space of $B$. Since $B$ has rank $\answer{2}$ then the null space of $B$ is $\answer{1}$-dimensional. Since $V$ gives an orthonormal basis for $\RR^3$, we can form the orthonormal basis for the null space of $B$ by the vector

    $$b_1=\begin{bmatrix}
        \answer[tolerance=.01]{-0.4660} \\
        \answer[tolerance=.01]{0.1297} \\
        \answer[tolerance=.01]{0.8752}
        \end{bmatrix}$$
    
    \item A basis of unit vectors for the eigenspace of the matrix \[
\begin{bmatrix} 
5.2713 & 0.0352 & 0.6612 & 0.1352 & -1.4593 & -0.0311 & 0.2197 & 0.7133 \\ 
0.0352 & 5.5933 & -0.2701 & -0.2153 & -0.8639 & -0.8503 & 0.5779 & -0.4228 \\ 
0.6612 & -0.2701 & 5.0584 & 0.3328 & 0.9953 & -0.5785 & -0.1108 & -1.2348 \\ 
0.1352 & -0.2153 & 0.3328 & 4.2787 & -0.3371 & -0.0054 & 0.7662 & 0.5801 \\ 
-1.4593 & -0.8639 & 0.9953 & -0.3371 & 4.3892 & -1.0024 & 0.3618 & 0.0643 \\ 
-0.0311 & -0.8503 & -0.5785 & -0.0054 & -1.0024 & 4.3743 & 0.6907 & -1.1862 \\ 
0.2197 & 0.5779 & -0.1108 & 0.7662 & 0.3618 & 0.6907 & 5.5454 & 0.2893 \\ 
0.7133 & -0.4228 & -1.2348 & 0.5801 & 0.0643 & -1.1862 & 0.2893 & 4.4894 
\end{bmatrix}
\] that corresponds to the eigenvalue of highest algebraic multiplicity.

\textbf{MATLAB Code:}
\begin{verbatim}
A = [5.2713 0.0352 0.6612 0.1352 -1.4593 -0.0311 0.2197 0.7133;
0.0352 5.5933 -0.2701 -0.2153 -0.8639 -0.8503 0.5779 -0.4228;
0.6612 -0.2701 5.0584 0.3328 0.9953 -0.5785 -0.1108 -1.2348;
0.1352 -0.2153 0.3328 4.2787 -0.3371 -0.0054 0.7662 0.5801;
-1.4593 -0.8639 0.9953 -0.3371 4.3892 -1.0024 0.3618 0.0643;
-0.0311 -0.8503 -0.5785 -0.0054 -1.0024 4.3743 0.6907 -1.1862;
0.2197 0.5779 -0.1108 0.7662 0.3618 0.6907 5.5454 0.2893;
0.7133 -0.4228 -1.2348 0.5801 0.0643 -1.1862 0.2893 4.4894];
\end{verbatim}


\textbf{Answer:}

The eigenvalue of highest multiplicity is $\lambda=\answer{6}$. It has algebraic multiplicity $\answer{4}$. It has geometric multiplicity $\answer{4}$, meaning the eigenvectors for $\lambda$ form a $\answer{4}$-dimensional subspace of $\RR^8$. A basis for this subspace (ordered by the first coordinate of each vector) and comprised only of unit vectors is

\[
b_1 = \begin{bmatrix} 
    \answer[tolerance=.01]{0.0471} \\ 
-0.0973 \\ 
0.2240 \\ 
-0.0576 \\ 
\answer[tolerance=.01]{-0.2443} \\ 
0.6548 \\ 
0.1308 \\ 
-0.6552 
\end{bmatrix}, \quad
b_2 = \begin{bmatrix} 
-0.1775 \\ 
\answer[tolerance=.01]{-0.5789} \\ 
\answer[tolerance=.01]{-0.3995} \\ 
-0.0210 \\ 
\answer[tolerance=.01]{-0.1796} \\ 
0.6529 \\ 
0.0124 \\ 
-0.1212 
\end{bmatrix}, \quad
b_3 = \begin{bmatrix} 
\answer[tolerance=.01]{-0.5632} \\ 
\answer[tolerance=.01]{0.0453} \\ 
-0.4700 \\ 
\answer[tolerance=.01]{-0.1151} \\ 
-0.0618 \\ 
0.5284 \\ 
\answer[tolerance=.01]{0.2654} \\ 
\answer[tolerance=.01]{-0.3053} 
\end{bmatrix}, \quad
b_4 = \begin{bmatrix} 
\answer[tolerance=.01]{0.3474} \\ 
-0.5025 \\ 
0.3365 \\ 
\answer[tolerance=.01]{0.4078} \\ 
\answer[tolerance=.01]{0.2344} \\ 
\answer[tolerance=.01]{-0.0815} \\ 
0.4131 \\ 
0.3393 
\end{bmatrix}
\]


\end{enumerate}

\end{example}



\section*{Linear Transformations}

\section*{Matrix Operations}


\section*{Systems of Equations}

\section*{Gaussian Elimination}


\section*{Matrix Inverses}

\begin{example}\label{ex:invertible}
    Which of the following matrices are invertible? Verify this with at least two methods, one of which should be the determinant.

    If the matrix has variable entries, find all values of the variables that make the matrix not invertible.

    (Note:, you may need to get a decimal approximation of the values yielding zero determinants).

    \begin{enumerate}
        \item \[
        A=\begin{bmatrix}  
            2 & -7 & -9 & 7 \\  
            8 & 6 & -2 & 10 \\  
            4 & 3 & -1 & -2 \\  
            -4 & -3 & 1 & -7  
            \end{bmatrix}  
            \]

\textbf{MATLAB Code:}
\begin{verbatim}
    [  2  -7  -9   7;
   8   6  -2  10;
   4   3  -1  -2;
  -4  -3   1  -7 ]
\end{verbatim}


    \textbf{Answer:}

    $A$ \wordChoice{\choice{is invertible}\choice[correct]{is not invertible}}. 

    \item \[
B=\begin{bmatrix}  
10 & -9 & 1 & 6 & -3 & 3 & 3 & -5 & -1 & -5 \\  
4 & -7 & -10 & -8 & -5 & 3 & -5 & 9 & 2 & 9 \\  
9 & 5 & 1 & 4 & 1 & 3 & -5 & 7 & 10 & 4 \\  
3 & 0 & -2 & 7 & -2 & 1 & 4 & 7 & 7 & 6 \\  
3 & -2 & 5 & 0 & 9 & -4 & 10 & 6 & 10 & -7 \\  
-2 & -4 & -7 & -10 & -3 & -8 & -5 & -1 & -3 & -8 \\  
-10 & -7 & -1 & -3 & -8 & 7 & 7 & 7 & -5 & -9 \\  
2 & 9 & 2 & 10 & -5 & -4 & 8 & 9 & -4 & 2 \\  
5 & -2 & -9 & -7 & -7 & 2 & 7 & -10 & -3 & 2 \\  
3 & -8 & 0 & 6 & 7 & -1 & 5 & 5 & 1 & -3  
\end{bmatrix}  
\]

\textbf{MATLAB Code:}
\begin{verbatim}
    [ 10  -9   1   6  -3   3   3  -5  -1  -5;
   4  -7 -10  -8  -5   3  -5   9   2   9;
   9   5   1   4   1   3  -5   7  10   4;
   3   0  -2   7  -2   1   4   7   7   6;
   3  -2   5   0   9  -4  10   6  10  -7;
  -2  -4  -7 -10  -3  -8  -5  -1  -3  -8;
 -10  -7  -1  -3  -8   7   7   7  -5  -9;
   2   9   2  10  -5  -4   8   9  -4   2;
   5  -2  -9  -7  -7   2   7 -10  -3   2;
   3  -8   0   6   7  -1   5   5   1  -3 ]
\end{verbatim}

    \textbf{Answer:}
    $B$ \wordChoice{\choice[correct]{is invertible}\choice{is not invertible}}.


    \item $B=[\vec{u}_1,\vec{u}_2,\vec{u}_1-\vec{u}_2]$, where $\vec{u}_1$ and $\vec{u}_2$ are linearly independent vectors in $\RR^3$.
    
    \textbf{Answer:}
    $B$ \wordChoice{\choice{is invertible}\choice[correct]{is not invertible}}.

        \item \[
        D = \begin{bmatrix} 
        x & -7 & 3y \\ 
        2z & y+1 & \frac{5}{2} \\ 
        3 & z-4 & 9 
        \end{bmatrix}
        \]

\textbf{MATLAB Code:}
\begin{verbatim}
    D = [x -7 3*y;
         2*z y+1 5/2;
         3 z-4 9];
\end{verbatim}

    \textbf{Answer:}
        $D$ is invertible for all values of $(x,y,z)$ (only considering real numbers) except for $(\answer[tolerance=.01]{105/38}, \answer{0}, \answer{4})$, and $(\answer{0}, \answer[tolerance=.01]{-431/216}, \answer[tolerance=.01]{5/12})$.


\item
        \[
        E = \begin{bmatrix} 
        \pi & e^y & x^2 \\ 
        y+3 & \log(y) & -2 \\ 
        \log(2) & 7z & x - y 
        \end{bmatrix}
        \]

\textbf{MATLAB Code:}
\begin{verbatim}
E = [pi exp(y) x^2;
     y+3 log(y) -2;
     log(2) 7*z x-y];
\end{verbatim}
            
    \textbf{Answer:}
        $E$ is invertible for except for the $(x,y,z)$ values $(\answer[tolerance=.01]{2.863}, \answer{\pi},\answer{0}), (\answer[tolerance=.01]{.6534}, \answer{1},\answer{0})$ and $(\answer[tolerance=.01]{-177.444}, \answer{\pi},\answer{0})$.
    \end{enumerate}
\end{example}

\section*{Range and Null Space}

See example \ref{ex:basesprob}.

\section*{Determinants}

See Example \ref{ex:invertible}.

\section*{Dot Products}

See Example \ref{ex:orthonormal}.

\section*{Projections}

\section*{Orthonormal Bases}

See Example \ref{ex:basesprob}.

\begin{example}\label{ex:orthonormal}

Which of the following sets of vectors form orthonormal bases for $\RR^n$?

\begin{enumerate}

    \item \[
Q = \begin{bmatrix} 
-0.5053 & 0.2870 & -0.8138 \\ 
-0.6260 & -0.7710 & 0.1168 \\ 
-0.5940 & 0.5684 & 0.5693 
\end{bmatrix}
\]

\textbf{MATLAB Code:}
\begin{verbatim}
Q = [-0.5053 0.2870 -0.8138;
-0.6260 -0.7710 0.1168;
-0.5940 0.5684 0.5693];
\end{verbatim}

\textbf{Answer:}
$Q$ \wordChoice{\choice[correct]{is}\choice{is not}} an orthonormal basis for $\RR^3$.


\item \[
Q = \begin{bmatrix} 
0.2794 & 0.6683 & 0.7999 & 0.4255 & 0.7797 \\
0.9119 & 0.3354 & 0.3311 & 0.0294 & 0.5980 \\
0.0598 & 0.0378 & 0.7374 & 0.7842 & 0.2846 \\
0.6160 & 0.2858 & 0.5674 & 0.9808 & 0.7463 \\
0.5639 & 0.9312 & 0.0161 & 0.6899 & 0.6939
\end{bmatrix}
\]

\textbf{MATLAB Code:}
\begin{verbatim}
Q = [0.2794 0.6683 0.7999 0.4255 0.7797;
0.9119 0.3354 0.3311 0.0294 0.5980;
0.0598 0.0378 0.7374 0.7842 0.2846;
0.6160 0.2858 0.5674 0.9808 0.7463;
0.5639 0.9312 0.0161 0.6899 0.6939];
\end{verbatim}

\textbf{Answer:}
$Q$ \wordChoice{\choice{is}\choice[correct]{is not}} an orthonormal basis for $\RR^5$.

\item \[
Q = \begin{bmatrix} 
0.9128 & 0.9632 & 0.1500 & 0.5183 & 0.0067 & 0.9448 & 0.1848 & 0.1445 & 0.2568 & 0.8795 \\
0.5342 & 0.8433 & 0.9862 & 0.7389 & 0.2149 & 0.2214 & 0.2278 & 0.8166 & 0.5919 & 0.9703 \\
0.4515 & 0.1139 & 0.8150 & 0.7440 & 0.0810 & 0.6006 & 0.8498 & 0.1817 & 0.3578 & 0.7523 \\
0.2856 & 0.9362 & 0.9122 & 0.4035 & 0.1615 & 0.4361 & 0.4943 & 0.8330 & 0.5481 & 0.0155 \\
0.6228 & 0.6884 & 0.6840 & 0.1609 & 0.0918 & 0.1736 & 0.9857 & 0.4063 & 0.7807 & 0.9923 \\
0.1818 & 0.3071 & 0.3030 & 0.0224 & 0.1341 & 0.1949 & 0.1722 & 0.4215 & 0.5589 & 0.4558 \\
0.9271 & 0.8607 & 0.7318 & 0.1059 & 0.0718 & 0.1891 & 0.9501 & 0.5333 & 0.3017 & 0.5205 \\
0.7140 & 0.9485 & 0.2762 & 0.9320 & 0.2688 & 0.9299 & 0.3823 & 0.4303 & 0.4606 & 0.0889 \\
0.8324 & 0.6818 & 0.6152 & 0.6155 & 0.0401 & 0.4905 & 0.4476 & 0.0024 & 0.9209 & 0.7851 \\
0.4965 & 0.5632 & 0.2949 & 0.6333 & 0.7441 & 0.7963 & 0.3599 & 0.6647 & 0.8862 & 0.7177
\end{bmatrix}
\]

\textbf{MATLAB Code:}
\begin{verbatim}
Q = [0.9128 0.9632 0.1500 0.5183 0.0067 0.9448 0.1848 0.1445 0.2568 0.8795;
0.5342 0.8433 0.9862 0.7389 0.2149 0.2214 0.2278 0.8166 0.5919 0.9703;
0.4515 0.1139 0.8150 0.7440 0.0810 0.6006 0.8498 0.1817 0.3578 0.7523;
0.2856 0.9362 0.9122 0.4035 0.1615 0.4361 0.4943 0.8330 0.5481 0.0155;
0.6228 0.6884 0.6840 0.1609 0.0918 0.1736 0.9857 0.4063 0.7807 0.9923;
0.1818 0.3071 0.3030 0.0224 0.1341 0.1949 0.1722 0.4215 0.5589 0.4558;
0.9271 0.8607 0.7318 0.1059 0.0718 0.1891 0.9501 0.5333 0.3017 0.5205;
0.7140 0.9485 0.2762 0.9320 0.2688 0.9299 0.3823 0.4303 0.4606 0.0889;
0.8324 0.6818 0.6152 0.6155 0.0401 0.4905 0.4476 0.0024 0.9209 0.7851;
0.4965 0.5632 0.2949 0.6333 0.7441 0.7963 0.3599 0.6647 0.8862 0.7177];
\end{verbatim}

\textbf{Answer:}
$Q$ \wordChoice{\choice{is}\choice[correct]{is not}} an orthonormal basis for $\RR^{10}$.

\item \[
Q = \begin{bmatrix} 
-0.3154 & 0.4547 & -0.4009 & 0.2905 & -0.3397 & -0.2158 & 0.2910 & -0.4204 & -0.0430 & 0.1530 \\
-0.3726 & -0.1591 & 0.2126 & -0.2886 & -0.2139 & -0.7216 & 0.1446 & 0.3174 & 0.1290 & -0.0499 \\
-0.2936 & -0.1364 & -0.2877 & -0.1685 & 0.8025 & -0.1267 & 0.2488 & -0.2022 & -0.1136 & -0.0913 \\
-0.2985 & -0.1039 & 0.6761 & 0.1929 & 0.1144 & -0.0271 & -0.1953 & -0.4820 & -0.1015 & 0.3333 \\
-0.3463 & -0.4610 & -0.2351 & -0.0080 & -0.1490 & 0.3106 & -0.0458 & -0.1204 & 0.6817 & 0.1107 \\
-0.1641 & -0.1217 & 0.0780 & -0.2572 & -0.2844 & 0.1491 & -0.0531 & -0.3968 & -0.2227 & -0.7570 \\
-0.3225 & -0.3635 & 0.0018 & 0.5610 & -0.0823 & 0.2045 & 0.2662 & 0.3892 & -0.4119 & -0.0924 \\
-0.3194 & 0.5471 & 0.2328 & 0.2675 & 0.2494 & 0.1184 & -0.1268 & 0.2588 & 0.3936 & -0.3986 \\
-0.3416 & 0.0393 & -0.3502 & -0.0687 & -0.0126 & -0.0591 & -0.7953 & 0.1460 & -0.2903 & 0.1127 \\
-0.3418 & 0.2798 & 0.1263 & -0.5563 & -0.0977 & 0.4877 & 0.2648 & 0.1940 & -0.1874 & 0.2988
\end{bmatrix}
\]

\textbf{MATLAB Code:}
\begin{verbatim}
Q = [-0.3154 0.4547 -0.4009 0.2905 -0.3397 -0.2158 0.2910 -0.4204 -0.0430 0.1530;
-0.3726 -0.1591 0.2126 -0.2886 -0.2139 -0.7216 0.1446 0.3174 0.1290 -0.0499;
-0.2936 -0.1364 -0.2877 -0.1685 0.8025 -0.1267 0.2488 -0.2022 -0.1136 -0.0913;
-0.2985 -0.1039 0.6761 0.1929 0.1144 -0.0271 -0.1953 -0.4820 -0.1015 0.3333;
-0.3463 -0.4610 -0.2351 -0.0080 -0.1490 0.3106 -0.0458 -0.1204 0.6817 0.1107;
-0.1641 -0.1217 0.0780 -0.2572 -0.2844 0.1491 -0.0531 -0.3968 -0.2227 -0.7570;
-0.3225 -0.3635 0.0018 0.5610 -0.0823 0.2045 0.2662 0.3892 -0.4119 -0.0924;
-0.3194 0.5471 0.2328 0.2675 0.2494 0.1184 -0.1268 0.2588 0.3936 -0.3986;
-0.3416 0.0393 -0.3502 -0.0687 -0.0126 -0.0591 -0.7953 0.1460 -0.2903 0.1127;
-0.3418 0.2798 0.1263 -0.5563 -0.0977 0.4877 0.2648 0.1940 -0.1874 0.2988];
\end{verbatim}

\textbf{Answer:}
$Q$ \wordChoice{\choice[correct]{is}\choice{is not}} an orthonormal basis for $\RR^{10}$.

\end{enumerate}

\end{example}

\section*{Cross Products}

\section*{Lines and Planes}

\begin{example}

    \begin{enumerate}
        \item Define the line passing through the plane spanned by vectors $\begin{bmatrix} 1 \\ 2 \\ 3 \end{bmatrix}$ and $\begin{bmatrix} 4 \\ 5 \\ 6 \end{bmatrix}$ going through the point $(1,2,3)$ and normal to the plane.
        \item %now a different example. Still a normal line but different vectors
        
    \end{enumerate}
\end{example}


\section*{SVD}


\section*{Eigenvalues and Eigenvectors}

\begin{example}
    Find the eigenvalues and eigenvectors of the matrix 

    
\end{example}

\section*{Multiplicity and Eigenspace}

\begin{example}
    What is the dimension of the eigenspace corresponding to the eigenvalue with the highest algebraic multiplicity for the matrix 



\end{example}




\end{document}