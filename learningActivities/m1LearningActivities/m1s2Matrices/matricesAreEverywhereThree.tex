\documentclass{ximera}
\graphicspath{     %% setup a global graphics path
{./}               %% look in the same-level directory
{./pictures/}      %% look in graphics
{../pictures/}     %% look up one directory, then in graphics
%{../../pictures/} %% look up two directories, then in graphics
}

\author{Zack Reed} %borrowed from Bart and PEter Selinger
\title{Learning Activity: Matrix Properties and Operations}
\begin{document}
\begin{abstract}
Here we introduce matrices similar to vectors
\end{abstract}
\maketitle
 
\section*{Matrix Properties and Operations}

\begin{remark}

  Because it is productive to think about matrices as arrays of vectors, we can take linear combinations of matrices in the same way we do with vectors. This is a powerful tool for data manipulation and analysis, and we'll gain even more useful relationships between matrices and vectors in the next chapter.

  For now, some definitions to make more precise the idea of a linear combination of matrices:

  \begin{definition}\name{Addition of matrices}
    Let $A$ and $B$ be two
    $m\times n$-matrices. Then $A+B=C$%
    \index{matrix!addition}%
    \index{sum|see{addition}}%
    \index{addition!of matrices} where $C$ is the $m\times n$-matrix
    $C$ defined by
    \begin{equation*}
      C_{ij}=A_{ij}+B_{ij}
    \end{equation*}

    Said differently, you add matrices by adding their corresponding entries.

    (IMPORTANT NOTE: You can only add matrices of the same size. If you have missing data, you need to make a decision on how to fill it in before you can add matrices together.)
  \end{definition}

  \begin{definition}\name{Scalar multiplication of matrices}
    Let $A$ be an $m\times n$-matrix and let $c$ be a scalar. Then $cA=D$%
    \index{matrix!scalar multiplication}%
    \index{scalar multiplication!of matrices} where $D$ is the $m\times n$-matrix
    $D$ defined by
    \begin{equation*}
      D_{ij}=cA_{ij}
    \end{equation*}

    Said differently, you multiply a matrix by a scalar by multiplying each entry of the matrix by the scalar.
  \end{definition}

\end{remark}

Let's finish by practicing on some basic matrices. If possible, perform the specified linear combinations.

\[
A = \begin{pmatrix}
2 & -1 & 0 \\
4 & 3 & -2 \\
\end{pmatrix}, \quad
B = \begin{pmatrix}
0 & 2 \\
-1 & 3
\end{pmatrix}, \quad
C = \begin{pmatrix}
4 & 0 & 1 \\
-2 & 2 & 5
\end{pmatrix}, \quad
D = \begin{pmatrix}
1 & -3 \\
2 & 1 \\
0 & 4
\end{pmatrix}
\]

\begin{enumerate}
\item Find $2A+3B$.


  $2A+3B$ is \wordChoice{
    \choice{possible}
    \choice[correct]{not possible}
  } because the matrices \wordChoice{
    \choice[correct]{are not}
    \choice{are}
  } the same size.


\item Find $-2C+5D$.



  $-2C+5D$ is \wordChoice{
    \choice{possible}
    \choice[correct]{not possible}
  } because the matrices \wordChoice{
    \choice[correct]{are not}
    \choice{are}
  } the same size.

\begin{problem}

  If we instead took the transpose of $C$, then 
  
  $$-2C^T+5D=\begin{pmatrix} \answer{-3} & -11 \\ 10 & \answer{1} \\ \answer{-2} & 10 \end{pmatrix}.$$

\end{problem}



\item Find $3A+5C-1D$.



  $3A+5C-1D$ is \wordChoice{
    \choice{possible}
    \choice[correct]{not possible}
  } because the matrices \wordChoice{
    \choice{are}
    \choice[correct]{are not}
  } the same size.

  \begin{problem}

    If we instead took the transpose of $D$, then 
    
    $$3A+5C-1D^T=\begin{pmatrix} 25 & \answer{-5} & \answer{5} \\ 5 & \answer{18} & 15 \end{pmatrix}.$$

  \end{problem}



\end{enumerate}


\end{document}