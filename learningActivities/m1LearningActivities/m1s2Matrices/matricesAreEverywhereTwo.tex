\documentclass{ximera}
\graphicspath{     %% setup a global graphics path
{./}               %% look in the same-level directory
{./pictures/}      %% look in graphics
{../pictures/}     %% look up one directory, then in graphics
%{../../pictures/} %% look up two directories, then in graphics
}

\author{Zack Reed} %borrowed from Bart and PEter Selinger
\title{Data Storing and Manipulation}
\begin{document}
\begin{abstract}

\end{abstract}
\maketitle

\begin{exploration}\name{Data Storing and Manipulation}

Now that we know the basics, onto the examples!

Matrices can be thought of as a ``mathematical spreadsheet.'' With a
spreadsheet you have rows and columns of data.  You can think of a
matrix a just a bunch of vectors stacked together either horizontally
or vertically.

 
\begin{example}[Population Counts] %https://worldpopulationreview.com/states/states-by-race
  The Midwest of the United States consists of $12$ states. We can
  express the $2023$
  \link[demographics]{https://worldpopulationreview.com/states/states-by-race}
  of each state as a vector represented by an ordered tuple. The
  ordered tuple for Ohio looks like:
  \[
  \vec{p}_{\texttt{OH}} = (\underset{\text{White}}{9394878},\underset{\text{Black}}{1442655},\underset{\text{American Indian}}{20442},\underset{\text{Asian}}{268527},\underset{\text{Hawaiian}}{3907},\underset{\text{Other}}{544866}).
  \]
  The ordered tuples for each state in the Midwest looks like:
\begin{align*}
  \vec{p}_{\texttt{IA}} &= (2806418,117035,10538,79296,3941,132783)\\
  \vec{p}_{\texttt{IL}} &= (8874067,1796660,33972,709567,5196,1296702)\\
  \vec{p}_{\texttt{IN}} &= (5510354,631923,14030,158705,2205,379676)\\
  \vec{p}_{\texttt{KA}} &= (2416165,165837,22278,87093,2344,218902)\\
  \vec{p}_{\texttt{MI}} &= (7735902,1360149,50035,316844,3117,507860)\\
  \vec{p}_{\texttt{MN}} &= (4572149,359817,54558,275242,2201,336199)\\
  \vec{p}_{\texttt{MO}} &= (4978046,698043,24274,123810,8887,291100)\\
  \vec{p}_{\texttt{ND}} &= (651470,23959,39165,11979,1004,32817)\\
  \vec{p}_{\texttt{NE}} &= (1641256,91896,16875,47944,1235,124620)\\
  \vec{p}_{\texttt{OH}} &= (9394878,1442655,20442,268527,3907,544866)\\
  \vec{p}_{\texttt{SD}} &= (735228,18836,74975,12413,544,37340)\\
  \vec{p}_{\texttt{WI}} &= (4895065,367889,48674,163396,2672,329279)
\end{align*}

Putting these into a matrix, we have: 

\[
  \begin{pmatrix}
  \vec{p}_{\texttt{IA}} \\
  \vec{p}_{\texttt{IL}} \\
  \vec{p}_{\texttt{IN}} \\
  \vec{p}_{\texttt{KA}} \\
  \vec{p}_{\texttt{MI}} \\
  \vec{p}_{\texttt{MN}} \\
  \vec{p}_{\texttt{MO}} \\
  \vec{p}_{\texttt{ND}} \\
  \vec{p}_{\texttt{NE}} \\
  \vec{p}_{\texttt{OH}} \\
  \vec{p}_{\texttt{SD}} \\
  \vec{p}_{\texttt{WI}}
  \end{pmatrix}
  =
  \begin{pmatrix}
    \text{White} & \text{Black} & \text{American Indian} & \text{Asian} & \text{Hawaiian} & \text{Other} \\
  2806418 & 117035 & 10538 & 79296 & 3941 & 132783\\
  8874067 & 1796660 & 33972 & 709567 & 5196 & 1296702\\
  5510354 & 631923 & 14030 & 158705 & 2205 & 379676\\
  2416165 & 165837 & 22278 & 87093 & 2344 & 218902\\
  7735902 & 1360149 & 50035 & 316844 & 3117 & 507860\\
  4572149 & 359817 & 54558 & 275242 & 2201 & 336199\\
  4978046 & 698043 & 24274 & 123810 & 8887 & 291100\\
  651470 & 23959 & 39165 & 11979 & 1004 & 32817\\
  1641256 & 91896 & 16875 & 47944 & 1235 & 124620\\
  9394878 & 1442655 & 20442 & 268527 & 3907 & 544866\\
  735228 & 18836 & 74975 & 12413 & 544 & 37340\\
  4895065 & 367889 & 48674 & 163396 & 2672 & 329279
  \end{pmatrix}
  \]

  We'll call this matrix $P$. By manipulating entries, rows, or columns of $P$, we can answer many questions about the data (even before doing more advanced linear algebraic methods).

  Before you do anything else, open up MATLAB and make a new live script. Load the data into MATLAB with 

  \begin{verbatim}
  
    load +linalg/midwest_2023.mat

  \end{verbatim}

  You should have access to the matrix \texttt{P}, and then the lists of labels as well.

  Using \texttt{state\_labels} and \texttt{demographic\_labels}, you can see the row and column numbers of the labels.

  %added to defs thms
  \begin{definition}\name{Notation Alert}

    This is a good opportunity to introduce more notation. If we want a specific entry in a matrix, we'll either say $P_{i,j}$ or $P(i,j)$ to mean the $(i,j)$-entry of the matrix $P$. Remember that the first number is the row number and the second number is the column number.

    So $P_{3,5}$ is the entry in the third row and fifth column of $P$, which is $2205$, the number of Hawaiian people living in Indiana in $2023$.

    When we want to specify a range of rows and/or columns at once, we use colons ``:'' to separate the starting and ending row or column. For example, $P_{2:4,3}$ means the entries in the second, third, and fourth rows of the third column of $P$, which are the American Indian population counts for Illinois, Indiana, and Kansas in $2023$.

    If we want to specify an entire row or column, we can use a colon by itself. For example, $P_{5,:}$ means the entire fifth row of $P$, which is the demographic data for Michigan in $2023$. 

    Finally, if we use just a single colon, that creates a $nx1$ vector of all the entries in the entire matrix, starting with the first row and moving down. 

  \end{definition}

\begin{enumerate}
\item What is the demographic submatrix of Indiana, Minnesota, and Wisconsin?
\item What is the submatrix of the Midwest that includes only the
  demographics of Black, American Indian, and Hawaiian?
\item What are the combined demographics of the states Michigan, Ohio,
  and Indiana?
\item Suppose that the annual percentage growth rate of the Midwest is
  currently $0.1\%$. Assuming this is even across all demographics,
  what would be the 2024 population matrix?
\item What are the average demographics of the Midwest?
\item What is each state's average demographic? 
\end{enumerate}

\begin{hint}

  If you're having trouble with the first two questions, check that you're not mixing up rows and columns, and that you're using the correct notation for submatrices.

\end{hint}

\begin{solution}
  \begin{enumerate}
  \item We need to put Indiana, Minnesota, and Wisconsin into a matrix. These are the third, sixth, and twelfth rows of $P$. Since none of them are connected, we need to pull out each individual rows and stack them together. We do this using the notation (I'll start and you finish):

    \[
    \texttt{P(}\answer[given]{3}\texttt{,:); P(}\answer[given]{6}\texttt{,:);}\answer[given,format=string]{P(12,:)}\texttt{]} 
    \]
    \item To find the submatrix of the Midwest that includes only the
      demographics of Black, American Indian, and Hawaiian, we similarly use the notation:

      \[
      \texttt{P(:,}\answer{2}\texttt{); P(:,}\answer{3}\texttt{);}\answer[format=string]{P(:,5)}\texttt{]}
      \]
  \item To find the combined demographics of Michigan, Ohio, and
    Indiana we treat the rows as vectors and add them together. You can do this a couple of ways, such as first defining each as a new vector and then adding them together, or by doing it all at once. For clarity, I'll first make them all vectors:

      \begin{Verbatim}
      
        p_MI = P(5,:);
        p_OH = P(10,:);
        p_IN = P(3,:);
        p_MI + p_OH + p_IN

      \end{Verbatim}

      If you entered this correctly, you should get the following:

    \[
    \vec{p}_{\texttt{MI}} + \vec{p}_{\texttt{OH}} + \vec{p}_{\texttt{IN}} = (\answer[given]{22641134},\answer[given]{3434727},\answer[given]{84507},\answer[given]{744076},\answer[given]{9229},\answer[given]{1432402})
    \]
  \item We want to define a new matrix for 2024 let's call it $\texttt{P\_2024}$.
  
  To find the population growth matrix, we can treat each row/column as if they were vectors, so we use the scalar multiplication, but we can do it on the entire matrix at once. 

  \begin{verbatim}
  
    .001*P

  \end{verbatim}

  Then, again since a matrix is an array of vectors, we can add matrices of the same size together.

    \begin{Verbatim}

      P_2024 = P + 0.001*P;

    \end{Verbatim}

    If you entered this correctly, you the Hawaiian population of South Dakota in 2024 should be $\answer{544.544}$, found at $P_{11,5}$.

\item Lucky for data anlysts everywhere, averaging is a linear combination, since it involves just adding and scaling vectors. In this case, we want to average across the states, so we treat the rows as vectors and find the average across the rows, returning a $1\times 6$ vector with the average for each demographic. 

The $\texttt{mean}$ function in MATLAB will do this for you, and you specify averaging across the rows or columns by using a $1$ or $2$ in the second argument. 

Using: 

\begin{verbatim}
  average_midwest = mean(P,1);
\end{verbatim}

You should get the following:

(Note, round up to the nearest hundred thousand, so $2.1$ means rounded to $2,100,000$)

\[
\texttt{average\_midwest} = [\answer{4.5},\answer{.6},\answer{0},\answer{0.2},\answer{0},\answer{0.4}]
\]

\item Finally, to find the average demographic of each state, we treat the columns as vectors and find the average across the columns, returning a $12\times 1$ vector with the average for each state. We do this by putting a $2$ in the second argument of the $\texttt{mean}$ function.

Using:

\begin{verbatim}
  average_state = mean(P,2);
\end{verbatim}

You should get the following:

\[
\texttt{average\_state} = \begin{pmatrix}
  \answer{.5}\\
  \answer{2.1}\\
  \answer{1.1}\\
  \answer{.5}\\
  \answer{1.7}\\
  \answer{.9}\\
  \answer{1}\\
  \answer{.1}\\
  \answer{.3}\\
  \answer{1.9}\\
  \answer{.1}\\
  \answer{1}
\end{pmatrix}
\]


  \end{enumerate}
\end{solution}
\end{example}

\end{exploration}

\end{document}