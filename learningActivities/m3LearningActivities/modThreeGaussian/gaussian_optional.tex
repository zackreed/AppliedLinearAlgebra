\documentclass{ximera}
\graphicspath{  %% When looking for images,
{./}            %% look here first,
{./pictures/}   %% then look for a pictures folder,
{../pictures/}  %% which may be a directory up.
{../../pictures/}  %% which may be a directory up.
{../../../pictures/}  %% which may be a directory up.
{../../../../pictures/}  %% which may be a directory up.
}

\usepackage{listings}
%\usepackage{circuitikz}
\usepackage{xcolor}
\usepackage{amsmath,amsthm}
\usepackage{subcaption}
\usepackage{graphicx}
\usepackage{tikz}
%\usepackage{tikz-3dplot}
\usepackage{amsfonts}
%\usepackage{mdframed} % For framing content
%\usepackage{tikz-cd}

  \renewcommand{\vector}[1]{\left\langle #1\right\rangle}
  \newcommand{\arrowvec}[1]{{\overset{\rightharpoonup}{#1}}}
  \newcommand{\ro}{\texttt{R}}%% row operation
  \newcommand{\dotp}{\bullet}%% dot product
  \renewcommand{\l}{\ell}
  \let\defaultAnswerFormat\answerFormatBoxed
  \usetikzlibrary{calc,bending}
  \tikzset{>=stealth}
  




%make a maroon color
\definecolor{maroon}{RGB}{128,0,0}
%make a dark blue color
\definecolor{darkblue}{RGB}{0,0,139}
%define the color fourier0 to be the maroon color
\definecolor{fourier0}{RGB}{128,0,0}
%define the color fourier1 to be the dark blue color
\definecolor{fourier1}{RGB}{0,0,139}
%define the color fourier 1t to be the light blue color
\definecolor{fourier1t}{RGB}{173,216,230}
%define the color fourier2 to be the dark green color
\definecolor{fourier2}{RGB}{0,100,0}
%define teh color fourier2t to be the light green color
\definecolor{fourier2t}{RGB}{144,238,144}
%define the color fourier3 to be the dark purple color
\definecolor{fourier3}{RGB}{128,0,128}
%define the color fourier3t to be the light purple color
\definecolor{fourier3t}{RGB}{221,160,221}
%define the color fourier0t to be the red color
\definecolor{fourier0t}{RGB}{255,0,0}
%define the color fourier4 to be the orange color
\definecolor{fourier4}{RGB}{255,165,0}
%define the color fourier4t to be the darker orange color
\definecolor{fourier4t}{RGB}{255,215,0}
%define the color fourier5 to be the yellow color
\definecolor{fourier5}{RGB}{255,255,0}
%define the color fourier5t to be the darker yellow color
\definecolor{fourier5t}{RGB}{255,255,100}
%define the color fourier6 to be the green color
\definecolor{fourier6}{RGB}{0,128,0}
%define the color fourier6t to be the darker green color
\definecolor{fourier6t}{RGB}{0,255,0}

%New commands for this doc for errors in copying
\newcommand{\eigenvar}{\lambda}
%\newcommand{\vect}[1]{\mathbf{#1}}
\renewcommand{\th}{^{\text{th}}}
\newcommand{\st}{^{\text{st}}}
\newcommand{\nd}{^{\text{nd}}}
\newcommand{\rd}{^{\text{rd}}}
\newcommand{\paren}[1]{\left(#1\right)}
\newcommand{\abs}[1]{\left|#1\right|}
\newcommand{\R}{\mathbb{R}}
\newcommand{\C}{\mathbb{C}}
\newcommand{\Hilb}{\mathbb{H}}
\newcommand{\qq}[1]{\text{#1}}
\newcommand{\Z}{\mathbb{Z}}
\newcommand{\N}{\mathbb{N}}
\newcommand{\q}[1]{\text{``#1''}}
%\newcommand{\mat}[1]{\begin{bmatrix}#1\end{bmatrix}}
\newcommand{\rref}{\text{reduced row echelon form}}
\newcommand{\ef}{\text{echelon form}}
\newcommand{\ohm}{\Omega}
\newcommand{\volt}{\text{V}}
\newcommand{\amp}{\text{A}}
\newcommand{\Seq}{\textbf{Seq}}
\newcommand{\Poly}{\textbf{P}}
\renewcommand{\quad}{\text{    }}
\newcommand{\roweq}{\simeq}
\newcommand{\rowop}{\simeq}
\newcommand{\rowswap}{\leftrightarrow}
\newcommand{\Mat}{\textbf{M}}
\newcommand{\Func}{\textbf{Func}}
\newcommand{\Hw}{\textbf{Hamming weight}}
\newcommand{\Hd}{\textbf{Hamming distance}}
\newcommand{\rank}{\text{rank}}
\newcommand{\longvect}[1]{\overrightarrow{#1}}
% Define the circled command
\newcommand{\circled}[1]{%
  \tikz[baseline=(char.base)]{
    \node[shape=circle,draw,inner sep=2pt,red,fill=red!20,text=black] (char) {#1};}%
}

% Define custom command \strikeh that just puts red text on the 2nd argument
\newcommand{\strikeh}[2]{\textcolor{red}{#2}}

% Define custom command \strikev that just puts red text on the 2nd argument
\newcommand{\strikev}[2]{\textcolor{red}{#2}}

%more new commands for this doc for errors in copying
\newcommand{\SI}{\text{SI}}
\newcommand{\kg}{\text{kg}}
\newcommand{\m}{\text{m}}
\newcommand{\s}{\text{s}}
\newcommand{\norm}[1]{\left\|#1\right\|}
\newcommand{\col}{\text{col}}
\newcommand{\sspan}{\text{span}}
\newcommand{\proj}{\text{proj}}
\newcommand{\set}[1]{\left\{#1\right\}}
\newcommand{\degC}{^\circ\text{C}}
\newcommand{\centroid}[1]{\overline{#1}}
\newcommand{\dotprod}{\boldsymbol{\cdot}}
%\newcommand{\coord}[1]{\begin{bmatrix}#1\end{bmatrix}}
\newcommand{\iprod}[1]{\langle #1 \rangle}
\newcommand{\adjoint}{^{*}}
\newcommand{\conjugate}[1]{\overline{#1}}
\newcommand{\eigenvarA}{\lambda}
\newcommand{\eigenvarB}{\mu}
\newcommand{\orth}{\perp}
\newcommand{\bigbracket}[1]{\left[#1\right]}
\newcommand{\textiff}{\text{ if and only if }}
\newcommand{\adj}{\text{adj}}
\newcommand{\ijth}{\emph{ij}^\text{th}}
\newcommand{\minor}[2]{M_{#2}}
\newcommand{\cofactor}{\text{C}}
\newcommand{\shift}{\textbf{shift}}
\newcommand{\startmat}[1]{
  \left[\begin{array}{#1}
}
\newcommand{\stopmat}{\end{array}\right]}
%a command to give a name to explorations and hints and theorems
\newcommand{\name}[1]{\begin{centering}\textbf{#1}\end{centering}}
\newcommand{\vect}[1]{\vec{#1}}
\newcommand{\dfn}[1]{\textbf{#1}}
\newcommand{\transpose}{\mathsf{T}}
\newcommand{\mtlb}[2][black]{\texttt{\textcolor{#1}{#2}}}
\newcommand{\RR}{\mathbb{R}} % Real numbers
\newcommand{\id}{\text{id}}
\newcommand{\coord}[1]{\langle#1\rangle}
\newcommand{\RREF}{\text{RREF}}
\newcommand{\Null}{\text{Null}}
\newcommand{\Nullity}{\text{Nullity}}
\newcommand{\Rank}{\text{Rank}}
\newcommand{\Col}{\text{Col}}
\newcommand{\Ef}{\text{EF}}
\newcommand{\boxprod}[3]{\abs{(#1\times#2)\cdot#3}}

\author{Zack Reed} %PEter Selinger
\title{Optional: Extra Gauss-Jordan Practice}
\begin{document}
\begin{abstract}
Here we introduce one of the most prevalent applications of matrices and vectors, the solving of systems of equations.
\end{abstract}
\maketitle

\section*{Gaussian Elimination and Rank}
\subsection*{Row Echelon and Reduced Row Echelon Forms}
 
Recall that a matrix (or augmented matrix) is in \dfn{row-echelon form} if:
\begin{itemize}
\item All entries {\it below} each leading entry are $0$.
\item Each leading entry is in a column to the right of the leading entries in the rows above it.
\item All rows of zeros, if there are any, are located below non-zero rows.
\end{itemize}
 
A matrix in row-echelon form is said to be in \dfn{reduced row-echelon form} if it has the following additional properties
\begin{itemize}
\item Each leading entry is a $1$
\item All entries {\it above} each leading $1$ are $0$
\end{itemize}
 
 
Matrices in row-echelon form and reduced row-echelon form are ``convenient" because their corresponding systems of equations are easy to solve. 
 
\begin{exploration}\label{init:gaussianelim1}
Solve the following system of equations.
$$\begin{array}{ccccccccc}
      x &+ &2y&-&3z&= &1 \\
     -5x& +&2y&-&3z&=&1\\
      x&- &2y&+&z&=&1
    \end{array}$$
 
We create an augmented matrix corresponding to the system and apply row operations until the matrix is in row-echelon form. 
$$\left[\begin{array}{ccc|c} 
 1&2&-3&1\\-5&2&-3&1\\1&-2&1&1
 \end{array}\right]
 \begin{array}{c}
 \\
 \xrightarrow{R_2+5R_1}\\
 \xrightarrow{R_3-R_1}\\
 \end{array}
\left[\begin{array}{ccc|c} 
 1&2&-3&1\\0&12&-18&6\\0&-4&4&0
 \end{array}\right]
 \begin{array}{c}
 \\
 \\
 \xrightarrow{R_3+\frac{1}{3}R_2}\\
 \end{array}$$
 \begin{equation}\label{eq:ref1}
 \left[\begin{array}{ccc|c} 
 1&2&-3&1\\0&12&-18&6\\0&0&-2&2
 \end{array}\right]
\end{equation}
 
Note that the elementary row operations that lead to (\ref{eq:ref1}) were not prescribed.  We may employ row-operations in a different manner and obtain a different matrix in row-echelon form.  For example, suppose for some reason we had begun by switching the first and third rows.
 
$$\left[\begin{array}{ccc|c} 
 1&2&-3&1\\-5&2&-3&1\\1&-2&1&1
 \end{array}\right]
 \begin{array}{c}
 \\
 \xrightarrow{R_1\leftrightarrow R_3}\\
\\
 \end{array}
\left[\begin{array}{ccc|c} 
 1&-2&1&1\\-5&2&-3&1\\1&2&-3&1
 \end{array}\right]
 \begin{array}{c}
 \\
\\
\\
 \end{array}$$
 
Next we would reduce this matrix to row-echelon form, perhaps in this way:
 
$$\left[\begin{array}{ccc|c} 
 1&-2&1&1\\-5&2&-3&1\\1&2&-3&1
 \end{array}\right]
 \begin{array}{c}
 \\
 \xrightarrow{R_2+5R_1}\\
 \xrightarrow{R_3-R_1}\\
 \end{array}
\left[\begin{array}{ccc|c} 
 1&-2&1&1\\0&-8&2&6\\0&4&-4&0
 \end{array}\right]
 \begin{array}{c}
 \\
 \\
 \xrightarrow{R_3+\frac{1}{2}R_2}\\
 \end{array}
$$
\begin{equation}\label{eq:ref2}
 \left[\begin{array}{ccc|c} 
 1&-2&1&1\\0&-8&2&6\\0&0&-3&3
 \end{array}\right]
\end{equation}
 
The augmented matrices in (\ref{eq:ref1}) and (\ref{eq:ref2}) are clearly not the same, but both are in row-echelon form.
 
If we write the systems of equations corresponding to (\ref{eq:ref1}) and (\ref{eq:ref2}), we can employ back substitution to solve them.  The matrix in (\ref{eq:ref1}) corresponds to
$$\begin{array}{ccccccccc}
      x &+ &2y&-&3z&= &1 \\
     & &12y&-&18z&=&6\\
      &&&&-2z&=&2
    \end{array}$$
     
The matrix in (\ref{eq:ref2}) corresponds to   
$$\begin{array}{ccccccccc}
      x &- &2y&+&z&= &1 \\
     & &-8y&+&2z&=&6\\
      &&&&-3z&=&3
    \end{array}$$
     
Because both systems are equivalent to the original system, it is not surprising that back substitution yields the same solution for both systems.
 
$$x=\answer{0},\quad y=\answer{-1},\quad z=\answer{-1}$$
 
%From the last equation we get $z=-1$. 
%Next, from the second equation we have $12y-18(-1)=6$, which yields $y=-1$.
%Finally, the first equation gives us $x+2(-1)-3(-1)=1$.  Solving for $x$ we get $x=0$. 
%Therefore, the solution to the system is $$x=0, y=-1, z=-1$$
%We arrive at a different matrix in row-echelon form, but back-substitution yields the same solution to the system (you should check this):  $$x=0, y=-1, z=-1$$
\end{exploration}
 
\begin{exploration}\label{init:gaussianelim2} In this problem we revisit the system
$$\begin{array}{ccccccccc}
      x &+ &2y&-&3z&= &1 \\
     -5x& +&2y&-&3z&=&1\\
      x&- &2y&+&z&=&1
    \end{array}$$
     
Following the steps we took to get (\ref{eq:ref1}), but taking the process a little further, we get the reduced row-echelon form.
$$\left[\begin{array}{ccc|c} 
 1&2&-3&1\\-5&2&-3&1\\1&-2&1&1
 \end{array}\right]
 \begin{array}{c}
 \\
 \xrightarrow{R_2+5R_1}\\
 \xrightarrow{R_3-R_1}\\
 \end{array}
\left[\begin{array}{ccc|c} 
 1&2&-3&1\\0&12&-18&6\\0&-4&4&0
 \end{array}\right]
 \begin{array}{c}
 \\
 \\
 \xrightarrow{R_3+\frac{1}{3}R_2}\\
 \end{array}$$
 $$\left[\begin{array}{ccc|c} 
 1&2&-3&1\\0&12&-18&6\\0&0&-2&2
 \end{array}\right]
  \color{blue}
 \begin{array}{c}
 \\
  \xrightarrow{\frac{1}{12}R_2}\\
 \xrightarrow{-\frac{1}{2}R_3}\\
 \end{array}
 \left[\begin{array}{ccc|c} 
 1&2&-3&1\\0&1&-\frac{3}{2}&\frac{1}{2}\\0&0&1&-1
 \end{array}\right]
  \begin{array}{c}
  \xrightarrow{R_1+3R_3}\\
 \xrightarrow{R_2+\frac{3}{2}R_3}\\
\\
 \end{array}$$
\begin{equation}\label{eq:rref3}  \color{blue}
 \left[\begin{array}{ccc|c} 
 1&2&0&-2\\0&1&0&-1\\0&0&1&-1
 \end{array}\right]
  \begin{array}{c}
  \xrightarrow{R_1-2R_2}\\
 \\
\\
 \end{array}
 \color{black}
  \left[\begin{array}{ccc|c} 
 1&0&0&0\\0&1&0&-1\\0&0&1&-1
 \end{array}\right]\end{equation}

If possible, find the elementary row operations that take (\ref{eq:ref2}) to the reduced row-echelon form in (\ref{eq:rref3}).
 
$$\left[\begin{array}{ccc|c} 
 1&2&-3&1\\-5&2&-3&1\\1&-2&1&1
 \end{array}\right]
 \begin{array}{c}
 \\
 \xrightarrow{R_1\leftrightarrow R_3}\\
\\
 \end{array}
\left[\begin{array}{ccc|c} 
 1&-2&1&1\\-5&2&-3&1\\1&2&-3&1
 \end{array}\right]$$
$$\begin{array}{c}
 \\
 \xrightarrow{R_2+5R_1}\\
 \xrightarrow{R_3-R_1}\\
 \end{array}
\left[\begin{array}{ccc|c} 
 1&-2&1&1\\0&-8&2&6\\0&4&-4&0
 \end{array}\right]
 \begin{array}{c}
 \\
 \\
 \xrightarrow{R_3+\frac{1}{2}R_2}\\
 \end{array}
$$
\begin{equation*}
  \left[\begin{array}{ccc|c} 
 1&-2&1&1\\0&-8&2&6\\0&0&-3&3
 \end{array}\right]
 \color{red}
 \begin{array}{c}
\rightsquigarrow\text{Elementary}\rightsquigarrow\\
\text{Row Ops.}
\end{array}
 \color{black}
  \left[\begin{array}{ccc|c} 
 1&0&0&0\\0&1&0&-1\\0&0&1&-1
 \end{array}\right]
\end{equation*}
\end{exploration}
 
The reduced row-echelon form of a matrix is an instance of a row-echelon form of the matrix.  While a given matrix may have multiple row-echelon forms, all row-echelon forms will share one characteristic: the number of nonzero rows in a row-echelon form of the given matrix will be the same.
We will prove this result in Theorem \ref{th:samenumberofnonzerorows}.
 
\section*{Gaussian and Gauss-Jordan Elimination}
 
\begin{definition}[Gaussian Elimination]\label{def:GaussianElimination}
The process of using the elementary row operations on a matrix to transform it into row-echelon form is called \dfn{Gaussian Elimination}.
\end{definition}

 The following algorithm takes any matrix (or augmented matrix) and transforms it into row-echelon form:
\begin{algorithm}[Gaussian Algorithm] \label{alg:gaussian}
Let $A$ be an $m\times n$ matrix.
 
Set $i=1$ initially.
\begin{itemize}
\item[] Step 1. If $A$ consists entirely of zeros, stop;  $A$ is already in row-echelon form.
 
\item[] Step 2. Otherwise, find the first column from the left containing a nonzero entry in row $i$ or below row $i$.  This column will be called a \dfn{pivot column}.  Go down the pivot column, beginning with row $i$. Pick the topmost nonzero entry and call it $a$. If $a$ is not in row $i$, switch rows so that $a$ moves to row $i$.  Now $a$ is the \dfn{leading entry} in its row.  We will also refer to $a$ as a \dfn{pivot}. 
 
\item[] Step 3. By subtracting multiples of the row containing $a$ from rows below it, make each entry below $a$ zero.
 
\item[] Step 4.  Set $i=i+1$.  If $i > m$ then stop; $A$ is in row-echelon form.
 
\end{itemize}
 
Repeat steps 1--4 on the matrix consisting of the remaining rows.
When the process stops, $A$ will be in row echelon form.
\end{algorithm}
Gaussian Algorithm guarantees that every matrix will have a row-echelon form. 
 
\begin{example}\label{ex:non-augmented}
 
Use the Gaussian Algorithm to find a row-echelon form of $A$ if $$A=\begin{bmatrix}2&4\\1&2\\-1&1\\3&5\end{bmatrix}$$
\begin{explanation}
Following Step 2, we choose the first entry, $2$, as our pivot.  We then perform step 3, using the top row to get zeros in all entries below the $2$.
$$\begin{bmatrix} \fbox{$2$}&4\\1&2\\-1&1\\3&5\end{bmatrix}
  \begin{array}{c}
  \\
  \xrightarrow{R_2-(1/2)R_1}\\
  \xrightarrow{R_3+(1/2)R_1}\\
 \xrightarrow{R_4-(3/2)R_1}\\
 \end{array}
\begin{bmatrix}2&4\\0&0\\0&3\\0&-1\end{bmatrix}
$$
The first row is now complete, and we repeat the process on the rows below it. We identify $3$ as a pivot entry in the second column and move the row containing $3$ to be directly below the first completed row.  We then use the $3$ to make each entry below the $3$ a zero. 
 
$$\begin{bmatrix}2&4\\0&0\\0&\fbox{$3$}\\0&-1\end{bmatrix}
\xrightarrow{R_2\leftrightarrow R_3}\\
\begin{bmatrix}2&4\\0&\fbox{$3$}\\0&0\\0&-1\end{bmatrix}
  \begin{array}{c}
 \\
\\
  \\
 \xrightarrow{R_4+\frac{1}{3}R_2}\\
 \end{array}
 \begin{bmatrix}2&4\\0&3\\0&0\\0&0\end{bmatrix}
$$
This time the algorithm terminates since row 3 and row 4 are zero rows.
\end{explanation}
\end{example}
 
\begin{definition}[Gauss-Jordan Elimination]\label{def:GaussJordanElimination}
The process of using the elementary row operations on a matrix to transform it into reduced row-echelon form is called \dfn{Gauss-Jordan elimination}.
\end{definition}
 
Given a matrix in row-echelon form, it is easy to bring it the reduced row-echelon form.  For example, continuing with Example \ref{ex:non-augmented}, we can start where we left off and compute $\mbox{rref}(A)$.  From our earlier computations we have:
 
$$\begin{bmatrix}2&4\\-1&1\\3&5\\1&2\end{bmatrix}\rightsquigarrow\begin{bmatrix}2&4\\0&3\\0&0\\0&0\end{bmatrix}$$
 
Now we create leading $1's$ and use them to to wipe out all non-zero entries above them.
$$\begin{bmatrix}2&4\\0&3\\0&0\\0&0\end{bmatrix}
  \begin{array}{c}
    \xrightarrow{(1/2)R_1}\\
  \xrightarrow{(1/3)R_2}\\
  \\
  \\
 \end{array}
\begin{bmatrix}1&2\\0&1\\0&0\\0&0\end{bmatrix}
  \begin{array}{c}
  \xrightarrow{R_1-2R_2}\\
\\
\\
 \\
 \end{array}
\begin{bmatrix}1&0\\0&1\\0&0\\0&0\end{bmatrix}=\mbox{rref}(A)$$
 
The following modification to the Gaussian Algorithm produces the reduced row-echelon form of a matrix.  This algorithm guarantees the existence of the reduced row-echelon form.
 
\begin{algorithm}[Gauss-Jordan Algorithm] \label{alg:gauss-jordan}
Let $A$ be an $m\times n$ matrix.
Follow the steps of the Gaussian Algorithm but modify Step 2 to create leading $1's$ by multiplying the row containing $a$ by $\frac{1}{a}$.
%Set $i=1$ initially.
%\begin{itemize}
%\item[] Step 1. If $A$ consists entirely of zeros, stop.  $A$ is already in row-echelon form.
 
%\item[] Step 2*. Otherwise, find the first column from the left containing a nonzero entry in row $i$ or below row $i$.  This column will be called a \dfn{pivot column}.  Scan the pivot column from top to bottom, starting with row $i$.  Pick the topmost nonzero entry and call it $a$.  Switch rows, if necessary, to move the row containing $a$ to row $i$.  Now $a$ is the \dfn{leading entry} in its row.  We will also refer to $a$ as a \dfn{pivot}.  Multiply the row containing $a$ by $\frac{1}{a}$ to create a leading $1$. 
 
%\item[] Step 3*. By subtracting multiples of the row containing the leading $1$ from rows {\it above} and below it, make each entry above and below the leading $1$ zero.
 
%\item[] Step 4.  Set $i=i+1$.  If $i&gt;m$ then stop, and $A$ will be in row-echelon form.
 
%\end{itemize}
 
%Repeat steps 1--4 on the matrix consisting of the remaining rows.
%When the process stops, $A$ will be in reduced row echelon form.
When the Gaussian Algorithm terminates, subtract multiples of the rows containing leading $1's$ from the rows above to make all entries above the pivots zero.
\end{algorithm}
 
 
\begin{example}\label{ex:gaussjordanalg}
Use the Gauss-Jordan Algorithm to solve the system
$$\begin{array}{ccccccccc}
      3x &+ &y&+&7z&= &7 \\
     5x& +&3y&+&9z&=&13\\
      2x&+ &y&+&4z&=&5
    \end{array}$$
\begin{explanation}
\begin{align*}&\left[\begin{array}{ccc|c} 
 \fbox{$3$}&1&7&7\\5&3&9&13\\2&1&4&5
 \end{array}\right]\\
 \begin{array}{c}
  \xrightarrow{\frac{1}{3}R_1}\\
\\
\\
 \end{array}
 &\left[\begin{array}{ccc|c} 
 \fbox{$1$}&1/3&7/3&7/3\\5&3&9&13\\2&1&4&5
 \end{array}\right]\\
 \begin{array}{c}
 \\
 \xrightarrow{R_2-5R_1}\\
\\
\end{array}
&\left[\begin{array}{ccc|c} 
 \fbox{$1$}&1/3&7/3&7/3\\0&4/3&-8/3&4/3\\2&1&4&5
 \end{array}\right]\\
 \begin{array}{c}
  \\
\\
 \xrightarrow{R_3-2R_1}\\
\end{array}&\left[\begin{array}{ccc|c} 
 1&1/3&7/3&7/3\\0&\fbox{$4/3$}&-8/3&4/3\\0&1/3&-2/3&1/3
 \end{array}\right]\\
 \begin{array}{c}
\\
 \xrightarrow{\frac{3}{4}R_2}\\
\\
\end{array}
&\left[\begin{array}{ccc|c} 
 1&1/3&7/3&7/3\\0&\fbox{$1$}&-2&1\\0&1/3&-2/3&1/3
 \end{array}\right]\\
 \begin{array}{c}
\\
\\
 \xrightarrow{R_3-\frac{1}{3}R_2}\\
\end{array}
&\left[\begin{array}{ccc|c} 
 1&1/3&7/3&7/3\\0&\fbox{$1$}&-2&1\\0&0&0&0
 \end{array}\right]\\
 \begin{array}{c}
 \xrightarrow{R_1-\frac{1}{3}R_2}\\
 \\
\\
\end{array}
&\left[\begin{array}{ccc|c} 
 1&0&3&2\\0&1&-2&1\\0&0&0&0
 \end{array}\right]
 \end{align*}
  
 We convert the reduced row-echelon form to a system of equations and find the solution.  The last equation contributes nothing to the system so we omit writing it down.
  
 $$\begin{array}{ccccccccc}
      x & &&+&3z&= &2 \\
     & &y&-&2z&=&1
    \end{array}$$
    The solution is
    $$x=2-3t,\quad y=1+2t,\quad z=t$$
\end{explanation}
\end{example}
The Gauss-Jordan Algorithm guarantees the existence of the reduced row-echelon form for all matrices. 
When doing computations by hand, however, the algorithm may not always be the optimal method of finding a row-echelon form or the reduced row-echelon form because the procedure often leads to fractions early in the process.
 
\end{document}