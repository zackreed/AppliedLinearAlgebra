\documentclass{ximera}
\graphicspath{  %% When looking for images,
{./}            %% look here first,
{./pictures/}   %% then look for a pictures folder,
{../pictures/}  %% which may be a directory up.
{../../pictures/}  %% which may be a directory up.
{../../../pictures/}  %% which may be a directory up.
{../../../../pictures/}  %% which may be a directory up.
}

\usepackage{listings}
%\usepackage{circuitikz}
\usepackage{xcolor}
\usepackage{amsmath,amsthm}
\usepackage{subcaption}
\usepackage{graphicx}
\usepackage{tikz}
%\usepackage{tikz-3dplot}
\usepackage{amsfonts}
%\usepackage{mdframed} % For framing content
%\usepackage{tikz-cd}

  \renewcommand{\vector}[1]{\left\langle #1\right\rangle}
  \newcommand{\arrowvec}[1]{{\overset{\rightharpoonup}{#1}}}
  \newcommand{\ro}{\texttt{R}}%% row operation
  \newcommand{\dotp}{\bullet}%% dot product
  \renewcommand{\l}{\ell}
  \let\defaultAnswerFormat\answerFormatBoxed
  \usetikzlibrary{calc,bending}
  \tikzset{>=stealth}
  




%make a maroon color
\definecolor{maroon}{RGB}{128,0,0}
%make a dark blue color
\definecolor{darkblue}{RGB}{0,0,139}
%define the color fourier0 to be the maroon color
\definecolor{fourier0}{RGB}{128,0,0}
%define the color fourier1 to be the dark blue color
\definecolor{fourier1}{RGB}{0,0,139}
%define the color fourier 1t to be the light blue color
\definecolor{fourier1t}{RGB}{173,216,230}
%define the color fourier2 to be the dark green color
\definecolor{fourier2}{RGB}{0,100,0}
%define teh color fourier2t to be the light green color
\definecolor{fourier2t}{RGB}{144,238,144}
%define the color fourier3 to be the dark purple color
\definecolor{fourier3}{RGB}{128,0,128}
%define the color fourier3t to be the light purple color
\definecolor{fourier3t}{RGB}{221,160,221}
%define the color fourier0t to be the red color
\definecolor{fourier0t}{RGB}{255,0,0}
%define the color fourier4 to be the orange color
\definecolor{fourier4}{RGB}{255,165,0}
%define the color fourier4t to be the darker orange color
\definecolor{fourier4t}{RGB}{255,215,0}
%define the color fourier5 to be the yellow color
\definecolor{fourier5}{RGB}{255,255,0}
%define the color fourier5t to be the darker yellow color
\definecolor{fourier5t}{RGB}{255,255,100}
%define the color fourier6 to be the green color
\definecolor{fourier6}{RGB}{0,128,0}
%define the color fourier6t to be the darker green color
\definecolor{fourier6t}{RGB}{0,255,0}

%New commands for this doc for errors in copying
\newcommand{\eigenvar}{\lambda}
%\newcommand{\vect}[1]{\mathbf{#1}}
\renewcommand{\th}{^{\text{th}}}
\newcommand{\st}{^{\text{st}}}
\newcommand{\nd}{^{\text{nd}}}
\newcommand{\rd}{^{\text{rd}}}
\newcommand{\paren}[1]{\left(#1\right)}
\newcommand{\abs}[1]{\left|#1\right|}
\newcommand{\R}{\mathbb{R}}
\newcommand{\C}{\mathbb{C}}
\newcommand{\Hilb}{\mathbb{H}}
\newcommand{\qq}[1]{\text{#1}}
\newcommand{\Z}{\mathbb{Z}}
\newcommand{\N}{\mathbb{N}}
\newcommand{\q}[1]{\text{``#1''}}
%\newcommand{\mat}[1]{\begin{bmatrix}#1\end{bmatrix}}
\newcommand{\rref}{\text{reduced row echelon form}}
\newcommand{\ef}{\text{echelon form}}
\newcommand{\ohm}{\Omega}
\newcommand{\volt}{\text{V}}
\newcommand{\amp}{\text{A}}
\newcommand{\Seq}{\textbf{Seq}}
\newcommand{\Poly}{\textbf{P}}
\renewcommand{\quad}{\text{    }}
\newcommand{\roweq}{\simeq}
\newcommand{\rowop}{\simeq}
\newcommand{\rowswap}{\leftrightarrow}
\newcommand{\Mat}{\textbf{M}}
\newcommand{\Func}{\textbf{Func}}
\newcommand{\Hw}{\textbf{Hamming weight}}
\newcommand{\Hd}{\textbf{Hamming distance}}
\newcommand{\rank}{\text{rank}}
\newcommand{\longvect}[1]{\overrightarrow{#1}}
% Define the circled command
\newcommand{\circled}[1]{%
  \tikz[baseline=(char.base)]{
    \node[shape=circle,draw,inner sep=2pt,red,fill=red!20,text=black] (char) {#1};}%
}

% Define custom command \strikeh that just puts red text on the 2nd argument
\newcommand{\strikeh}[2]{\textcolor{red}{#2}}

% Define custom command \strikev that just puts red text on the 2nd argument
\newcommand{\strikev}[2]{\textcolor{red}{#2}}

%more new commands for this doc for errors in copying
\newcommand{\SI}{\text{SI}}
\newcommand{\kg}{\text{kg}}
\newcommand{\m}{\text{m}}
\newcommand{\s}{\text{s}}
\newcommand{\norm}[1]{\left\|#1\right\|}
\newcommand{\col}{\text{col}}
\newcommand{\sspan}{\text{span}}
\newcommand{\proj}{\text{proj}}
\newcommand{\set}[1]{\left\{#1\right\}}
\newcommand{\degC}{^\circ\text{C}}
\newcommand{\centroid}[1]{\overline{#1}}
\newcommand{\dotprod}{\boldsymbol{\cdot}}
%\newcommand{\coord}[1]{\begin{bmatrix}#1\end{bmatrix}}
\newcommand{\iprod}[1]{\langle #1 \rangle}
\newcommand{\adjoint}{^{*}}
\newcommand{\conjugate}[1]{\overline{#1}}
\newcommand{\eigenvarA}{\lambda}
\newcommand{\eigenvarB}{\mu}
\newcommand{\orth}{\perp}
\newcommand{\bigbracket}[1]{\left[#1\right]}
\newcommand{\textiff}{\text{ if and only if }}
\newcommand{\adj}{\text{adj}}
\newcommand{\ijth}{\emph{ij}^\text{th}}
\newcommand{\minor}[2]{M_{#2}}
\newcommand{\cofactor}{\text{C}}
\newcommand{\shift}{\textbf{shift}}
\newcommand{\startmat}[1]{
  \left[\begin{array}{#1}
}
\newcommand{\stopmat}{\end{array}\right]}
%a command to give a name to explorations and hints and theorems
\newcommand{\name}[1]{\begin{centering}\textbf{#1}\end{centering}}
\newcommand{\vect}[1]{\vec{#1}}
\newcommand{\dfn}[1]{\textbf{#1}}
\newcommand{\transpose}{\mathsf{T}}
\newcommand{\mtlb}[2][black]{\texttt{\textcolor{#1}{#2}}}
\newcommand{\RR}{\mathbb{R}} % Real numbers
\newcommand{\id}{\text{id}}
\newcommand{\coord}[1]{\langle#1\rangle}
\newcommand{\RREF}{\text{RREF}}
\newcommand{\Null}{\text{Null}}
\newcommand{\Nullity}{\text{Nullity}}
\newcommand{\Rank}{\text{Rank}}
\newcommand{\Col}{\text{Col}}
\newcommand{\Ef}{\text{EF}}
\newcommand{\boxprod}[3]{\abs{(#1\times#2)\cdot#3}}

\author{Zack Reed}
%borrowed from selinger linear algebra
\title{Application: Cryptography}
\begin{document}
\begin{abstract}

\end{abstract}
\maketitle


\section*{Application: Cryptography}

We've spent much time already talking about how matrices encode linear transformations. That is, with matrices we can transform space under certain operations such as rotation, stretching, scaling, etc. We're now going to focus on when the underlying matrix transformations are reversible. That is, if we use $A$ to change a vector $\vec{x}$ into a vector $\vec{b}$ by the multiplication $A\vec{x}=\vec{b}$, can we find a new matrix $M$ to reconstruct $\vec{x}$ from $\vec{b}$? 

That is, if we know $A$ and $\vec{b}$, and that $A\vec{x}=\vec{b}$, is there a matrix $M$ such that $\vec{x}=M\vec{b}$? If so, we'd call $M$ the \emph{inverse} of $A$, and use the notation $A^{-1}$.

We will begin these questions with a motivating example from cryptography, where one might use a matrix $A$ to encode a message such as ``Linear Algebra is Cool'' into an otherwise unintelligible string ``JPAPWWASMO,FS,AMWHC,S .A''. Using a basic numeric encoding of the alphabet (see below), you might take the original message ``Linear Algebra is Cool'' to a vector $\vec{x}$ and then the encrypted message ``JPAPWWASMO,FS,AMWHC,S .A'' as the output vector $A\vec{x}=\vec{b}$.

The goal of a cryptographer (a decoder of these messages) (perhaps an eavesdropping spy who wants to gain an advantage by understanding secretive communication!) is to determine the matrix $M$ by which the multiplication $M\vec{b}$ of the encrpted message $\vec{b}$ becomes the original message $\vec{x}$.

We'll call the original message the
\textbf{plaintext} and the encrypted message is called the
\textbf{ciphertext}. The process of turning a plaintext into the
corresponding ciphertext is called \textbf{encryption}, and the process of turning a ciphertext
into the corresponding plaintext is called \textbf{decryption}. 

An encryption and decryption method is also
called a \textbf{cipher}. A cipher should be designed so that decryption is
easy for a person who knows the key, but difficult for everybody
else. The art of designing ciphers is called \textbf{cryptography}, and the art of breaking ciphers is called
\textbf{cryptanalysis}.

In order to be able to define ciphers using algebraic operations, we
start by encoding strings as sequences of numbers. To that end, we
assign a number to each letter of the alphabet, as well as the special
symbols ``space'', ``comma'', and ``period'', according to the
following scheme.
\begin{center}
  \tabcolsep=2.5ex
  \begin{tabular}{|c|c|c|c|c|c|c|c|c|}
    \hline
    Space & \qq{A} & \qq{B} & \qq{C} & \qq{D} & \ldots & \qq{Z} & Comma & Period \\\hline
    0 & 1 & 2 & 3 & 4 & \ldots & 26 & 27 & 28 \\\hline
  \end{tabular}
\end{center}

In practical applications, one would probably use a larger set of
symbols.



\begin{example}\name{Representing strings as sequences of numbers}\label{ex:string-encoding}
  Convert the string ``Attack at dawn'' to a sequence of
  numbers. 
  
  Convert the sequence of numbers
  $9,0,12,9,11,5,0,3,15,4,5,19,28$ to a string.



\begin{solution}
  We have $\qq{A}=1$, $\qq{T}=20$, $\qq{T}=20$, $\qq{A}=1$, $\qq{C}=3$,
  $\qq{K}=11$, $\mbox{Space}=0$, and so on. Continuing in this way, the
  encoding of ``Attack at dawn'' is
  $1,\answer{20},20,1,3,\answer{11},0,1,20,0,\answer{4},1,\answer{23},14$.

  The course function $\texttt{linalg.string\_to\_vector}$ will do this for you.
  
  Conversely, we have $9=\qq{I}$,
  $0=\mbox{Space}$, $12=\qq{L}$, $9=\qq{I}$, $11=\qq{K}$, $5=\qq{E}$, and so
  on. We find that the decoded string is ``$\answer[format=string]{I LIKE CODES.}$''

  The course function $\texttt{linalg.vectors\_to\_string}$ will do this for you.
\end{solution}

\end{example}


There are many different ways to define ciphers. Some of the oldest
known ciphers date back thousands of years.  Modern ciphers are designed
to satisfy a property called \textbf{diffusion}: changing one letter of the plaintext should
change many letters of the ciphertext. We will examine one such diffusive cipher, called a ``Hill cipher'', which uses linear algebra for the encryption process.

\begin{definition}\name{Hill cipher}\label{def:hill-cipher}
  The \textbf{Hill cipher} of block size $n$ has as its key an
  invertible $n\times n$-matrix $A$ with scalars from $[1, 2, \ldots, 27, 28]$. Each
  ciphertext block $c_1,\ldots,c_n$ is computed from the corresponding
  plaintext block $p_1,\ldots,p_n$ by matrix multiplication:
  \begin{equation*}
    \startmat{c}c_1\\\vdots\\c_n\stopmat
    ~=~ A\startmat{c}p_1\\\vdots\\p_n\stopmat.
  \end{equation*}
  The matrix $A$ is called the \textbf{encryption matrix} of the cipher.

  \begin{remark}
    We need to define $A$ such that we always get integers back, so rotations or fractional scalings won't work. Also, after matrix multiplication some numbers may be outside of the range $[1, 2, \ldots, 27, 28]$. The course function $\texttt{linalg.string\_to\_vector}$ accounts for this through a process called ``modular arithmetic'', where $k$ is converted to the remainder $r$ in the formula $k=29*i+r$ for some suitable $i$. We won't worry about this too much, but modding often comes up in computer applications.
  \end{remark}
\end{definition}

\begin{example}\name{Hill cipher: encryption}\label{ex:hill-cipher-encryption}
  Encrypt the message ``Meet me tomorrow'' using the Hill cipher with
  block size $3$ and encryption matrix
  \begin{equation*}
    A ~=~ \startmat{ccc}
      2 & 4 & 1 \\
      3 & 1 & 5 \\
      1 & 3 & 2 \\
    \stopmat.
  \end{equation*}


\begin{solution}
  We start by converting the message ``Meet me tomorrow'' to a sequence
  of scalars. We have $\qq{M}=13$, $\qq{E}=5$, and so on. 
  
  If we start with the string and then convert to a vector,
  
  \begin{verbatim}
    phrase="Meet me tomorrow"
    plaintext=linalg.string_to_vector(phrase)
  \end{verbatim}
  
  The encoded
  plaintext is $13,5,5,\answer{20},0,13,\answer{5},0,20,\answer{15},13,15,18,18,15,\answer{23}$. 
  
  Next, we
  divide the plaintext into blocks of length 3. Since the length of
  the plaintext is not a multiple of three, we pad the final block
  with spaces, i.e., with zeros.
  \begin{equation*}
    \mbox{Plaintext blocks:}\quad
    (13,5,5),\
    (20,0,13),\
    (5,0,20),\
    (15,13,15),\
    (18,18,15),\
    (23,0,0).
  \end{equation*}

  This can be automatically done with the course function $\texttt{linalg.split\_vector(plaintext,3)}$.

  To compute the ciphertext, we regard each plaintext block as a
  $3$-dimensional column vector and multiply by the encryption matrix
  $A$. 
  
  For example, for the
  first block, we have
  \begin{equation*}
    A \startmat{c} 13 \\ 5 \\ 5 \stopmat
    ~=~ \startmat{ccc}
      2 & 4 & 1 \\
      3 & 1 & 5 \\
      1 & 3 & 2 \\
    \stopmat
    \startmat{c} 13 \\ 5 \\ 5 \stopmat
    ~=~ \startmat{c} 51 \\ 69 \\ 38  \stopmat,
  \end{equation*}
  so the first ciphertext block is $(51 ,69 ,38 )$. We repeat the same
  with the remaining plaintext blocks.

  This can be done succinctly with vector-matrix multiplication via the following code:

  \begin{verbatim}
    plain_text_blocks=linalg.split_vector(plaintext,3)
    cipher_blocks=A*plain_text_blocks
  \end{verbatim}


  \begin{align*}
    &A \startmat{c} 20 \\ 0 \\ 13 \stopmat
    = \startmat{c} 53\\\answer{125}\\46 \stopmat,
    \quad
    A \startmat{c} 5 \\ 0 \\ 20 \stopmat
    = \startmat{c} 30\\115\\\answer{45} \stopmat,
    \quad
    A \startmat{c} 15 \\ 13 \\ 15 \stopmat
    = \startmat{c} 97\\\answer{133}\\84 \stopmat,
    \\\\[-2ex]
    &A \startmat{c} 18 \\ 18 \\ 15 \stopmat
    = \startmat{c} \answer{123}\\147\\102 \stopmat,
    \quad
    A \startmat{c} 23 \\ 0 \\ 0 \stopmat
    = \startmat{c} 46\\69\\\answer{32} \stopmat.
  \end{align*}

  Note that the ciphertext vectors are not numbers between $0$ and $28$, as required to convert back to letters, but the course function $\texttt{linalg.vectors\_to\_string}$ will perform this modular arithmetic for us.
  
  Therefore, we convert the ciphertext using 

  \begin{verbatim}
    linalg.vectors_to_string(cipher_blocks)
  \end{verbatim}
  
  to yield a list of symbols
  $\answer[format=string]{VKIXIQA.PJQZGBOQKW}$.
\end{solution}
\end{example}

We've seen that encoding a message using a Hill cipher entails using a matrix to transform the plaintext blocks in space. \emph{Decrypting} the message thus requires the reverse transformation. Said differently, if the encryption of a message with plaintext vectors $\lbrace \vec{x}\rbrace$ produces ciphertext vectors $\lbrace \vec{b}\rbrace$ through matrix multiplication $A\vec{x}=\vec{b}$, then to decript the message we need to reverse the transformation with a matrix $M$ to produce $\vec{x}=M\vec{b}$.

\begin{example}
  Suppose the encryption matrix re-ordered the coordinate axes according to the matrix $A=\begin{bmatrix}
    1&0&0\\0&0&1\\0&1&0
  \end{bmatrix}$, producing the ciphertext "UES HTEF OCRE  ", find the transformation-reversing matrix $M$ and decript the message.
\end{example}

\begin{solution}
  The matrix $A$ swaps the ordering of the 2nd and 3rd unit vectors, effectively switching $y$ and $z$. If you've guessed the message just by looking at the ciphertext, you can intuit this effect in the message itself.

  The ciphertext is mapped by the vectors $\begin{bmatrix}
    21 \\5\\19\end{bmatrix}$, $\begin{bmatrix}
    0 \\8\\20 \end{bmatrix}$, $\begin{bmatrix}
    5\\6\\0\end{bmatrix}$, $\begin{bmatrix}
    15\\3\\18\end{bmatrix}$, and $\begin{bmatrix}
    5\\0\\0\end{bmatrix}$.

  We reverse the transformation of $A$ by again swapping the order of the 2nd and 3rd axes, so $A$ itself is actually the decryption matrix.

  If we apply $A$ to the ciphertext vectors (which can be succinctly done by storing the vectors all as a matrix) we get the plaintext blocks 

  $\begin{bmatrix}
    21 \\\answer{19}\\\answer{5}\end{bmatrix}$, $\begin{bmatrix}
    0 \\\answer{20}\\8\end{bmatrix}$, $\begin{bmatrix}
    5\\0\\\answer{6}\end{bmatrix}$, $\begin{bmatrix}
    15\\\answer{18}\\\answer{3}\end{bmatrix}$, and $\begin{bmatrix}
    5\\0\\0\end{bmatrix}$.

  These vectors (run through the course function $\texttt{linalg.vectors\_to\_string})$ produce the original phrase (input all caps and don't enter the spaces at the end) ``$\answer[format=string]{USE THE FORCE}$''.

\end{solution}

This process of transformation reversal is called \emph{finding the inverse} of matrix $A$, and is the focus of this chapter. You'll learn what inverse matrices are, learn their properties, and begin to learn about ways that we can tell whether or not matrices even have inverses.

\end{document}