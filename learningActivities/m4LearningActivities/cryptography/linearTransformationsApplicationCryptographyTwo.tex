\documentclass{ximera}
\graphicspath{  %% When looking for images,
{./}            %% look here first,
{./pictures/}   %% then look for a pictures folder,
{../pictures/}  %% which may be a directory up.
{../../pictures/}  %% which may be a directory up.
{../../../pictures/}  %% which may be a directory up.
{../../../../pictures/}  %% which may be a directory up.
}

\usepackage{listings}
%\usepackage{circuitikz}
\usepackage{xcolor}
\usepackage{amsmath,amsthm}
\usepackage{subcaption}
\usepackage{graphicx}
\usepackage{tikz}
%\usepackage{tikz-3dplot}
\usepackage{amsfonts}
%\usepackage{mdframed} % For framing content
%\usepackage{tikz-cd}

  \renewcommand{\vector}[1]{\left\langle #1\right\rangle}
  \newcommand{\arrowvec}[1]{{\overset{\rightharpoonup}{#1}}}
  \newcommand{\ro}{\texttt{R}}%% row operation
  \newcommand{\dotp}{\bullet}%% dot product
  \renewcommand{\l}{\ell}
  \let\defaultAnswerFormat\answerFormatBoxed
  \usetikzlibrary{calc,bending}
  \tikzset{>=stealth}
  




%make a maroon color
\definecolor{maroon}{RGB}{128,0,0}
%make a dark blue color
\definecolor{darkblue}{RGB}{0,0,139}
%define the color fourier0 to be the maroon color
\definecolor{fourier0}{RGB}{128,0,0}
%define the color fourier1 to be the dark blue color
\definecolor{fourier1}{RGB}{0,0,139}
%define the color fourier 1t to be the light blue color
\definecolor{fourier1t}{RGB}{173,216,230}
%define the color fourier2 to be the dark green color
\definecolor{fourier2}{RGB}{0,100,0}
%define teh color fourier2t to be the light green color
\definecolor{fourier2t}{RGB}{144,238,144}
%define the color fourier3 to be the dark purple color
\definecolor{fourier3}{RGB}{128,0,128}
%define the color fourier3t to be the light purple color
\definecolor{fourier3t}{RGB}{221,160,221}
%define the color fourier0t to be the red color
\definecolor{fourier0t}{RGB}{255,0,0}
%define the color fourier4 to be the orange color
\definecolor{fourier4}{RGB}{255,165,0}
%define the color fourier4t to be the darker orange color
\definecolor{fourier4t}{RGB}{255,215,0}
%define the color fourier5 to be the yellow color
\definecolor{fourier5}{RGB}{255,255,0}
%define the color fourier5t to be the darker yellow color
\definecolor{fourier5t}{RGB}{255,255,100}
%define the color fourier6 to be the green color
\definecolor{fourier6}{RGB}{0,128,0}
%define the color fourier6t to be the darker green color
\definecolor{fourier6t}{RGB}{0,255,0}

%New commands for this doc for errors in copying
\newcommand{\eigenvar}{\lambda}
%\newcommand{\vect}[1]{\mathbf{#1}}
\renewcommand{\th}{^{\text{th}}}
\newcommand{\st}{^{\text{st}}}
\newcommand{\nd}{^{\text{nd}}}
\newcommand{\rd}{^{\text{rd}}}
\newcommand{\paren}[1]{\left(#1\right)}
\newcommand{\abs}[1]{\left|#1\right|}
\newcommand{\R}{\mathbb{R}}
\newcommand{\C}{\mathbb{C}}
\newcommand{\Hilb}{\mathbb{H}}
\newcommand{\qq}[1]{\text{#1}}
\newcommand{\Z}{\mathbb{Z}}
\newcommand{\N}{\mathbb{N}}
\newcommand{\q}[1]{\text{``#1''}}
%\newcommand{\mat}[1]{\begin{bmatrix}#1\end{bmatrix}}
\newcommand{\rref}{\text{reduced row echelon form}}
\newcommand{\ef}{\text{echelon form}}
\newcommand{\ohm}{\Omega}
\newcommand{\volt}{\text{V}}
\newcommand{\amp}{\text{A}}
\newcommand{\Seq}{\textbf{Seq}}
\newcommand{\Poly}{\textbf{P}}
\renewcommand{\quad}{\text{    }}
\newcommand{\roweq}{\simeq}
\newcommand{\rowop}{\simeq}
\newcommand{\rowswap}{\leftrightarrow}
\newcommand{\Mat}{\textbf{M}}
\newcommand{\Func}{\textbf{Func}}
\newcommand{\Hw}{\textbf{Hamming weight}}
\newcommand{\Hd}{\textbf{Hamming distance}}
\newcommand{\rank}{\text{rank}}
\newcommand{\longvect}[1]{\overrightarrow{#1}}
% Define the circled command
\newcommand{\circled}[1]{%
  \tikz[baseline=(char.base)]{
    \node[shape=circle,draw,inner sep=2pt,red,fill=red!20,text=black] (char) {#1};}%
}

% Define custom command \strikeh that just puts red text on the 2nd argument
\newcommand{\strikeh}[2]{\textcolor{red}{#2}}

% Define custom command \strikev that just puts red text on the 2nd argument
\newcommand{\strikev}[2]{\textcolor{red}{#2}}

%more new commands for this doc for errors in copying
\newcommand{\SI}{\text{SI}}
\newcommand{\kg}{\text{kg}}
\newcommand{\m}{\text{m}}
\newcommand{\s}{\text{s}}
\newcommand{\norm}[1]{\left\|#1\right\|}
\newcommand{\col}{\text{col}}
\newcommand{\sspan}{\text{span}}
\newcommand{\proj}{\text{proj}}
\newcommand{\set}[1]{\left\{#1\right\}}
\newcommand{\degC}{^\circ\text{C}}
\newcommand{\centroid}[1]{\overline{#1}}
\newcommand{\dotprod}{\boldsymbol{\cdot}}
%\newcommand{\coord}[1]{\begin{bmatrix}#1\end{bmatrix}}
\newcommand{\iprod}[1]{\langle #1 \rangle}
\newcommand{\adjoint}{^{*}}
\newcommand{\conjugate}[1]{\overline{#1}}
\newcommand{\eigenvarA}{\lambda}
\newcommand{\eigenvarB}{\mu}
\newcommand{\orth}{\perp}
\newcommand{\bigbracket}[1]{\left[#1\right]}
\newcommand{\textiff}{\text{ if and only if }}
\newcommand{\adj}{\text{adj}}
\newcommand{\ijth}{\emph{ij}^\text{th}}
\newcommand{\minor}[2]{M_{#2}}
\newcommand{\cofactor}{\text{C}}
\newcommand{\shift}{\textbf{shift}}
\newcommand{\startmat}[1]{
  \left[\begin{array}{#1}
}
\newcommand{\stopmat}{\end{array}\right]}
%a command to give a name to explorations and hints and theorems
\newcommand{\name}[1]{\begin{centering}\textbf{#1}\end{centering}}
\newcommand{\vect}[1]{\vec{#1}}
\newcommand{\dfn}[1]{\textbf{#1}}
\newcommand{\transpose}{\mathsf{T}}
\newcommand{\mtlb}[2][black]{\texttt{\textcolor{#1}{#2}}}
\newcommand{\RR}{\mathbb{R}} % Real numbers
\newcommand{\id}{\text{id}}
\newcommand{\coord}[1]{\langle#1\rangle}
\newcommand{\RREF}{\text{RREF}}
\newcommand{\Null}{\text{Null}}
\newcommand{\Nullity}{\text{Nullity}}
\newcommand{\Rank}{\text{Rank}}
\newcommand{\Col}{\text{Col}}
\newcommand{\Ef}{\text{EF}}
\newcommand{\boxprod}[3]{\abs{(#1\times#2)\cdot#3}}

\author{Zack Reed}
%borrowed from selinger linear algebra
\title{Application: Return to Cryptography}
\begin{document}
\begin{abstract}

\end{abstract}
\maketitle


\section*{A Return to Cryptography}

Now that we've been introduced to matrix inverses and have obtained some initial means of finding them, we can return to the application of cryptography and see why the Hill cipher is not a secure means of encripting messages.

The way that one decrypts an encrypted message in the case of Hill ciphers is by applying the inverse transformation to the ciphertext. From a linear algebra perspective this makes sense. The cipher enacts a transformation the plaintext vectors, and so doing the reverse transformation (i.e. applying the inverse matrix) to the ciphertext should return the plaintext.

\begin{center}
  \youtube{ci-bG7a01f4}
\end{center}

\begin{example}\name{Hill cipher: decryption}\label{ex:hill-cipher-decryption}

  Decrypt the message $\q{RNOLFPHHCIGH DE}$ using the Hill cipher with
  block size $3$ and encryption matrix
  \begin{equation*}
    A ~=~ \startmat{ccc}
      2 & 4 & 1 \\
      3 & 1 & 5 \\
      1 & 3 & 2 \\
    \stopmat.
  \end{equation*}

  Solution:

  The process is analogous to encryption, except that we need to use
  the decryption matrix $A^{-1}$ instead of $A$. We first calculate
  $A^{-1}$.
  The method is the same as in the previous section. We first find the $rref(A|I)$
  \begin{equation*}
    \startmat{c}A\mid I\stopmat
    ~=~
    \startmat{ccc|ccc}
      2 & 4 & 1  &  1 & 0 & 0 \\
      3 & 1 & 5  &  0 & 1 & 0 \\
      1 & 3 & 2  &  0 & 0 & 1 \\
    \stopmat
    ~\simeq~\ldots~\simeq~
    \startmat{ccc|ccc}
    1 & 0 & 0  & \answer[tolerance=.01]{0.5909} & \answer[tolerance=.01]{0.2273} & \answer[tolerance=.01]{-0.8636} \\
    0 & 1 & 0  &   \answer[tolerance=.01]{0.0455} & \answer[tolerance=.01]{-0.1364} & \answer[tolerance=.01]{0.3182} \\
    0 & 0 & 1 & \answer[tolerance=.01]{-0.3636} & \answer[tolerance=.01]{0.0909} & \answer[tolerance=.01]{0.4545}
    \stopmat
  \end{equation*}

  Using the course function \texttt{linalg.mod\_p(M,29)}, (where $M$ is the augmented matrix), we get the following augmented matrix that converts the inverse to the numbers $0$ to $28$ (needed for the cipher).

  This yields:

  $\startmat{ccc|ccc}
      1 & 0 & 0  &  23 & 20 & 11 \\
      0 & 1 & 0  &   4 & 17 & 28 \\
      0 & 0 & 1  &  26 &  8 & 11 \\
    \stopmat
    ~=~
    \startmat{c}I\mid A^{-1}\stopmat.$

  Next, we convert the 15 ciphertext symbols $\q{RNOLFPHHCIGH DE}$ to
  scalars and divide them into blocks of length 3:
  \begin{equation*}
    \mbox{Ciphertext blocks:}\quad
    (18,14,15),\
    (12,6,16),\
    (8,8,3),\
    (9,7,8),\
    (0,4,5).
  \end{equation*}
  Now we decrypt each ciphertext block by a matrix multiplication
  with $A^{-1}$. The symbol $\stackrel{29}{=}$ before the final vector in each calculation indicates that we have converted the vector back to be between $0$ and $28$ (i.e. mod 29).

  \begin{hint}
    For efficient decryption, just write out one matrix $C$ with columns as the ciphertext and then multiply $C$ by $A^{-1}$. 
  \end{hint}

  \begin{align*}
    &A^{-1} \startmat{c} 18 \\ 14 \\ 15 \stopmat
    = \begin{bmatrix}
      \answer{859} \\
      \answer{730} \\
      \answer{745}
      \end{bmatrix}\stackrel{29}{=}\startmat{c} 18 \\ 5 \\ 20 \stopmat,
    \quad
    A^{-1} \startmat{c} 12 \\ 6 \\ 16 \stopmat
    = \begin{bmatrix}
      572 \\
      598 \\
      536
      \end{bmatrix}\stackrel{29}{=}\startmat{c} 21 \\ 18 \\ 14 \stopmat,
    \quad
    A^{-1} \startmat{c} 8 \\ 8 \\ 3 \stopmat
    = \begin{bmatrix}
      \answer{377} \\
      \answer{252} \\
      \answer{305}
      \end{bmatrix}\stackrel{29}{=}\startmat{c} 0 \\ 20 \\ 15 \stopmat,
    \\\\[-2ex]
    &A^{-1} \startmat{c} 9 \\ 7 \\ 8 \stopmat
    = \begin{bmatrix}
      435 \\
      379 \\
      378
      \end{bmatrix}\stackrel{29}{=}\startmat{c} 0 \\ 2 \\ 1 \stopmat,
    \quad
    A^{-1} \startmat{c} 0 \\ 4 \\ 5 \stopmat
    = \begin{bmatrix}
      \answer{135} \\
      \answer{208} \\
      \answer{87}
      \end{bmatrix}\stackrel{29}{=}\startmat{c} 19 \\ 5 \\ 0 \stopmat.
  \end{align*}

  Converting these plaintext blocks to letters, and omitting the trailing space (using the course function \texttt{linalg.vectors\_to\_string}),
  we find that the plaintext (in all caps) is ``$\answer[format=string]{RETURN TO BASE}$''.
\end{example}

\section*{Breaking Hill Ciphers}

Despite its complexity, Hill ciphers can be easily broken with relatively little information. 

For
instance, because $A\vect{0}=\vect{0}$, a block of spaces in the plaintext
will always be encrypted as a block of spaces in the ciphertext,
regardless of the encryption matrix $A$. 

More importantly, if an eavesdropper intercepts
some ciphertext for which a small amount of the corresponding
plaintext happens to be known, it is possible to recover
the key and therefore decrypt the rest of the ciphertext. 

\begin{center}
  \youtube{nS8KcjnqtA0}
\end{center}

\begin{example}\name{Cryptanalysis of the Hill cipher: known plaintext attack}\label{ex:hill-cipher-cryptanalysis}
  Eve intercepts the following encrypted message sent by Alice:%
  \index{cryptanalysis}
  \begin{center}
    \q{EFNOR.AHIFNEPL.TSZS,RSKT.ZBBRFVUPFVZLFHNTV}.
  \end{center}
  Eve knows that Alice uses a Hill cipher with block length 3, but she
  does not know the secret encryption matrix. 
  
  Eve also knows that
  Alice begins all of her correspondence with ``My dear
  love''. 
  
  Decrypt the message.

  Solution:

  The first three blocks of the ciphertext are $\q{EFNOR.AHI}$, i.e.,
  \begin{equation*}
    \mbox{Ciphertext blocks:}\quad
    (5,6,14),\
    (15,18,28),\
    (1,8,9).
  \end{equation*}

  Eve also knows that the first three blocks of the plaintext are
  $\q{MY DEAR L}$, i.e.,

  \begin{equation*}
    \mbox{Plaintext blocks:}\quad
    (13,25,0),\
    (4,5,1),\
    (18,0,12).
  \end{equation*}

  These facts allow Eve to deduce the following information about the
  unknown decryption matrix $A^{-1}$:
  \begin{equation*}
    A^{-1} \startmat{c} 5 \\ 6 \\ 14 \stopmat
    = \startmat{c} 13 \\ 25 \\ 0 \stopmat,\quad
    A^{-1} \startmat{c} 15 \\ 18 \\ 28 \stopmat
    = \startmat{c} 4 \\ 5 \\ 1 \stopmat,\quad
    A^{-1} \startmat{c} 1 \\ 8 \\ 9 \stopmat
    = \startmat{c} 18 \\ 0 \\ 12 \stopmat.
  \end{equation*}

  Because multiplication of $A^{-1}$ by the matrix 
  
  \begin{equation*}
    \startmat{ccc}
      5 & 15 & 1 \\
      6 & 18 & 8 \\
      14 & 28 & 9 \\
    \stopmat
  \end{equation*}

  is the same as multiplication of each column vector by $A^{-1}$, we now have a matrix equation with two known matrices and one unkown matrix:
  
  \begin{equation*}
    A^{-1} \startmat{ccc}
      5 & 15 & 1 \\
      6 & 18 & 8 \\
      14 & 28 & 9 \\
    \stopmat
    = \startmat{ccc}
      13 & 4 & 18 \\
      25 & 5 & 0 \\
      0 & 1 & 12 \\
    \stopmat.\quad
  \end{equation*}

  We can write this more succinctly as $A^{-1}C=P$. ($C$
 for ``ciphertext'' and $P$ for ``plaintext''). 
 
  So long as $C$ is invertible, we can right-multiply 
  both sides of the equation by $C^{-1}$. This solves for $A^{-1}$ to get
  $A^{-1} = PC^{-1}$. Thus, assuming that $C$ is invertible, Eve can
  easily compute the decryption matrix $A^{-1}$. 

  Let's check if $C$ is invertible. If it is, then we can proceed to solve for $A^{-1}$ and break the encryption. If $C$ is not invertible, then \wordChoice{
    \choice{the encryption is unbreakable.}
    \choice{Eve can proceed anyways if she find $rref(C|I)$.}
    \choice[correct]{Eve will need to find other plaintext/ciphertext matches until the ciphertext matrix is invertible.}
  }

  $C$ is the matrix $\startmat{ccc}
  5 & 15 & 1 \\
  6 & 18 & 8 \\
  14 & 28 & 9 \\
\stopmat$, and the $rref(C|I)$ is $\begin{bmatrix}
  1&0&0&\answer[tolerance=.01]{-0.1303} & \answer[tolerance=.01]{-0.2248} & \answer[tolerance=.01]{0.2143} \\
  0&1&0 & \answer[tolerance=.01]{0.1218} & \answer[tolerance=.01]{0.0651} & \answer[tolerance=.01]{-0.0714} \\
  0&0&1 & \answer[tolerance=.01]{-0.1765} & \answer[tolerance=.01]{0.1471} & \answer{0}
  \end{bmatrix}$.

  Thus, $C$ is invertible. Using \texttt{linalg.mod\_p} (with the 2nd argument 29) on the found inverse, we get.
  
  \begin{equation*}
    C^{-1}
    =
    \startmat{ccc}
      19 & 8 & 23 \\
      0 & 5 & 2 \\
      22 & 1 & 0 \\
    \stopmat.
  \end{equation*}
  This allows Eve to compute the decryption matrix (do the matrix multiplication and then don't forget to use \texttt{linalg.mod\_p(M,29)}, where $M$ is the resulting product.):
  \begin{equation*}
    A^{-1}
    =
    PC^{-1}
    =
    \startmat{ccc}
      13 & 4 & 18 \\
      25 & 5 & 0 \\
      0 & 1 & 12 \\
    \stopmat
    \startmat{ccc}
      19 & 8 & 23 \\
      0 & 5 & 2 \\
      22 & 1 & 0 \\
    \stopmat
    =
    \startmat{ccc}
    \answer{5} & \answer{26} & \answer{17} \\
    \answer{11} & \answer{22} & \answer{5} \\
    \answer{3} & \answer{17} & \answer{2}
    \stopmat.
  \end{equation*}

  Armed with the decryption matrix $A^{-1}$, Eve can now decrypt
  Alice's entire message. The plaintext is (in all caps) ``$\answer[format=string]{MY DEAR LOVE, RUN AWAY WITH ME AT MIDNIGHT}$''.
\end{example}

As the example shows, the Hill cipher is not secure at all. The main
problem is that the cipher is {\em linear}, i.e., each component of a
ciphertext block is a simple linear combination of the components of
the plaintext block. This linearity property enables Eve to break the
cipher by solving a system of linear equations.

For this reason, all modern block ciphers have a non-linear
component. Often this takes the form of so-called \textbf{S-boxes}. An S-box is an operation that scrambles the
symbols of the alphabet in a non-linear way. Consider
the following S-box:
\begin{center}
  \tabcolsep=0.4ex\def\arraystretch{1.4}
  \begin{tabular}{|c|c|c|c|c|c|c|c|c|c|c|c|c|c|c|c|c|c|c|c|c|c|c|c|c|c|c|c|c|c|}
    \hline
    $x$ & 0 & 1 & 2 & 3 & 4 & 5 & 6 & 7 & 8 & 9 & 10 & 11 & 12 & 13 & 14 & 15 & 16 & 17 & 18 & 19 & 20 & 21 & 22 & 23 & 24 & 25 & 26 & 27 & 28 \\\hline
    ~$S(x)$~ & 17 & 9 & 27 & 2 & 20 & 12 & 21 & 26 & 16 & 18 & 4 & 24 & 23 & 7 & 19 & 14 & 28 & 29 & 1 & 15 & 10 & 22 & 6 & 5 & 25 & 11 & 13 & 3 & 8 \\\hline
  \end{tabular}
\end{center}
The inputs of the S-box are shown in the top row, and the
corresponding outputs in the bottom row.  For example, this S-box maps
the input $7$ to the output $26$. We write $S(7)=26$.

In the real world, ciphers incorporate diffusion (multiplication by a matrix), the use of an $S$-box, and other similar steps in combination to produce more secure systems of encryption. 

Like much of what you'll encounter in this course, linear algebra is often a fundamental aspect of many applications but does not typically constitute the entirety of the application.

\end{document}