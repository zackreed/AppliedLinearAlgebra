\documentclass{ximera}
\graphicspath{  %% When looking for images,
{./}            %% look here first,
{./pictures/}   %% then look for a pictures folder,
{../pictures/}  %% which may be a directory up.
{../../pictures/}  %% which may be a directory up.
{../../../pictures/}  %% which may be a directory up.
{../../../../pictures/}  %% which may be a directory up.
}

\usepackage{listings}
%\usepackage{circuitikz}
\usepackage{xcolor}
\usepackage{amsmath,amsthm}
\usepackage{subcaption}
\usepackage{graphicx}
\usepackage{tikz}
%\usepackage{tikz-3dplot}
\usepackage{amsfonts}
%\usepackage{mdframed} % For framing content
%\usepackage{tikz-cd}

  \renewcommand{\vector}[1]{\left\langle #1\right\rangle}
  \newcommand{\arrowvec}[1]{{\overset{\rightharpoonup}{#1}}}
  \newcommand{\ro}{\texttt{R}}%% row operation
  \newcommand{\dotp}{\bullet}%% dot product
  \renewcommand{\l}{\ell}
  \let\defaultAnswerFormat\answerFormatBoxed
  \usetikzlibrary{calc,bending}
  \tikzset{>=stealth}
  




%make a maroon color
\definecolor{maroon}{RGB}{128,0,0}
%make a dark blue color
\definecolor{darkblue}{RGB}{0,0,139}
%define the color fourier0 to be the maroon color
\definecolor{fourier0}{RGB}{128,0,0}
%define the color fourier1 to be the dark blue color
\definecolor{fourier1}{RGB}{0,0,139}
%define the color fourier 1t to be the light blue color
\definecolor{fourier1t}{RGB}{173,216,230}
%define the color fourier2 to be the dark green color
\definecolor{fourier2}{RGB}{0,100,0}
%define teh color fourier2t to be the light green color
\definecolor{fourier2t}{RGB}{144,238,144}
%define the color fourier3 to be the dark purple color
\definecolor{fourier3}{RGB}{128,0,128}
%define the color fourier3t to be the light purple color
\definecolor{fourier3t}{RGB}{221,160,221}
%define the color fourier0t to be the red color
\definecolor{fourier0t}{RGB}{255,0,0}
%define the color fourier4 to be the orange color
\definecolor{fourier4}{RGB}{255,165,0}
%define the color fourier4t to be the darker orange color
\definecolor{fourier4t}{RGB}{255,215,0}
%define the color fourier5 to be the yellow color
\definecolor{fourier5}{RGB}{255,255,0}
%define the color fourier5t to be the darker yellow color
\definecolor{fourier5t}{RGB}{255,255,100}
%define the color fourier6 to be the green color
\definecolor{fourier6}{RGB}{0,128,0}
%define the color fourier6t to be the darker green color
\definecolor{fourier6t}{RGB}{0,255,0}

%New commands for this doc for errors in copying
\newcommand{\eigenvar}{\lambda}
%\newcommand{\vect}[1]{\mathbf{#1}}
\renewcommand{\th}{^{\text{th}}}
\newcommand{\st}{^{\text{st}}}
\newcommand{\nd}{^{\text{nd}}}
\newcommand{\rd}{^{\text{rd}}}
\newcommand{\paren}[1]{\left(#1\right)}
\newcommand{\abs}[1]{\left|#1\right|}
\newcommand{\R}{\mathbb{R}}
\newcommand{\C}{\mathbb{C}}
\newcommand{\Hilb}{\mathbb{H}}
\newcommand{\qq}[1]{\text{#1}}
\newcommand{\Z}{\mathbb{Z}}
\newcommand{\N}{\mathbb{N}}
\newcommand{\q}[1]{\text{``#1''}}
%\newcommand{\mat}[1]{\begin{bmatrix}#1\end{bmatrix}}
\newcommand{\rref}{\text{reduced row echelon form}}
\newcommand{\ef}{\text{echelon form}}
\newcommand{\ohm}{\Omega}
\newcommand{\volt}{\text{V}}
\newcommand{\amp}{\text{A}}
\newcommand{\Seq}{\textbf{Seq}}
\newcommand{\Poly}{\textbf{P}}
\renewcommand{\quad}{\text{    }}
\newcommand{\roweq}{\simeq}
\newcommand{\rowop}{\simeq}
\newcommand{\rowswap}{\leftrightarrow}
\newcommand{\Mat}{\textbf{M}}
\newcommand{\Func}{\textbf{Func}}
\newcommand{\Hw}{\textbf{Hamming weight}}
\newcommand{\Hd}{\textbf{Hamming distance}}
\newcommand{\rank}{\text{rank}}
\newcommand{\longvect}[1]{\overrightarrow{#1}}
% Define the circled command
\newcommand{\circled}[1]{%
  \tikz[baseline=(char.base)]{
    \node[shape=circle,draw,inner sep=2pt,red,fill=red!20,text=black] (char) {#1};}%
}

% Define custom command \strikeh that just puts red text on the 2nd argument
\newcommand{\strikeh}[2]{\textcolor{red}{#2}}

% Define custom command \strikev that just puts red text on the 2nd argument
\newcommand{\strikev}[2]{\textcolor{red}{#2}}

%more new commands for this doc for errors in copying
\newcommand{\SI}{\text{SI}}
\newcommand{\kg}{\text{kg}}
\newcommand{\m}{\text{m}}
\newcommand{\s}{\text{s}}
\newcommand{\norm}[1]{\left\|#1\right\|}
\newcommand{\col}{\text{col}}
\newcommand{\sspan}{\text{span}}
\newcommand{\proj}{\text{proj}}
\newcommand{\set}[1]{\left\{#1\right\}}
\newcommand{\degC}{^\circ\text{C}}
\newcommand{\centroid}[1]{\overline{#1}}
\newcommand{\dotprod}{\boldsymbol{\cdot}}
%\newcommand{\coord}[1]{\begin{bmatrix}#1\end{bmatrix}}
\newcommand{\iprod}[1]{\langle #1 \rangle}
\newcommand{\adjoint}{^{*}}
\newcommand{\conjugate}[1]{\overline{#1}}
\newcommand{\eigenvarA}{\lambda}
\newcommand{\eigenvarB}{\mu}
\newcommand{\orth}{\perp}
\newcommand{\bigbracket}[1]{\left[#1\right]}
\newcommand{\textiff}{\text{ if and only if }}
\newcommand{\adj}{\text{adj}}
\newcommand{\ijth}{\emph{ij}^\text{th}}
\newcommand{\minor}[2]{M_{#2}}
\newcommand{\cofactor}{\text{C}}
\newcommand{\shift}{\textbf{shift}}
\newcommand{\startmat}[1]{
  \left[\begin{array}{#1}
}
\newcommand{\stopmat}{\end{array}\right]}
%a command to give a name to explorations and hints and theorems
\newcommand{\name}[1]{\begin{centering}\textbf{#1}\end{centering}}
\newcommand{\vect}[1]{\vec{#1}}
\newcommand{\dfn}[1]{\textbf{#1}}
\newcommand{\transpose}{\mathsf{T}}
\newcommand{\mtlb}[2][black]{\texttt{\textcolor{#1}{#2}}}
\newcommand{\RR}{\mathbb{R}} % Real numbers
\newcommand{\id}{\text{id}}
\newcommand{\coord}[1]{\langle#1\rangle}
\newcommand{\RREF}{\text{RREF}}
\newcommand{\Null}{\text{Null}}
\newcommand{\Nullity}{\text{Nullity}}
\newcommand{\Rank}{\text{Rank}}
\newcommand{\Col}{\text{Col}}
\newcommand{\Ef}{\text{EF}}
\newcommand{\boxprod}[3]{\abs{(#1\times#2)\cdot#3}}

\author{Zack Reed}
%borrowed from chrichton ogle Linear
\title{Elementary Matrices: Row Reduction, Matrix Decomposition, and Finding the Inverse of a Matrix}
\begin{document}
\begin{abstract}

    In this section we re-examine row operations, but as matrices in addition to transformations of systems of equations. We will find that the sequence of row operations that reduce a matrix to reduced row-echelon form give us the inverse of a matrix (if it exists). This is one useful instance of the idea of matrix decomposition, writing a matrix as a product of nice matrices.

\end{abstract}
\maketitle

%THE MAIN IDEA IS THAT WE CONSTRUCT THE INVERSE OF A MATRIX BY USING ELEMENTARY MATRICES, AND SO THAT'S THE POINT OF REDUCED ROW ECHELON FORM AND STUFF. 
%ALSO SET UP AN RREF IN GEOGEBRA WITH "Reduced" function as part of it, just so they can 
%See the full process, then use rref in maltab and stuff for higher dimensions.

\section*{Inverse Problems}

We spent a lot of time last chapter finding solutions to systems of equations (when they existed).

As we said, solving a system of equations was analogous to solving the matrix equation $A\vec{x}=\vec{b}$ for possible $\vec{x}$ vectors that satisfy the equation. 

Let's press more into solving the equation $A\vec{x}=\vec{b}$, as this is an example of an \emph{inverse} question in mathematics. Inverse questions are very important, and give us many tools.

\begin{example}

  For nonzero scalars $s$, multiplication inverse problems amount to finding a number $r$ such that $r\cdot s=1$.

  For instance, $2*.5=1$, $3\cdot .\overline{333}=1$, $\pi\cdot \frac{1}{\pi}=1$. 

  In this simple case, we can explicitly find the inverse of any real number $s$ (multiplicatively) by taking $\frac{1}{s}$, and hence $s\cdot\frac{1}{s}=1$.

\end{example}

\begin{example}

  For functions, in calculus, we concerned ourselves with finding the derivatives and integrals of inverse functions.

  \emph{Inverse functions} compose with each other to make the identity function (e.g. $y=x$). 

  \begin{enumerate}
  
    \item Exponential and Logarithms: $a^x$ and $\log_a(x)$ are inverses because $a^{\log_a(x)}=log_a(a^x)=x$
    
    \item Quadratic functions $q(x)=x^2$ don't fully have inverse functions because square root functions $sqrt(x)=\sqrt{x}$ are only defined for nonnegative numbers.
    
    So $\sqrt{x^2}$ and $\sqrt{x}^2$ don't equalt $y=x$, but instead $\sqrt{x^2}=\abs{x}$ and $\sqrt{x}^2=x$ only for $x\geq 0$.

  \end{enumerate}

\end{example}

So, the conditions for mathematical objects to have inverses is quite stringent, and we'll see this stringency matter for matrices. 

\begin{exploration}{For some $\vec{x}$? For all $\vec{x}$?}

  You might be tempted to characterize the inverses of matrices in terms of the matrix equation $A\vec{x}=\vec{b}$. 
  
  This is somewhat intuitive, since inverse scalars help us solve problems like $3x=5\rightarrow x=5/3$. 

  Let's explore this a little bit, and start with the matrix 

  $$A=\begin{bmatrix}1&2&3\\2&4&6\end{bmatrix}.$$

  We know from before (take the \texttt{rref(A)}) that $\mbox{im}(A)=\mbox{span}\left(\begin{bmatrix}1\\2\end{bmatrix}\right)$. In essence, $A$ takes all of space to a line in $\RR^2$.

  What inverse problems can we solve here? Specifically, for what vectors $\vec{b}$ can we find $\vec{x}$ to satisfy $A\vec{x}=\vec{b}$?

  Use MATLAB to set up and quickly solve many matrix-vector equations $A\vec{x}=\vec{b}$, for different values of $\vec{b}$. You'll want to 



\end{exploration}

 
\section*{A return to elementary matrices}

We return once again to \emph{elementary matrices}, but now with an eye towards their reversibility, or rather their \emph{invertibility}.

\begin{itemize}

  \item The \textbf{row swapping matrix} $E_{ij}$, the identity matrix with rows $i$ and $j$ swapped.
  
  $E_{12}$ performed on a $3\times 3$ matrix would be the matrix

  \begin{equation*}
    \begin{bmatrix}
      0 & 1 & 0 \\
      1 & 0 & 0 \\
      0 & 0 & 1
    \end{bmatrix}
  \end{equation*}

  \item The \textbf{scaling matrix} $E_{i}(\lambda)$ is the matrix that results from multiplying row $i$ of the identity matrix by the scalar $c$.
  
  $E_{2}(3)$ performed on a $3\times 3$ matrix would be the matrix

  \begin{equation*}
    \begin{bmatrix}
      1 & 0 & 0 \\
      0 & 3 & 0 \\
      0 & 0 & 1
    \end{bmatrix}.
  \end{equation*}

  \item The \textbf{row addition matrix} $E_{ij}(\lambda)$ is the matrix that results from adding $c$ times row $i$ to row $j$ of the identity matrix.
  
  $E_{3,2}(2)$ performed on a $3\times 3$ matrix would be the matrix

  \begin{equation*}
    \begin{bmatrix}
      1 & 0 & 0 \\
      0 & 1 & 0 \\
      0 & 2 & 1
    \end{bmatrix}.
  \end{equation*}

  This will add $2$ times the second row to the third row.

\end{itemize}

It turns out these row operations can be realized by left
multiplication by a certain type of matrix, and these matrices have
uses beyond that of performing row operations. To explain how matrix
multiplication comes into play, let us write $\cal R(_-)$ for a
particular row operation on $m\times n$ matrices, so that the given
operation is represented by $A\mapsto {\cal R}(A)$. It turns out that
for any of the three types of row operations we have considered above,
one has the identity
\[
{\cal R}(A) = {\cal R}(I*A) = {\cal R}(I)*A
\]
In other words, {\it the row operation ${\cal R}(_-)$, applied to $A$, can be realized in terms of left multiplication by the $m\times m$ matrix ${\cal R}(I)$ gotten by applying $\cal R$ to the $m\times m$ identity matrix}.
 
As one would then expect, one has - for each row operation - a
corresponding {\it elementary matrix} derived from the identity matrix
of the appropriate dimension by application of that given operation.
 
These matrices, and the notation used to define them, can be recorded
in an expanded version of the table above in which we indicated the
types of operations and their representation:
 
\begin{center}
    \begin{tabular}{|c|c|c|c|}\hline\hline
    \phantom{x} & \phantom{x} & \phantom{x} & \phantom{x}\\
    \Large{Type} &\Large{What it does} &\Large{Indicated by} & \Large{Elementary matrix}\\ \hline
    \phantom{x} & {\large Switches} & \phantom{\Huge X} & \phantom{X}\\
    \large{Type I} &{\large $i^{th}$ and $j^{th}$} & {\large $R_i\leftrightarrow R_j$} & {\large ${\cal R}(I) = P_{ij}$}\\
    \phantom{x} & {\large rows} & \phantom{x} & \phantom{x}\\ \hline
    \phantom{x} &{\large Multiplies} & \phantom{\Huge X} & \phantom{x}\\
    \large{Type II} &{\large $i^{th}$ row} & {\large $r\cdot R_i$} & {\large ${\cal R}(I) = D_i(r)$}\\
    \phantom{x} & {\large by $r\ne 0$} & \phantom{x} & \phantom{x}\\ \hline
    \phantom{x} &{\large Adds} & \phantom{\Huge X} & \phantom{x}\\
    \large{Type III} &{\large $a$ times the $i^{th}$ row} & {\large $a\cdot R_i$ added to $R_j$} & {\large ${\cal R}(I) = E_{ji}(a)$}\\
    \phantom{x} & {\large to the $j^{th}$ row} & \phantom{x} & \phantom{x}\\\hline\hline
    \end{tabular}
    \end{center}
\vskip.2in
 
%BADBAD: Examples to be included
 
In an analogous fashion, one can also perform column operations on a
matrix. As with row operations there are three types, and each type
can be achieved via {\it right} multiplication with the corresponding
matrix. In other words, if ${\cal C}(_-)$ indicates the given column
operation, then for each type one has
\[
{\cal C}(A) = {\cal C}(A*I) = A*{\cal C}(I)
\]
Denoting the $k^{th}$ column by $C_k$, we have
 
\begin{center}
    \begin{tabular}{|c|c|c|c|}\hline\hline
    \phantom{x} & \phantom{x} & \phantom{x} & \phantom{x}\\
    \Large{Type} &\Large{What it does} &\Large{Indicated by} & \Large{Elementary matrix}\\ \hline
    \phantom{x} & {\large Switches} & \phantom{\Huge X} & \phantom{X}\\
    \large{Type I} &{\large $i^{th}$ and $j^{th}$} & {\large $C_i\leftrightarrow C_j$} & {\large ${\cal C}(I) = P_{ij}$}\\
    \phantom{x} & {\large colmns} & \phantom{x} & \phantom{x}\\ \hline
    \phantom{x} &{\large Multiplies} & \phantom{\Huge X} & \phantom{x}\\
    \large{Type II} &{\large $i^{th}$ column} & {\large $r\cdot C_i$} & {\large ${\cal C}(I) = D_i(r)$}\\
    \phantom{x} & {\large by $r\ne 0$} & \phantom{x} & \phantom{x}\\ \hline
    \phantom{x} &{\large Adds} & \phantom{\Huge X} & \phantom{x}\\
    \large{Type III} &{\large $a$ times the $i^{th}$ column} & {\large $a\cdot C_i$ added to $C_j$} & {\large ${\cal C}(I) = E_{ij}(a)$}\\
    \phantom{x} & {\large to the $j^{th}$ column} & \phantom{x} & \phantom{x}\\\hline\hline
    \end{tabular}
    \end{center}
\vskip.2in
 
% BADBAD: [Examples to be included]
 
Each elementary matrix is invertible, and of the same type. The
following indicates how each elementary matrix behaves under i)
inversion and ii) transposition:
\begin{align}
P_{ij}^{-1} = P_{ij},&\qquad P_{ij}^T = P_{ij}\\
D_i(r)^{-1} = D_i(r^{-1}),&\qquad D_i(r)^T = D_i(r)\\
E_{ij}(a)^{-1} = E_{ij}(-a),&\qquad E_{ij}(a)^T = E_{ji}(a)
\end{align}
\vskip.4in
 
Elementary matrices are useful in problems where one wants to express the inverse of a matrix explicitly as a product of elementary matrices. We have already seen that a square matrix $A$ is invertible iff is is row equivalent to the identity matrix. By keeping track of the row operations used and then realizing them in terms of left multiplication by elementary matrices, we can write down a product of matrices ending in $A$ that equals $I$. The product of elementary matrices appearing to the left of $A$ must then be equal to $A^{-1}$. With such an expression for $A^{-1}$ we can also represent $A$ itself as a product of elementray matrices.
\vskip.2in
 
\begin{example} Suppose $A = \begin{bmatrix} 2 & 3\\3 & 5\end{bmatrix}$. The following (non-unique) sequence of row operations reduces $A$ to $I$:
\[\begin{split}
\begin{bmatrix} 2 & 3\\3 & 5\end{bmatrix}
\xrightarrow{R_2 := R_2 + (-1)*R_1}
\begin{bmatrix} 2 & 3\\1 & 2\end{bmatrix}
\xrightarrow{R_1 := R_1 + (-2)*R_2}
\begin{bmatrix} 0 & -1\\1 & 2
\end{bmatrix}\\
\xrightarrow{R_2 := R_2 + 2*R_1}
\begin{bmatrix} 0 & -1\\1 & 0
\end{bmatrix}
\xrightarrow{R_1\leftrightarrow R_2}\begin{bmatrix} 1 & 0\\0 & -1
\end{bmatrix}\xrightarrow{(-1)*R_2}\begin{bmatrix} 1 & 0\\0 & 1\end{bmatrix}
\end{split}
\]
Remembering that elementary matrices realize row operations via left-multiplication, transforming the above sequence into a product yields the equation
\[
D_2(-1)*P_{12}*E_{21}(2)*E_{12}(-2)*E_{21}(-1)*A = I
\]
Setting $B = D_2(-1)*P_{12}*E_{21}(2)*E_{12}(-2)*E_{21}(-1)$, we see that $B*A = I$. But we have already seen that this implies $B = A^{-1}$, the unique inverse of $A$.
\vskip.2in
 
Now if we want to also represent $A$ as a product of elementary matrices, we could do so by the sequence of equalities
\begin{align*}
A &= \big(A^{-1}\big)^{-1}\\
  &= \big(D_2(-1)*P_{12}*E_{21}(2)*E_{12}(-2)*E_{21}(-1)\big)^{-1}\\
  &= E_{21}(-1)^{-1}*E_{12}(-2)^{-1}*E_{21}(2)^{-1}*P_{12}^{-1}*D_2(-1)^{-1}\\
  &= E_{21}(1)*E_{12}(2)*E_{21}(-2)*P_{12}*D_2(-1)
\end{align*}
\end{example}
\vskip.2in
 
\begin{exercise} Suppose $B = \begin{bmatrix} 5 & 1\\4 & 2\end{bmatrix}$. Find a sequence of row operations that row reduce $B$ to $I$. Then use this to express both $B^{-1}$ an $B$ as a product of elementary matrices.
 
\end{exercise}

\subsection{Inverses of elementary matrices}

Suppose we have applied a row operation to a matrix $A$. Consider the
row operation required to return $A$ to its original form, i.e., to
undo the row operation. It turns out that this action is described by
the inverse of an elementary matrix. The following theorem ensures
that the inverse of each elementary matrix is itself an elementary
matrix.

\begin{theorem}{Inverses of elementary matrices}{inverse-elementary-matrix}
  Every elementary matrix is invertible and its inverse is also an
  elementary matrix%
  \index{elementary matrix!inverse}%
  \index{matrix!elementary matrix!inverse}.
\end{theorem}

In fact, the inverse of an elementary matrix is constructed by doing
the {\em reverse} row operation on $I$. $E^{-1}$ is obtained by
performing the row operation which would carry $E$ back to $I$.

\begin{itemize}
\item If $E$ is obtained by switching rows $i$ and $j$, then $E^{-1}$
  is also obtained by switching rows $i$ and $j$.
\item If $E$ is obtained by multiplying row $i$ by the scalar $k$,
  then $E^{-1}$ is obtained by multiplying row $i$ by the scalar
  $\frac{1}{k}$.
\item If $E$ is obtained by adding $k$ times row $i$ to row $j$, then
  $E^{-1}$ is obtained by subtracting $k$ times row $i$ from row $j$.
\end{itemize}

\begin{example}{Inverse of an elementary matrix}{inverse-elementary-matrix}
  Find $E^{-1}$, where $E$ is the elementary matrix
  \begin{equation*}
    E
    =
    \startmat{cc}
      1 & 0 \\
      0 & 2
    \stopmat
  \end{equation*}
\end{example}

\begin{solution}
  $E$ is obtained from the $2\times 2$ identity matrix by multiplying
  the second row by $2$. In order to carry $E$ back to the identity,
  we need to multiply the second row of $E$ by $\frac{1}{2}$.  Hence,
  $E^{-1}$ is given by
  \begin{equation*}
    E^{-1}
    =
    \startmat{cc}
      1 & 0 \\
      0 & \frac{1}{2}
    \stopmat.
  \end{equation*}
\end{solution}

\subsection{More properties of inverses}

% ======================================================================

In this section, we will use elementary matrices to prove a useful
theorem about the inverse of a square matrix. We start with an
observation about the {\ef} of a right invertible matrix.

\begin{lemma}\name{\Ef of a right invertible matrix}
  Suppose that $A$ is right invertible. Then the {\rref} of $A$ does
  not have a row of zeros.
\end{lemma}

\begin{proof}
  Let $R$ be the {\rref} of $A$. Then by Theorem~\ref{thm:form-rua},
  we can write $R=UA$ for some invertible square matrix $U$. By
  assumption, we have $AB=I$, and therefore
  \begin{equation*}
    R(BU^{-1})
    ~=~
    (UA)(BU^{-1})
    ~=~
    U(AB)U^{-1}
    ~=~
    UIU^{-1}
    ~=~
    UU^{-1}
    ~=~
    I.
  \end{equation*}
  If $R$ had a row of zeros, then so would the product
  $R(BU^{-1})$. But since the identity matrix $I$ does not have a row
  of zeros, neither does $R$.
\end{proof}

\begin{theorem}{Right invertible square matrices are invertible}{right-square-left}
  Suppose $A$ and $B$ are square matrices such that $AB=I$. Then it
  follows that $BA=I$, and therefore $B=A^{-1}$. In particular, a
  square matrix is right invertible if and only if it is left
  invertible if and only if it is invertible.
\end{theorem}

\begin{proof}
  Assume $A$ and $B$ are square matrices such that $AB=I$. Let $R$ be
  the {\rref} of $A$. Then by Theorem~\ref{thm:form-rua}, we can write
  $R=UA$ where $U$ is an invertible matrix. Since $AB=I$, we know by
  Lemma~\ref{lem:no-zero-row} that $R$ does not have a row of
  zeros. Since $R$ is a square {\rref} with no row of zeros, each
  column must be a pivot column, and it follows that $R=I$. Hence,
  $UA=I$, and therefore $A$ is left invertible. Moreover, we have
  \begin{equation*}
    B ~=~ IB ~=~ (UA)B ~=~ U(AB) ~=~ UI ~=~ U,
  \end{equation*}
  and therefore $B=U$. It follows that $BA=UA=I$, as claimed.

  To prove the last claim, note that we just proved that for square
  matrices, $AB=I$ implies $BA=I$. Therefore, every right inverse of
  $A$ is also a left inverse, and therefore an inverse. But of course,
  we also have that $BA=I$ implies $AB=I$. This is just the same
  theorem, with the roles of $A$ and $B$ interchanged. Therefore,
  every left inverse of $A$ is also a right inverse.
\end{proof}

This theorem is very useful, because it allows us to test only one of
the products $AB$ or $BA$ in order to check that $B$ is the inverse of
$A$, saving us half of the work. It is important to stress, however,
that this only works for {\em square} matrices. As we saw in
Example~\ref{exa:right-inverse}, non-square matrices can be right
invertible without being left invertible, or vice versa.

\subsection{Writing an invertible matrix as a product of elementary matrices}

Recall from Algorithm~\ref{algo:matrix-inversion-algorithm} that an
$n\times n$-matrix $A$ is invertible if and only if $A$ can be carried
to the $n\times n$ identity matrix using elementary row
operations. Combining this with our discussion of elementary matrices
we see that $A$ is invertible if and only if it can be written as a
product of elementary matrices. This is the content of the following
theorem.

\begin{theorem}{Product of elementary matrices}{prod-elementary}
  Let $A$ be an $n \times n$-matrix. Then $A$ is invertible if and
  only if it can be written as a product of elementary matrices.
\end{theorem}

\begin{proof}
  If $A$ is an invertible $n\times n$-matrix, then its {\rref} is the
  $n\times n$ identity matrix $I$. By Theorem~\ref{thm:form-rua}, we
  can write $I=UA$, where $U = E_k\cdots E_2E_1$ is a product of
  elementary matrices. Then
  \begin{equation*}
    A ~=~ U^{-1} ~=~ E_1^{-1}E_2^{-1}\cdots E_k^{-1}.
  \end{equation*}
  By Theorem~\ref{thm:inverse-elementary-matrix}, if $E_i$ is an
  elementary matrix, then so is $E_i^{-1}$. Therefore, $A$ has been
  written as a product of elementary matrices. Conversely, if $A$ can
  be written as a product of elementary matrices, then $A$ is clearly
  invertible, because each elementary matrix is invertible.
\end{proof}

\begin{example}{Product of elementary matrices}{prod-elementary}
  Let $A = \startmat{rrr}
    0 &  1 & 0 \\
    1 &  1 & 0 \\
    0 & -2 & 1
  \stopmat$.
  Write $A$ as a product of elementary matrices.
\end{example}

\begin{solution}
  Following the process of Theorem~\ref{thm:prod-elementary}, we first
  row-reduce $A$ to its {\rref}, recording each row operation as an
  elementary matrix.
  \begin{equation*}
    \startmat{rrr}
      0 & 1 & 0 \\
      1 & 1 & 0 \\
      0 & -2 & 1 \\
    \stopmat
    \quad\stackrel{R_1\rowswap R_2}{\roweq}\quad
    \startmat{rrr}
      1 & 1 & 0 \\
      0 & 1 & 0 \\
      0 & -2 & 1 \\
    \stopmat
    \quad\mbox{with elementary matrix}\quad
    E_1 ~=~ \startmat{rrr}
      0 & 1 & 0 \\
      1 & 0 & 0 \\
      0 & 0 & 1
    \stopmat,
  \end{equation*}
  \begin{equation*}
    \startmat{rrr}
      1 & 1 & 0 \\
      0 & 1 & 0 \\
      0 & -2 & 1 \\
    \stopmat
    \quad\stackrel{R_1\rowop R_1-R_2}{\roweq}\quad
    \startmat{rrr}
      1 & 0 & 0 \\
      0 & 1 & 0 \\
      0 & -2 & 1 \\
    \stopmat
    \quad\mbox{with elementary matrix}\quad
    E_2 ~=~  \startmat{rrr}
      1 & -1 & 0 \\
      0 & 1 & 0 \\
      0 & 0 & 1
    \stopmat,
  \end{equation*}
  \begin{equation*}
    \startmat{rrr}
      1 & 0 & 0 \\
      0 & 1 & 0 \\
      0 & -2 & 1 \\
    \stopmat
    \quad\stackrel{R_3\rowop R_3+2R_2}{\roweq}\quad
    \startmat{rrr}
      1 & 0 & 0 \\
      0 & 1 & 0 \\
      0 & 0 & 1 \\
    \stopmat
    \quad\mbox{with elementary matrix}\quad
    E_3 ~=~ \startmat{rrr}
      1 & 0 & 0 \\
      0 & 1 & 0 \\
      0 & 2 & 1 \\
    \stopmat.
  \end{equation*}
  Notice that the {\rref} of $A$ is $I$. Hence $I = UA$ where
  $U=E_3E_2E_1$. It follows that
  $A = U^{-1} = E_1^{-1}E_2^{-1}E_3^{-1}$, and so we have succeeded in
  writing $A$ as a product of elementary matrices
  \begin{equation*}
    A
    ~=~ E_1^{-1}E_2^{-1}E_3^{-1}
    ~=~
    \startmat{rrr}
      0 & 1 & 0 \\
      1 & 0 & 0 \\
      0 & 0 & 1
    \stopmat
    \startmat{rrr}
      1 & 1 & 0 \\
      0 & 1 & 0 \\
      0 & 0 & 1
    \stopmat
    \startmat{rrr}
      1 & 0 & 0 \\
      0 & 1 & 0 \\
      0 & -2 & 1
    \stopmat.
  \end{equation*}
  In particular, it follows that $A$ is invertible.
\end{solution}

\subsection{Elementary matrices and row operations}

Recall from Definition~\ref{def:row-operations} that there are three
kinds of elementary row operations%
\index{matrix!row operation}%
\index{matrix!elementary row operation}%
\index{row operation}%
\index{elementary row operation} on matrices:
\begin{enumerate}
\item Switch two rows.
\item Multiply a row by a non-zero number.
\item Add a multiple of one row to another row.
\end{enumerate}
The purpose of this section is to show that each of these row
operations corresponds to a special type of invertible matrix called
an \textbf{elementary matrix}%
\index{elementary matrix}%
\index{matrix!elementary matrix}.

\begin{example}{Elementary matrix for switching two rows}{elementary-matrix-1}
  Let
  \begin{equation*}
    E ~=~ \startmat{lll}
      1 & 0 & 0 \\
      0 & 0 & 1 \\
      0 & 1 & 0 \\
    \stopmat.
  \end{equation*}
  What is the effect of multiplying $E$ by an arbitrary $3\times
  n$-matrix $A$?
\end{example}

\begin{solution}
  Consider an arbitrary $3\times n$-matrix
  \begin{equation*}
    A ~=~ \startmat{cccc}
      a_{11} & a_{12} & \cdots & a_{1n} \\
      a_{21} & a_{22} & \cdots & a_{2n} \\
      a_{31} & a_{32} & \cdots & a_{3n} \\
    \stopmat.
  \end{equation*}
  We compute the product $EA$ by the row method:
  \begin{equation*}
    EA ~=~ \startmat{lll}
      1 & 0 & 0 \\
      0 & 0 & 1 \\
      0 & 1 & 0 \\
    \stopmat
    \startmat{cccc}
      a_{11} & a_{12} & \cdots & a_{1n} \\
      a_{21} & a_{22} & \cdots & a_{2n} \\
      a_{31} & a_{32} & \cdots & a_{3n} \\
    \stopmat
    ~=~
    \startmat{cccc}
      a_{11} & a_{12} & \cdots & a_{1n} \\
      a_{31} & a_{32} & \cdots & a_{3n} \\
      a_{21} & a_{22} & \cdots & a_{2n} \\
    \stopmat.
  \end{equation*}
  So the effect of multiplying $A$ by $E$ on the left is exactly the
  same as switching rows 2 and 3. We say that $E$ is the
  \textbf{elementary matrix for switching rows 2 and 3}.
\end{solution}

\begin{example}{Elementary matrix for multiplying a row by a non-zero number}{elementary-matrix-2}
  Let
  \begin{equation*}
    E ~=~ \startmat{lll}
      1 & 0 & 0 \\
      0 & k & 0 \\
      0 & 0 & 1 \\
    \stopmat.
  \end{equation*}
  What is the effect of multiplying $E$ by an arbitrary $3\times
  n$-matrix $A$?
\end{example}

\begin{solution}
  We compute the product $EA$ by the row method:
  \begin{equation*}
    EA ~=~ \startmat{lll}
      1 & 0 & 0 \\
      0 & k & 0 \\
      0 & 0 & 1 \\
    \stopmat
    \startmat{cccc}
      a_{11} & a_{12} & \cdots & a_{1n} \\
      a_{21} & a_{22} & \cdots & a_{2n} \\
      a_{31} & a_{32} & \cdots & a_{3n} \\
    \stopmat
    ~=~
    \startmat{rrrr}
      a_{11} & a_{12} & \cdots & a_{1n} \\
      ka_{21} & ka_{22} & \cdots & ka_{2n} \\
      a_{31} & a_{32} & \cdots & a_{3n} \\
    \stopmat.
  \end{equation*}
  So the effect of multiplying $A$ by $E$ on the left is exactly the
  same as multiplying row 2 by the scalar $k$. We say that $E$ is the
  \textbf{elementary matrix for multiplying row 2 by $k$}.
\end{solution}

\begin{example}{Elementary matrix for adding a multiple of one row to another row}{elementary-matrix-3}
  Let
  \begin{equation*}
    E ~=~ \startmat{lll}
      1 & 0 & 0 \\
      0 & 1 & 0 \\
      0 & k & 1 \\
    \stopmat.
  \end{equation*}
  What is the effect of multiplying $E$ by an arbitrary $3\times
  n$-matrix $A$?
\end{example}

\begin{solution}
  Once again we compute the product $EA$:
  \begin{equation*}
    EA ~=~ \startmat{lll}
      1 & 0 & 0 \\
      0 & 1 & 0 \\
      0 & k & 1 \\
    \stopmat
    \startmat{cccc}
      a_{11} & a_{12} & \cdots & a_{1n} \\
      a_{21} & a_{22} & \cdots & a_{2n} \\
      a_{31} & a_{32} & \cdots & a_{3n} \\
    \stopmat
    ~=~
    \startmat{cccc}
      a_{11} & a_{12} & \cdots & a_{1n} \\
      a_{21} & a_{22} & \cdots & a_{2n} \\
      a_{31}+ka_{21} & a_{32}+ka_{22} & \cdots & a_{3n}+ka_{2n} \\
    \stopmat.
  \end{equation*}
  So the effect of multiplying $A$ by $E$ on the left is exactly the
  same as adding $k$ times row 2 to row 3. We say that $E$ is the
  \textbf{elementary matrix for adding $k$ times row 2 to row 3}.
\end{solution}

As these examples show, performing each type of elementary row
operation is the same as multiplying (on the left) by a certain
invertible matrix. Thesse matrices are called the \textbf{elementary
  matrices}%
\index{elementary matrix}%
\index{matrix!elementary matrix}. In the
above examples, we have only considered $3\times 3$-elementary
matrices, but they exist for other sizes too. The following definition
makes this precise. It also shows how to calculate the elementary
matrix corresponding to any elementary row operation.

\begin{definition}{Elementary matrices and row operations}{elementary-matrices-and-row-operations}
  Let $E$ be an $n\times n$-matrix. Then $E$ is an \textbf{elementary
    matrix}%
  \index{elementary matrix}%
  \index{matrix!elementary matrix}
  if it is the result of applying one elementary row operation to the
  $n\times n$ identity matrix.
\end{definition}

\begin{example}{Finding an elementary matrix}{finding-elementary-matrix}
  Consider the elementary row operation of adding $5$ times row 3 to
  row 1 of a $4\times n$-matrix. Find the elementary matrix $E$
  corresponding to this row operation.
\end{example}

\begin{solution}
  Following Definition~\ref{def:elementary-matrices-and-row-operations}, all
  we have to do is apply the desired row operation to the
  $4\times 4$-identity matrix:
  \begin{equation*}
    \startmat{rrrr}
      1 & 0 & 0 & 0 \\
      0 & 1 & 0 & 0 \\
      0 & 0 & 1 & 0 \\
      0 & 0 & 0 & 1 \\
    \stopmat
    \quad
    \stackrel{R_1\rowop R_1+5R_3}{\roweq}
    \quad
    \startmat{rrrr}
      1 & 0 & 5 & 0 \\
      0 & 1 & 0 & 0 \\
      0 & 0 & 1 & 0 \\
      0 & 0 & 0 & 1 \\
    \stopmat
    ~=~ E.
  \end{equation*}
\end{solution}

We can double-check that multiplying $E$ by any $4\times n$-matrix
does indeed have the desired effect:
\begin{equation*}
  \startmat{rrrr}
    1 & 0 & 5 & 0 \\
    0 & 1 & 0 & 0 \\
    0 & 0 & 1 & 0 \\
    0 & 0 & 0 & 1 \\
  \stopmat
  \startmat{cccc}
    a_{11} & a_{12} & \cdots & a_{1n} \\
    a_{21} & a_{22} & \cdots & a_{2n} \\
    a_{31} & a_{32} & \cdots & a_{3n} \\
    a_{41} & a_{42} & \cdots & a_{4n} \\
  \stopmat
  ~=~
  \startmat{cccc}
    a_{11}+5a_{31} & a_{12}+5a_{32} & \cdots & a_{1n}+5a_{3n} \\
    a_{21} & a_{22} & \cdots & a_{2n} \\
    a_{31} & a_{32} & \cdots & a_{3n} \\
    a_{41} & a_{42} & \cdots & a_{4n} \\
  \stopmat.
\end{equation*}
The fact that this always works is the content of the following
theorem.

\begin{theorem}{Multiplication by an elementary matrix and row operations}{multiplication-by-elementary-matrix}
  Performing any of the three elementary row operations on a matrix $A$ is the
  same as taking the product $EA$, where $E$ is the elementary matrix
  obtained by applying the desired row operation to the identity
  matrix.
\end{theorem}



\end{document}