\documentclass{ximera}
\graphicspath{     %% setup a global graphics path
{./}               %% look in the same-level directory
{./pictures/}      %% look in graphics
{../pictures/}     %% look up one directory, then in graphics
%{../../pictures/} %% look up two directories, then in graphics
}


\author{Zack Reed}
%borrowed from Dirk Colbry's msu python stuff
\title{Rank and Nullity}
\begin{document}
\begin{abstract}

In this activity, we will explore the rank and nullity of a matrix.

\end{abstract}
\maketitle

\subsection*{Rank}
 
As stated in Theorem \ref{th:uniquenessofrref}, the reduced row-echelon form of a matrix $A$ is uniquely determined by $A$. That is, no matter which series of row operations is used to transform $A$ to its reduced row-echelon matrix, the result will always be the same matrix. In contrast, this is not true for row-echelon matrices: different sequences of row operations can transform the same matrix $A$ to different row-echelon matrices. However, it is true that the number $r$ of nonzero rows must be the same in each of these row-echelon matrices, as we will see in Theorem \ref{th:dimofrowA}. Hence, the number $r$ depends only on $A$ and not on the way in which $A$ is carried to row-echelon form. 
 
\begin{example}\label{ex:rowechofA}
Matrices (\ref{eq:ref1}) and (\ref{eq:ref2}) of Exploration \ref{init:gaussianelim1} are both row-echelon forms of $A$.  Both matrices have three nonzero rows.  The same is true for $\mbox{rref}(A)$. 
\end{example}
 
\begin{definition}\label{def:rankofamatrix}
The \dfn{rank} of matrix $A$, denoted by $\mbox{rank}(A)$, is the number of nonzero rows that remain after we reduce $A$ to row-echelon form by elementary row operations.
\end{definition}
 
\begin{example}\label{ex:rankofA1}
Compute the rank of
$$A = 
\begin{bmatrix}
    1 & 1 & -1 & 4 \\
    2 & 1 &  3 & 0 \\
    0 & 1 & -5 & 8
\end{bmatrix}$$
 
\begin{explanation}
A reduction of $A$ to row-echelon form is
$$
A = 
\begin{bmatrix}
1 & 1 & -1 & 4 \\
2 & 1 &  3 & 0 \\
0 & 1 & -5 & 8
\end{bmatrix} \rightsquigarrow\begin{bmatrix}
1 & 1 & -1 & 4 \\
0 & -1 &  5 & -8 \\
0 &  0 & 0 & 0
\end{bmatrix}
$$
Because the row-echelon form has two nonzero rows, $\mbox{rank}(A) = 2$.
\end{explanation}
\end{example}
 
\begin{theorem}\label{th:rankandsolutions}
Suppose a system of $m$ equations in $n$ variables is consistent, and that the rank of the {\it coefficient} matrix is $r$.
 
\begin{enumerate}
\item The set of solutions involves exactly $n - r$ parameters, corresponding to $n-r$ free variables.
 
\item If $r < n$, the system has infinitely many solutions.
 
\item If $r = n$, the system has a unique solution.
 
\end{enumerate}
\end{theorem}


 
\begin{proof}
The fact that the rank of the coefficient matrix is $r$ means that there are exactly $r$ leading variables in the coefficient matrix, and hence exactly $n - r$ nonleading variables. The nonleading variables are called free variables.  All free variables are assigned parameters, so the set of solutions involves exactly $n - r$ parameters. Hence if $r < n$, there is at least one parameter, and so infinitely many solutions. If $r = n$, there are no parameters and the resulting solution is unique.
\end{proof}
 
Theorem \ref{th:rankandsolutions} shows that, for any system of linear equations, exactly three possibilities exist:
 
\begin{enumerate}
 
\item Unique solution. This occurs when every variable is a leading variable.
 
\item Infinitely many solutions. This occurs when the system is consistent and there is at least one nonleading variable, so at least one parameter is involved.
 
\item No solution.  This occurs when a row $\left[\begin{array}{cccc|c}  0&0&\ldots &0&1
 \end{array}\right]$ appears in the row-echelon form. Such a row corresponds to an equation with no solutions. (See Example \ref{ex:nosolutionssys}.)
 
\end{enumerate}

\end{document}