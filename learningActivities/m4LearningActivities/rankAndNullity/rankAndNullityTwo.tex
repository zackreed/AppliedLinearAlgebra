\documentclass{ximera}
\graphicspath{  %% When looking for images,
{./}            %% look here first,
{./pictures/}   %% then look for a pictures folder,
{../pictures/}  %% which may be a directory up.
{../../pictures/}  %% which may be a directory up.
{../../../pictures/}  %% which may be a directory up.
{../../../../pictures/}  %% which may be a directory up.
}

\usepackage{listings}
%\usepackage{circuitikz}
\usepackage{xcolor}
\usepackage{amsmath,amsthm}
\usepackage{subcaption}
\usepackage{graphicx}
\usepackage{tikz}
%\usepackage{tikz-3dplot}
\usepackage{amsfonts}
%\usepackage{mdframed} % For framing content
%\usepackage{tikz-cd}

  \renewcommand{\vector}[1]{\left\langle #1\right\rangle}
  \newcommand{\arrowvec}[1]{{\overset{\rightharpoonup}{#1}}}
  \newcommand{\ro}{\texttt{R}}%% row operation
  \newcommand{\dotp}{\bullet}%% dot product
  \renewcommand{\l}{\ell}
  \let\defaultAnswerFormat\answerFormatBoxed
  \usetikzlibrary{calc,bending}
  \tikzset{>=stealth}
  




%make a maroon color
\definecolor{maroon}{RGB}{128,0,0}
%make a dark blue color
\definecolor{darkblue}{RGB}{0,0,139}
%define the color fourier0 to be the maroon color
\definecolor{fourier0}{RGB}{128,0,0}
%define the color fourier1 to be the dark blue color
\definecolor{fourier1}{RGB}{0,0,139}
%define the color fourier 1t to be the light blue color
\definecolor{fourier1t}{RGB}{173,216,230}
%define the color fourier2 to be the dark green color
\definecolor{fourier2}{RGB}{0,100,0}
%define teh color fourier2t to be the light green color
\definecolor{fourier2t}{RGB}{144,238,144}
%define the color fourier3 to be the dark purple color
\definecolor{fourier3}{RGB}{128,0,128}
%define the color fourier3t to be the light purple color
\definecolor{fourier3t}{RGB}{221,160,221}
%define the color fourier0t to be the red color
\definecolor{fourier0t}{RGB}{255,0,0}
%define the color fourier4 to be the orange color
\definecolor{fourier4}{RGB}{255,165,0}
%define the color fourier4t to be the darker orange color
\definecolor{fourier4t}{RGB}{255,215,0}
%define the color fourier5 to be the yellow color
\definecolor{fourier5}{RGB}{255,255,0}
%define the color fourier5t to be the darker yellow color
\definecolor{fourier5t}{RGB}{255,255,100}
%define the color fourier6 to be the green color
\definecolor{fourier6}{RGB}{0,128,0}
%define the color fourier6t to be the darker green color
\definecolor{fourier6t}{RGB}{0,255,0}

%New commands for this doc for errors in copying
\newcommand{\eigenvar}{\lambda}
%\newcommand{\vect}[1]{\mathbf{#1}}
\renewcommand{\th}{^{\text{th}}}
\newcommand{\st}{^{\text{st}}}
\newcommand{\nd}{^{\text{nd}}}
\newcommand{\rd}{^{\text{rd}}}
\newcommand{\paren}[1]{\left(#1\right)}
\newcommand{\abs}[1]{\left|#1\right|}
\newcommand{\R}{\mathbb{R}}
\newcommand{\C}{\mathbb{C}}
\newcommand{\Hilb}{\mathbb{H}}
\newcommand{\qq}[1]{\text{#1}}
\newcommand{\Z}{\mathbb{Z}}
\newcommand{\N}{\mathbb{N}}
\newcommand{\q}[1]{\text{``#1''}}
%\newcommand{\mat}[1]{\begin{bmatrix}#1\end{bmatrix}}
\newcommand{\rref}{\text{reduced row echelon form}}
\newcommand{\ef}{\text{echelon form}}
\newcommand{\ohm}{\Omega}
\newcommand{\volt}{\text{V}}
\newcommand{\amp}{\text{A}}
\newcommand{\Seq}{\textbf{Seq}}
\newcommand{\Poly}{\textbf{P}}
\renewcommand{\quad}{\text{    }}
\newcommand{\roweq}{\simeq}
\newcommand{\rowop}{\simeq}
\newcommand{\rowswap}{\leftrightarrow}
\newcommand{\Mat}{\textbf{M}}
\newcommand{\Func}{\textbf{Func}}
\newcommand{\Hw}{\textbf{Hamming weight}}
\newcommand{\Hd}{\textbf{Hamming distance}}
\newcommand{\rank}{\text{rank}}
\newcommand{\longvect}[1]{\overrightarrow{#1}}
% Define the circled command
\newcommand{\circled}[1]{%
  \tikz[baseline=(char.base)]{
    \node[shape=circle,draw,inner sep=2pt,red,fill=red!20,text=black] (char) {#1};}%
}

% Define custom command \strikeh that just puts red text on the 2nd argument
\newcommand{\strikeh}[2]{\textcolor{red}{#2}}

% Define custom command \strikev that just puts red text on the 2nd argument
\newcommand{\strikev}[2]{\textcolor{red}{#2}}

%more new commands for this doc for errors in copying
\newcommand{\SI}{\text{SI}}
\newcommand{\kg}{\text{kg}}
\newcommand{\m}{\text{m}}
\newcommand{\s}{\text{s}}
\newcommand{\norm}[1]{\left\|#1\right\|}
\newcommand{\col}{\text{col}}
\newcommand{\sspan}{\text{span}}
\newcommand{\proj}{\text{proj}}
\newcommand{\set}[1]{\left\{#1\right\}}
\newcommand{\degC}{^\circ\text{C}}
\newcommand{\centroid}[1]{\overline{#1}}
\newcommand{\dotprod}{\boldsymbol{\cdot}}
%\newcommand{\coord}[1]{\begin{bmatrix}#1\end{bmatrix}}
\newcommand{\iprod}[1]{\langle #1 \rangle}
\newcommand{\adjoint}{^{*}}
\newcommand{\conjugate}[1]{\overline{#1}}
\newcommand{\eigenvarA}{\lambda}
\newcommand{\eigenvarB}{\mu}
\newcommand{\orth}{\perp}
\newcommand{\bigbracket}[1]{\left[#1\right]}
\newcommand{\textiff}{\text{ if and only if }}
\newcommand{\adj}{\text{adj}}
\newcommand{\ijth}{\emph{ij}^\text{th}}
\newcommand{\minor}[2]{M_{#2}}
\newcommand{\cofactor}{\text{C}}
\newcommand{\shift}{\textbf{shift}}
\newcommand{\startmat}[1]{
  \left[\begin{array}{#1}
}
\newcommand{\stopmat}{\end{array}\right]}
%a command to give a name to explorations and hints and theorems
\newcommand{\name}[1]{\begin{centering}\textbf{#1}\end{centering}}
\newcommand{\vect}[1]{\vec{#1}}
\newcommand{\dfn}[1]{\textbf{#1}}
\newcommand{\transpose}{\mathsf{T}}
\newcommand{\mtlb}[2][black]{\texttt{\textcolor{#1}{#2}}}
\newcommand{\RR}{\mathbb{R}} % Real numbers
\newcommand{\id}{\text{id}}
\newcommand{\coord}[1]{\langle#1\rangle}
\newcommand{\RREF}{\text{RREF}}
\newcommand{\Null}{\text{Null}}
\newcommand{\Nullity}{\text{Nullity}}
\newcommand{\Rank}{\text{Rank}}
\newcommand{\Col}{\text{Col}}
\newcommand{\Ef}{\text{EF}}
\newcommand{\boxprod}[3]{\abs{(#1\times#2)\cdot#3}}


\author{Zack Reed}
%borrowed from Dirk Colbry's msu python stuff
\title{Rank and Nullity}
\begin{document}
\begin{abstract}

In this activity, we will explore the rank and nullity of a matrix.

\end{abstract}
\maketitle

\subsection*{Rank}
 
As stated in Theorem \ref{th:uniquenessofrref}, the reduced row-echelon form of a matrix $A$ is uniquely determined by $A$. That is, no matter which series of row operations is used to transform $A$ to its reduced row-echelon matrix, the result will always be the same matrix. In contrast, this is not true for row-echelon matrices: different sequences of row operations can transform the same matrix $A$ to different row-echelon matrices. However, it is true that the number $r$ of nonzero rows must be the same in each of these row-echelon matrices, as we will see in Theorem \ref{th:dimofrowA}. Hence, the number $r$ depends only on $A$ and not on the way in which $A$ is carried to row-echelon form. 
 
\begin{example}\label{ex:rowechofA}
Matrices (\ref{eq:ref1}) and (\ref{eq:ref2}) of Exploration \ref{init:gaussianelim1} are both row-echelon forms of $A$.  Both matrices have three nonzero rows.  The same is true for $\mbox{rref}(A)$. 
\end{example}
 
\begin{definition}\label{def:rankofamatrix}
The \dfn{rank} of matrix $A$, denoted by $\mbox{rank}(A)$, is the number of nonzero rows that remain after we reduce $A$ to row-echelon form by elementary row operations.
\end{definition}
 
\begin{example}\label{ex:rankofA1}
Compute the rank of
$$A = 
\begin{bmatrix}
    1 & 1 & -1 & 4 \\
    2 & 1 &  3 & 0 \\
    0 & 1 & -5 & 8
\end{bmatrix}$$
 
\begin{explanation}
A reduction of $A$ to row-echelon form is
$$
A = 
\begin{bmatrix}
1 & 1 & -1 & 4 \\
2 & 1 &  3 & 0 \\
0 & 1 & -5 & 8
\end{bmatrix} \rightsquigarrow\begin{bmatrix}
1 & 1 & -1 & 4 \\
0 & -1 &  5 & -8 \\
0 &  0 & 0 & 0
\end{bmatrix}
$$
Because the row-echelon form has two nonzero rows, $\mbox{rank}(A) = 2$.
\end{explanation}
\end{example}
 
\begin{theorem}\label{th:rankandsolutions}
Suppose a system of $m$ equations in $n$ variables is consistent, and that the rank of the {\it coefficient} matrix is $r$.
 
\begin{enumerate}
\item The set of solutions involves exactly $n - r$ parameters, corresponding to $n-r$ free variables.
 
\item If $r < n$, the system has infinitely many solutions.
 
\item If $r = n$, the system has a unique solution.
 
\end{enumerate}
\end{theorem}


 
\begin{proof}
The fact that the rank of the coefficient matrix is $r$ means that there are exactly $r$ leading variables in the coefficient matrix, and hence exactly $n - r$ nonleading variables. The nonleading variables are called free variables.  All free variables are assigned parameters, so the set of solutions involves exactly $n - r$ parameters. Hence if $r < n$, there is at least one parameter, and so infinitely many solutions. If $r = n$, there are no parameters and the resulting solution is unique.
\end{proof}
 
Theorem \ref{th:rankandsolutions} shows that, for any system of linear equations, exactly three possibilities exist:
 
\begin{enumerate}
 
\item Unique solution. This occurs when every variable is a leading variable.
 
\item Infinitely many solutions. This occurs when the system is consistent and there is at least one nonleading variable, so at least one parameter is involved.
 
\item No solution.  This occurs when a row $\left[\begin{array}{cccc|c}  0&0&\ldots &0&1
 \end{array}\right]$ appears in the row-echelon form. Such a row corresponds to an equation with no solutions. (See Example \ref{ex:nosolutionssys}.)
 
\end{enumerate}

\end{document}