\documentclass{ximera}
\graphicspath{     %% setup a global graphics path
{./}               %% look in the same-level directory
{./pictures/}      %% look in graphics
{../pictures/}     %% look up one directory, then in graphics
%{../../pictures/} %% look up two directories, then in graphics
}

\author{Zack Reed}
%borrowed from anna davis
\title{Dot Product}
\begin{document}
\begin{abstract}

\end{abstract}
\maketitle

\section*{Dot Product and its Properties}
 
%added to defs
\begin{definition}\label{def:dotproduct}
  Let $\vec{u}$ and $\vec{v}$ be vectors in $\RR^n$.  The \dfn{dot
    product} of $\vec{u}$ and $\vec{v}$, denoted by
  $\vec{u}\dotp \vec{v}$, is given by
  \begin{align*}
    \vec{u}\dotp\vec{v}=\begin{bmatrix}u_1\\u_2\\\vdots\\u_n\end{bmatrix}\dotp\begin{bmatrix}v_1\\v_2\\\vdots\\v_n\end{bmatrix}=u_1v_1+u_2v_2+\ldots+u_nv_n.
  \end{align*}
\end{definition}
 
\begin{example}\label{ex:dotex}
  Find $\vec{u}\dotp \vec{v}$ if
  $\vec{u}=\begin{bmatrix}-2\\0\\1\end{bmatrix}$ and
  $\vec{v}=\begin{bmatrix}3\\2\\-4\end{bmatrix}$.
 
  \begin{explanation}
    $$\vec{u}\dotp\vec{v}=\begin{bmatrix}-2\\0\\1\end{bmatrix}\dotp\begin{bmatrix}3\\2\\-4\end{bmatrix}=(-2)(3)+(0)(2)+(1)(-4)=-6-4=-10$$
  \end{explanation}
\end{example}
 
Note that the dot product of two vectors is a scalar.  For this reason, the dot product is sometimes called a \dfn{scalar product}.
 
\subsection*{Properties of the Dot Product}
 
A quick examination of Example \ref{ex:dotex} will convince you that the dot product is \dfn{commutative}. In other words, $\vec{u}\dotp\vec{v}=\vec{v}\dotp\vec{u}$.  This and other properties of the dot product are stated below.
 
\begin{theorem}\label{th:dotproductproperties} The following properties hold for
  vectors $\vec{u}$, $\vec{v}$ and $\vec{w}$ in $\RR^n$ and scalar
  $k$ in $\RR$.
  \begin{enumerate}
  \item\label{item:commutative}
    $\vec{u}\dotp\vec{v}=\vec{v}\dotp\vec{u}$
    
  \item\label{item:distributive} $(\vec{u}+\vec{v})\dotp \vec{w}=\vec{u}\dotp \vec{w}+\vec{v}\dotp \vec{w}$
    
  \item\label{item:distributive-again} $\vec{u}\dotp (\vec{v}+\vec{w})=\vec{u}\dotp\vec{v}+\vec{u}\dotp \vec{w}$
    
  \item\label{item:scalar} $(k\vec{u})\dotp \vec{v}=k(\vec{u}\dotp\vec{v})=\vec{u}\dotp (k\vec{v})$
    
  \item \label{item:positive} $\vec{u}\dotp\vec{u}\geq 0$, and $\vec{u}\dotp\vec{u}=0$ if and only if $\vec{u}={\bf 0}$.
    
  \item \label{item:norm}
    $\norm{\vec{u}}^2=\vec{u}\dotp\vec{u}$
  \end{enumerate}
\end{theorem}
 
We will prove Property ~\ref{item:distributive}.  The remaining properties are left as exercises.
 
\begin{proof}[Proof of Property~\ref{item:distributive}:]
 
\begin{align*}
\left(\vec{u}+\vec{v}\right)\dotp \vec{w}&=\left(\begin{bmatrix} u_1\\ u_2\\ \vdots\\ u_n \end{bmatrix}+\begin{bmatrix} v_1\\ v_2\\ \vdots\\ v_n \end{bmatrix}\right)\dotp \begin{bmatrix}w_1\\w_2\\\vdots\\w_n\end{bmatrix}=\begin{bmatrix}
u_1+v_1\\
u_2+v_2\\
\vdots\\
u_n+v_n
\end{bmatrix}\dotp \begin{bmatrix}w_1\\w_2\\\vdots\\w_n\end{bmatrix}\\
&=(u_1+v_1)w_1+
(u_2+v_2)w_2+
\ldots+
(u_n+v_n)w_n\\
&=u_1w_1+v_1w_1+
u_2w_2+v_2w_2+
\ldots+
u_nw_n+v_nw_n\\
&=(u_1w_1+
u_2w_2\ldots+u_nw_n)+(v_1w_1+v_2w_2+
\ldots
+v_nw_n)\\
&=\begin{bmatrix}
u_1\\
u_2\\
\vdots\\
u_n
\end{bmatrix}\dotp\begin{bmatrix}w_1\\w_2\\\vdots\\w_n\end{bmatrix}+\begin{bmatrix}
v_1\\
v_2\\
\vdots\\
v_n
\end{bmatrix}\dotp \begin{bmatrix}w_1\\w_2\\\vdots\\w_n\end{bmatrix}
=\vec{u}\dotp\vec{w}+\vec{v}\dotp\vec{w}
\end{align*}
\end{proof}
 
We will illustrate Property~\ref{item:norm} with an example.
\begin{example}\label{ex:exprop6}
  Let $\vec{u}=\begin{bmatrix}-2\\3\end{bmatrix}$.  Use $\vec{u}$ to illustrate Property~\ref{item:norm} of Theorem~\ref{th:dotproductproperties}.
  \begin{explanation}
   
  $$\norm{\vec{u}}^2=(-2)^2+3^2=(-2)(-2)+(3)(3)=\begin{bmatrix}-2\\3\end{bmatrix}\dotp\begin{bmatrix}-2\\3\end{bmatrix}=\vec{u}\dotp\vec{u}$$
  \end{explanation}
\end{example}
 
If we take the square root of both sides of the equation in Property~\ref{item:norm}, we get an alternative way to think of the \href{https://ximera.osu.edu/appliedlinearalgebra/c1ChapterOne/learningActivities/m1LearningActivities/m1s1Vectors/vectorsAreEverywhereTwo}{length of a vector}.
 
\begin{corollary}[Length of a Vector]\label{cor:length_via_dotprod}
    Let $\vec{v}$ be a vector in $\RR^n$, then the \dfn{length}, or the \dfn{magnitude}, of $\vec{v}$ is given by
\begin{equation*} \label{eq:norm_dotp}
\norm{\vec{v}}=\sqrt{\vec{v} \dotp \vec{v}}
\end{equation*}
\end{corollary}


\end{document}