\documentclass{ximera}
\graphicspath{     %% setup a global graphics path
{./}               %% look in the same-level directory
{./pictures/}      %% look in graphics
{../pictures/}     %% look up one directory, then in graphics
%{../../pictures/} %% look up two directories, then in graphics
}

\author{Zack Reed}
%borrowed from anna davis
\title{Unit Vectors}
\begin{document}
\begin{abstract}

\end{abstract}
\maketitle

\section*{Unit Vector in the Direction of a Given Vector}
Recall that a {\it unit} vector is a vector of length 1.  Given a non-zero vector $\vec{v}$, we can find a unit vector in the same direction by multiplying $\vec{v}$ by an appropriate scalar.  For example, if $\vec{v}=\begin{bmatrix}a\\b\end{bmatrix}$ and $\norm{\vec{v}}=3$, then a unit vector $\vec{u}$ in the same direction is given by $\vec{u}=\begin{bmatrix}a/3\\b/3\end{bmatrix}=\begin{bmatrix}a/\norm{\vec{v}}\\b/\norm{\vec{v}}\end{bmatrix}$.


 
\begin{center}
\begin{tikzpicture}


  \draw (-1,0)--(4,0);
  \draw (0,-1)--(0,2);



  \draw[line width=1pt,-stealth, red](0,0)--(3,1) node[above right]{$\vec{v}=\begin{bmatrix}a \\ b\end{bmatrix}$};

  \draw[line width=0.5pt,dashed, red](3,1)--(3,0);
   \draw[line width=0.5pt,dashed, red](3,1)--(0,1);

    \node[red] at (3, -0.2)   (b) {$a$};
    \node[red] at (-0.2, 1)   (b) {$b$};
    \node[red] at (2.2, 0.3)   (b) {$\norm{\vec{v}}=3$};

\end{tikzpicture}
\end{center}


 
\begin{center}
\begin{tikzpicture}
  \draw (-1,0)--(4,0);
  \draw (0,-1)--(0,2);
  \draw[line width=1pt,-stealth, red](0,0)--(3,1) node[above right]{$\vec{v}=\begin{bmatrix}a\\b\end{bmatrix}$};
  \draw[line width=0.5pt,dashed, red](3,1)--(3,0);
   \draw[line width=0.5pt,dashed, red](3,1)--(0,1);
\node[red] at (3, -0.2)   (b) {$a$};
    \node[red] at (-0.2, 1)   (b) {$b$};
    \draw[line width=1.2pt,-stealth, blue](0,0)--(1,1/3);
  \draw[line width=0.5pt,dashed, blue](1,1/3)--(1,0);
   \draw[line width=0.5pt,dashed, blue](1,1/3)--(0,1/3);
   \node[blue] at (1, -0.5)   (b) {$\frac{1}{3}a$};
    \node[blue] at (-0.4, 0.4)   (b) {$\frac{1}{3}b$};
  \end{tikzpicture}
\end{center}
 
In general, dividing a non-zero vector by its own magnitude produces a unit vector in the same direction.  We summarize this observation in a theorem.


 
 
  \begin{theorem}\label{th:unit} Let $\vec{v}=\begin{bmatrix}v_1\\v_2\\\vdots\\v_n\end{bmatrix}$ be a non-zero vector in $\mathbb{R}^n$. Vector $\vec{u}$ given by
  \begin{equation*}
 \vec{u}=\frac{1}{\norm{\vec{v}}}\vec{v}=\begin{bmatrix}v_1/\norm{\vec{v}}\\v_2/\norm{\vec{v}}\\\vdots\\v_n/\norm{\vec{v}}\end{bmatrix}
\end{equation*}
is a unit vector in the direction of $\vec{v}$.
\end{theorem}



 
\begin{proof}
Because $\vec{u}$ is a positive scalar multiple of $\vec{v}$, $\vec{u}$ points in the direction of $\vec{v}$.  We now show that $\norm{\vec{u}}=1$.
\begin{eqnarray*}
\norm{\vec{u}}&=&\sqrt{ \Big(\frac{v_1}{\norm{\vec{v}}}\Big)^2+\Big(\frac{v_2}{\norm{\vec{v}}}\Big)^2+\ldots +\Big(\frac{v_n}{\norm{\vec{v}}}\Big)^2}\\
&=&\frac{1}{\norm{\vec{v}}}\sqrt{v_1^2+v_2^2+\ldots +v_n^2}\\
&=&\frac{\norm{\vec{v}}}{\norm{\vec{v}}}=1
\end{eqnarray*}
\end{proof}
 

 
\begin{example}\label{reference}
Find a unit vector in the direction of $\vec{v}=\begin{bmatrix}2\\-3\\1\\0\\1\end{bmatrix}$.
 
\begin{explanation}
We first compute $\norm{\vec{v}}$.
$$\norm{\vec{v}}=\sqrt{4+9+1+1}=\sqrt{15}$$
By Theorem \ref{th:unit},
$$\vec{u}=\begin{bmatrix}2/\sqrt{15}\\-3/\sqrt{15}\\1/\sqrt{15}\\0\\1/\sqrt{15}\end{bmatrix}=\frac{1}{\sqrt{15}}\begin{bmatrix}2\\-3\\1\\0\\1\end{bmatrix}$$
\end{explanation}
\end{example}


\end{document}