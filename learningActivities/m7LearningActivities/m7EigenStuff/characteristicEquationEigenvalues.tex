\documentclass{ximera}
\graphicspath{  %% When looking for images,
{./}            %% look here first,
{./pictures/}   %% then look for a pictures folder,
{../pictures/}  %% which may be a directory up.
{../../pictures/}  %% which may be a directory up.
{../../../pictures/}  %% which may be a directory up.
{../../../../pictures/}  %% which may be a directory up.
}

\usepackage{listings}
%\usepackage{circuitikz}
\usepackage{xcolor}
\usepackage{amsmath,amsthm}
\usepackage{subcaption}
\usepackage{graphicx}
\usepackage{tikz}
%\usepackage{tikz-3dplot}
\usepackage{amsfonts}
%\usepackage{mdframed} % For framing content
%\usepackage{tikz-cd}

  \renewcommand{\vector}[1]{\left\langle #1\right\rangle}
  \newcommand{\arrowvec}[1]{{\overset{\rightharpoonup}{#1}}}
  \newcommand{\ro}{\texttt{R}}%% row operation
  \newcommand{\dotp}{\bullet}%% dot product
  \renewcommand{\l}{\ell}
  \let\defaultAnswerFormat\answerFormatBoxed
  \usetikzlibrary{calc,bending}
  \tikzset{>=stealth}
  




%make a maroon color
\definecolor{maroon}{RGB}{128,0,0}
%make a dark blue color
\definecolor{darkblue}{RGB}{0,0,139}
%define the color fourier0 to be the maroon color
\definecolor{fourier0}{RGB}{128,0,0}
%define the color fourier1 to be the dark blue color
\definecolor{fourier1}{RGB}{0,0,139}
%define the color fourier 1t to be the light blue color
\definecolor{fourier1t}{RGB}{173,216,230}
%define the color fourier2 to be the dark green color
\definecolor{fourier2}{RGB}{0,100,0}
%define teh color fourier2t to be the light green color
\definecolor{fourier2t}{RGB}{144,238,144}
%define the color fourier3 to be the dark purple color
\definecolor{fourier3}{RGB}{128,0,128}
%define the color fourier3t to be the light purple color
\definecolor{fourier3t}{RGB}{221,160,221}
%define the color fourier0t to be the red color
\definecolor{fourier0t}{RGB}{255,0,0}
%define the color fourier4 to be the orange color
\definecolor{fourier4}{RGB}{255,165,0}
%define the color fourier4t to be the darker orange color
\definecolor{fourier4t}{RGB}{255,215,0}
%define the color fourier5 to be the yellow color
\definecolor{fourier5}{RGB}{255,255,0}
%define the color fourier5t to be the darker yellow color
\definecolor{fourier5t}{RGB}{255,255,100}
%define the color fourier6 to be the green color
\definecolor{fourier6}{RGB}{0,128,0}
%define the color fourier6t to be the darker green color
\definecolor{fourier6t}{RGB}{0,255,0}

%New commands for this doc for errors in copying
\newcommand{\eigenvar}{\lambda}
%\newcommand{\vect}[1]{\mathbf{#1}}
\renewcommand{\th}{^{\text{th}}}
\newcommand{\st}{^{\text{st}}}
\newcommand{\nd}{^{\text{nd}}}
\newcommand{\rd}{^{\text{rd}}}
\newcommand{\paren}[1]{\left(#1\right)}
\newcommand{\abs}[1]{\left|#1\right|}
\newcommand{\R}{\mathbb{R}}
\newcommand{\C}{\mathbb{C}}
\newcommand{\Hilb}{\mathbb{H}}
\newcommand{\qq}[1]{\text{#1}}
\newcommand{\Z}{\mathbb{Z}}
\newcommand{\N}{\mathbb{N}}
\newcommand{\q}[1]{\text{``#1''}}
%\newcommand{\mat}[1]{\begin{bmatrix}#1\end{bmatrix}}
\newcommand{\rref}{\text{reduced row echelon form}}
\newcommand{\ef}{\text{echelon form}}
\newcommand{\ohm}{\Omega}
\newcommand{\volt}{\text{V}}
\newcommand{\amp}{\text{A}}
\newcommand{\Seq}{\textbf{Seq}}
\newcommand{\Poly}{\textbf{P}}
\renewcommand{\quad}{\text{    }}
\newcommand{\roweq}{\simeq}
\newcommand{\rowop}{\simeq}
\newcommand{\rowswap}{\leftrightarrow}
\newcommand{\Mat}{\textbf{M}}
\newcommand{\Func}{\textbf{Func}}
\newcommand{\Hw}{\textbf{Hamming weight}}
\newcommand{\Hd}{\textbf{Hamming distance}}
\newcommand{\rank}{\text{rank}}
\newcommand{\longvect}[1]{\overrightarrow{#1}}
% Define the circled command
\newcommand{\circled}[1]{%
  \tikz[baseline=(char.base)]{
    \node[shape=circle,draw,inner sep=2pt,red,fill=red!20,text=black] (char) {#1};}%
}

% Define custom command \strikeh that just puts red text on the 2nd argument
\newcommand{\strikeh}[2]{\textcolor{red}{#2}}

% Define custom command \strikev that just puts red text on the 2nd argument
\newcommand{\strikev}[2]{\textcolor{red}{#2}}

%more new commands for this doc for errors in copying
\newcommand{\SI}{\text{SI}}
\newcommand{\kg}{\text{kg}}
\newcommand{\m}{\text{m}}
\newcommand{\s}{\text{s}}
\newcommand{\norm}[1]{\left\|#1\right\|}
\newcommand{\col}{\text{col}}
\newcommand{\sspan}{\text{span}}
\newcommand{\proj}{\text{proj}}
\newcommand{\set}[1]{\left\{#1\right\}}
\newcommand{\degC}{^\circ\text{C}}
\newcommand{\centroid}[1]{\overline{#1}}
\newcommand{\dotprod}{\boldsymbol{\cdot}}
%\newcommand{\coord}[1]{\begin{bmatrix}#1\end{bmatrix}}
\newcommand{\iprod}[1]{\langle #1 \rangle}
\newcommand{\adjoint}{^{*}}
\newcommand{\conjugate}[1]{\overline{#1}}
\newcommand{\eigenvarA}{\lambda}
\newcommand{\eigenvarB}{\mu}
\newcommand{\orth}{\perp}
\newcommand{\bigbracket}[1]{\left[#1\right]}
\newcommand{\textiff}{\text{ if and only if }}
\newcommand{\adj}{\text{adj}}
\newcommand{\ijth}{\emph{ij}^\text{th}}
\newcommand{\minor}[2]{M_{#2}}
\newcommand{\cofactor}{\text{C}}
\newcommand{\shift}{\textbf{shift}}
\newcommand{\startmat}[1]{
  \left[\begin{array}{#1}
}
\newcommand{\stopmat}{\end{array}\right]}
%a command to give a name to explorations and hints and theorems
\newcommand{\name}[1]{\begin{centering}\textbf{#1}\end{centering}}
\newcommand{\vect}[1]{\vec{#1}}
\newcommand{\dfn}[1]{\textbf{#1}}
\newcommand{\transpose}{\mathsf{T}}
\newcommand{\mtlb}[2][black]{\texttt{\textcolor{#1}{#2}}}
\newcommand{\RR}{\mathbb{R}} % Real numbers
\newcommand{\id}{\text{id}}
\newcommand{\coord}[1]{\langle#1\rangle}
\newcommand{\RREF}{\text{RREF}}
\newcommand{\Null}{\text{Null}}
\newcommand{\Nullity}{\text{Nullity}}
\newcommand{\Rank}{\text{Rank}}
\newcommand{\Col}{\text{Col}}
\newcommand{\Ef}{\text{EF}}
\newcommand{\boxprod}[3]{\abs{(#1\times#2)\cdot#3}}

\author{Zack Reed}
%borrowed from selinger linear algebra
\title{The Characteristic Equation: Eigenvalues}
\begin{document}
\begin{abstract}

\end{abstract}
\maketitle


\section*{The Characteristic Equation}

    
Let $A$ be an $n \times n$ matrix.  In \href{https://ximera.osu.edu/appliedlinearalgebra/c7ChapterSeven/learningActivities/m7LearningActivities/m7EigenStuff/describingEigenstuff}{the previous section} we learned that the eigenvectors and eigenvalues of $A$ are vectors $\vec{x}$ and scalars $\lambda$ that satisfy the equation 
$A \vec{v} = \lambda \vec{v}$

We listed a few reasons why we are interested in finding eigenvalues and eigenvectors, but we did not give any process for finding them.  In this section we will focus on a process which can be used for small matrices.  %For larger matrices, the best methods we have are iterative methods, and we will explore some of these in \href{https://ximera.osu.edu/oerlinalg/LinearAlgebra/EIG-0070/main}{The Power Method and the Dominant Eigenvalue}.
    
For an $n \times n$ matrix, we will see that the eigenvalues are the roots of a polynomial called the \dfn{characteristic polynomial}.  So finding eigenvalues is equivalent to solving a polynomial equation of degree $n$.  Finding the corresponding eigenvectors turns out to be a matter of computing the null space of a matrix, as the following exploration demonstrates.
    
\begin{exploration}\label{exp:slowdown}
If a vector $\vec{x}$ is an eigenvector satisfying the equation $A\vec{x}=\lambda \vec{x}$, then clearly it also satisfies  $A\vec{x}-\lambda \vec{x} =$ \wordChoice{\choice{0} \choice[correct]{$\vec{0}$}}.
    
It seems natural at this point to try to factor.  We would love to ``factor out'' $\vec{x}$.  Here is the procedure:
\begin{align*}
A\vec{x}-\lambda \vec{x} &= \vec{0} \\
A\vec{x}-\lambda I\vec{x} &= \vec{0} \\
(A-\lambda I)\vec{x} &= \vec{0}
\end{align*}
The middle step was necessary before factoring because \wordChoice{\choice[correct]{we cannot subtract a $1 \times 1$ scalar $\lambda$ from an $n \times n$ matrix $A$} \choice{$\lambda$ is a Greek letter}}.
    
This shows that any eigenvector $\vec{x}$ of $A$ is in the \wordChoice{\choice{row space}\choice{column space}\choice[correct]{null space}}  of the related matrix, $A-\lambda I$.
    
Since eigenvectors are non-zero vectors, this means that $A$ will have eigenvectors if and only if the null space of $A-\lambda I$ is nontrivial.  The only way that $\mbox{null}(A-\lambda I)$ can be nontrivial is if $\mbox{rank}(A-\lambda I)$ \wordChoice{\choice{$=$}\choice[correct]{$<$}\choice{$>$}} $n$.
    
If the rank of an $n \times n$ matrix is less than $n$, then the matrix is singular.  Since $(A-\lambda I)$ must be singular for any eigenvalue $\lambda$, we see that $\lambda$ is an eigenvalue of $A$ if and only if
\begin{equation}\label{eqn:chareqn}
\mbox{det}(A-\lambda I) = \answer{0}
\end{equation}
\end{exploration}
In theory, Exploration \ref{exp:slowdown} offers us a way to find eigenvalues.  To find the eigenvalues of $A$, one can solve Equation (\ref{eqn:chareqn}) for $\lambda$.
    
\subsection*{Eigenvalues}
    
\begin{definition}\label{def:chareqcharpoly}
The equation
$$\mbox{det}(A-\lambda I) = 0$$ is called the \dfn{characteristic equation} of $A$.
\end{definition}
    
\begin{example}\label{ex:2x2eig}
Let $A=\begin{bmatrix} 2& 1\\ 1&2
\end{bmatrix}$.  Compute the eigenvalues of this matrix using the characteristic equation. (List your answers in an increasing order.)
\begin{explanation}
    \begin{align*}\det(A-\lambda I)=\begin{vmatrix} \answer{2}-\lambda & \answer{1} \\ \answer{1} &\answer{2}-\lambda \end{vmatrix} &= (\lambda{2}-\lambda)^2 - \answer{1} \\
    &= \lambda^2 - \answer{4} \lambda + \answer{3} \\
    &= (\lambda - \answer{1})(\lambda - \answer{3})
    \end{align*}
    The characteristic equation \((\lambda - \answer{1})(\lambda - \answer{3}) = \answer{0}\) has solutions \(\lambda_1 = \answer{1}\) and \(\lambda_2 = \answer{3}\). These are the eigenvalues of \(A\).
    \end{explanation}
\end{example}
    
\begin{example}\label{ex:2x2eig2}
Let $B=\begin{bmatrix} 2& 1\\ 4&2
\end{bmatrix}$.  Compute the eigenvalues of $B$ using the characteristic equation.  (List your answers in an increasing order.)
    
$\lambda_1=\answer{0}$ and $\lambda_2=\answer{4}$
\end{example}


    
\begin{example}\label{ex:3x3eig}
Let $C=\begin{bmatrix} 2 & 1 & 1\\ 1 & 2 & 1\\ 1 & 1 & 2\end{bmatrix}$.  Compute the eigenvalues of $C$ using the characteristic equation.
\begin{explanation}
    \begin{align*}\det(C-\lambda I)&=\begin{vmatrix} \answer{2}-\lambda & \answer{1} & \answer{1} \\ \answer{1} & \answer{2}-\lambda & \answer{1} \\ \answer{1} & \answer{1} & \answer{2}-\lambda \end{vmatrix}\\
    &=\answer{4} - \answer{9} \lambda + \answer{6} \lambda^2 - \lambda^3\\
    &=-(\lambda - \answer{4})(\lambda - \answer{1})^2
    \end{align*}
    Matrix \(C\) has eigenvalues \(\lambda_1 = \answer{1}\) and \(\lambda_2 = \answer{4}\).
    \end{explanation}
\end{example}
    
In Example \ref{ex:3x3eig}, the factor  $(\lambda-1)$ appears twice.  This repeated factor gives rise to the eigenvalue $\lambda_1=1$.  We say that the eigenvalue $\lambda_1=1$ has \dfn{algebraic multiplicity} $2$.

There are actually two kinds of multiplicities associated with eigenvalues that we want to consider, particulalry in the next chapter. 

\begin{definition}\name{Eigenvalue multiplicities and Eigenspaces}\label{def:geommulteig}

The \emph{algebraic multiplicity} of an eigenvalue $\lambda$ is the number of times that it occurs as a root of the characteristic polynomial.

The eigenspace $\mathcal{S}_\lambda$ is the span of all eigenvectors associated with the eigenvalue $\lambda$.

The \emph{geometric multiplicity} of an eigenvalue $\lambda$ is the dimension of the corresponding eigenspace $\mathcal{S}_\lambda$. 
\end{definition}

Consider now the following lemma.

\begin{lemma}\label{lemma:dimeigenspace}
Let $A$ be an $n\times n$ matrix, and let $\mathcal{S}_{\lambda_1}$ be the eigenspace corresponding to the eigenvalue $\lambda_1$ which has algebraic multiplicity $m$.  Then
$$\mbox{dim}(\mathcal{S}_{\lambda_1})\leq m$$
\end{lemma}

In other words, the geometric multiplicity of an eigenvalue is less than or equal to the algebraic multiplicity of that same eigenvalue.

    
The three examples above are a bit contrived.  It is not always possible to completely factor the characteristic polynomial using only real numbers.  However, a fundamental fact from algebra is that every degree $n$ polynomial has $n$ roots (counting multiplicity) provided that we allow complex numbers.
    
\begin{example}\label{ex:3x3_complex_eig}
Let $D=\begin{bmatrix} 0&0&0\\ 0 &1&1\\ 0 & -1&1\end{bmatrix}$.  Compute the eigenvalues of this matrix.
    
\begin{explanation}
    \begin{align*}
    \det(D-\lambda I) &= \begin{vmatrix} 
    \answer{-1}\lambda & 0 & 0\\ 
    0 & \answer{1}+\answer{-1}\lambda & \answer{1}\\ 
    0 & \answer{-1} & \answer{1}+\answer{-1}\lambda 
    \end{vmatrix} \\
    &= \answer{-1}\lambda(\lambda^2 + \answer{-2}\lambda + \answer{2})
    \end{align*}
    So one of the eigenvalues of $D$ is $\lambda=\answer{0}$.  To get the other eigenvalues we must solve $\lambda^2+\answer{-2}\lambda+\answer{2}=0$.  We compute that $\lambda_1=\answer{1}+i$ and $\lambda_2=\answer{1}-i$ are also eigenvalues of $D$.
    \end{explanation}
\end{example}


    
\begin{exploration}\label{init:3x3tri}
Let $T=\begin{bmatrix} 1 & 2 & 3\\ 0 & 5 & 6\\ 0 & 0 & 9\end{bmatrix}$.  Compute the eigenvalues of this matrix.  (List your answers in an increasing order.)
    
$$\lambda_1=\answer{1},\quad\lambda_2=\answer{5},\quad\text{and}\quad\lambda_3=\answer{9}$$
    
Note that the eigenvalues of $T$ are the entries on the main diagonal. This is a special property of triangular matrices, and it follows directly from the very simple to calculate determinant of a triangular matrix.
\end{exploration}
    
The matrix in Exploration Problem \ref{init:3x3tri} is a triangular matrix, and the property you observed holds in general.
    
\begin{theorem}\label{th:eigtri}
Let $T$ be a triangular matrix.  Then the eigenvalues of $T$ are the entries on the main diagonal.
\end{theorem}
    
    
\begin{corollary}\label{th:eigdiag}
Let $D$ be a diagonal matrix.  Then the eigenvalues of $D$ are the entries on the main diagonal.
\end{corollary}
    
\begin{remark}
    One final note about eigenvalues. The relationship to the characteristic polynomial is primarily for conceptual grounding. In practice, much like with the SVD from the last chapter, we compute eigenvalues in a much more efficient way than is presented here. Understanding the connection to the characteristic equation is still an important idea, however.
\end{remark}

\end{document}