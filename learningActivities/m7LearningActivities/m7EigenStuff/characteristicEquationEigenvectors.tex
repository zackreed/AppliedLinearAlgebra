\documentclass{ximera}
\graphicspath{  %% When looking for images,
{./}            %% look here first,
{./pictures/}   %% then look for a pictures folder,
{../pictures/}  %% which may be a directory up.
{../../pictures/}  %% which may be a directory up.
{../../../pictures/}  %% which may be a directory up.
{../../../../pictures/}  %% which may be a directory up.
}

\usepackage{listings}
%\usepackage{circuitikz}
\usepackage{xcolor}
\usepackage{amsmath,amsthm}
\usepackage{subcaption}
\usepackage{graphicx}
\usepackage{tikz}
%\usepackage{tikz-3dplot}
\usepackage{amsfonts}
%\usepackage{mdframed} % For framing content
%\usepackage{tikz-cd}

  \renewcommand{\vector}[1]{\left\langle #1\right\rangle}
  \newcommand{\arrowvec}[1]{{\overset{\rightharpoonup}{#1}}}
  \newcommand{\ro}{\texttt{R}}%% row operation
  \newcommand{\dotp}{\bullet}%% dot product
  \renewcommand{\l}{\ell}
  \let\defaultAnswerFormat\answerFormatBoxed
  \usetikzlibrary{calc,bending}
  \tikzset{>=stealth}
  




%make a maroon color
\definecolor{maroon}{RGB}{128,0,0}
%make a dark blue color
\definecolor{darkblue}{RGB}{0,0,139}
%define the color fourier0 to be the maroon color
\definecolor{fourier0}{RGB}{128,0,0}
%define the color fourier1 to be the dark blue color
\definecolor{fourier1}{RGB}{0,0,139}
%define the color fourier 1t to be the light blue color
\definecolor{fourier1t}{RGB}{173,216,230}
%define the color fourier2 to be the dark green color
\definecolor{fourier2}{RGB}{0,100,0}
%define teh color fourier2t to be the light green color
\definecolor{fourier2t}{RGB}{144,238,144}
%define the color fourier3 to be the dark purple color
\definecolor{fourier3}{RGB}{128,0,128}
%define the color fourier3t to be the light purple color
\definecolor{fourier3t}{RGB}{221,160,221}
%define the color fourier0t to be the red color
\definecolor{fourier0t}{RGB}{255,0,0}
%define the color fourier4 to be the orange color
\definecolor{fourier4}{RGB}{255,165,0}
%define the color fourier4t to be the darker orange color
\definecolor{fourier4t}{RGB}{255,215,0}
%define the color fourier5 to be the yellow color
\definecolor{fourier5}{RGB}{255,255,0}
%define the color fourier5t to be the darker yellow color
\definecolor{fourier5t}{RGB}{255,255,100}
%define the color fourier6 to be the green color
\definecolor{fourier6}{RGB}{0,128,0}
%define the color fourier6t to be the darker green color
\definecolor{fourier6t}{RGB}{0,255,0}

%New commands for this doc for errors in copying
\newcommand{\eigenvar}{\lambda}
%\newcommand{\vect}[1]{\mathbf{#1}}
\renewcommand{\th}{^{\text{th}}}
\newcommand{\st}{^{\text{st}}}
\newcommand{\nd}{^{\text{nd}}}
\newcommand{\rd}{^{\text{rd}}}
\newcommand{\paren}[1]{\left(#1\right)}
\newcommand{\abs}[1]{\left|#1\right|}
\newcommand{\R}{\mathbb{R}}
\newcommand{\C}{\mathbb{C}}
\newcommand{\Hilb}{\mathbb{H}}
\newcommand{\qq}[1]{\text{#1}}
\newcommand{\Z}{\mathbb{Z}}
\newcommand{\N}{\mathbb{N}}
\newcommand{\q}[1]{\text{``#1''}}
%\newcommand{\mat}[1]{\begin{bmatrix}#1\end{bmatrix}}
\newcommand{\rref}{\text{reduced row echelon form}}
\newcommand{\ef}{\text{echelon form}}
\newcommand{\ohm}{\Omega}
\newcommand{\volt}{\text{V}}
\newcommand{\amp}{\text{A}}
\newcommand{\Seq}{\textbf{Seq}}
\newcommand{\Poly}{\textbf{P}}
\renewcommand{\quad}{\text{    }}
\newcommand{\roweq}{\simeq}
\newcommand{\rowop}{\simeq}
\newcommand{\rowswap}{\leftrightarrow}
\newcommand{\Mat}{\textbf{M}}
\newcommand{\Func}{\textbf{Func}}
\newcommand{\Hw}{\textbf{Hamming weight}}
\newcommand{\Hd}{\textbf{Hamming distance}}
\newcommand{\rank}{\text{rank}}
\newcommand{\longvect}[1]{\overrightarrow{#1}}
% Define the circled command
\newcommand{\circled}[1]{%
  \tikz[baseline=(char.base)]{
    \node[shape=circle,draw,inner sep=2pt,red,fill=red!20,text=black] (char) {#1};}%
}

% Define custom command \strikeh that just puts red text on the 2nd argument
\newcommand{\strikeh}[2]{\textcolor{red}{#2}}

% Define custom command \strikev that just puts red text on the 2nd argument
\newcommand{\strikev}[2]{\textcolor{red}{#2}}

%more new commands for this doc for errors in copying
\newcommand{\SI}{\text{SI}}
\newcommand{\kg}{\text{kg}}
\newcommand{\m}{\text{m}}
\newcommand{\s}{\text{s}}
\newcommand{\norm}[1]{\left\|#1\right\|}
\newcommand{\col}{\text{col}}
\newcommand{\sspan}{\text{span}}
\newcommand{\proj}{\text{proj}}
\newcommand{\set}[1]{\left\{#1\right\}}
\newcommand{\degC}{^\circ\text{C}}
\newcommand{\centroid}[1]{\overline{#1}}
\newcommand{\dotprod}{\boldsymbol{\cdot}}
%\newcommand{\coord}[1]{\begin{bmatrix}#1\end{bmatrix}}
\newcommand{\iprod}[1]{\langle #1 \rangle}
\newcommand{\adjoint}{^{*}}
\newcommand{\conjugate}[1]{\overline{#1}}
\newcommand{\eigenvarA}{\lambda}
\newcommand{\eigenvarB}{\mu}
\newcommand{\orth}{\perp}
\newcommand{\bigbracket}[1]{\left[#1\right]}
\newcommand{\textiff}{\text{ if and only if }}
\newcommand{\adj}{\text{adj}}
\newcommand{\ijth}{\emph{ij}^\text{th}}
\newcommand{\minor}[2]{M_{#2}}
\newcommand{\cofactor}{\text{C}}
\newcommand{\shift}{\textbf{shift}}
\newcommand{\startmat}[1]{
  \left[\begin{array}{#1}
}
\newcommand{\stopmat}{\end{array}\right]}
%a command to give a name to explorations and hints and theorems
\newcommand{\name}[1]{\begin{centering}\textbf{#1}\end{centering}}
\newcommand{\vect}[1]{\vec{#1}}
\newcommand{\dfn}[1]{\textbf{#1}}
\newcommand{\transpose}{\mathsf{T}}
\newcommand{\mtlb}[2][black]{\texttt{\textcolor{#1}{#2}}}
\newcommand{\RR}{\mathbb{R}} % Real numbers
\newcommand{\id}{\text{id}}
\newcommand{\coord}[1]{\langle#1\rangle}
\newcommand{\RREF}{\text{RREF}}
\newcommand{\Null}{\text{Null}}
\newcommand{\Nullity}{\text{Nullity}}
\newcommand{\Rank}{\text{Rank}}
\newcommand{\Col}{\text{Col}}
\newcommand{\Ef}{\text{EF}}
\newcommand{\boxprod}[3]{\abs{(#1\times#2)\cdot#3}}

\author{Zack Reed}
%borrowed from selinger linear algebra
\title{The Characteristic Equation: Eigenvectors}
\begin{document}
\begin{abstract}

\end{abstract}
\maketitle

    
\subsection*{Eigenvectors}
Once we have computed an eigenvalue $\lambda$ of an $n \times n$ matrix $A$, the next step is to compute the associated eigenvectors.  In other words, we seek vectors $\vec{x}$ such that $A\vec{x}=\lambda \vec{x}$, or equivalently,
\begin{equation}\label{eqn:nullspace}
    (A-\lambda I) \vec{x}=\vec{0}  
\end{equation}
For any given eigenvalue $\lambda$ there are infinitely many eigenvectors associated with it. This underlies the notion of an eigenspace that was defined earlier. We again state the definition below.

    
\begin{definition}\label{def:eigspace}
The set of all eigenvectors associated with a given eigenvalue of a matrix is known as the \dfn{eigenspace} associated with that eigenvalue, and is a subspace of $\RR^n$.
\end{definition}
    
So given an eigenvalue $\lambda$, there is an associated eigenspace $\mathcal{S}$, and our goal is to find a basis of $\mathcal{S}$, for then any eigenvector $\vec{x}$ will be a linear combination of the vectors in that basis.  Moreover, we are trying to find a basis for the set of vectors that satisfy Equation \ref{eqn:nullspace}, which means we seek a basis for $\mbox{null}(A-\lambda I)$. 
    
Let's return to our previous examples to find eigenvectors for the matrices $A$, $B$, $C$, and $D$.
    
\begin{example}\label{ex:eigvect2x2eig} (Finding eigenvectors for Example \ref{ex:2x2eig} )
    
Recall that $A=\begin{bmatrix} 2& 1\\ 1&2
\end{bmatrix}$ has eigenvalues $\lambda_1=1$ and $\lambda_2=3$.  Compute a basis for the eigenspace associated with each of these eigenvalues.
\begin{explanation}
Eigenvectors associated with the eigenvalue $\lambda_1=1$ are in the null space of $A-I$.  So we seek a basis for $\mbox{null}(A-I)$.  We compute:
\begin{align*}\mbox{rref}(A-I)=\mbox{rref}\left(\begin{bmatrix}1&1\\1&1\end{bmatrix}\right)&=\begin{bmatrix}\answer{1}&\answer{1}\\\answer{0}&\answer{0}\end{bmatrix},
\end{align*}
From this we see that the eigenspace $\mathcal{S}_1$ associated with $\lambda_1=1$ consists of vectors of the form $\begin{bmatrix}\answer{-1}\\\answer{1}\end{bmatrix}t$.

This means that $\left\{\begin{bmatrix}\answer{-1}\\\answer{1}\end{bmatrix}\right\}$
is one possible basis for $\mathcal{S}_1$.
    
In a similar way, we compute a basis for $\mathcal{S}_3$, the subspace of all eigenvectors associated with the eigenvalue $\lambda_2=3$.  Now we compute:
\begin{align*}\mbox{rref}(A-3I)=\mbox{rref}\left(\begin{bmatrix}-1&1\\1&-1\end{bmatrix}\right)&=\begin{bmatrix}\answer{1}&\answer{-1}\\\answer{0}&\answer{0}\end{bmatrix},
\end{align*}
Vectors in the null space have the form $\begin{bmatrix}\answer{1}\\\answer{1}\end{bmatrix}t$.  So one possible basis for the eigenspace $\mathcal{S}_3$ is given by $\left\{\begin{bmatrix}\answer{1}\\\answer{1}\end{bmatrix}\right\}$.
\end{explanation}
\end{example}
    
\begin{example}\label{ex:eigvectors2x2eig2} (Finding eigenvectors for Example \ref{ex:2x2eig2})
We know from Example \ref{ex:2x2eig2} that $B=\begin{bmatrix} 2& 1\\ 4&2
\end{bmatrix}$ has eigenvalues $\lambda_1=0$ and $\lambda_2=4$.  Compute a basis for the eigenspace associated with each of these eigenvalues.
\begin{explanation}
Let's begin by finding a basis for the eigenspace $\mathcal{S}_0$, which is the subspace of $\RR^n$ consisting of eigenvectors corresponding to the eigenvalue $\lambda_1=0$.  We need to compute a basis for $\mbox{null}(B-0I) = \mbox{null}(B)$.  We compute:
\begin{align*}\mbox{rref}(B)=\mbox{rref}\left(\begin{bmatrix}2&1\\4&2\end{bmatrix}\right)&=\begin{bmatrix}\answer{1}&\answer{1/2}\\0&0\end{bmatrix},
\end{align*}
From this we see that an eigenvector in $\mathcal{S}_0$ has the form $\begin{bmatrix}\answer{-1/2}\\\answer{1}\end{bmatrix}t$.
This means that $\left\{\begin{bmatrix}\answer{-1/2}\\\answer{1}\end{bmatrix}\right\}$
     is one possible basis for the eigenspace $\mathcal{S}_0$.  By letting $t=-2$, we obtain an arbitrary but arguably nicer-looking basis: $\left\{\begin{bmatrix}\answer{1}\\\answer{-2}\end{bmatrix}\right\}$.

    
To compute a basis for $\mathcal{S}_4$, the subspace of all eigenvectors associated to the eigenvalue $\lambda_2=4$, we compute:
$$\mbox{rref}(B-4I)=\mbox{rref}\left(\begin{bmatrix}-2&1\\4&-2\end{bmatrix}\right)=\begin{bmatrix}\answer{1}&\answer{-1/2}\\0&0\end{bmatrix}$$

    
From this we find that $\left\{\begin{bmatrix}1\\\answer{2}\end{bmatrix}\right\}$ is one possible basis for the eigenspace $\mathcal{S}_4$.
\end{explanation}
\end{example}
    
\begin{example}\label{ex:eigvectors3x3eig} (Finding eigenvectors for Example \ref{ex:3x3eig})
    We know from Example \ref{ex:3x3eig} that $C=\begin{bmatrix} 2 & 1 & 1\\ 1 & 2 & 1\\ 1 & 1 & 2\end{bmatrix}$ has eigenvalues $\lambda_1=1$ and $\lambda_2=4$. Compute a basis for the eigenspace associated to each of these eigenvalues.
    \begin{explanation}
    We first find a basis for the eigenspace $\mathcal{S}_1$.  We need to compute a basis for $\mbox{null}(C-\answer{1}I)$.  We compute:
    \begin{align*}
    \mbox{rref}(C-\answer{1}I) &= \mbox{rref}\left(\begin{bmatrix} \answer{1} & \answer{1} & \answer{1}\\ \answer{1} & \answer{1} & \answer{1}\\ \answer{1} & \answer{1} & \answer{1}\end{bmatrix}\right) \\
    &= \begin{bmatrix} \answer{1} & \answer{1} & \answer{1}\\ 0 & 0 & 0\\ 0 & 0 & 0\end{bmatrix},
    \end{align*}
    Notice that there are two free variables. The eigenvectors in $\mathcal{S}_1$ have the form
    $$\begin{bmatrix}-s-t\\s\\t\end{bmatrix} = s\begin{bmatrix}-\answer{1}\\\answer{1}\\\answer{0}\end{bmatrix} + t\begin{bmatrix}-\answer{1}\\\answer{0}\\\answer{1}\end{bmatrix}$$
        
    So one possible basis for the eigenspace $\mathcal{S}_1$ is given by $\left\{\begin{bmatrix}-\answer{1}\\\answer{1}\\\answer{0}\end{bmatrix}, \begin{bmatrix}-\answer{1}\\\answer{0}\\\answer{1}\end{bmatrix}\right\}$.
        
    Next we find a basis for the eigenspace $\mathcal{S}_4$.  We need to compute a basis for $\mbox{null}(C-\answer{4}I)$.  We compute:
    \begin{align*}
    \mbox{rref}(C-\answer{4}I) &= \mbox{rref}\left(\begin{bmatrix} -\answer{2} & \answer{1} & \answer{1}\\ \answer{1} & -\answer{2} & \answer{1}\\ \answer{1} & \answer{1} & -\answer{2}\end{bmatrix}\right) \\
    &= \begin{bmatrix} \answer{1} & \answer{0} & -\answer{1}\\ 0 & \answer{1} & -\answer{1}\\ 0 & 0 & 0\end{bmatrix}
    \end{align*}
    This time there is one free variable. The eigenvectors in $\mathcal{S}_4$ have the form $\begin{bmatrix}\answer{1}\\\answer{1}\\\answer{1}\end{bmatrix}t$, so a possible basis for the eigenspace $\mathcal{S}_4$ is given by $\left\{\begin{bmatrix}\answer{1}\\\answer{1}\\\answer{1}\end{bmatrix}\right\}$.
    \end{explanation}
    \end{example}
    
    \begin{example}\label{ex:3x3_complex_ev} (Finding eigenvectors for Example \ref{ex:3x3_complex_eig})
        We know from Example \ref{ex:3x3_complex_eig} that $D=\begin{bmatrix} 0&0&0\\ 0 &1&1\\ 0 & -1&1\end{bmatrix}$ has eigenvalues $\lambda=0$, $\lambda_1=1+i$, and $\lambda_2=1-i$.  Compute a basis for the eigenspace associated with each eigenvalue.
        \begin{explanation}
        We first find a basis for the eigenspace $\mathcal{S}_0$.  We need to compute a basis for $\mbox{null}(D-\answer{0}I)=\mbox{null}(D)$.  We compute:
        \begin{align*}
        \mbox{rref}(D) &= \mbox{rref}\left(\begin{bmatrix} 0 & 0 & 0\\ 0 & 1 & 1\\ 0 & -1 & 1\end{bmatrix}\right) \\
        &= \begin{bmatrix} \answer{0} & \answer{1} & \answer{1}\\ \answer{0} & \answer{0} & \answer{1}\\ \answer{0} & \answer{0} & \answer{0}\end{bmatrix},
        \end{align*}
        From this we see that for any eigenvector $\begin{bmatrix}x_1\\x_2\\x_3\end{bmatrix}$ in $\mathcal{S}_0$ we have $x_2=\answer{0}$ and $x_3=\answer{0}$, but $x_1$ is a free variable.
        So one possible basis for the eigenspace $\mathcal{S}_0$ is given by $$\left\{\begin{bmatrix}\answer{1}\\\answer{0}\\\answer{0}\end{bmatrix}\right\}$$

        Next we find a basis for the eigenspace $\mathcal{S}_{1+i}$.  We need to compute a basis for $\mbox{null}(D-(1+i)I)$.  We compute:
        \begin{align*}
        \mbox{rref}(D-(1+i)I) &= \mbox{rref}\left(\begin{bmatrix} -(1+i)&0&0\\ 0 &1-(1+i)&1\\ 0 & -1&1-(1+i)\end{bmatrix}\right) \\
        &= \begin{bmatrix} \answer{1} & \answer{0} &\answer{0}\\ \answer{0} & \answer{1} & i\\ \answer{0} & \answer{0} & \answer{0}\end{bmatrix}
        \end{align*}
        There is one free variable.  Setting $x_3=t$, we get $x_1=\answer{0}$ and $x_2=ti$.  From this we see that eigenvectors in $\mathcal{S}_{1+i}$ have the form $\begin{bmatrix}\answer{0}\\i\\\answer{1}\end{bmatrix}t$, so a possible basis for the eigenspace $\mathcal{S}_{1+i}$ is given by $\left\{\begin{bmatrix}\answer{0}\\i\\\answer{1}\end{bmatrix}\right\}$.
        \end{explanation}
        \end{example}
    
We conclude this section by establishing the significance of a matrix having an eigenvalue of zero.
    
\begin{theorem}\label{th:zero_ew}
A square matrix has an eigenvalue of zero if and only if it is singular.
\end{theorem}
    
\begin{proof}
A square matrix $A$ is singular if and only if $\det{A}=0$.  But $\det{A}=0$ if and only if $\det{A-0I}=0$, which is true if and only if zero is an eigenvalue of $A$.
\end{proof}

    

\end{document}