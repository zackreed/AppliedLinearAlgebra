\documentclass{ximera}
\graphicspath{  %% When looking for images,
{./}            %% look here first,
{./pictures/}   %% then look for a pictures folder,
{../pictures/}  %% which may be a directory up.
{../../pictures/}  %% which may be a directory up.
{../../../pictures/}  %% which may be a directory up.
{../../../../pictures/}  %% which may be a directory up.
}

\usepackage{listings}
%\usepackage{circuitikz}
\usepackage{xcolor}
\usepackage{amsmath,amsthm}
\usepackage{subcaption}
\usepackage{graphicx}
\usepackage{tikz}
%\usepackage{tikz-3dplot}
\usepackage{amsfonts}
%\usepackage{mdframed} % For framing content
%\usepackage{tikz-cd}

  \renewcommand{\vector}[1]{\left\langle #1\right\rangle}
  \newcommand{\arrowvec}[1]{{\overset{\rightharpoonup}{#1}}}
  \newcommand{\ro}{\texttt{R}}%% row operation
  \newcommand{\dotp}{\bullet}%% dot product
  \renewcommand{\l}{\ell}
  \let\defaultAnswerFormat\answerFormatBoxed
  \usetikzlibrary{calc,bending}
  \tikzset{>=stealth}
  




%make a maroon color
\definecolor{maroon}{RGB}{128,0,0}
%make a dark blue color
\definecolor{darkblue}{RGB}{0,0,139}
%define the color fourier0 to be the maroon color
\definecolor{fourier0}{RGB}{128,0,0}
%define the color fourier1 to be the dark blue color
\definecolor{fourier1}{RGB}{0,0,139}
%define the color fourier 1t to be the light blue color
\definecolor{fourier1t}{RGB}{173,216,230}
%define the color fourier2 to be the dark green color
\definecolor{fourier2}{RGB}{0,100,0}
%define teh color fourier2t to be the light green color
\definecolor{fourier2t}{RGB}{144,238,144}
%define the color fourier3 to be the dark purple color
\definecolor{fourier3}{RGB}{128,0,128}
%define the color fourier3t to be the light purple color
\definecolor{fourier3t}{RGB}{221,160,221}
%define the color fourier0t to be the red color
\definecolor{fourier0t}{RGB}{255,0,0}
%define the color fourier4 to be the orange color
\definecolor{fourier4}{RGB}{255,165,0}
%define the color fourier4t to be the darker orange color
\definecolor{fourier4t}{RGB}{255,215,0}
%define the color fourier5 to be the yellow color
\definecolor{fourier5}{RGB}{255,255,0}
%define the color fourier5t to be the darker yellow color
\definecolor{fourier5t}{RGB}{255,255,100}
%define the color fourier6 to be the green color
\definecolor{fourier6}{RGB}{0,128,0}
%define the color fourier6t to be the darker green color
\definecolor{fourier6t}{RGB}{0,255,0}

%New commands for this doc for errors in copying
\newcommand{\eigenvar}{\lambda}
%\newcommand{\vect}[1]{\mathbf{#1}}
\renewcommand{\th}{^{\text{th}}}
\newcommand{\st}{^{\text{st}}}
\newcommand{\nd}{^{\text{nd}}}
\newcommand{\rd}{^{\text{rd}}}
\newcommand{\paren}[1]{\left(#1\right)}
\newcommand{\abs}[1]{\left|#1\right|}
\newcommand{\R}{\mathbb{R}}
\newcommand{\C}{\mathbb{C}}
\newcommand{\Hilb}{\mathbb{H}}
\newcommand{\qq}[1]{\text{#1}}
\newcommand{\Z}{\mathbb{Z}}
\newcommand{\N}{\mathbb{N}}
\newcommand{\q}[1]{\text{``#1''}}
%\newcommand{\mat}[1]{\begin{bmatrix}#1\end{bmatrix}}
\newcommand{\rref}{\text{reduced row echelon form}}
\newcommand{\ef}{\text{echelon form}}
\newcommand{\ohm}{\Omega}
\newcommand{\volt}{\text{V}}
\newcommand{\amp}{\text{A}}
\newcommand{\Seq}{\textbf{Seq}}
\newcommand{\Poly}{\textbf{P}}
\renewcommand{\quad}{\text{    }}
\newcommand{\roweq}{\simeq}
\newcommand{\rowop}{\simeq}
\newcommand{\rowswap}{\leftrightarrow}
\newcommand{\Mat}{\textbf{M}}
\newcommand{\Func}{\textbf{Func}}
\newcommand{\Hw}{\textbf{Hamming weight}}
\newcommand{\Hd}{\textbf{Hamming distance}}
\newcommand{\rank}{\text{rank}}
\newcommand{\longvect}[1]{\overrightarrow{#1}}
% Define the circled command
\newcommand{\circled}[1]{%
  \tikz[baseline=(char.base)]{
    \node[shape=circle,draw,inner sep=2pt,red,fill=red!20,text=black] (char) {#1};}%
}

% Define custom command \strikeh that just puts red text on the 2nd argument
\newcommand{\strikeh}[2]{\textcolor{red}{#2}}

% Define custom command \strikev that just puts red text on the 2nd argument
\newcommand{\strikev}[2]{\textcolor{red}{#2}}

%more new commands for this doc for errors in copying
\newcommand{\SI}{\text{SI}}
\newcommand{\kg}{\text{kg}}
\newcommand{\m}{\text{m}}
\newcommand{\s}{\text{s}}
\newcommand{\norm}[1]{\left\|#1\right\|}
\newcommand{\col}{\text{col}}
\newcommand{\sspan}{\text{span}}
\newcommand{\proj}{\text{proj}}
\newcommand{\set}[1]{\left\{#1\right\}}
\newcommand{\degC}{^\circ\text{C}}
\newcommand{\centroid}[1]{\overline{#1}}
\newcommand{\dotprod}{\boldsymbol{\cdot}}
%\newcommand{\coord}[1]{\begin{bmatrix}#1\end{bmatrix}}
\newcommand{\iprod}[1]{\langle #1 \rangle}
\newcommand{\adjoint}{^{*}}
\newcommand{\conjugate}[1]{\overline{#1}}
\newcommand{\eigenvarA}{\lambda}
\newcommand{\eigenvarB}{\mu}
\newcommand{\orth}{\perp}
\newcommand{\bigbracket}[1]{\left[#1\right]}
\newcommand{\textiff}{\text{ if and only if }}
\newcommand{\adj}{\text{adj}}
\newcommand{\ijth}{\emph{ij}^\text{th}}
\newcommand{\minor}[2]{M_{#2}}
\newcommand{\cofactor}{\text{C}}
\newcommand{\shift}{\textbf{shift}}
\newcommand{\startmat}[1]{
  \left[\begin{array}{#1}
}
\newcommand{\stopmat}{\end{array}\right]}
%a command to give a name to explorations and hints and theorems
\newcommand{\name}[1]{\begin{centering}\textbf{#1}\end{centering}}
\newcommand{\vect}[1]{\vec{#1}}
\newcommand{\dfn}[1]{\textbf{#1}}
\newcommand{\transpose}{\mathsf{T}}
\newcommand{\mtlb}[2][black]{\texttt{\textcolor{#1}{#2}}}
\newcommand{\RR}{\mathbb{R}} % Real numbers
\newcommand{\id}{\text{id}}
\newcommand{\coord}[1]{\langle#1\rangle}
\newcommand{\RREF}{\text{RREF}}
\newcommand{\Null}{\text{Null}}
\newcommand{\Nullity}{\text{Nullity}}
\newcommand{\Rank}{\text{Rank}}
\newcommand{\Col}{\text{Col}}
\newcommand{\Ef}{\text{EF}}
\newcommand{\boxprod}[3]{\abs{(#1\times#2)\cdot#3}}

\author{Zack Reed}
%borrowed from selinger linear algebra
\title{Diagonalization}
\begin{document}
\begin{abstract}

\end{abstract}
\maketitle

\section*{Diagonalization: Similarity and Eigenconnections}

\subsection*{Diagonalization}

We ended last chapter by exploring matrix similarity and its connections to eigenvectors and eigenvalues. Specifically, two matrices $A$ and $B$ are similar if for some invertible matrix, $B=P^{-1}*A*P$. We also learned that similar matrices share the same eigenvalues. We now further examine similarity and its connection to eigentheory, specifically for matrices that are similar to a diagonal matrix. 

\begin{definition}\name{Diagonalizability}
  A matrix $A$ is \emph{diagonalizable} if it is similar to a diagonal matrix, $D$. 

  That is, $A$ is said to be \dfn{diagonalizable} if there exists an invertible matrix $P$ such that
\begin{equation*}
P^{-1}AP=D
\end{equation*}
where $D$ is a diagonal matrix.
\end{definition}

Similarity of matrices in and of itself is not necessarily always useful, but similarity with diagonal matrices in particular come with some nice properties, and exploring these properties has important implications for various other concepts, such as the SVD.

First, here are two videos by Grant Sanderson that set the stage for the connections between eigentheory and diagonalization. This first video is on change of basis, which we've seen before informally with the SVD, but he now goes into a little more detail about how to think about change of basis and connect to similarity.

\begin{center}
\youtube{P2LTAUO1TdA?si=5_IxPtqQ97cUMBdr}
\end{center}

Now, we re-visit his video on eigentheory. Feel free to watch the whole thing as a review, but the video only covers new material on diagonalization after 13:03.

\begin{center}
\youtube{PFDu9oVAE-g?si=a6C_O0_jfXJk8Z32}
\end{center}

\begin{exploration}

    Since $P^{-1}AP=D$, we can left-multiply by $P$ to get $AP=PD$. Let's call the columns of $P$ $\vec{x}_i$ and the diagonal entries of $D$ $\lambda_i$. Thinking about this multiplication column-wise, we get

    $$A\vec{x}_i=\lambda_i\vec{x}_i.$$

    This should look familiar! If $A$ is diagonalizable, then $\vec{x}_i$ are eigenvectors of $A$ and $\lambda_i$ are eigenvalues of $A$! Moreoer, since $P$ is invertible, it has full rank, and so we get a complete reconstruction of $A$ entirely in terms of eigenvectors and eigenvalues. In fact, some simple right-multiplication by $P^{-1}$ re-states diagonalizability as

    $$A=PDP^{-1}.$$


\end{exploration}

This is another decomposition, in fact very similar to the SVD, where $A$ is factored into matrices formed by its eigenvalues and eigenvectors. This is actually a very restrictive theorem:

\begin{theorem}

  An $n\times n$ matrix $A$ is diagonalizable only when its eigenvectors form an $n\times n$ invertible matrix. When this happens, $A$ is similar by $P$ to a diagonal matrix $D$ whose entries are its eigenvalues.

  Said differently, a matrix $A$ is only diagonalizable when it has $n$ linearly independent eigenvectors. We call the decomposition $PDP^{-1}$ an \emph{eigen decomposition} of $A$.

\end{theorem}

Note, diagonalizability is much more restrictive than the SVD in that it only applies to square matrices with $n$ linearly independent eigenvectors. This is a trade off and both decompositions are useful. The broad applicability of the SVD (and its numerous applications) are very useful, but often we utilize the very nice properties of diagonalizable matrices for other purposes. We will see this in the next section. 

\begin{example}\label{ex:diagonalizematrix}
  Let
  \begin{equation*}
  A=\begin{bmatrix}
  2 & 0 & 0 \\
  1 & 4 & -1 \\
  -2 & -4 & 4
  \end{bmatrix}
  \end{equation*}
   Find an invertible matrix $P$ and a diagonal matrix $D$ such that $P^{-1}AP=D$.
   
  \begin{explanation}
  We will use eigenvectors of $A$ as the columns of $P$, and
  the corresponding eigenvalues of $A$ as the diagonal entries of $D$. Proceeding in MATLAB as we did in the previous chapter, we find the eigenvalues of $A$ are $\lambda_1 =2,\lambda_2 = 2$, and $\lambda_3 = 6$. 
   
  One possible basis for $\mathcal{S}_2$, the eigenspace corresponding to $2$, is
  $\left\{
  \begin{bmatrix}
  -2 \\
  1 \\
  0
  \end{bmatrix},
  \begin{bmatrix}
  1 \\
  0 \\
  1
  \end{bmatrix}
  \right\}$,
  while a basis for $\mathcal{S}_6$ is given by
  $\left\{\begin{bmatrix}
  0 \\
  1 \\
  -2
  \end{bmatrix}\right\}$.
   
  We construct the matrix $P$ by using these basis elements as columns.
  \begin{equation*}
  P=\begin{bmatrix}
  -2 & 1 & 0 \\
  1 & 0 & 1 \\
  0 & 1 & -2
  \end{bmatrix}
  \end{equation*}
  
  Checking using on of many techniques in MATLAB reveals that $P$ is invertible, with

  \begin{equation*}
  P^{-1}=\begin{bmatrix}
  -1/4 & 1/2 & 1/4 \\
  1/2 & 1 & 1/2 \\
  1/4 & 1/2 & -1/4
  \end{bmatrix}
  \end{equation*}
  Thus,
  \begin{eqnarray*}
  P^{-1}AP &=&\begin{bmatrix}
  -1/4 & 1/2 & 1/4 \\
  1/2 & 1 & 1/2 \\
  1/4 & 1/2 & -1/4
  \end{bmatrix} \begin{bmatrix}
  2 & 0 & 0 \\
  1 & 4 & -1 \\
  -2 & -4 & 4
  \end{bmatrix} \begin{bmatrix}
  -2 & 1 & 0 \\
  1 & 0 & 1 \\
  0 & 1 & -2
  \end{bmatrix} \\
  &=&\begin{bmatrix}
  2 & 0 & 0 \\
  0 & 2 & 0 \\
  0 & 0 & 6
  \end{bmatrix}
  \end{eqnarray*}
  You can see that the result here is a diagonal matrix where the entries on the main diagonal are the eigenvalues of $A$. Notice that eigenvalues on the main diagonal {\it must} be in the same order as the corresponding eigenvectors in $P$.
  \end{explanation}
  \end{example}
   
  It is often easier to work with matrices that are diagonalizable, as the next Exploration demonstrates. 
   
  \begin{exploration}\name{Powers of Diagonalizable Matrices}

  Let $A=\begin{bmatrix}
  2 & 0 & 0 \\
  1 & 4 & -1 \\
  -2 & -4 & 4
  \end{bmatrix}$ and let $D=\begin{bmatrix}
  2 & 0 & 0 \\
  0 & 2 & 0 \\
  0 & 0 & 6
  \end{bmatrix}$. 
  Would it be easier to compute $A^5$ or $D^5$ if you had to do so by hand, without a computer?  Certainly $D^5$ is easier, due to the number of zero entries!  
   
  In fact, for any diagonal matrix $D$ and an integer power $n$, $D^n$ is just a diagonal matrix whose diagonal entries are the $nth$ power of the diagonal entries of $D$.
   
  Since $A$ is not diagonal, a standard calculation of $A^5$ requires many more calculations.  However, $A$ is similar to $D$, and we can use this to make our computation easier.  This is because
  \begin{align*}
      A^5&=\left(PDP^{-1}\right)^5 \\
         &=(PDP^{-1})(PDP^{-1})(PDP^{-1})(PDP^{-1})(PDP^{-1}) \\
         &=PD(P^{-1}P)D(P^{-1}P)D(P^{-1}P)D(P^{-1}P)DP^{-1} \\
         &=PD(I)D(I)D(I)D(I)DP^{-1} \\
         &=PD^5P^{-1}
  \end{align*}
  With this in mind, it is not as daunting to calculate $A^5$.  We can compute the product $PD^5$ quite easily since $D^5$ is diagonal.  That leaves just one product of $3 \times 3$ matrices to compute by hand to compute $A^5$.  And the savings in work would certainly be more pronounced for larger matrices or for powers larger that 5.
\end{exploration}
   
  Because matrix $A$ was diagonalizable, we were able to cut down on computations.  When we chose to work with $D$ and $P$ instead of $A$ we worked with the eigenvalues and eigenvectors of $A$.  Each column of $P$ is an eigenvector of $A$, and so we repeatedly made use of the following theorem (with $m=5$).
   
  \begin{theorem}\label{th:eigpowers}
  Let $A$ be an $n \times n$ matrix and suppose $A\vec{x}=\lambda \vec{x}$.  Then $A^m \vec{x} = \lambda^m \vec{x}$
  \end{theorem}
   
  Matrix $A$ had a repeated eigenvalue of 2.  The next theorem and corollary show that matrices which have \emph{distinct} eigenvalues (where none are repeated) have desirable properties.
   
  \begin{theorem}\label{th:linindepeigenvectors}
  Let $A$ be an $n\times n$ matrix, and suppose that $A$
  has distinct eigenvalues $\lambda_1, \lambda_2, \ldots, \lambda_m$.
  For each $i$, let $\vec{x}_i$ be a $\lambda_i$-eigenvector of $A$.
  Then $\{ \vec{x}_1, \vec{x}_2, \ldots, \vec{x}_m\}$ is
  linearly independent.
  \end{theorem}
   
  Since diagonalizability requires linearly independent eigenvectors, distinct eigenvalues ensures diagonalizability.
   
  \begin{corollary}\label{th:distincteigenvalues}
  Let $A$ be an $n \times n$ matrix and suppose it has $n$ distinct eigenvalues. Then it follows that $A$ is diagonalizable.
  \end{corollary}
   
  %The corollary tells us that many matrices can be diagonalized in this way.
  \begin{remark}
      Note that Corollary \ref{th:distincteigenvalues} is NOT an ``if and only if statement".  This means that if $A$ has repeated eigenvalues it is still sometimes possible to diagonalize $A$, as seen in Example \ref{ex:diagonalizematrix}.
  \end{remark}
   
  Not every matrix has an eigenvalue decomposition. Sometimes we cannot find an invertible matrix $P$ such that $P^{-1}AP=D$.  Consider the following example.
   
  \begin{example}\label{ex:impossiblediagonalize}
  Let
  \begin{equation*}
  A =
  \begin{bmatrix}
  1 & 1 \\
  0 & 1
  \end{bmatrix}
  \end{equation*}
  If possible, find an invertible matrix $P$ and a diagonal matrix $D$ so that $P^{-1}AP=D$.
   
   
  \begin{explanation}
  The characteristic equation of $A$ is fairly simply calculated and reveals that the eigenvalues of $A$ are $\lambda_1 =1$ and  $\lambda_2=1$.
  To find $P$, the next step would be to find a basis for the corresponding eigenspace $\mathcal{S}_1$.  We solve the equation $\left( A - \lambda I \right) \vec{x} = \vec{0}$.
  Writing this equation as an augmented matrix, we already have a matrix in row echelon form:
  \begin{equation*}
  \left[\begin{array}{cc|c} 
  0 & -1 & 0 \\
  0 & 0 & 0
   \end{array}\right]
  \end{equation*}
  We see that the eigenvectors in $\mathcal{S}_1$ are of the form
  $$
  \begin{bmatrix}
  t \\
  0
  \end{bmatrix}
  =t\begin{bmatrix}
  1 \\
  0
  \end{bmatrix},
  $$
  so a basis for the eigenspace $\mathcal{S}_1$ is given by
  $\begin{bmatrix}
  1 \\
  0
  \end{bmatrix}$.
  It is easy to see that we cannot form an invertible matrix $P$, because any two eigenvectors will be of the form
  $\begin{bmatrix}
  t \\
  0
  \end{bmatrix}$,
  and so the second row of $P$ would have a row of zeros, and $P$ could not be invertible.  Hence $A$ cannot be diagonalized.
  \end{explanation}
  \end{example}
  
  Not all hope is lost for matrices with repeated eigenvalues, however. We do have a farily useful test to determine if matrices with repeated eigenvalues are diagonalizable.
   
  Recall the \dfn{algebraic multiplicity} of an eigenvalue $\lambda$ is the number of times that it occurs as a root of the characteristic polynomial and the \dfn{geometric multiplicity} of an eigenvalue $\lambda$ is the dimension of the corresponding eigenspace $\mathcal{S}_\lambda$. These multiplicity measures give us a useful criterion for determining the diagonalizability of matrices. 
   
  \begin{theorem}\label{th:diagonalizability}
  Let $A$ be an $n \times n$ matrix $A$. Then $A$ is diagonalizable if and only if for each eigenvalue $\lambda$ of $A$, the algebraic multiplicity of $\lambda$ is equal to the geometric multiplicity of $\lambda$.
  \end{theorem}
   
  \begin{proof}
  Suppose $A$ is diagonalizable and let $\lambda_1, \ldots, \lambda_t$ be the distinct eigenvalues of $A$, with algebraic multiplicities $m_1, \ldots, m_t$, respectively and geometric multiplicities $k_1, \ldots, k_t$, respectively.  Since $A$ is diagonalizable, Theorem \ref{th:eigenvectorsanddiagonalizable} implies that $ k_1+\cdots+k_t=n$.  By applying Lemma \ref{lemma:dimeigenspace} $t$ times, we have
  $$n = k_1+\cdots+k_t \le m_1+\cdots+m_t = n,$$
  which is only possible if $k_i=m_i$ for $i=1,\ldots,t$.
   
  Conversely, if the geometric multiplicity equals the algebraic multiplicity of each eigenvalue, then obtaining a basis for each eigenspace yields $n$ eigenvectors.  Applying Theorem \ref{th:linindepeigenvectors}, we know that these $n$ eigenvectors are linearly independent, so Theorem \ref{th:eigenvectorsanddiagonalizable} implies that $A$ is diagonalizable.
  \end{proof}



\end{document}