\documentclass{ximera}
\graphicspath{     %% setup a global graphics path
{./}               %% look in the same-level directory
{./pictures/}      %% look in graphics
{../pictures/}     %% look up one directory, then in graphics
%{../../pictures/} %% look up two directories, then in graphics
}

\author{Zack Reed}
%borrowed from anna davis
\title{SVD Revisited}
\begin{document}
\begin{abstract}

\end{abstract}
\maketitle

\section*{Redefining the SVD}
We return to the Singular Value Decomposition to make some important connections between the SVD and eigenvectors and eigenvalues. This will help further uncover the reason that each matrix has an SVD.
 
\begin{definition}\label{singularvalues}
Let $A$ be an $m\times n$ matrix. The \dfn{singular values} of $A$ are the square roots of the positive
eigenvalues of $A^TA.$
\end{definition}
 
Noting that $A^TA$ is always a real symmetric matrix, hopefully the last two sections have primed you to anticipate all of the nice properties that $A^TA$ will have.

The Singular Value Decomposition (SVD) can be thought of as
a generalization of orthogonal diagonalization of a symmetric matrix
to an arbitrary $m\times n$ matrix.
 
\begin{lemma}\label{lem:samenonzeroeigenvalues}
Let $A$ be an $m \times n$ matrix. Then $A^TA$ and $AA^T$ have the same \textbf{nonzero eigenvalues}.
\end{lemma}
 
Suppose $A$ is an $m\times n$ matrix, and suppose that  $\lambda$ is a nonzero eigenvalue of $A^TA$.
Then there exists a nonzero vector $\vec{x} \in \RR^n$ such that
\begin{equation}\label{nonzero}
(A^TA)\vec{x}=\lambda \vec{x}
\end{equation}
Multiplying both sides of this equation by $A$ yields:
\begin{eqnarray*}
A(A^TA)\vec{x} & = & A\lambda \vec{x}\\
(AA^T)(A\vec{x}) & = & \lambda (A\vec{x})
\end{eqnarray*}
Since $\lambda\neq 0$ and $\vec{x}\neq 0_n$, $\lambda \vec{x}\neq 0_n$,
and thus by equation~(\ref{nonzero}),
$(A^TA)\vec{x}\neq 0_m$; thus $A^T(A\vec{x})\neq 0_m$,
implying that $A\vec{x}\neq 0_m$.
 
Therefore $A\vec{x}$ is an eigenvector of $AA^T$ corresponding to eigenvalue
$\lambda$.  An analogous argument can be used to show that every
nonzero eigenvalue of $AA^T$ is an eigenvalue of $A^TA$.

 
Given an $m\times n$ matrix $A$, recall that the SVD of a matrix $A$ is the product
\[ A=US V^T.\]

We will see that in addition to all of the nice properties of $U$, $S$, and $V$ noted in Chapter 6, we have the additional properties that

\begin{itemize}
\item $U$ is an $m\times m$ orthogonal matrix whose columns are
eigenvectors of $AA^T$.
\item $V$ is an $n\times n$ orthogonal matrix whose columns are
eigenvectors of $A^TA$.
\item $S$ is an $m\times n$ matrix whose only nonzero values
lie on its main diagonal, and are the singular values of $A$ (i.e. the square root of the eigenvalues of $A^TA$).
\end{itemize}

How can we find such a decomposition? We are aiming to decompose $A$ in the following form:
\begin{equation*}
A=U\left[
\begin{array}{cc}
S & 0 \\
0 & 0
\end{array}
\right] V^T
\end{equation*}
where $S $ is a block matrix of the form
\[
S =\left[
\begin{array}{ccc}
S _{1} &  & 0 \\
& \ddots &  \\
0 &  & S _{k}
\end{array}
\right]
\]
Thus $A^T=V\left[
\begin{array}{cc}
S & 0 \\
0 & 0
\end{array}
\right] U^T$ and it follows that
\begin{equation*}
A^TA=V\left[
\begin{array}{cc}
S & 0 \\
0 & 0
\end{array}
\right] U^TU\left[
\begin{array}{cc}
S & 0 \\
0 & 0
\end{array}
\right] V^T=V\left[
\begin{array}{cc}
S ^{2} & 0 \\
0 & 0
\end{array}
\right] V^T
\end{equation*}
and so $A^TAV=V\left[
\begin{array}{cc}
S ^{2} & 0 \\
0 & 0
\end{array}
\right] .$ Similarly, $AA^TU=U\left[
\begin{array}{cc}
S ^{2} & 0 \\
0 & 0
\end{array}
\right] .$ Therefore, you would find an orthonormal basis of eigenvectors
for $AA^T$ make them the columns of a matrix such that the
corresponding eigenvalues are decreasing. This gives $U.$ You could then do
the same for $A^TA$ to get $V$.
 
We formalize this discussion in the following theorem.
 
\begin{theorem}[Singular Value Decomposition]\label{th:singvaldecomp}
Let $A$ be an $m\times n$ matrix. Then the following are true: 

\begin{enumerate}
\item $A^TA$ is diagonalizable with an orthogonal eigenvector matrix $V$ and an eigenvalue matrix $S=\begin{bmatrix}
    \sigma^2&0\\0&0\end{bmatrix}$. 
\item $AA^T$ diagonalizable with an orthogonal eigenvector matrix $U$ and an eigenvalue matrix $S=\begin{bmatrix}
    \sigma^2&0\\0&0\end{bmatrix}$. 
\item The eigenvector matrices $U$ and $V$ are the left and right (respectively) singular vector matrices of the SVD, and the block matrix $S=\begin{bmatrix}
    \sigma&0\\0&0\end{bmatrix}$ give the singular values of $A$.
\end{enumerate}

\end{theorem}

This gives us a more readiliy accessible means of finding the SVD of a matrix $A$ than was discussed in \href{https://ximera.osu.edu/appliedlinearalgebra/c6ChapterSix/learningActivities/m6LearningActivities/leastSquares/PCASVDIntro}{Chapter 6}. Rather than iteratively maximizing $\norm{A\vec{v}}$ over orthogonal unit vectors $\vec{v}$, we simply compute $A^TA$ and find the guaranteed eigenvalues and eigenvectors. This also further clarifies why the SVD exists for \emph{any} matrix $A$, which was perhaps unclear from the algorithm discussed in \href{https://ximera.osu.edu/appliedlinearalgebra/c6ChapterSix/learningActivities/m6LearningActivities/leastSquares/PCASVDIntro}{Chapter 6}. That is, we didn't quite discuss whether or why the algorithm guaranteed a full set of singular values and singular vectors. We now, however, know for certain that $A^TA$ is orthogonally diagonalizable, which guarantees the existence and orthogonal properties of the SVD.
 
Let's compute the SVD of a simple matrix.
 
\begin{example}\label{ex:SVD2x3}
Let
$A=\left[\begin{array}{rrr} 1 & -1 & 3 \\ 3 & 1 & 1 \end{array}\right]$.
Find the SVD of $A$.
 
\begin{explanation}
To begin, we compute $AA^T$ and $A^TA$.
\[ AA^T = \left[\begin{array}{rrr} 1 & -1 & 3 \\ 3 & 1 & 1 \end{array}\right]
\left[\begin{array}{rr} 1 & 3 \\ -1 & 1 \\ 3 & 1  \end{array}\right]
= \left[\begin{array}{rr} 11 & 5 \\ 5 & 11  \end{array}\right]\]
 
\[ A^TA = \left[\begin{array}{rr} 1 & 3 \\ -1 & 1 \\ 3 & 1  \end{array}\right]
\left[\begin{array}{rrr} 1 & -1 & 3 \\ 3 & 1 & 1 \end{array}\right]
= \left[\begin{array}{rrr} 10 & 2 & 6 \\ 2 & 2 & -2\\
6 & -2 & 10 \end{array}\right]\]
Since $AA^T$ is $2\times 2$ while $A^T A$ is $3\times 3$, and $AA^T$
and $A^TA$ have the same {\em nonzero} eigenvalues (by Lemma
\ref{lem:samenonzeroeigenvalues}), we compute the characteristic polynomial  $c_{AA^T}(x)$ (because it is
easier to compute than $c_{A^TA}(x)$).
\begin{eqnarray*}
c_{AA^T}(z)& = &\det(zI-AA^T)=  \det \left[\begin{array}{cc}
z-11 & -5 \\ -5 & z-11 \end{array}\right]\\
& = &(z-11)^2 - 25 \\
& = & z^2-22z+121-25\\
& = & z^2-22z+96\\
& = & (z-16)(z-6)
\end{eqnarray*}
Therefore, the eigenvalues of $AA^T$ are $\lambda_1=16$ and $\lambda_2=6$.
 
The eigenvalues of $A^TA$ are $\lambda_1=16$, $\lambda_2=6$, and
$\lambda_3=0$, and the singular values of $A$ are $S_1=\sqrt{16}=4$ and
$S_2=\sqrt{6}$.
By convention, we list the eigenvalues (and corresponding singular values)
in non increasing order (i.e., from largest to smallest).
 
\textbf{To find the matrix $V$}:
 
To construct the matrix $V$ we need to find eigenvectors for $A^TA$.
Since the eigenvalues of $AA^T$ are distinct, the corresponding
eigenvectors are orthogonal, and we need only normalize them.
 
$\lambda_1=16$: solve $(16I-A^TA)\vec{x}_1= \vec{0}$.
 
\[ \left[\begin{array}{rrr|r}
6 & -2 & -6 & 0 \\ -2 & 14 & 2 & 0 \\ -6 & 2 & 6 & 0
\end{array}\right]
\rightarrow
\left[\begin{array}{rrr|r}
1 & 0 & -1 & 0 \\ 0 & 1 & 0 & 0 \\ 0 & 0 & 0 & 0
\end{array}\right],
\mbox{ so }
\vec{x}_1 =\left[\begin{array}{r} t \\ 0 \\ t \end{array}\right]
=t\left[\begin{array}{r} 1 \\ 0 \\ 1 \end{array}\right],
t\in \RR. \]
 
$\lambda_2=6$: solve $(6I-A^TA)\vec{x}_2= \vec{0}$.
 
\[ \left[\begin{array}{rrr|r}
-4 & -2 & -6 & 0 \\ -2 & 4 & 2 & 0 \\ -6 & 2 & -4 & 0
\end{array}\right]
\rightarrow
\left[\begin{array}{rrr|r}
1 & 0 & 1 & 0 \\ 0 & 1 & 1 & 0 \\ 0 & 0 & 0 & 0
\end{array}\right],
\mbox{ so }
\vec{x}_2=\left[\begin{array}{r} -s \\ -s \\ s \end{array}\right]
=s\left[\begin{array}{r} -1 \\ -1 \\ 1 \end{array}\right],
s\in \RR. \]
 
$\lambda_3=0$: solve $(-A^TA)\vec{x}_3= \vec{0}$.
\[ \left[\begin{array}{rrr|r}
-10 & -2 & -6 & 0 \\ -2 & -2 & 2 & 0 \\ -6 & 2 & -10 & 0
\end{array}\right]
\rightarrow
\left[\begin{array}{rrr|r}
1 & 0 & 1 & 0 \\ 0 & 1 & -2 & 0 \\ 0 & 0 & 0 & 0
\end{array}\right],
\mbox{ so }
\vec{x}_3=\left[\begin{array}{r} -r \\ 2r \\ r \end{array}\right]
=r\left[\begin{array}{r} -1 \\ 2 \\ 1 \end{array}\right],
r\in \RR. \]
Let
\[ \vec{v}_1=\frac{1}{\sqrt{2}}\left[\begin{array}{r} 1\\ 0\\ 1 \end{array}\right],
\vec{v}_2=\frac{1}{\sqrt{3}}\left[\begin{array}{r} -1\\ -1\\ 1 \end{array}\right],
\vec{v}_3=\frac{1}{\sqrt{6}}\left[\begin{array}{r} -1\\ 2\\ 1 \end{array}\right]\]
Then
\[ V=\frac{1}{\sqrt{6}}\left[\begin{array}{rrr}
\sqrt 3 & -\sqrt 2 & -1  \\
0 & -\sqrt 2 & 2 \\
\sqrt 3 & \sqrt 2 & 1 \end{array}\right]\]
Also,
\[ S = \left[\begin{array}{rrr} 4 & 0 & 0 \\
0 & \sqrt 6 & 0 \end{array}\right]\]
and we use $A$, $V^T$, and $S$ to find $U$. Since $V$ is orthogonal and $A=US V^T$, it follows that $AV=US$.
Let $V=\left[\begin{array}{ccc} \vec{v}_1 & \vec{v}_2 & \vec{v}_3 \end{array}\right]$, and
let $U=\left[\begin{array}{cc} \vec{u}_1 & \vec{u}_2 \end{array}\right]$, where
$\vec{u}_1$ and $\vec{u}_2$ are the two columns of $U$. Then we have
\begin{eqnarray*}
A\left[\begin{array}{ccc} \vec{v}_1 & \vec{v}_2 & \vec{v}_3 \end{array}\right]
&=& \left[\begin{array}{cc} \vec{u}_1 & \vec{u}_2 \end{array}\right]S\\
\left[\begin{array}{ccc} A\vec{v}_1 & A\vec{v}_2 & A\vec{v}_3 \end{array}\right]
&=& \left[\begin{array}{ccc} S_1\vec{u}_1 + 0\vec{u}_2 &
0\vec{u}_1 + S_2 \vec{u}_2 & 0 \vec{u}_1 + 0\vec{u}_2 \end{array}\right] \\
&=& \left[\begin{array}{ccc} S_1\vec{u}_1 & S_2 \vec{u}_2 &
0 \end{array}\right]
\end{eqnarray*}
which implies that $A\vec{v}_1=S_1\vec{u}_1 = 4\vec{u}_1$ and
$A\vec{v}_2=S_2\vec{u}_2 = \sqrt 6 \vec{u}_2$. Thus,
\[ \vec{u}_1 = \frac{1}{4}A\vec{v}_1
= \frac{1}{4}
\left[\begin{array}{rrr} 1 & -1 & 3 \\ 3 & 1 & 1 \end{array}\right]
\frac{1}{\sqrt{2}}\left[\begin{array}{r} 1\\ 0\\ 1 \end{array}\right]
= \frac{1}{4\sqrt 2}\left[\begin{array}{r} 4\\ 4 \end{array}\right]
= \frac{1}{\sqrt 2}\left[\begin{array}{r} 1\\ 1 \end{array}\right]\]
and
\[ \vec{u}_2 = \frac{1}{\sqrt 6}A\vec{v}_2
= \frac{1}{\sqrt 6}
\left[\begin{array}{rrr} 1 & -1 & 3 \\ 3 & 1 & 1 \end{array}\right]
\frac{1}{\sqrt{3}}\left[\begin{array}{r} -1\\ -1\\ 1 \end{array}\right]
=\frac{1}{3\sqrt 2}\left[\begin{array}{r} 3\\ -3 \end{array}\right]
=\frac{1}{\sqrt 2}\left[\begin{array}{r} 1\\ -1 \end{array}\right]
\]
Therefore,
\[ U=\frac{1}{\sqrt{2}}\left[\begin{array}{rr} 1 & 1 \\
1 & -1 \end{array}\right]\]
and
\begin{eqnarray*}
A & = & \left[\begin{array}{rrr} 1 & -1 & 3 \\ 3 & 1 & 1 \end{array}\right]\\
& = & \left(\frac{1}{\sqrt{2}}\left[\begin{array}{rr} 1 & 1 \\
1 & -1 \end{array}\right]\right)
\left[\begin{array}{rrr} 4 & 0 & 0 \\
0 & \sqrt 6 & 0 \end{array}\right]
\left(\frac{1}{\sqrt{6}}\left[\begin{array}{rrr}
\sqrt 3 & 0 & \sqrt 3  \\
-\sqrt 2 & -\sqrt 2 & \sqrt2 \\
-1 & 2 & 1 \end{array}\right]\right)
\end{eqnarray*}
\end{explanation}
\end{example}
 
Consider another example.
 
\begin{example}\label{SVDanother2x3}
Find an SVD for the matrix
\begin{equation*}
A= \left[
\begin{array}{ccc}
\frac{2}{5}\sqrt{2}\sqrt{5} & \frac{4}{5}\sqrt{2}\sqrt{5} & 0 \\
\frac{2}{5}\sqrt{2}\sqrt{5} & \frac{4}{5}\sqrt{2}\sqrt{5} & 0
\end{array}
\right]
\end{equation*}
 
\begin{explanation}
First consider $A^TA$
\begin{equation*}
\left[
\begin{array}{ccc}
\frac{16}{5} & \frac{32}{5} & 0 \\
\frac{32}{5} & \frac{64}{5} & 0 \\
0 & 0 & 0
\end{array}
\right]
\end{equation*}
What are some eigenvalues and eigenvectors? Some computing shows that the eigenvalues are $0$ and $16$.  Furthermore, we can find a basis for each eigenspace.
 
\begin{equation*}
\mathcal{S}_0=\mbox{span}\left( \left[
\begin{array}{c}
0 \\
0 \\
1
\end{array}
\right] ,\left[ 
\begin{array}{c}
-\frac{2}{5}\sqrt{5} \\
\frac{1}{5}\sqrt{5} \\
0
\end{array}
\right] \right),
\quad\mathcal{S}_{16}=\mbox{span}\left( \left[ 
\begin{array}{c}
\frac{1}{5}\sqrt{5} \\
\frac{2}{5}\sqrt{5} \\
0
\end{array}
\right] \right)
\end{equation*}
Thus the matrix $V$ is given by
\begin{equation*}
V=\left[
\begin{array}{ccc}
\frac{1}{5}\sqrt{5} & -\frac{2}{5}\sqrt{5} & 0 \\
\frac{2}{5}\sqrt{5} & \frac{1}{5}\sqrt{5} & 0 \\
0 & 0 & 1
\end{array}
\right]
\end{equation*}
Next consider $AA^T$
\begin{equation*}
\left[
\begin{array}{cc}
8 & 8 \\
8 & 8
\end{array}
\right]
\end{equation*}
Eigenvalues are $0$ and $16$, and eigenspaces are
 
\begin{equation*}
\mathcal{S}_0=\mbox{span}\left(\left[ 
\begin{array}{c}
-\frac{1}{2}\sqrt{2} \\
\frac{1}{2}\sqrt{2}
\end{array}
\right] \right),\quad\mathcal{S}_{16}=\mbox{span}\left( \left[ 
\begin{array}{c}
\frac{1}{2}\sqrt{2} \\
\frac{1}{2}\sqrt{2}
\end{array}
\right] \right)
\end{equation*}
Thus you can let $U$ be given by
\begin{equation*}
U=\left[ 
\begin{array}{cc}
\frac{1}{2}\sqrt{2} & -\frac{1}{2}\sqrt{2} \\
\frac{1}{2}\sqrt{2} & \frac{1}{2}\sqrt{2}
\end{array}
\right]
\end{equation*}
Let's check this. $U^TAV=$
\begin{equation*}
\left[ 
\begin{array}{cc}
\frac{1}{2}\sqrt{2} & \frac{1}{2}\sqrt{2} \\
-\frac{1}{2}\sqrt{2} & \frac{1}{2}\sqrt{2}
\end{array}
\right] \left[  
\begin{array}{ccc}
\frac{2}{5}\sqrt{2}\sqrt{5} & \frac{4}{5}\sqrt{2}\sqrt{5} & 0 \\
\frac{2}{5}\sqrt{2}\sqrt{5} & \frac{4}{5}\sqrt{2}\sqrt{5} & 0
\end{array}
\right] \left[
\begin{array}{ccc}
\frac{1}{5}\sqrt{5} & -\frac{2}{5}\sqrt{5} & 0 \\
\frac{2}{5}\sqrt{5} & \frac{1}{5}\sqrt{5} & 0 \\
0 & 0 & 1
\end{array}
\right]
\end{equation*}
\begin{equation*}
=\left[
\begin{array}{ccc}
4 & 0 & 0 \\
0 & 0 & 0
\end{array}
\right]
\end{equation*}
\end{explanation}
\end{example}
 
 You will do one more example in Chapter 9, after learning about standard processes for finding orthonormal bases from a list of linearly independent vectors. 

 Finally, let's re-visit the use of SVD to find the best-fitting subspaces for the data given in $\texttt{+linalg/subspace\_fitting\_affine\_data.mat}$, depicted here:

 \begin{example}\name{Affine subspace fitting problem}\label{ex:affine-subspace-fitting}
    Consider the following collection of points in $\R^2$:
    \begin{equation*}
      \set{
        \startmat{r} 10 \\ -6 \stopmat,
        \startmat{r} 2 \\ 10 \stopmat,
        \startmat{r} 5 \\ -1 \stopmat,
        \startmat{r} 8 \\ 3 \stopmat,
        \startmat{r} 2 \\ 5 \stopmat,
        \startmat{r} 3 \\ 3 \stopmat,
        \startmat{r} 4 \\ 11 \stopmat,
        \startmat{r} 10 \\ -1 \stopmat,
        \startmat{r} 1 \\ 12 \stopmat
      }.
    \end{equation*}
    Find the 1-dimensional affine subspace that best approximates this
    collection of points. What is the total squared distance of the
    points to the subspace?
  \end{example}
  
  \begin{solution}
    We start by computing the centroid:
    \begin{equation*}
      \centroid{\vect{v}} =
      \frac{1}{9}(\vect{v}_1+\ldots+\vect{v}_9)
      = \frac{1}{9}\startmat{r} 45 \\ 36 \stopmat
      = \startmat{r} 5 \\ 4 \stopmat.
    \end{equation*}
    Next, we shift all vectors by $-\centroid{\vect{v}}$ to get a new
    collection of vectors $\vect{w}_1,\ldots,\vect{w}_9$ centered at the
    origin.
    For example,
    \begin{eqnarray*}
      \vect{w}_1 ~~=~~ \vect{v}_1 - \centroid{\vect{v}}
      &=& \startmat{r} 10 \\ -6 \stopmat
      - \startmat{r} 5 \\ 4 \stopmat
      ~~=~~ \startmat{r} 5 \\ -10 \stopmat,
      \\
      \vect{w}_2 ~~=~~ \vect{v}_2 - \centroid{\vect{v}}
      &=& \startmat{r} 2 \\ 10 \stopmat
      - \startmat{r} 5 \\ 4 \stopmat
      ~~=~~ \startmat{r} -3 \\ 6 \stopmat,
    \end{eqnarray*}
    and so on. We get
    \begin{equation*}
      \set{\vect{w}_1,\ldots,\vect{w}_9} =
      \set{
        \startmat{r} 5 \\ -10 \stopmat,
        \startmat{r} -3 \\ 6 \stopmat,
        \startmat{r} 0 \\ -5 \stopmat,
        \startmat{r} 3 \\ -1 \stopmat,
        \startmat{r} -3 \\ 1 \stopmat,
        \startmat{r} -2 \\ -1 \stopmat,
        \startmat{r} -1 \\ 7 \stopmat,
        \startmat{r} 5 \\ -5 \stopmat,
        \startmat{r} -4 \\ 8 \stopmat
      }.
    \end{equation*}
    Next we, we proceed by finding $A^T$. We have

    \begin{equation*}
      A^T = \startmat{rrrrrrrrr}
        5 & -3 & 0 & 3 & -3 & -2 & -1 & 5 & -4 \\
        -10 & 6 & -5 & -1 & 1 & -1 & 7 & -5 & 8 \\
      \stopmat
    \end{equation*}
    and
    \begin{equation*}
      B = A^TA = \startmat{rr}
        98 & -136 \\
        -136 & 302 \\
      \stopmat.
    \end{equation*}

    The eigenvalues of $B$ are $\eigenvar_1 = 370$ and $\eigenvar_2 =
    30$, with corresponding eigenvectors
    \begin{equation*}
      \vect{u}_1 = \startmat{r} 1 \\ -2 \stopmat
      \quad\mbox{and}\quad
      \vect{u}_2 = \startmat{r} 2 \\ 1 \stopmat
    \end{equation*}

    For simplicity in this example, we won't proceed to normalize the singular vectors and will simply maintain the squared singular values (the first of which gives the total square spread of the data on the subspace). 
    
    Thus, the best-fitting 1-dimensional subspace for
    $\vect{w}_1,\ldots,\vect{w}_9$ is $W = \sspan\set{\vect{u}_1}$, and
    the best-fitting 1-dimensional affine subspace for
    $\vect{v}_1,\ldots,\vect{v}_9$ is
    \begin{equation*}
      \centroid{\vect{v}} + W
      = \set{\centroid{\vect{v}} + \vect{w} \mid \vect{w}\in W}
      = \set{\left.\startmat{r} 5 \\ 4 \stopmat +
          t\startmat{r} 1 \\ -2 \stopmat ~\right\vert~ t\in\R}.
    \end{equation*}
    Note that this is the equation of a line passing through the
    centroid $\centroid{\vect{v}}$, and with direction vector
    $\vect{u}_1$. The points $\vect{v}_1,\ldots,\vect{v}_9$, their
    centroid, and the affine subspace $\centroid{\vect{v}} + W$ are
    shown in the following illustration:
    \begin{center}
      \begin{tikzpicture}[scale=0.2]
        \draw[thin,->] (-12,0) -- (12,0);
        \draw[thin,->] (0,-12) -- (0,12);
        \draw[thin] (0,10) -- (-1,10) node[left] {$10$};
        \draw[thin] (10,0) -- (10,-1) node[below] {$10$};
        \draw[thick,blue] (5,4) +(-6,12) -- +(6,-12);
        \fill[color=blue] (5,4) circle (0.6);
        \draw[blue, ->] (5,4) +(8,4) node[right,yshift=4] {centroid} -- +(0.4,0.2);
        \fill[color=red] (10, -6) circle (0.3);
        \fill[color=red] (10, -1) circle (0.3);
        \fill[color=red] (5, -1) circle (0.3);
        \fill[color=red] (8, 3) circle (0.3);
        \fill[color=red] (3, 3) circle (0.3);
        \fill[color=red] (2, 5) circle (0.3);
        \fill[color=red] (2, 10) circle (0.3);
        \fill[color=red] (4, 11) circle (0.3);
        \fill[color=red] (1, 12) circle (0.3);
      \end{tikzpicture}
    \end{center}
    \vspace{-4ex}\par  
  \end{solution}


\end{document}