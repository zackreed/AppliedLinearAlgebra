\documentclass{ximera}
\graphicspath{     %% setup a global graphics path
{./}               %% look in the same-level directory
{./pictures/}      %% look in graphics
{../pictures/}     %% look up one directory, then in graphics
%{../../pictures/} %% look up two directories, then in graphics
}

\author{Zack Reed} %borrowed from Bart and PEter Selinger
\title{Matrices are Everywhere!}
\begin{document}
\begin{abstract}
Here we introduce matrices similar to vectors
\end{abstract}
\maketitle


\section{Matrices store and transform data}

Keeping with the convention that \emph{vectors} are arrays of numbers that represent data, we often want to consider not just individual vectors at one time, but collections of vectors. \emph{matrices} allow us to methodologically store and manipulate data vectors.

We'll start with some intuition and basics for now, and then next chapter we'll go into much greater detail about the geometric connections between matrices and vectors.

First, a very basic definition or two:

\begin{remark}

\begin{definition}

A \emph{matrix} is just a big rectangular array of numbers
\[
M =
\underset{\displaystyle\boldsymbol{5}~\textbf{columns}}{\begin{pmatrix}
  a_{1,1} & a_{1,2} & a_{1,3} & a_{1,4} & a_{1,5} \\
  a_{2,1} & a_{2,2} & a_{2,3} & a_{2,4} & a_{2,5} \\
  a_{3,1} & a_{3,2} & a_{3,3} & a_{1,4} & a_{3,5} \\
  a_{4,1} & a_{4,2} & a_{4,3} & a_{4,4} & a_{4,5}
\end{pmatrix}}
\boldsymbol{4}~\textbf {rows}
\]

We give the \emph{dimensions of a matrix} by stating its number of rows
and columns. Rows come first and columns come second, so $M$ above is
a $(4\times 5)$-matrix. We can use similar notation to talk about
specific entries of a matrix. Above, $a_{i,j}$ is the
$\boldsymbol{(i,j)}${\bf-}\emph{entry} of the matrix $M$. Sometimes,
people write $M_{i,j}$ to mean the $(i,j)$-entry of the matrix $M$.

\end{definition}

For instance, the following is a $4\times 5$ matrix:

\[
M =
\underset{\displaystyle\boldsymbol{5}\textbf{columns}}{
\begin{pmatrix}
4.56 & -2.32 & 3.12 & 0.78 & 1.92 \\
7.13 & 1.25 & -0.67 & 2.89 & -1.05 \\
3.45 & 8.62 & 0.34 & 6.78 & -9.12 \\
2.98 & -5.44 & 3.91 & 4.23 & 0.15
\end{pmatrix}}
\boldsymbol{4}\textbf{rows}
\]

\begin{definition}

    We might also say that a matrix is an \emph{array of vectors}. 
    
    That is, we can think of vectors as either being the rows or columns of a matrix, and thus the matrix is an array of these vectors.

\end{definition}

Because of this, we might at times specify vectors in reference to their position in a matrix, by identifying whether we care to think of the rows as vectors, or the columns as vectors, or both. This often gives the matrix some interpretability.

\begin{example}

    If we consider the columns of $M$ as vectors, we might write the columns of matrix $M$ as: 

    \[
v_1 = \begin{pmatrix} 4.56 \\ 7.13 \\ 3.45 \\ 2.98 \end{pmatrix}, \quad
v_2 = \begin{pmatrix} -2.32 \\ 1.25 \\ 8.62 \\ -5.44 \end{pmatrix}, \quad
v_3 = \begin{pmatrix} 3.12 \\ -0.67 \\ 0.34 \\ 3.91 \end{pmatrix}, \quad
v_4 = \begin{pmatrix} 0.78 \\ 2.89 \\ 6.78 \\ 4.23 \end{pmatrix}, \quad
v_5 = \begin{pmatrix} 1.92 \\ -1.05 \\ -9.12 \\ 0.15 \end{pmatrix}
\]

where the matrix is the array of vectors

    \[
M =
\left[\begin{array}{ccccc}
  v_1 & v_2 & v_3 & v_4 & v_5
\end{array}
\right]
\].

Alternatively, if we take the rows of the matrix, we can write $M$ as:

\[
M =
\left[\begin{array}{c}
  r_1 \\
  r_2 \\
  r_3 \\
  r_4
\end{array}
\right]
\]

where

\[
r_1 = \begin{pmatrix} 4.56 & -2.32 & 3.12 & 0.78 & 1.92 \end{pmatrix}, \quad
r_2 = \begin{pmatrix} 7.13 & 1.25 & -0.67 & 2.89 & -1.05 \end{pmatrix}, \quad \]
\[
r_3 = \begin{pmatrix} 3.45 & 8.62 & 0.34 & 6.78 & -9.12 \end{pmatrix}, \quad
r_4 = \begin{pmatrix} 2.98 & -5.44 & 3.91 & 4.23 & 0.15 \end{pmatrix}
\]\


\end{example}

Either way, the matrix $M$ has $4$ rows and $5$ columns, and so we say that $M$ is a $(4\times 5)$-matrix.

\end{remark}

\begin{exploration}

    The distinction between matrices storing rows of vectors and columns of vectors motivates the following, at first simple, characterizations:

    \begin{description}
        \item[As a row vector] A row vector is a vector whose entries have been specifically arranged in row format.
        
        Suppose a bookstore sells a variety of books in a month, say
          $141$ Science Fiction,
          $304$ Fantasy,
          $249$ Mystery,
          $199$ Romance,
          $251$ Historical.
          We can express this data as a row vector:
          \[
          \vec{s} = \begin{pmatrix}141 & 304 & 249 & 199 & 251 \end{pmatrix}
          \]
        \item[As a column vector] A column vector is just like a row vector,
          except is it written vertically:
          \[
          \vec{w}_{\texttt{TR}} = \begin{pmatrix}
            0.5\\ 4 \\ 0 \\ 1\end{pmatrix}
            \qquad
            \begin{array}{l}
            \text{Emails}\\
            \text{Classes}\\
            \text{Projects}\\
            \text{Exercising}
          \end{array}
          \]
          Above, we could suppose that $\vec{w}_{\texttt{TR}}$ represents the
          fact that on Tuesdays and Thursdays, someone spends $0.5$ hours on
          emails, $4$ hours in class, $0$ hours on projects, and $1$ hour
          exercising.
        \end{description}
        
        
        Of course, since mathematics is a human endeavor you will find variations on the notation above, as a common example, some folks use different brackets like these
        \[
        \langle a, b, c\rangle \qquad\text{or}\qquad
        \begin{bmatrix}
          a\\
          b\\
          c
        \end{bmatrix}
        \]
        for an ordered-tuple vectors and column vectors. Ordered tuples and
        row vectors are more convient to write in a line, because they are
        horizontal. On the other hand, column vectors take up less horizontal
        space, and are more convenient when you have vectors with many entries.
        
        
        \begin{definition}
          To switch a row vector into a column vector and vice versa, we use what is called the \emph{transpose} operation:
          \[
          \begin{pmatrix} 1 &  2 & 3 \end{pmatrix}^\transpose =
          \begin{pmatrix} 1 \\ 2 \\ 3 \end{pmatrix}
          \quad\text{and}\quad
          \begin{pmatrix} 1 \\ 2 \\ 3 \end{pmatrix}^\transpose =
          \begin{pmatrix} 1 &  2 & 3 \end{pmatrix}
          \]
        \end{definition}

        At times it will also be beneficial to re-orient a matrix by swapping its rows and columns. As with vectors, we will also call this a \emph{transpose} operation. 
        
        \begin{example}

            Below we give a $3\times 4$ matrix $A$ and its transpose $A^\transpose$:

            \[
A=\begin{pmatrix}
    2.12 & 4.56 & -1.32 & 3.78 \\
    -0.67 & 1.29 & 5.43 & -3.12 \\
    7.14 & -2.28 & 0.98 & 6.13
\end{pmatrix}
\quad \text{and} \quad
A^\transpose=\begin{pmatrix}
    2.12 & -0.67 & 7.14 \\
    4.56 & 1.29 & -2.28 \\
    -1.32 & 5.43 & 0.98 \\
    3.78 & -3.12 & 6.13
\end{pmatrix}
\]

The dimension of $A^\transpose$ is $\answer{4}\times \answer{3}$ because we swapped the rows for columns and vice versa.

        \end{example}

        \begin{question}
          Consider $\vec{s} = \begin{pmatrix}141 & 304 & 249 & 199 & 251 \end{pmatrix}$. Compute $\vec{s}^\transpose$.
          \begin{prompt}
          \[
          \vec{s}^\transpose  = \begin{pmatrix}\answer{141} \\ \answer{304} \\ \answer{249} \\ \answer{199} \\ \answer{251} \end{pmatrix}
          \]
          \end{prompt}
        \end{question}
        
        Once we have data represented as vectors encoded as ordered tuples,
        row vectors, or column vectors, we can describe some general
        information about the vectors. In particular, we can think about their
        \textit{dimension} and their \textit{components}.

        \begin{question}
          Which matrix below is a $3\times 2$ matrix?
          \begin{multipleChoice}
            \choice{
                $\begin{pmatrix}
                    1 & 2 & 3 \\
                    4 & 5 & 6 \\
                    7 & 8 & 9
                \end{pmatrix}$}
        
            \choice[correct]{
                $\begin{pmatrix}
                    1 & 2 \\
                    3 & 4 \\
                    5 & 6
                \end{pmatrix}$}
            
            \choice{
                $\begin{pmatrix}
                    1 & 2 & 3 \\
                    4 & 5 & 6
                \end{pmatrix}$}
            
        \end{multipleChoice}
        \end{question}
        
        $n$-dimensional row vectors are just $1\times n$ matrices and
        $n$-dimensional column vectors are just $n\times 1$ matrices.

\end{exploration}

\begin{exploration}\name{Images as Matrices}

  A manifestation of matrices in every day life is in the digital world. Any image or movie rendered on a computer screen is coming from a matrix (or for a movie, a sequence of matrices).
  
  Images thus give us a very intuitive way to make meaning from matrices, and we will repeatedly return to images throughout the course as a way to visually ground things we do with and to matrices.
  
  \begin{example}
  
    This is also a great opportunity for you to get started with MATLAB! MATLAB will be a major learning tool in this course, as we can offload many lengthy calculations to the computer, allowing us to quickly, efficiently, and consistently gain better insights into vectors and matrices.
  
    If you haven't yet installed MALTAB and got set up with the course folder and files, watch this video here to get started:
  
    %\youtube{https://www.youtube.com/watch?v=3Z1cZk5J2jw}
  
    Once you're set up, open MATLAB and make a new live script. Make sure that MATLAB is set to load from the same folder that contains the course folder, $\texttt{+linalg}$.
  
    All we'll do right now is load a simple $21\times 21$ image of a smiley face into MATLAB, display it, alter some values in the matrix to see the visual effect of matrix manipulation, and then you'll do more work with images in the homework and in the Chapter 2 mini-project.
  
    First, we load the image and take a look at the matrix:
  
    \begin{verbatim}
    
      load +linalg/smiley.mat
      smiley
  
    \end{verbatim}
  
    Notice that $\texttt{smiley}$ is a $21\times 21$ matrix. Its values vary between $0$ and $255$, where $0$ is black and $255$ is white. For now we'll stick to ``grayscale'' images, but color images can be thought of as three matrices stacked on top of each other.
  
    Now, because the image is quite small, there's a course-specific command that will render the image in a larger window. 
  
    \begin{verbatim}
    
      linalg.smiley_show(smiley)
  
    \end{verbatim}
  
    You should see the following figure render:
  
    \begin{center}
      \includegraphics{smiley_mat.png}
    \end{center}
  
    For now, all we'll do is change one of the rows of the image to be all white, and one of the columns to be all black. This can help us see that there's some alignment between matrix notation and how matrices are manipulated in software, and also reinforce the basic idea that matrices are just rows and/or columns of vectors.
  
    First, let's change the $11$th row to be all white. We do this by setting all the values in the $11$th row to $255$. In matrix notation, $11$ is the first index. In MATLAB, we use a colon ``\:'' specify that we want to take all possible columns for the $11$th row.
  
    \begin{verbatim}
    
      smiley(11,:) = 255;
  
    \end{verbatim}
  
    By saying that the $11$th row is equal to $255$, we re-assign the value of each entry to be $255$.
  
    Now let's change the $11$th column to be all black. Using very similar notation, we do the following:
  
    \begin{verbatim}
    
      smiley(:,11) = 0;
  
    \end{verbatim}
  
    Now if we render the figure again, we should see the following:
  
    \begin{center}
      \includegraphics{smiley_mat_changed.png}
    \end{center}
  
    As you can see, the middle row has become white, and the middle column has become black. This is a simple example of how we can manipulate images using matrices.
  
  \end{example}
  
  \end{exploration}

\begin{exploration}\name{Data Storing and Manipulation}

Now that we know the basics, onto the examples!

Matrices can be thought of as a ``mathematical spreadsheet.'' With a
spreadsheet you have rows and columns of data.  You can think of a
matrix a just a bunch of vectors stacked together either horizontally
or vertically.

 
\begin{example}[Population Counts] %https://worldpopulationreview.com/states/states-by-race
  The Midwest of the United States consists of $12$ states. We can
  express the $2023$
  \link[demographics]{https://worldpopulationreview.com/states/states-by-race}
  of each state as a vector represented by an ordered tuple. The
  ordered tuple for Ohio looks like:
  \[
  \vec{p}_{\texttt{OH}} = (\underset{\text{White}}{9394878},\underset{\text{Black}}{1442655},\underset{\text{American Indian}}{20442},\underset{\text{Asian}}{268527},\underset{\text{Hawaiian}}{3907},\underset{\text{Other}}{544866}).
  \]
  The ordered tuples for each state in the Midwest looks like:
\begin{align*}
  \vec{p}_{\texttt{IA}} &= (2806418,117035,10538,79296,3941,132783)\\
  \vec{p}_{\texttt{IL}} &= (8874067,1796660,33972,709567,5196,1296702)\\
  \vec{p}_{\texttt{IN}} &= (5510354,631923,14030,158705,2205,379676)\\
  \vec{p}_{\texttt{KA}} &= (2416165,165837,22278,87093,2344,218902)\\
  \vec{p}_{\texttt{MI}} &= (7735902,1360149,50035,316844,3117,507860)\\
  \vec{p}_{\texttt{MN}} &= (4572149,359817,54558,275242,2201,336199)\\
  \vec{p}_{\texttt{MO}} &= (4978046,698043,24274,123810,8887,291100)\\
  \vec{p}_{\texttt{ND}} &= (651470,23959,39165,11979,1004,32817)\\
  \vec{p}_{\texttt{NE}} &= (1641256,91896,16875,47944,1235,124620)\\
  \vec{p}_{\texttt{OH}} &= (9394878,1442655,20442,268527,3907,544866)\\
  \vec{p}_{\texttt{SD}} &= (735228,18836,74975,12413,544,37340)\\
  \vec{p}_{\texttt{WI}} &= (4895065,367889,48674,163396,2672,329279)
\end{align*}

Putting these into a matrix, we have: 

\[
  \begin{pmatrix}
  \vec{p}_{\texttt{IA}} \\
  \vec{p}_{\texttt{IL}} \\
  \vec{p}_{\texttt{IN}} \\
  \vec{p}_{\texttt{KA}} \\
  \vec{p}_{\texttt{MI}} \\
  \vec{p}_{\texttt{MN}} \\
  \vec{p}_{\texttt{MO}} \\
  \vec{p}_{\texttt{ND}} \\
  \vec{p}_{\texttt{NE}} \\
  \vec{p}_{\texttt{OH}} \\
  \vec{p}_{\texttt{SD}} \\
  \vec{p}_{\texttt{WI}}
  \end{pmatrix}
  =
  \begin{pmatrix}
    \text{White} & \text{Black} & \text{American Indian} & \text{Asian} & \text{Hawaiian} & \text{Other} \\
  2806418 & 117035 & 10538 & 79296 & 3941 & 132783\\
  8874067 & 1796660 & 33972 & 709567 & 5196 & 1296702\\
  5510354 & 631923 & 14030 & 158705 & 2205 & 379676\\
  2416165 & 165837 & 22278 & 87093 & 2344 & 218902\\
  7735902 & 1360149 & 50035 & 316844 & 3117 & 507860\\
  4572149 & 359817 & 54558 & 275242 & 2201 & 336199\\
  4978046 & 698043 & 24274 & 123810 & 8887 & 291100\\
  651470 & 23959 & 39165 & 11979 & 1004 & 32817\\
  1641256 & 91896 & 16875 & 47944 & 1235 & 124620\\
  9394878 & 1442655 & 20442 & 268527 & 3907 & 544866\\
  735228 & 18836 & 74975 & 12413 & 544 & 37340\\
  4895065 & 367889 & 48674 & 163396 & 2672 & 329279
  \end{pmatrix}
  \]

  We'll call this matrix $P$. By manipulating entries, rows, or columns of $P$, we can answer many questions about the data (even before doing more advanced linear algebraic methods).

  Before you do anything else, open up MATLAB and make a new live script. Load the data into MATLAB with 

  \begin{verbatim}
  
    load +linalg/midwest_2023.mat

  \end{verbatim}

  You should have access to the matrix \texttt{P}, and then the lists of labels as well.

  Using \texttt{state\_labels} and \texttt{demographic\_labels}, you can see the row and column numbers of the labels.

  \begin{definition}\name{Notation Alert}

    This is a good opportunity to introduce more notation. If we want a specific entry in a matrix, we'll either say $P_{i,j}$ or $P(i,j)$ to mean the $(i,j)$-entry of the matrix $P$. Remember that the first number is the row number and the second number is the column number.

    So $P_{3,5}$ is the entry in the third row and fifth column of $P$, which is $2205$, the number of Hawaiian people living in Indiana in $2023$.

    When we want to specify a range of rows and/or columns at once, we use colons ``:'' to separate the starting and ending row or column. For example, $P_{2:4,3}$ means the entries in the second, third, and fourth rows of the third column of $P$, which are the American Indian population counts for Illinois, Indiana, and Kansas in $2023$.

    If we want to specify an entire row or column, we can use a colon by itself. For example, $P_{5,:}$ means the entire fifth row of $P$, which is the demographic data for Michigan in $2023$. 

    Finally, if we use just a single colon, that creates a $nx1$ vector of all the entries in the entire matrix, starting with the first row and moving down. 

  \end{definition}

\begin{enumerate}
\item What is the demographic submatrix of Indiana, Minnesota, and Wisconsin?
\item What is the submatrix of the Midwest that includes only the
  demographics of Black, American Indian, and Hawaiian?
\item What are the combined demographics of the states Michigan, Ohio,
  and Indiana?
\item Suppose that the annual percentage growth rate of the Midwest is
  currently $0.1\%$. Assuming this is even across all demographics,
  what would be the 2024 population matrix?
\item What are the average demographics of the Midwest?
\item What is each state's average demographic? 
\end{enumerate}

\begin{hint}

  If you're having trouble with the first two questions, check that you're not mixing up rows and columns, and that you're using the correct notation for submatrices.

\end{hint}

\begin{solution}
  \begin{enumerate}
  \item We need to put Indiana, Minnesota, and Wisconsin into a matrix. These are the third, sixth, and twelfth rows of $P$. Since none of them are connected, we need to pull out each individual rows and stack them together. We do this using the notation (I'll start and you finish):

    \[
    \texttt{P(}\answer[given]{3}\texttt{,:); P(}\answer[given]{6}\texttt{,:);}\answer[given, format=string]{P(12,:)}\texttt{]} 
    \]
    \item To find the submatrix of the Midwest that includes only the
      demographics of Black, American Indian, and Hawaiian, we similarly use the notation:

      \[
      \texttt{P(:,}\answer{2}\texttt{); P(:,}\answer{3}\texttt{);}\answer[format=string]{P(:,5)}\texttt{]}
      \]
  \item To find the combined demographics of Michigan, Ohio, and
    Indiana we treat the rows as vectors and add them together. You can do this a couple of ways, such as first defining each as a new vector and then adding them together, or by doing it all at once. For clarity, I'll first make them all vectors:

      \begin{Verbatim}
      
        p_MI = P(5,:);
        p_OH = P(10,:);
        p_IN = P(3,:);
        p_MI + p_OH + p_IN

      \end{Verbatim}

      If you entered this correctly, you should get the following:

    \[
    \vec{p}_{\texttt{MI}} + \vec{p}_{\texttt{OH}} + \vec{p}_{\texttt{IN}} = (\answer[given]{22641134},\answer[given]{3434727},\answer[given]{84507},\answer[given]{744076},\answer[given]{9229},\answer[given]{1432402})
    \]
  \item We want to define a new matrix for 2024 let's call it $\texttt{P\_2024}$.
  
  To find the population growth matrix, we can treat each row/column as if they were vectors, so we use the scalar multiplication, but we can do it on the entire matrix at once. 

  \begin{verbatim}
  
    .001*P

  \end{verbatim}

  Then, again since a matrix is an array of vectors, we can add matrices of the same size together.

    \begin{Verbatim}

      P_2024 = P + 0.001*P;

    \end{Verbatim}

    If you entered this correctly, you the Hawaiian population of South Dakota in 2024 should be $\answer{544.544}$, found at $P_{11,5}$.

\item Lucky for data anlysts everywhere, averaging is a linear combination, since it involves just adding and scaling vectors. In this case, we want to average across the states, so we treat the rows as vectors and find the average across the rows, returning a $1\times 6$ vector with the average for each demographic. 

The $\texttt{mean}$ function in MATLAB will do this for you, and you specify averaging across the rows or columns by using a $1$ or $2$ in the second argument. 

Using: 

\begin{verbatim}
  average_midwest = mean(P,1);
\end{verbatim}

You should get the following:

(Note, round up to the nearest hundred thousand, so $2.1$ means rounded to $2,100,000$)

\[
\texttt{average\_midwest} = [\answer{4.5},\answer{.6},\answer{0},\answer{0.2},\answer{0},\answer{0.4}]
\]

\item Finally, to find the average demographic of each state, we treat the columns as vectors and find the average across the columns, returning a $12\times 1$ vector with the average for each state. We do this by putting a $2$ in the second argument of the $\texttt{mean}$ function.

Using:

\begin{verbatim}
  average_state = mean(P,2);
\end{verbatim}

You should get the following:

\[
\texttt{average\_state} = \begin{pmatrix}
  \answer{.5}\\
  \answer{2.1}\\
  \answer{1.1}\\
  \answer{.5}\\
  \answer{1.7}\\
  \answer{.9}\\
  \answer{1}\\
  \answer{.1}\\
  \answer{.3}\\
  \answer{1.9}\\
  \answer{.1}\\
  \answer{1}
\end{pmatrix}
\]


  \end{enumerate}
\end{solution}
\end{example}

\end{exploration}
 
\begin{exploration}\name{Matrix Properties and Operations}

\begin{remark}

  Because it is productive to think about matrices as arrays of vectors, we can take linear combinations of matrices in the same way we do with vectors. This is a powerful tool for data manipulation and analysis, and we'll gain even more useful relationships between matrices and vectors in the next chapter.

  For now, some definitions to make more precise the idea of a linear combination of matrices:

  \begin{definition}\name{Addition of matrices}
    Let $A$ and $B$ be two
    $m\times n$-matrices. Then $A+B=C$%
    \index{matrix!addition}%
    \index{sum|see{addition}}%
    \index{addition!of matrices} where $C$ is the $m\times n$-matrix
    $C$ defined by
    \begin{equation*}
      C_{ij}=A_{ij}+B_{ij}
    \end{equation*}

    Said differently, you add matrices by adding their corresponding entries.

    (IMPORTANT NOTE: You can only add matrices of the same size. If you have missing data, you need to make a decision on how to fill it in before you can add matrices together.)
  \end{definition}

  \begin{definition}\name{Scalar multiplication of matrices}
    Let $A$ be an $m\times n$-matrix and let $c$ be a scalar. Then $cA=D$%
    \index{matrix!scalar multiplication}%
    \index{scalar multiplication!of matrices} where $D$ is the $m\times n$-matrix
    $D$ defined by
    \begin{equation*}
      D_{ij}=cA_{ij}
    \end{equation*}

    Said differently, you multiply a matrix by a scalar by multiplying each entry of the matrix by the scalar.
  \end{definition}

\end{remark}

Let's finish by practicing on some basic matrices. If possible, perform the specified linear combinations.

\[
A = \begin{pmatrix}
2 & -1 & 0 \\
4 & 3 & -2 \\
\end{pmatrix}, \quad
B = \begin{pmatrix}
0 & 2 \\
-1 & 3
\end{pmatrix}, \quad
C = \begin{pmatrix}
4 & 0 & 1 \\
-2 & 2 & 5
\end{pmatrix}, \quad
D = \begin{pmatrix}
1 & -3 \\
2 & 1 \\
0 & 4
\end{pmatrix}
\]

\begin{enumerate}
\item Find $2A+3B$.
\begin{solution}

  $2A+3B$ is \wordChoice{
    \choice{possible}
    \choice[correct]{not possible}
  } because the matrices \wordChoice{
    \choice[correct]{are not}
    \choice{are}
  } the same size.

\end{solution}
\item Find $-2C+5D$.

\begin{solution}

  $-2C+5D$ is \wordChoice{
    \choice{possible}
    \choice[correct]{not possible}
  } because the matrices \wordChoice{
    \choice[correct]{are not}
    \choice{are}
  } the same size.

\begin{problem}

  If we instead took the transpose of $C$, then 
  
  $$-2C^T+5D=\begin{pmatrix} \answer{-3} & -11 \\ 10 & \answer{1} \\ \answer{-2} & 10 \end{pmatrix}.$$

\end{problem}

\end{solution}

\item Find $3A+5C-1D$.

\begin{solution}

  $3A+5C-1D$ is \wordChoice{
    \choice{possible}
    \choice[correct]{not possible}
  } because the matrices \wordChoice{
    \choice{are}
    \choice[correct]{are not}
  } the same size.

  \begin{problem}

    If we instead took the transpose of $D$, then 
    
    $$3A+5C-1D^T=\begin{pmatrix} 25 & \answer{-5} & \answer{5} \\ 5 & \answer{18} & 15 \end{pmatrix}.$$

  \end{problem}

\end{solution}

\end{enumerate}

\end{exploration}

\end{document}