\documentclass{ximera}

\graphicspath{     %% setup a global graphics path
{./}               %% look in the same-level directory
{./pictures/}      %% look in graphics
{../pictures/}     %% look up one directory, then in graphics
%{../../pictures/} %% look up two directories, then in graphics
}

\author{Zack Reed}
%%Snapp, B. (2024). Ximera la-carte: An open-source platform for interactive textbooks. https://ximera.osu.edu
%%OpenAI. (2023). ChatGPT (Mar 14 version) [Large language model]. https://chat.openai.com/chat

\title{Learning Activity: Adding and Scaling}

\begin{document}
\begin{abstract}
    An introduction to how vectors provide a lens for understanding the world around us.
\end{abstract}
\maketitle

\begin{exploration}\name{Adding and Scaling}

  Wrapping up this initial discussion of vectors, let's consider other quantities for which it does or does not make sense to \emph{add} and \emph{scale}.

  \begin{example}\name{When adding doesn't make sense}

\begin{description}
\item[Personal Identification Numbers] Summing things like social
  security numbers, phone numbers, or zip codes doesn't provide
  meaningful information.
\item[Temperatures] While you can mathematically add temperatures
  together, doing so usually doesn't provide useful information.
\item[Time] Adding specific points in time, like dates or hours of the
  day, usually doesn't yield meaningful results.
\item[Geographic Coordinates] Adding the latitude and longitude of two
  locations doesn't give you a location that has any real-world
  significance.
\end{description}
\end{example}

For some of the categories above, the \textit{average} can be
meaningful; or perhaps if each quantity is thought of as
`displacement' or `duration,' summing might be meaningful. However,
without additional (pun intended!) stipulations, summing the types of
quantities above is not meaningful.

\begin{example}

On the other hand, there are lots of
times that it makes sense to add data:


\begin{description}
\item[Population Counts] Adding the populations of different regions,
  cities, or countries to find the total population of a larger
  area. Answers to questions like these help us understand our
  society and are important to many people.
\item[Financial Transactions] Summing daily sales to find total
  monthly sales, or adding up all expenses to find total costs. These
  are real numbers that represent actual amounts of money, and their
  total gives meaningful information about financial status or
  performance.
\item[Time Spent on Tasks] Adding the duration of time spent on
  various activities on day gives a total time spent. We all have busy
  lives, and time is a very precious commodity. Hence, it is good to
  understand how we spend our time.
\item[Navigation] Aircraft navigate by knowing the direction and speed
  that they are traveling. When interpreted correctly, we add the
  speed and direction of many of many different ``course changes'' to
  find the ultimate position of the aircraft.
\end{description}

\end{example}

Often when it make sense to \textit{add} quantities, it make sense to
\textit{scale} them as well. For example with population, you could
ask for the population of a city with $3$ times the population, or
half the population. Similar \textit{scaling} examples exist for all
the examples above where it make sense to add data.

\end{exploration}

In the next section, we introduce the idea of a \emph{matrix}, which for now we will roughly discuss as a specific way to organize multiple data vectors together. This will be the first step in understanding how to manipulate and analyze data in a more sophisticated way than we have so far.





\end{document}